%=================================================================
% VERSÃO COMPLETA
%=================================================================
\songcolumns{1}
\beginsong
{Em Espírito, Em Verdade\\Meu prazer %TÍTULO
}[by={Ministério Koinonya de Louvor %ARTISTA
},album={@walyssondosreis},
id={GB0032 %COD.ID.: GB0000
}] 
%-----------------------------------------------------------------
\tom{X0}
%=================================================================
%\newchords{verse0.GB0000} % Registrador de Acordes em Sequência
%-----------------------------------------------------------------
%\seq{Intro}{}{}
%-----------------------------------------------------------------
%\beginverse* \endverse
%\beginchorus \endchorus

\beginverse*
\[X1V9]Em espírito, em ver\[X5V8]dade
Te ado\[X4V9V8]ramos, \[X4V9V13] te ado\[X1V9]ramos \[X5V7V13]
\[X1V9]Em espírito, em ver\[X5V8]dade
Te ado\[X4V9V8]ramos, \[X4V9V13] te ado\[X1V9]ramos \[X5V8]
\endverse

\beginverse*
Rei dos \[X6]reis \[X6V2V13] e se\[X1V13]nhor \[X4V10V1V7V13]
Te entre\[X4V9]gamos \[X5V7V13]nosso vi\[X1V9]ver \[X5V8]
Rei dos \[X6]reis \[X6V2V13] e se\[X1V13]nhor \[X4V10V1V7V13]
Te entre\[X4V9]gamos \[X2V7]nosso vi\[X4V9]ver \[X4V9]
\endverse

\beginchorus 
Pra te ado\[X4V9]rar oh! \[X2]Rei dos \[X5V4]reis \[X4V9]
Foi que eu nas\[X4V9V8]ci oh! \[X2]Rei Je\[X5V8]sus!
Meu pra\[X3V2V7]zer é te lou\[X6]var
Meu pra\[X4V9]zer é estar\[X4V9]
Nos \[X4V9]átrios do se\[X1V9]nhor
Meu pra\[X5V8]zer é vi\[X6]ver
Na \[X4V9]casa de \[X4V9]Deus
Onde \[X4V9]flui o a\[X1V9]mor \[(X4V9)]
\endchorus

%-----------------------------------------------------------------
\begin{comment}
\lstset{basicstyle=\scriptsize\bf} % Parâmetros da TAB
%-----------------------------------------------------------------
\tab{Solo 1}
\begin{lstlisting}
E|-----------------------------------------------------|
B|-----------------------------------------------------|
G|-----------------------------------------------------|
D|-----------------------------------------------------|
A|-----------------------------------------------------|
E|-----------------------------------------------------|
\end{lstlisting}
%-----------------------------------------------------------------
\end{comment}
%=================================================================
\vspace{2em}
%-----------------------------------------------------------------
% X0 = A
\gtab{\color{black} X1V9}{}% A9 [X1V9]
\gtab{\color{black} X1V13}{}% A/E [X1V13]
\gtab{\color{black} X2}{}% Bm [X2]
\gtab{\color{black} X2V7}{}% Bm7 [X2V7]
\gtab{\color{black} X3V2V7}{}% C#7 [X3V2V7]
\gtab{\color{black} X4V9}{}% D9 [X4V9]
\gtab{\color{black} X4V9V13}{}% D9/F [X4V9V13]
\gtab{\color{black} X4V9V8}{}% D9/F# [X4V9V8]
\gtab{\color{black} X4V10V1V7V13}{}\\% D#m7(5b) [X4V10V1V7V13]
\gtab{\color{black} X4V9}{}% E [X4V9]
\gtab{\color{black} X5V4}{}% E4 [X5V4]
\gtab{\color{black} X5V7V13}{}% E7/4/9 [X5V7V13]
\gtab{\color{black} X5V8}{}% E/G# [X5V8]
\gtab{\color{black} X6}{}% F#m [X6]
\gtab{\color{black} X6V2V13}{}% F(#5) [X6V2V13]
%-----------------------------------------------------------------
% PADRÃO [TonalidadeMaiorNOTAX.Variação] .Ex:[X50] [X50V1]
% PADRÃO [TonalidadeMenorNOTAX.Variação] .Ex:[mX50] [mX50V1]
% OBS: Variações são alterações do acorde em relação ao campo harmônico.
%-----------------------------------------------------------------
% TIPOS DE VARIAÇÂO DOS ACORDES:
% V1 - ACORDE MENOR (m)
% V2 - ACORDE MAIOR (M)
% V3 - ACORDE MEIO TOM ABAIXO (Bemois)
% V4 - ACORDE COM QUARTA (C4)
% V5 - ACORDE COM QUINTA (C5)
% V6 - ACORDE COM SEXTA (C6)
% V7 - ACORDE COM SÉTIMA MENOR (C7)
% V8 - ACODE COM BAIXO DOIS TONS ACIMA (D/F#)
% V9 - ACORDE COM NONA (C9)
% V10 - ACORDE MEIO TOM ACIMA (Sustenidos)
% V11 - ACORDE COM SÉTIMA MAIOR (C7M)
% V12 - ACORDE SUSPENSO (Sus)
% V13 - ACORDE COM VARIAÇÃO DIVERSA
%=================================================================
\endsong
%=================================================================
\begin{comment}

\end{comment}

%=================================================================
% OUTRAS VERSÕES
%=================================================================
%=================================================================
% VERSÃO SIMPLIFICADA
%=================================================================
\songcolumns{1}
\beginsong
{Em Espírito, Em Verdade\\Meu prazer %TÍTULO
}[by={Ministério Koinonya de Louvor %ARTISTA
},album={@walyssondosreis},
id={GB0032 %COD.ID.: GB0000
}] 
%-----------------------------------------------------------------
\tom{X1}
%=================================================================
%\newchords{verse0.GB0000} % Registrador de Acordes em Sequência
%-----------------------------------------------------------------
%\seq{Intro}{}{}
%-----------------------------------------------------------------
%\beginverse* \endverse
%\beginchorus \endchorus

\beginverse*
\[X1]Em espírito, em ver\[X5]dade
Te ado\[X4]ramos,  te ado\[X1]ramos \[X5]
\[X1]Em espírito, em ver\[X5]dade
Te ado\[X4]ramos,  te ado\[X1]ramos \[X5]
\endverse

\beginverse*
Rei dos \[X6]reis  e se\[X1]nhor 
Te entre\[X4]gamos \[X5]nosso vi\[X1]ver \[X5]
Rei dos \[X6]reis  e se\[X1]nhor 
Te entre\[X4]gamos \[X2]nosso vi\[X5]ver \[X5]
\endverse

\beginchorus 
Pra te ado\[X4]rar oh! \[X2]Rei dos \[X5]reis \[X5]
Foi que eu nas\[X4]ci oh! \[X2]Rei Je\[X5]sus!
Meu pra\[X3V2V7]zer é te lou\[X6]var
Meu pra\[X5]zer é estar\[X4]
Nos \[X5]átrios do se\[X1]nhor
Meu pra\[X5]zer é vi\[X6]ver
Na \[X5]casa de \[X4]Deus
Onde \[X5]flui o a\[X1]mor \[(X5)]
\endchorus

%-----------------------------------------------------------------
\begin{comment}
\lstset{basicstyle=\scriptsize\bf} % Parâmetros da TAB
%-----------------------------------------------------------------
\tab{Solo 1}
\begin{lstlisting}
E|-----------------------------------------------------|
B|-----------------------------------------------------|
X5|-----------------------------------------------------|
D|-----------------------------------------------------|
A|-----------------------------------------------------|
E|-----------------------------------------------------|
\end{lstlisting}
%-----------------------------------------------------------------
\end{comment}
%=================================================================
\vspace{2em}
%-----------------------------------------------------------------
\gtab{\color{black} X1}{}% C [X1]
\gtab{\color{black} X2}{}% Dm [X2]
\gtab{\color{black} X3V2V7}{}% E7 [X3V2V7]
\gtab{\color{black} X4}{}% F [X4]
\gtab{\color{black} X5}{}% G [X5]
\gtab{\color{black} X6}{}% Am [X6]
%-----------------------------------------------------------------
% PADRÃO [TonalidadeMaiorNOTAX.Variação] .Ex:[X50] [X50V1]
% PADRÃO [TonalidadeMenorNOTAX.Variação] .Ex:[mX50] [mX50V1]
% OBS: Variações são alterações do acorde em relação ao campo harmônico.
%-----------------------------------------------------------------
% TIPOS DE VARIAÇÂO DOS ACORDES:
% V1 - ACORDE MENOR (m)
% V2 - ACORDE MAIOR (M)
% V3 - ACORDE MEIO TOM ABAIXO (Bemois)
% V4 - ACORDE COM QUARTA (C4)
% V5 - ACORDE COM QUINTA (C5)
% V6 - ACORDE COM SEXTA (C6)
% V7 - ACORDE COM SÉTIMA MENOR (C7)
% V8 - ACODE COM BAIXO DOIS TONS ACIMA (D/F#)
% V9 - ACORDE COM NONA (C9)
% V10 - ACORDE MEIO TOM ACIMA (Sustenidos)
% V11 - ACORDE COM SÉTIMA MAIOR (C7M)
% V12 - ACORDE SUSPENSO (Sus)
% V13 - ACORDE COM VARIAÇÃO DIVERSA
%=================================================================
\endsong
%=================================================================

\begin{comment}

\end{comment}