%=================================================================
\songcolumns{1}
\beginsong
{Quando o Mundo Cai Ao Meu Redor %TÍTULO
}[by={Livres Para Adorar %ARTISTA
},album={@walyssondosreis},
id={GB0091 %COD.ID.: GB0000
},rev={0}, %REVISÃO
qr={https://drive.google.com/open?id=11huwqynSXswMIDO02-HTmcS2GButEHYb %LINK
}]
%-----------------------------------------------------------------
\tom{X1}{C}
%=================================================================
%\newchords{verse1.GB0000X} % Registrador de Acordes em Sequência
%\newchords{chorus1.GB0000X} % Registrador de Acordes em Sequência
%-----------------------------------------------------------------
%\seq{Intro}{}{}
%-----------------------------------------------------------------
%\beginverse* \endverse
%\beginchorus \endchorus
\beginverse*
Disseste que nunca me deixaria
Disseste que a vida me abalaria
Disseste que no mundo eu teria aflições
Mas eu sei
\endverse
\beginchorus
Quando o mundo cai ao meu redor
Teus braços me seguram
Quando o mundo cai ao meu redor
És a esperança pra mim
Quando o mundo cai ao meu redor
E as forças se vão, encontro abrigo em Ti
Segura-me, segura em minhas mãos
\endchorus
\beginverse*
Quando eu vejo a escuridão da noite
Quando eu vejo a tragédia vindo sobre mim
Mesmo assim eu descanso em Teus braços de amor
Pois eu sei
\endverse
\beginverse*
O choro dura uma noite
Mas com o dia vem o sol
As tempestades virão
Mas sobre a morte Ele venceu
Ele venceu!
\endverse

%-----------------------------------------------------------------
\begin{comment}
\lstset{basicstyle=\scriptsize\bf} % Parâmetros da TAB
%-----------------------------------------------------------------
\tab{Solo 1}
\begin{lstlisting}
E|-----------------------------------------------------|
B|-----------------------------------------------------|
G|-----------------------------------------------------|
D|-----------------------------------------------------|
A|-----------------------------------------------------|
E|-----------------------------------------------------|
\end{lstlisting}
%-----------------------------------------------------------------
\end{comment}
%=================================================================
\vspace{2em} 
%-----------------------------------------------------------------
\color{act-gray}\gtab{\color{act-gray} X}{}% 
\color{act-gray}\gtab{\color{act-gray} X}{}% 
\color{act-gray}\gtab{\color{act-gray} X}{}% 
\color{act-gray}\gtab{\color{act-gray} X}{}% 
%-----------------------------------------------------------------
% PADRÃO: [TonalidadeMaior+NOTAX+Variações] .Ex:[X50] [X57V1V7]
% OBS: Variações são alterações do acorde em relação ao campo harmônico.
%-----------------------------------------------------------------
% Tipos de Variações de Acordes:
% V0 - Variação Diversa
% V1 - Menor (m)
% V2 - Maior (M)
% V3 - Meio Tom Abaixo (Bemol)
% V4 - Com Quarta (ex:C4)
% V5 - Com Quinta (ex:C5)
% V6 - Com Sexta (ex:C6)
% V7 - Com Sétima Menor (ex:C7)
% V8 - Com baixo dois Tons Acima (ex:D/F#)
% V9 - Com Nona (ex:C9)
% V10 - Meio Tom Acima (Sustenido)
% V11 - Com Sétima Maior (ex:C7M)
% V12 - Suspenso (Sus)
% V13 - Com baixo dois Tons e Meio Acima (ex:A/E)
% V14 - Com baixo um Tom e Meio Acima (ex:D9/F) 
% V15 - Meio-Diminuto (m7b5)
% N15 - NÃO Meio-Diminuto
% V16 - Diminuto (º)
% N16 - NÃO Diminuto
%=================================================================
\endsong
%=================================================================
\begin{comment}

\end{comment}