%=================================================================
\songcolumns{2}
\beginsong
{Nunca Foi Sobre Nós %TÍTULO
}[by={Ministério Zoe %ARTISTA
},album={@walyssondosreis},
id={GB0150 %COD.ID.: GB0000
},rev={0}, %REVISÃO
qr={https://drive.google.com/open?id=1U4QMpEuxsAKh7EFYSVx_vNpxRuIWVwsr %LINK
}]
%-----------------------------------------------------------------
\tom{X1}{Eb}
%=================================================================
%\newchords{verse1.GB0000X} % Registrador de Acordes em Sequência
%\newchords{chorus1.GB0000X} % Registrador de Acordes em Sequência
%-----------------------------------------------------------------
%\seq{Intro}{}{}
%-----------------------------------------------------------------
%\beginverse* \endverse
%\beginchorus \endchorus
\beginverse*
Oh profundidade das riquezas
Da sabedoria e do conhecimento de Deus
Quão insondáveis são os Teus juízos
Quão impenetráveis os Teus caminhos
\endverse
\beginverse*
Quem pois conheceu a mente do Senhor?
Ou quem foi o Seu conselheiro?
Ou quem primeiro deu a Ele
Para ser restituído?
\endverse
\beginchorus
Porque d'Ele
Por meio d'Ele
Para Ele, são todas as coisas
\endchorus
\beginverse*
Nunca foi sobre nós
Nem sobre o que podemos fazer
É tudo sobre você
Tudo para você, Jesus
\endverse
\beginverse*
E quem sou eu?
E o que eu tenho pra Te oferecer?
É tudo sobre você
Tudo para você, Jesus
\endverse
\beginverse*
Eu descobri que sem Ti
Sem Ti eu nada posso fazer
É tudo sobre você
Tudo para você, Jesus
\endverse
\beginverse*
Você é Santo, Santo, Santo
Santo, Santo, Santo
É tudo sobre você
Tudo para você, Jesus
\endverse
\beginverse*
Hosana, Hosana
Hosana nas maiores alturas
\endverse
\beginverse*
Nunca foi sobre nós
\endverse
%-----------------------------------------------------------------
\vspace{4em} % Regulador de Espaçamento
%-----------------------------------------------------------------
\begin{comment}
\lstset{basicstyle=\scriptsize\bf} % Parâmetros da TAB
%-----------------------------------------------------------------
\tab{Solo 1}
\begin{lstlisting}
E|-----------------------------------------------------|
B|-----------------------------------------------------|
G|-----------------------------------------------------|
D|-----------------------------------------------------|
A|-----------------------------------------------------|
E|-----------------------------------------------------|
\end{lstlisting}
%-----------------------------------------------------------------
\end{comment}
%=================================================================
 
%-----------------------------------------------------------------
\color{drawChord}\gtab{\color{nameChord} X}{}% 
\color{drawChord}\gtab{\color{nameChord} X}{}% 
\color{drawChord}\gtab{\color{nameChord} X}{}% 
\color{drawChord}\gtab{\color{nameChord} X}{}% 
%-----------------------------------------------------------------
% PADRÃO: [TonalidadeMaior+NOTAX+Variações] .Ex:[X50] [X57V1V7]
% OBS: Variações são alterações do acorde em relação ao campo harmônico.
%-----------------------------------------------------------------
% Tipos de Variações de Acordes:
% V0 - Variação Diversa
% V1 - Menor (m)
% V2 - Maior (M)
% V3 - Meio Tom Abaixo (Bemol)
% V4 - Com Quarta (ex:C4)
% V5 - Com Quinta (ex:C5)
% V6 - Com Sexta (ex:C6)
% V7 - Com Sétima Menor (ex:C7)
% V8 - Com baixo dois Tons Acima (ex:D/F#)
% V9 - Com Nona (ex:C9)
% V10 - Meio Tom Acima (Sustenido)
% V11 - Com Sétima Maior (ex:C7M)
% V12 - Suspenso (Sus)
% V13 - Com baixo dois Tons e Meio Acima (ex:A/E)
% V14 - Com baixo um Tom e Meio Acima (ex:D9/F) 
% V15 - Meio-Diminuto (m7b5)
% N15 - NÃO Meio-Diminuto
% V16 - Diminuto (º)
% N16 - NÃO Diminuto
%=================================================================
\endsong
%=================================================================
\begin{comment}

\end{comment}