%=================================================================
\songcolumns{1}
\beginsong
{Poderoso Deus %TÍTULO
}[by={Santa Geração %ARTISTA
},album={@walyssondosreis},
id={GB0087 %COD.ID.: GB0000
(Rev.0) %REVISÃO.: 0...N
}]
%-----------------------------------------------------------------
\tom{X1}
%=================================================================
%\newchords{verse1.GB0000X} % Registrador de Acordes em Sequência
%\newchords{chorus1.GB0000X} % Registrador de Acordes em Sequência
%-----------------------------------------------------------------
%\seq{Intro}{}{}
%-----------------------------------------------------------------
%\beginverse* \endverse
%\beginchorus \endchorus
Ao que está assentado no trono
E ao Cordeiro seja o louvor
Seja a honra, seja a glória, seja o domínio
Pelos séculos dos séculos
Poderoso Deus, poderoso Deus
Poderoso Deus, minh'alma anseia por Ti

Oh! Deus de majestade, honra e poder
Quão glorioso é o Teu Santo Nome!
Te exaltamos, enquanto nos prostramos
Pois não existe outro tão glorioso como Tu
Muitas vezes falhamos em contemplar o Teu poder
A Tua glória enquanto reinas
Em Teu trono celestial na presença dos serafins
Querubins e dos santos que já estão diante de Ti
Toda a criação e por toda a eternidade
Canto os Teus louvores
Quem somos nós para que nos ame tanto assim?
Quantas vezes falhamos em dar-te reverência e honra
Temos perdido o temor do Senhor
Ou trocado por um breve abraço
Ou um beijinho em Tua presença
Ensina-nos o temor do Teu nome
Ajuda-nos a andar em Tua santidade
Para que nos acheguemos a Ti
Para que Te adoremos
E juntamente com as hostes celestiais
Cantemos dando misericórdia da Tua bondade
E do Teu amor sem fim
Toda honra e glória pertencem a Ti
Só Tu és digno, oh Deus, Todo-poderoso
Criador dos confins da Terra

Ao que está assentado no trono
E ao Cordeiro seja o louvor
Seja a honra, seja a glória, seja o domínio
Pelo século dos séculos
Poderoso Deus, poderoso Deus
Poderoso Deus, minh'alma anseia por Ti

Minh'alma anseia por Ti
Minh'alma anseia por Ti
Como a corsa suspira pelas águas
Minh'alma anseia por Ti
Abro minha boca e suspiro
Pois tenho sede de Ti

Como a corsa suspira pelas águas
(Suspiro por Ti)
Minh'alma anseia por Ti
Abro minha boca e suspiro
Pois tenho sede de Ti
Eu tenho sede de Ti


% Verso de preenchimento
\beginverse*
.
.
.
\endverse
%-----------------------------------------------------------------
\begin{comment}
\lstset{basicstyle=\scriptsize\bf} % Parâmetros da TAB
%-----------------------------------------------------------------
\tab{Solo 1}
\begin{lstlisting}
E|-----------------------------------------------------|
B|-----------------------------------------------------|
G|-----------------------------------------------------|
D|-----------------------------------------------------|
A|-----------------------------------------------------|
E|-----------------------------------------------------|
\end{lstlisting}
%-----------------------------------------------------------------
\end{comment}
%=================================================================
\vspace{2em} 
%-----------------------------------------------------------------
\gtab{\color{black} X}{}% 
\gtab{\color{black} X}{}% 
\gtab{\color{black} X}{}% 
\gtab{\color{black} X}{}% 
%-----------------------------------------------------------------
% PADRÃO: [TonalidadeMaior+NOTAX+Variações] .Ex:[X50] [X57V1V7]
% OBS: Variações são alterações do acorde em relação ao campo harmônico.
%-----------------------------------------------------------------
% Tipos de Variações de Acordes:
% V0 - Variação Diversa
% V1 - Menor (m)
% V2 - Maior (M)
% V3 - Meio Tom Abaixo (Bemol)
% V4 - Com Quarta (ex:C4)
% V5 - Com Quinta (ex:C5)
% V6 - Com Sexta (ex:C6)
% V7 - Com Sétima Menor (ex:C7)
% V8 - Com baixo dois Tons Acima (ex:D/F#)
% V9 - Com Nona (ex:C9)
% V10 - Meio Tom Acima (Sustenido)
% V11 - Com Sétima Maior (ex:C7M)
% V12 - Suspenso (Sus)
% V13 - Com baixo dois Tons e Meio Acima (ex:A/E)
% V14 - Com baixo um Tom e Meio Acima (ex:D9/F) 
% V15 - Meio-Diminuto (m7b5)
% N15 - NÃO Meio-Diminuto
% V16 - Diminuto (º)
% N16 - NÃO Diminuto
%=================================================================
\endsong
%=================================================================
\begin{comment}

\end{comment}