%=================================================================
\songcolumns{2}
\beginsong
{No Dia em que sai de casa %TÍTULO
}[by={Zezé Di Camargo e Luciano %ARTISTA
},album={@walyssondosreis},
id={\href{https://music.youtube.com/watch?v=iSH1B-zK1nc&feature=share %LINK
}{ST0028 %COD.ID.: XXNNNN
}},rev={5}, %REVISÃO
qr={https://music.youtube.com/watch?v=iSH1B-zK1nc&feature=share %LINK
}]
%-----------------------------------------------------------------
\tom{}{A}
%=================================================================
%\newchords{verse1.XX0000X} % Registrador de Acordes em Sequência
%\newchords{chorus1.XX0000X} % Registrador de Acordes em Sequência
%-----------------------------------------------------------------
\seq{Intro}{...}{1x}
%\act{}{}{}
%-----------------------------------------------------------------
%\beginverse \endverse
%\beginchorus \endchorus
\beginverse
No dia em que eu saí de casa
Minha mãe me disse: Filho, vem cá
Passou a mão em meus cabelos
Olhou em meus olhos
Começou falar
Por onde você for, eu sigo com meu pensamento
Sempre onde estiver
Em minhas orações, eu vou pedir a Deus
Que ilumine os passos seus 
\endverse

\beginchorus
Eu sei que ela nunca compreendeu
Os meus motivos de sair de lá
Mas ela sabe que depois que cresce
O filho vira passarinho e quer voar 
Eu bem queria continuar ali
Mas o destino quis me contrariar
E o olhar de minha mãe na porta
Eu deixei, chorando, a me abençoar
\endchorus
\act{Riff Intro}{Repetir}{1x}
\beginverse
A minha mãe naquele dia
Me falou do mundo como ele é
Parece que ela conhecia
Cada pedra que eu iria pôr o pé
Sempre ao lado do meu pai
Da pequena cidade, ela jamais saiu
Ela me disse assim: Meu filho, vá com Deus
Que esse mundo inteiro é seu 
\endverse
\act{Refrão}{Repetir}{1x}
\beginverse
... E o olhar de minha mãe na porta
Eu deixei, chorando, a me abençoar
\endverse
\act{Verso 3}{Repetir}{+1x}

%-----------------------------------------------------------------
\vspace{4em} % Regulador de Espaçamento
%-----------------------------------------------------------------
\begin{comment}
\lstset{basicstyle=\scriptsize\bf} % Parâmetros da TAB
%-----------------------------------------------------------------
\tab{Solo 1}
\begin{lstlisting}
E|-----------------------------------------------------|
B|-----------------------------------------------------|
G|-----------------------------------------------------|
D|-----------------------------------------------------|
A|-----------------------------------------------------|
E|-----------------------------------------------------|
\end{lstlisting}
%-----------------------------------------------------------------
\end{comment}
%=================================================================
\begin{comment}

\color{drawChord}\gtab{\color{nameChord} X}{}% 
\color{drawChord}\gtab{\color{nameChord} X}{}% 
\color{drawChord}\gtab{\color{nameChord} X}{}% 
\color{drawChord}\gtab{\color{nameChord} X}{}% 

\end{comment}
%=================================================================
% PADRÃO: [TonalidadeMaior+NOTAX+Variações] .Ex:[X50] [X57V1V7]
% OBS: Variações são alterações do acorde em relação ao campo harmônico.
%-----------------------------------------------------------------
% Tipos de Variações de Acordes:
% V0 - Variação Diversa
% V1 - Menor (m)
% V2 - Maior (M)
% V3 - Meio Tom Abaixo (Bemol)
% V4 - Com Quarta (ex:C4)
% V5 - Com Quinta (ex:C5)
% V6 - Com Sexta (ex:C6)
% V7 - Com Sétima Menor (ex:C7)
% V8 - Com baixo dois Tons Acima (ex:D/F#)
% V9 - Com Nona (ex:C9)
% V10 - Meio Tom Acima (Sustenido)
% V11 - Com Sétima Maior (ex:C7M)
% V12 - Suspenso (Sus)
% V13 - Com baixo dois Tons e Meio Acima (ex:A/E)
% V14 - Com baixo um Tom e Meio Acima (ex:D9/F) 
% V15 - Meio-Diminuto (m7b5)
% N15 - NÃO Meio-Diminuto
% V16 - Diminuto (º)
% N16 - NÃO Diminuto
% V17 - Com baixo um Tom Acima (ex: C/D)
% V18 - Com baixo um Tom Abaixo (ex: Em/D)
% V19 - Com baixo dois Tons e meio Abaixo (ex: G/D)
%=================================================================
%\vspace{4em} % Regulador de Espaçamento
%=================================================================
\endsong
%=================================================================
