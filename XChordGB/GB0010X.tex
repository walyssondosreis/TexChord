%=================================================================
\songcolumns{1}
\beginsong
{Amizade %TÍTULO
}[by={Quatro Por Um %ARTISTA
},album={@walyssondosreis},
id={GB0010 %COD.ID.: GB0000
},rev={3}, %REVISÃO
qr={https://drive.google.com/open?id=1FPPj0ps6TuPa631JNSVxWLh59OqA5D1A %LINK
}]
%-----------------------------------------------------------------
\tom{X1}{A}
%=================================================================
%\newchords{verse1.GB0000X} % Registrador de Acordes em Sequência
%\newchords{chorus1.GB0000X} % Registrador de Acordes em Sequência
%-----------------------------------------------------------------
\seq{Intro}{X1 X5 X4 X6 X5}{} 
%-----------------------------------------------------------------
%\beginverse* \endverse
%\beginchorus \endchorus
\beginverse
\[X1] Que bom te \[X5]ter aqui co\[X4]migo, \[X6]\[X5]
\[X1] Pra conver\[X5]sar e te conhe\[X4]cer \[X6]\[X5]
\[X1] Entra na \[X5]roda e vem co\[X4]migo \[X6]\[X5]
\[X1] Só é fe\[X5]liz quem tem a\[X4]migos \[X6]\[X5]
\endverse
\beginverse
\[X4V9] Aproveitar esse mo\[X1]mento lindo
\[X2] Cantar, sorrir, fa\[X5]zer amigos
\[X4V9] Celebrando a Deus que \[X1]nos uniu
\[X2] Como foi bom te \[X5]conhecer
\endverse
\beginchorus
\[X1] Que bom te \[X1V8]conhecer
\[X4] Pra mim foi \[X5]um prazer
\[X1] Viver em \[X1V8]comunhão
A\[X4]migos \[X1]mais che\[X2]gados \[X5]que ir\[X1]mãos \[(X5)]
\endchorus
\act{Repetir}{Refrão}{+1x}
\seq{Riff Intro}{X1 X5 X4 X6 X5}{} 
\act{Retomar}{Verso 1}{1x}
\act{Repetir}{Refrão}{+2x}
\beginverse
...\[(X6)] A\[X4]migos \[X1]mais che\[X2]gados \[X5]que ir\[X1]mãos
\[(X6)] A\[X4]migos \[X1]mais che\[X2]gados \[X5]que ir\[X1]mãos
\endverse

%-----------------------------------------------------------------
\begin{comment}
\lstset{basicstyle=\scriptsize\bf} % Parâmetros da TAB
%-----------------------------------------------------------------
\tab{Solo 1}
\begin{lstlisting}
E|-----------------------------------------------------|
B|-----------------------------------------------------|
G|-----------------------------------------------------|
D|-----------------------------------------------------|
A|-----------------------------------------------------|
E|-----------------------------------------------------|
\end{lstlisting}
%-----------------------------------------------------------------
\end{comment}
%=================================================================
\vspace{2em} 
%-----------------------------------------------------------------
\color{drawChord}\gtab{\color{nameChord} X1}{}% 
\color{drawChord}\gtab{\color{nameChord} X1V8}{}%
\color{drawChord}\gtab{\color{nameChord} X2}{}% 
\color{drawChord}\gtab{\color{nameChord} X4}{}% 
\color{drawChord}\gtab{\color{nameChord} X4V9}{}%
\color{drawChord}\gtab{\color{nameChord} X5}{}%
\color{drawChord}\gtab{\color{nameChord} X6}{}%
%-----------------------------------------------------------------
% PADRÃO: [TonalidadeMaior+NOTAX+Variações] .Ex:[X50] [X57V1V7]
% OBS: Variações são alterações do acorde em relação ao campo harmônico.
%-----------------------------------------------------------------
% Tipos de Variações de Acordes:
% V0 - Variação Diversa
% V1 - Menor (m)
% V2 - Maior (M)
% V3 - Meio Tom Abaixo (Bemol)
% V4 - Com Quarta (ex:C4)
% V5 - Com Quinta (ex:C5)
% V6 - Com Sexta (ex:C6)
% V7 - Com Sétima Menor (ex:C7)
% V8 - Com baixo dois Tons Acima (ex:D/F#)
% V9 - Com Nona (ex:C9)
% V10 - Meio Tom Acima (Sustenido)
% V11 - Com Sétima Maior (ex:C7M)
% V12 - Suspenso (Sus)
% V13 - Com baixo dois Tons e Meio Acima (ex:A/E)
% V14 - Com baixo um Tom e Meio Acima (ex:D9/F) 
% V15 - Meio-Diminuto (m7b5)
% N15 - NÃO Meio-Diminuto
% V16 - Diminuto (º)
% N16 - NÃO Diminuto
%=================================================================
\endsong
%=================================================================
\begin{comment}

\end{comment}