%=================================================================
\songcolumns{1}
\beginsong
{Telefone Mudo %TÍTULO
}[by={Trio Parada Dura %ARTISTA
},album={@walyssondosreis},
id={\href{https://music.youtube.com/watch?v=pq2JKVI6F5s&feature=share %LINK
}{ST0085 %COD.ID.: XXNNNN
}},rev={5}, %REVISÃO
qr={https://music.youtube.com/watch?v=pq2JKVI6F5s&feature=share %LINK
}]
%-----------------------------------------------------------------
\tom{A}{A}
%=================================================================
%\newchords{verse1.XX0000X} % Registrador de Acordes em Sequência
%\newchords{chorus1.XX0000X} % Registrador de Acordes em Sequência
%-----------------------------------------------------------------
\seq{Intro}{D A E D A}{2x}
%\act{}{}{}
%-----------------------------------------------------------------
%\beginverse \endverse
%\beginchorus \endchorus
\beginverse
Eu \[D]quero que risque o meu nome da sua a\[A]genda
Es\[E]queça o meu tele\[D]fone, não me ligue \[A]mais
Por\[D]que já estou cansado de ser o re\[A]médio
Pra curar o seu \[E]tédio
Quando seus a\[D]mores não lhe satis\[A]faz
\endverse
\beginverse
Can\[E]sei de ser o seu palhaço
Fa\[D]zer o que sempre \[A]quis
Cansei de curar sua \[E]fossa
Quando você \[D]não se sentia fe\[A]liz
\endverse
\beginverse
Por \[E]isso é que decidi
O \[D]meu telefone \[A]cortar
Você vai discar várias \[E]vezes
Telefone \[D]mudo não \[E]pode cha\[A]mar
\endverse
\act{Riff Intro}{Repetir}{1x}
\act{Verso 1}{Retomar}{1x}
%-----------------------------------------------------------------
\vspace{4em} % Regulador de Espaçamento
%-----------------------------------------------------------------
\begin{comment}
\lstset{basicstyle=\scriptsize\bf} % Parâmetros da TAB
%-----------------------------------------------------------------
\tab{Solo 1}
\begin{lstlisting}
E|-----------------------------------------------------|
B|-----------------------------------------------------|
G|-----------------------------------------------------|
D|-----------------------------------------------------|
A|-----------------------------------------------------|
E|-----------------------------------------------------|
\end{lstlisting}
%-----------------------------------------------------------------
\end{comment}
%=================================================================

\color{drawChord}\gtab{\color{nameChord} A}{~:X02220}% 
\color{drawChord}\gtab{\color{nameChord} D}{~:XX0232}% 
\color{drawChord}\gtab{\color{nameChord} E}{~:022100}% 

%=================================================================
% PADRÃO: [TonalidadeMaior+NOTAX+Variações] .Ex:[E0] [E7V1V7]
% OBS: Variações são alterações do acorde em relação ao campo harmônico.
%-----------------------------------------------------------------
% Tipos de Variações de Acordes:
% V0 - Variação Diversa
% V1 - Menor (m)
% V2 - Maior (M)
% V3 - Meio Tom Abaixo (Bemol)
% V4 - Com Quarta (ex:C4)
% V5 - Com Quinta (ex:C5)
% V6 - Com Sexta (ex:C6)
% V7 - Com Sétima Menor (ex:C7)
% V8 - Com baixo dois Tons Acima (ex:D/F#)
% V9 - Com Nona (ex:C9)
% V10 - Meio Tom Acima (Sustenido)
% V11 - Com Sétima Maior (ex:C7M)
% V12 - Suspenso (Sus)
% V13 - Com baixo dois Tons e Meio Acima (ex:A/E)
% V14 - Com baixo um Tom e Meio Acima (ex:D9/F) 
% V15 - Meio-Diminuto (m7b5)
% N15 - NÃO Meio-Diminuto
% V16 - Diminuto (º)
% N16 - NÃO Diminuto
% V17 - Com baixo um Tom Acima (ex: C/D)
% V18 - Com baixo um Tom Abaixo (ex: Em/D)
% V19 - Com baixo dois Tons e meio Abaixo (ex: G/D)
%=================================================================
%\vspace{4em} % Regulador de Espaçamento
%=================================================================
\endsong
%=================================================================
