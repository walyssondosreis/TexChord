%=================================================================
\songcolumns{1}
\beginsong
{Medley\\Pra Sempre, Meu Prazer, Consagração %TÍTULO
}[by={Vários Artistas %ARTISTA
},album={@walyssondosreis},
id={GB[88.32.153] %COD.ID.: GB0000
},rev={3}, %REVISÃO
qr={ %LINK
}]
%-----------------------------------------------------------------
\tom{X1}{C}
%=================================================================
%\newchords{verse0.GB0000} % Registrador de Acordes em Sequência
%-----------------------------------------------------------------
%\seq{Intro}{}{}
%-----------------------------------------------------------------
%\beginverse* \endverse
%\beginchorus \endchorus
\act{Pra Sempre}{Toda música}{}
\beginverse*
\[X1]O universo \[X5]chora, o Sol se apa\[X6V7]gou
Ali estava \[X4]morto o Salva\[X1]dor
\[X1]Seu corpo lá na \[X5]cruz, seu sangue derra\[X6V7]mou
O peso do pe\[X4]cado ele le\[X1]vou
Le\[X5]vou, le\[X2V7]vou \[X4] 
\[\{ X1 X5 X6V7 X4 \}]{. . . . . . . . . . .}
\endverse
\act{Meu Prazer}{Refrão}{}
\beginchorus
\[(X5)] Pra te ado\[X4]rar oh! \[X6]Rei dos \[X5]reis
Foi que eu nas\[X4]ci oh! \[X6]Rei Je\[X5]sus!
Meu pra\[X3V2V7]zer é te lou\[X6]var
Meu pra\[X5]zer é estar\[X4]
Nos \[X5]átrios do se\[X1]nhor
Meu pra\[X5]zer é vi\[X6]ver
Na \[X5]casa de \[X4]Deus
Onde \[X5]flui o a\[X1]mor \[(X5)]
\endchorus
\act{Pra Sempre}{Refrão}{}
\beginchorus
A \[X1]honra, a \[X5V8]glória, a \[X6]força
E \[X3]o poder ao rei Je\[X4]sus \[X1V8]
E o lou\[X2]vor \[X1] ao \[X7N15V3]rei \[X4] Je\[X5]sus
\endchorus
%-----------------------------------------------------------------
\begin{comment}
\lstset{basicstyle=\scriptsize\bf} % Parâmetros da TAB
%-----------------------------------------------------------------
\tab{Solo 1}
\begin{lstlisting}
E|-----------------------------------------------------|
B|-----------------------------------------------------|
X5|-----------------------------------------------------|
D|-----------------------------------------------------|
A|-----------------------------------------------------|
E|-----------------------------------------------------|
\end{lstlisting}
%-----------------------------------------------------------------
\end{comment}
%=================================================================
\vspace{2em}
%-----------------------------------------------------------------
\color{drawChord}\gtab{\color{nameChord} X1}{}% X1
\color{drawChord}\gtab{\color{nameChord} X1V8}{}% X1V8
\color{drawChord}\gtab{\color{nameChord} X2V7}{}% X2V7
\color{drawChord}\gtab{\color{nameChord} X2}{}% X2
\color{drawChord}\gtab{\color{nameChord} X3V2V7}{}% X3V2V7
\color{drawChord}\gtab{\color{nameChord} X3}{}% X3
\color{drawChord}\gtab{\color{nameChord} X4}{}% X4
\color{drawChord}\gtab{\color{nameChord} X5}{}% X5
\color{drawChord}\gtab{\color{nameChord} X5V8}{}% X5V8
\color{drawChord}\gtab{\color{nameChord} X6}{}% X6
\color{drawChord}\gtab{\color{nameChord} X6V7}{}% X6V7
\color{drawChord}\gtab{\color{nameChord} X7N15V3}{}% X7N15V3
%-----------------------------------------------------------------
% PADRÃO: [TonalidadeMaior+NOTAX+Variações] .Ex:[X50] [X57V1V7]
% OBS: Variações são alterações do acorde em relação ao campo harmônico.
%-----------------------------------------------------------------
% Tipos de Variações de Acordes:
% V0 - Variação Diversa
% V1 - Menor (m)
% V2 - Maior (M)
% V3 - Meio Tom Abaixo (Bemol)
% V4 - Com Quarta (ex:C4)
% V5 - Com Quinta (ex:C5)
% V6 - Com Sexta (ex:C6)
% V7 - Com Sétima Menor (ex:C7)
% V8 - Com baixo dois Tons Acima (ex:D/X4#)
% V9 - Com Nona (ex:C9)
% V10 - Meio Tom Acima (Sustenido)
% V11 - Com Sétima Maior (ex:C7M)
% V12 - Suspenso (Sus)
% V13 - Com baixo dois Tons e Meio Acima (ex:A/E)
% V14 - Com baixo um Tom e Meio Acima (ex:D9/X4) 
% V15 - Meio-Diminuto (m7b5)
% N15 - NÃO Meio-Diminuto
% V16 - Diminuto (º)
% N16 - NÃO Diminuto
%=================================================================
\endsong
%=================================================================
\begin{comment}

\end{comment}