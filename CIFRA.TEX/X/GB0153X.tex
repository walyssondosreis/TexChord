%=================================================================
\songcolumns{2}
\beginsong
{Consagração %TÍTULO
}[by={Aline Barros %ARTISTA
},album={@walyssondosreis},
id={GB0153 %COD.ID.: GB0000
}] 
%-----------------------------------------------------------------
\tom{X1}
%=================================================================
%\newchords{verse0.GB0000} % Registrador de Acordes em Sequência
%-----------------------------------------------------------------
%\seq{Intro}{}{}
%-----------------------------------------------------------------
%\beginverse* \endverse
%\beginchorus \endchorus

\beginverse*
Ao \[X1]Rei dos reis con\[X5V8]sagro 
\[X5V1V14]Tudo o que \[X4V8]sou
De \[X4V1]gratos louvores trans\[X1]borda o meu \[X2]coração \[X5]
A \[X1]minha vida eu en\[X5V8]trego 
Nas \[X5V1V14]tuas mãos, meu Se\[X4V8]nhor
\[X4V1]Pra te exal\[X1]tar com todo \[X2]meu amor \[X5]
\[X1]Eu te louva\[X5V8]rei conforme a \[X5V1V14]tua jus\[X4V8]tiça
E \[X4V1]cantarei lou\[X1]vores, pois tu \[X2V2]és altíssi\[X5]mo
\endverse

\beginverse*
Celebra\[X4]rei a \[X5]ti, ó Deus, 
Com \[X6]meu viver \[X5]\[X6]
Canta\[X4]rei e conta\[X5]rei as tuas \[X1]obras \[X1V7]
Pois por \[X4]tuas mãos \[X5]foram criados
\[X3V2]Terra, céu e \[X6]mar e todo \[X4]ser \[X5]que neles \[X6V2V7]há
Toda a \[X4]terra celebra a \[X5]ti
Com \[X1]cânti\[X5V8]cos de \[X6]júbilo
Pois tu \[X4]és o \[X5]Deus cria\[X6V2V7]dor
Toda a \[X4]terra celebra a \[X5]ti
Com \[X1]cânti\[X5V8]cos de \[X6]júbilo
Pois tu \[X4]és o \[X5]Deus cria\[X1]dor
\{ Pois tu \[X4]és o \[X5]Deus cria\[X1]dor
Pois tu \[X4]és o \[X5]Deus cria\[X1]dor \[(X1V4)] \}
\endverse

\beginchorus
A \[X1]honra, a \[X5V8]glória, a \[X6]força
E \[X3]o poder ao rei Je\[X4]sus \[X1V8]
E o lou\[X2]vor \[X1] ao \[X7V2]rei \[X4V8] Je\[X5]sus
\endchorus
%-----------------------------------------------------------------
\begin{comment}
\lstset{basicstyle=\scriptsize\bf} % Parâmetros da TAB
%-----------------------------------------------------------------
\tab{Solo 1}
\begin{lstlisting}
E|-----------------------------------------------------|
B|-----------------------------------------------------|
G|-----------------------------------------------------|
D|-----------------------------------------------------|
A|-----------------------------------------------------|
E|-----------------------------------------------------|
\end{lstlisting}
%-----------------------------------------------------------------
\end{comment}
%=================================================================
\vspace{2em}
%-----------------------------------------------------------------
\gtab{\color{black} X1}{}% A [X1]
\gtab{\color{black} X1V8}{}% A/C# [X1V8]
\gtab{\color{black} X1V4}{}% A4 [X1V4]
\gtab{\color{black} X1V7}{}% A7 [X1V7]
\gtab{\color{black} X2}{}% Bm [X2]
\gtab{\color{black} X2V2}{}% B [X2V2]
\gtab{\color{black} X3}{}\\% C#m [X3]
\gtab{\color{black} X3V2}{}% C# [X3V2]
\gtab{\color{black} X4}{}% D [X4]
\gtab{\color{black} X4V8}{}% D/F# [X4V8]
\gtab{\color{black} X4V1}{}% Dm [X4V1]
\gtab{\color{black} X5}{}% E [X5]
\gtab{\color{black} X5V8}{}% E/G# [X5V8]
\gtab{\color{black} X5V1V14}{}\\% Em/G [X5V1V14]
\gtab{\color{black} X6}{}% F#m [X6]
\gtab{\color{black} X6V2V7}{}% F#7 [X6V2V7]
\gtab{\color{black} X7V2}{}% G [X7V2]
%-----------------------------------------------------------------
% PADRÃO [TonalidadeMaiorNOTAX.Variação] .Ex:[X50] [X50V1]
% PADRÃO [TonalidadeMenorNOTAX.Variação] .Ex:[mX50] [mX50V1]
% OBS: Variações são alterações do acorde em relação ao campo harmônico.
%-----------------------------------------------------------------
% TIPOS DE VARIAÇÂO DOS ACORDES:
% V0 - ACORDE COM VARIAÇÃO DIVERSA
% V1 - ACORDE MENOR (m)
% V2 - ACORDE MAIOR (M)
% V3 - ACORDE MEIO TOM ABAIXO (Bemois)
% V4 - ACORDE COM QUARTA (C4)
% V5 - ACORDE COM QUINTA (C5)
% V6 - ACORDE COM SEXTA (C6)
% V7 - ACORDE COM SÉTIMA MENOR (C7)
% V8 - ACORDE COM BAIXO DOIS TONS ACIMA (D/F#)
% V9 - ACORDE COM NONA (C9)
% V10 - ACORDE MEIO TOM ACIMA (Sustenidos)
% V11 - ACORDE COM SÉTIMA MAIOR (C7M)
% V12 - ACORDE SUSPENSO (Sus)
% V13 - ACORDE COM BAIXO DOIS TONS E MEIO ACIMA (A/E)
% V14 - ACORDE UM TOM E MEIO ACIMA (D9/F)
%=================================================================
\endsong
%=================================================================
\begin{comment}

\end{comment}