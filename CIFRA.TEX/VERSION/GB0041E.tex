%=================================================================
\songcolumns{2}
\beginsong
{Eu Vou (África) %TÍTULO
}[by={Fernanda Brum %ARTISTA
},album={@walyssondosreis},
id={GB0041 %COD.ID.: GB0000
}] 
%-----------------------------------------------------------------
\tom{E}
%=================================================================
\newchords{verse1.GB0041E} % Registrador de Acordes em Sequência
%\newchords{chorus1.GB0000X} % Registrador de Acordes em Sequência
%-----------------------------------------------------------------
\seq{Intro}{E}{}
%-----------------------------------------------------------------
%\beginverse* \endverse
%\beginchorus \endchorus
\beginverse*\memorize[verse1.GB0041E]
\[E]Alguém da Ásia me \[A]disse "Vem me aju\[E]dar" \[A]
\[E]Posso ouvir a \[A]África pedir por so\[B4]corro \[B]
Ru\[C\#m]anda, Somália, Ni\[A]géria
Clamando por um \[C\#m]pouco de amor \[A]
\[C\#m]Vou fazer tudo que eu \[A]posso fazer, eu \[F\#4]vou \[F\#]
Tenho \[A]muito pra dar, eu \[F\#4]vou \[F\#]
O evan\[A]gelho pregar
\endverse
\beginchorus
\[E]Como ouvi\[A]rão se não há quem \[B4]pregue \[B]
\[E]Como pregar se nin\[A]guém se dispõe a \[D]ir \[C\#m]\[B]
\[E]Como crerão na\[A]queles
De quem nada ou\[B4]viram \[B]
Eis-me a\[F\#m]qui\[A], eu \[C\#m]vou\[B]
Eis-me a\[F\#m]qui\[A], Se\[C\#m]nhor\[B]
Eis-me \[F\#m]aqui \[E/G\#]\[A]
\{ Eu \[E]vou, eu vou
Eu \[D]vou, eu vou
Eu \[A]vou, eu vou
Eu \[C]vou \}
\endchorus
\beginverse*\replay[verse1.GB0041E]
^Vi um menino na ^Rússia olhar para o ^céu ^
^El Salvador tá cho^rando por salva^ção ^
Ro^mênia, Arábia Sau^dita, o Iraque
Es^peram por nós ^
Co^lômbia, Indonésia, Al^bânia, pra China
Eu ^vou ^
Pelo ^chão do Brasil, eu ^vou ^
Deus me ^quer pras nações
\endverse


%-----------------------------------------------------------------
\begin{comment}
\lstset{basicstyle=\scriptsize\bf} % Parâmetros da TAB
%-----------------------------------------------------------------
\tab{Solo 1}
\begin{lstlisting}
E|-----------------------------------------------------|
B|-----------------------------------------------------|
G|-----------------------------------------------------|
D|-----------------------------------------------------|
A|-----------------------------------------------------|
E|-----------------------------------------------------|
\end{lstlisting}
%-----------------------------------------------------------------
\end{comment}
%=================================================================
\vspace{2em} 
%-----------------------------------------------------------------
\gtab{\color{black} E}{}% E [E]
\gtab{\color{black} E/G\#}{}% E/G# [E/G\#]
\gtab{\color{black} F\#m}{}% F#m [F\#m]
\gtab{\color{black} F\#}{}% F# [F\#]
\gtab{\color{black} F\#4}{}% F#4 [F\#4]
\gtab{\color{black} A}{}\\% A [A] 
\gtab{\color{black} B}{}% B [B]
\gtab{\color{black} B4}{}% B4 [B4]
\gtab{\color{black} C\#m}{}% C#m [C\#m]
\gtab{\color{black} C}{}% C [C]
\gtab{\color{black} D}{}% D [D]
%-----------------------------------------------------------------
% PADRÃO [TonalidadeMaiorNOTAX.Variação] .Ex:[X50H] [X50V1]
% PADRÃO [TonalidadeMenorNOTAX.Variação] .Ex:[mX50H] [mX50V1]
% OBS: Variações são alterações do acorde em relação ao campo harmônico.
% OBS: 'H' Denomina acordes que são naturais do campo harmônico.
%-----------------------------------------------------------------
% TIPOS DE VARIAÇÂO DOS ACORDES:
% V0 - ACORDE COM VARIAÇÃO DIVERSA
% V1 - ACORDE MENOR (m)
% V2 - ACORDE MAIOR (M)
% V3 - ACORDE MEIO TOM ABAIXO (Bemois)
% V4 - ACORDE COM QUARTA (C4)
% V5 - ACORDE COM QUINTA (C5)
% V6 - ACORDE COM SEXTA (C6)
% V7 - ACORDE COM SÉTIMA MENOR (C7)
% V8 - ACORDE COM BAIXO DOIS TONS ACIMA (D/F#)
% V9 - ACORDE COM NONA (C9)
% V10 - ACORDE MEIO TOM ACIMA (Sustenidos)
% V11 - ACORDE COM SÉTIMA MAIOR (C7M)
% V12 - ACORDE SUSPENSO (Sus)
% V13 - ACORDE COM BAIXO DOIS TONS E MEIO ACIMA (A/E)
% V14 - ACORDE UM TOM E MEIO ACIMA (D9/F)
%=================================================================
\endsong
%=================================================================
\begin{comment}

\end{comment}