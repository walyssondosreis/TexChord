%=================================================================
\songcolumns{2}
\beginsong
{Sonho de Um Caminhoneiro %TÍTULO
}[by={Milionário e José Rico %ARTISTA
},album={@walyssondosreis},
id={\href{https://music.youtube.com/watch?v=_e69eLtfC6w&feature=share %LINK
}{ST0076 %COD.ID.: XXNNNN
}},rev={5}, %REVISÃO
qr={https://music.youtube.com/watch?v=_e69eLtfC6w&feature=share %LINK
}]
%-----------------------------------------------------------------
\tom{A}{A}
%=================================================================
%\newchords{verse1.XX0000X} % Registrador de Acordes em Sequência
%\newchords{chorus1.XX0000X} % Registrador de Acordes em Sequência
%-----------------------------------------------------------------
\seq{Intro}{D A E A}{1x}
%\act{}{}{}
%-----------------------------------------------------------------
%\beginverse \endverse
%\beginchorus \endchorus
\beginverse
\[A] Eram dois amigos insepa\[E]ráveis
Lu\[D]tando pela vida e o \[A]pão
Levando um sonho de cidade em ci\[E]dade
De serem \[D]donos de seu cami\[A]nhão\[A7]
Com muita \[D]luta e sacrifício
Para pa\[B7]gar em dia a presta\[E]ção
Se reali\[D]zava o sonho final\[A]mente
O empre\[E]gado passa a ser pa\[A]trão
\endverse
\beginverse
^ Suas viagens eram intermi^náveis
De can^saço de poeira e ^chão
E um dos amigos um recém ca^sado
Ia ser ^pai do primeiro va^rão ^
Com ale^gria vinham pela estrada
Não ^vendo a hora de che^gar
E o caminho^neiro disse ao a^migo
Vou lhe dar meu ^filho para bati^zar
\endverse
\beginverse
^ Mas o destino cruel e traiço^eiro
Mar^cou a hora e o lu^gar
A chuva fina e a pista mo^lhada
Com uma ca^rreta foram se cho^car ^
Mas como ^todos tem a sua sina
^Um a morte não le^vou
E o agoni^zante no braços do a^migo disse
Vá conhecer meu ^filho porque eu não ^vou
\endverse
\act{Riff Intro}{Repetir}{1x}
\act{Verso 3}{Retomar}{1x}


%-----------------------------------------------------------------
\vspace{4em} % Regulador de Espaçamento
%-----------------------------------------------------------------
\begin{comment}
\lstset{basicstyle=\scriptsize\bf} % Parâmetros da TAB
%-----------------------------------------------------------------
\tab{Solo 1}
\begin{lstlisting}
E|-----------------------------------------------------|
B|-----------------------------------------------------|
G|-----------------------------------------------------|
D|-----------------------------------------------------|
A|-----------------------------------------------------|
E|-----------------------------------------------------|
\end{lstlisting}
%-----------------------------------------------------------------
\end{comment}
%=================================================================

\color{drawChord}\gtab{\color{nameChord} A}{~:X02220}% 
\color{drawChord}\gtab{\color{nameChord} A7}{~:X02020}% 
\color{drawChord}\gtab{\color{nameChord} B7}{2:X02020}% 
\color{drawChord}\gtab{\color{nameChord} D}{5:X02220}% 
\color{drawChord}\gtab{\color{nameChord} E}{~:022100}% 

%=================================================================
% PADRÃO: [TonalidadeMaior+NOTAX+Variações] .Ex:[E0] [E7V1V7]
% OBS: Variações são alterações do acorde em relação ao campo harmônico.
%-----------------------------------------------------------------
% Tipos de Variações de Acordes:
% V0 - Variação Diversa
% V1 - Menor (m)
% V2 - Maior (M)
% V3 - Meio Tom Abaixo (Bemol)
% V4 - Com Quarta (ex:C4)
% V5 - Com Quinta (ex:C5)
% V6 - Com Sexta (ex:C6)
% V7 - Com Sétima Menor (ex:C7)
% V8 - Com baixo dois Tons Acima (ex:D/F#)
% V9 - Com Nona (ex:C9)
% V10 - Meio Tom Acima (Sustenido)
% V11 - Com Sétima Maior (ex:C7M)
% V12 - Suspenso (Sus)
% V13 - Com baixo dois Tons e Meio Acima (ex:A/E)
% V14 - Com baixo um Tom e Meio Acima (ex:D9/F) 
% V15 - Meio-Diminuto (m7b5)
% N15 - NÃO Meio-Diminuto
% V16 - Diminuto (º)
% N16 - NÃO Diminuto
% V17 - Com baixo um Tom Acima (ex: C/D)
% V18 - Com baixo um Tom Abaixo (ex: Em/D)
% V19 - Com baixo dois Tons e meio Abaixo (ex: G/D)
%=================================================================
%\vspace{4em} % Regulador de Espaçamento
%=================================================================
\endsong
%=================================================================
