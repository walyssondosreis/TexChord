%=================================================================
\songcolumns{1}
\beginsong
{Santo %TÍTULO
}[by={Livres Para Adorar  %ARTISTA
},album={@walyssondosreis},
id={GB0103 %COD.ID.: GB0000
(Rev.1) %REVISÃO.: 0...N
}]
%-----------------------------------------------------------------
\tom{X1}
%=================================================================
%\newchords{verse0.GB0000} % Registrador de Acordes em Sequência
%-----------------------------------------------------------------
\seq{Intro}{X6 X2 X1 X3}{}
%-----------------------------------------------------------------
%\beginverse* \endverse
%\beginchorus \endchorus

\beginverse 
Eu não co\[X6]nheço outra can\[X2]ção
Que lhe des\[X1]creva em perfei\[X3]ção
\endverse
\act{Repetir}{Verso 1}{+1x}
\beginchorus
\[X6] Santo, \[X2] santo, \[X1] Deus pode\[X3]roso
\[X6] Santo, \[X2] santo, \[X1] Deus pode\[X3]roso
\[X6]\[X2] Deus pode\[X1]roso, 
\[X3] Deus pode\[X6]roso 
\[X2] Deus pode\[X1]roso, não não \[X3]há
\endchorus
\act{Retomar}{Verso 1}{1x}

\beginverse
^Não há outro i^gual
Tu és ^santo, san^to
\endverse
\act{Repetir}{Verso 2}{+1x}
\act{Repetir}{Refrão}{1x}
\act{Executar}{Solo 1}{}
\act{Repetir}{Verso 2}{2x}
\beginverse
\[X6] Santo, \[X2] santo, \[X1] Deus pode\[X3]roso
\[X6] Santo, \[X2] santo, \[X1] Deus pode\[X3]roso
\[X6]\[X2] Deus pode\[X1]roso \[(X6)]
\endverse
\act{Repetir}{Verso 3}{+1x}
\beginverse*
.
.
\endverse
%=================================================================
\vspace{2em}
%-----------------------------------------------------------------
\gtab{\color{black} X1}{}% D [X1]
\gtab{\color{black} X2}{}% Em [X2]
\gtab{\color{black} X3}{}% F#m [X3]
\gtab{\color{black} X6}{}% Bm [X6]
%-----------------------------------------------------------------
% PADRÃO [TonalidadeMaiorNOTAX.Variação] .Ex:[X50] [X50V1]
% PADRÃO [TonalidadeMenorNOTAX.Variação] .Ex:[mX50] [mX50V1]
% OBS: Variações são alterações do acorde em relação ao campo harmônico.
%-----------------------------------------------------------------
% TIPOS DE VARIAÇÂO DOS ACORDES:
% V0 - ACORDE COM VARIAÇÃO DIVERSA
% V1 - ACORDE MENOR (m)
% V2 - ACORDE MAIOR (M)
% V3 - ACORDE MEIO TOM ABAIXO (Bemois)
% V4 - ACORDE COM QUARTA (C4)
% V5 - ACORDE COM QUINTA (C5)
% V6 - ACORDE COM SEXTA (C6)
% V7 - ACORDE COM SÉTIMA MENOR (C7)
% V8 - ACORDE COM BAIXO DOIS TONS ACIMA (D/F#)
% V9 - ACORDE COM NONA (C9)
% V10 - ACORDE MEIO TOM ACIMA (Sustenidos)
% V11 - ACORDE COM SÉTIMA MAIOR (C7M)
% V12 - ACORDE SUSPENSO (Sus)
% V13 - ACORDE COM BAIXO DOIS TONS E MEIO ACIMA (A/E)
% V14 - ACORDE UM TOM E MEIO ACIMA (D9/F)
%=================================================================
\endsong
%=================================================================
\begin{comment}

\end{comment}