%=================================================================
\songcolumns{2}
\beginsong
{Deu Medo %TÍTULO
}[by={Leandro e Leonardo %ARTISTA
},album={@walyssondosreis},
id={\href{https://music.youtube.com/watch?v=KhyotY0gZ1E&feature=share %LINK
}{ST0018 %COD.ID.: XXNNNN
}},rev={5}, %REVISÃO
qr={https://music.youtube.com/watch?v=KhyotY0gZ1E&feature=share %LINK
}]
%-----------------------------------------------------------------
\tom{}{A}
%=================================================================
%\newchords{verse1.XX0000X} % Registrador de Acordes em Sequência
%\newchords{chorus1.XX0000X} % Registrador de Acordes em Sequência
%-----------------------------------------------------------------
\seq{Intro}{...}{1x}
%\act{}{}{}
%-----------------------------------------------------------------
%\beginverse \endverse
%\beginchorus \endchorus
\beginverse
Deu medo
Quando você pediu que a gente desse um tempo
Meu coração bateu sofrendo em silêncio
Me deu vontade de pedir, não vá embora
\endverse

\beginverse
Deu medo
Aí me segurei pra não chorar por dentro
Quase cheguei a te odiar nesse momento
Meu coração quase parou naquela hora
\endverse

\beginverse
E matamos nosso medo
Abraçados corpo, alma e coração
E o Sol nasceu mais cedo
Acendendo outra vez nossa paixão
\endverse

\beginchorus
Pode o mundo acabar, pode o tempo parar
Tudo acontecer, que eu não vou te perder
Pode o céu desabar, pode secar o mar
Só não pode morrer meu amor por você
Por você
\endchorus
\seq{Solo}{...}{1x}
\act{Verso 3}{Retomar}{1x}
\act{Refrão}{Repitir}{1x}

\beginverse
...Por você
\endverse
%-----------------------------------------------------------------
\vspace{4em} % Regulador de Espaçamento
%-----------------------------------------------------------------
\begin{comment}
\lstset{basicstyle=\scriptsize\bf} % Parâmetros da TAB
%-----------------------------------------------------------------
\tab{Solo 1}
\begin{lstlisting}
E|-----------------------------------------------------|
B|-----------------------------------------------------|
G|-----------------------------------------------------|
D|-----------------------------------------------------|
A|-----------------------------------------------------|
E|-----------------------------------------------------|
\end{lstlisting}
%-----------------------------------------------------------------
\end{comment}
%=================================================================
\begin{comment}

\color{drawChord}\gtab{\color{nameChord} X}{}% 
\color{drawChord}\gtab{\color{nameChord} X}{}% 
\color{drawChord}\gtab{\color{nameChord} X}{}% 
\color{drawChord}\gtab{\color{nameChord} X}{}% 

\end{comment}
%=================================================================
% PADRÃO: [TonalidadeMaior+NOTAX+Variações] .Ex:[X50] [X57V1V7]
% OBS: Variações são alterações do acorde em relação ao campo harmônico.
%-----------------------------------------------------------------
% Tipos de Variações de Acordes:
% V0 - Variação Diversa
% V1 - Menor (m)
% V2 - Maior (M)
% V3 - Meio Tom Abaixo (Bemol)
% V4 - Com Quarta (ex:C4)
% V5 - Com Quinta (ex:C5)
% V6 - Com Sexta (ex:C6)
% V7 - Com Sétima Menor (ex:C7)
% V8 - Com baixo dois Tons Acima (ex:D/F#)
% V9 - Com Nona (ex:C9)
% V10 - Meio Tom Acima (Sustenido)
% V11 - Com Sétima Maior (ex:C7M)
% V12 - Suspenso (Sus)
% V13 - Com baixo dois Tons e Meio Acima (ex:A/E)
% V14 - Com baixo um Tom e Meio Acima (ex:D9/F) 
% V15 - Meio-Diminuto (m7b5)
% N15 - NÃO Meio-Diminuto
% V16 - Diminuto (º)
% N16 - NÃO Diminuto
% V17 - Com baixo um Tom Acima (ex: C/D)
% V18 - Com baixo um Tom Abaixo (ex: Em/D)
% V19 - Com baixo dois Tons e meio Abaixo (ex: G/D)
%=================================================================
%\vspace{4em} % Regulador de Espaçamento
%=================================================================
\endsong
%=================================================================
