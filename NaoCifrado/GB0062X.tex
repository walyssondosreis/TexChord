%=================================================================
\songcolumns{1}
\beginsong
{Me Ama %TÍTULO
}[by={Diante do Trono %ARTISTA
},album={@walyssondosreis},
id={GB0062 %COD.ID.: GB0000
(Rev.0) %REVISÃO.: 0...N
}]
%-----------------------------------------------------------------
\tom{X1}
%=================================================================
%\newchords{verse1.GB0000X} % Registrador de Acordes em Sequência
%\newchords{chorus1.GB0000X} % Registrador de Acordes em Sequência
%-----------------------------------------------------------------
%\seq{Intro}{}{}
%-----------------------------------------------------------------
%\beginverse* \endverse
%\beginchorus \endchorus
Tem ciúmes de mim
O Seu amor é como um furacão
E eu me rendo ao vento de Sua misericórdia

Então, de repente, não vejo mais minhas aflições
Eu só vejo a glória
E percebo quão maravilhoso Ele é
E o tanto que Ele me quer

Ô, Ele me amou
Ô, Ele me ama
Ele me amou

Tem ciúmes de mim
O Seu amor é como um furacão
E eu me rendo ao vento de sua misericórdia

Então, de repente, não vejo mais minhas aflições
Eu só vejo a glória
E percebo quão maravilhoso Ele é
E o tanto que Ele me quer

Ô, Ele me amou
Ô, Ele me ama
Ele me amou

Me ama, Ele me ama
Ele me ama, Ele me ama
Me ama, Ele me ama
Ele me ama, Ele me ama

Somos sua herança e Ele é o nosso galardão
Seu olhar de graça nos atrai à redenção
Se a graça é um oceano, estamos afogando

O céu se une à terra como um beijo apaixonado
E meu coração dispara em meu peito acelerado
Não tenho tempo pra perder com ressentimentos
Quando penso que Ele

Me ama, Ele me ama
Ele me ama, Ele me ama
Me ama, Ele me ama
Ele me ama, Ele me ama

Me ama, Ele me ama
Ele me ama, Ele me ama
Me ama, Ele me ama
Ele me ama, Ele me ama

Ô, Ele me amou
Ô, Ele me ama
Ele me amou


% Verso de preenchimento
\beginverse*
.
.
.
\endverse
%-----------------------------------------------------------------
\begin{comment}
\lstset{basicstyle=\scriptsize\bf} % Parâmetros da TAB
%-----------------------------------------------------------------
\tab{Solo 1}
\begin{lstlisting}
E|-----------------------------------------------------|
B|-----------------------------------------------------|
G|-----------------------------------------------------|
D|-----------------------------------------------------|
A|-----------------------------------------------------|
E|-----------------------------------------------------|
\end{lstlisting}
%-----------------------------------------------------------------
\end{comment}
%=================================================================
\vspace{2em} 
%-----------------------------------------------------------------
\gtab{\color{black} X}{}% 
\gtab{\color{black} X}{}% 
\gtab{\color{black} X}{}% 
\gtab{\color{black} X}{}% 
%-----------------------------------------------------------------
% PADRÃO: [TonalidadeMaior+NOTAX+Variações] .Ex:[X50] [X57V1V7]
% OBS: Variações são alterações do acorde em relação ao campo harmônico.
%-----------------------------------------------------------------
% Tipos de Variações de Acordes:
% V0 - Variação Diversa
% V1 - Menor (m)
% V2 - Maior (M)
% V3 - Meio Tom Abaixo (Bemol)
% V4 - Com Quarta (ex:C4)
% V5 - Com Quinta (ex:C5)
% V6 - Com Sexta (ex:C6)
% V7 - Com Sétima Menor (ex:C7)
% V8 - Com baixo dois Tons Acima (ex:D/F#)
% V9 - Com Nona (ex:C9)
% V10 - Meio Tom Acima (Sustenido)
% V11 - Com Sétima Maior (ex:C7M)
% V12 - Suspenso (Sus)
% V13 - Com baixo dois Tons e Meio Acima (ex:A/E)
% V14 - Com baixo um Tom e Meio Acima (ex:D9/F) 
% V15 - Meio-Diminuto (m7b5)
% N15 - NÃO Meio-Diminuto
% V16 - Diminuto (º)
% N16 - NÃO Diminuto
%=================================================================
\endsong
%=================================================================
\begin{comment}

\end{comment}