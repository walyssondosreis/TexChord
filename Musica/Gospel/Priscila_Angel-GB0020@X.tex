%=================================================================
\songcolumns{2}
\beginsong
{Corpo e Família %TÍTULO
}[by={Priscila Angel %ARTISTA
},album={@walyssondosreis},
id={GB0020 %COD.ID.: GB0000
},rev={3}, %REVISÃO
qr={https://drive.google.com/open?id=1blUg5nqzoMF1siC1EBDmNyKgxa8e1hg1 %LINK
}]
%-----------------------------------------------------------------
\tom{X1}{F}
%=================================================================
%\newchords{verse1.GB0000X} % Registrador de Acordes em Sequência
%\newchords{chorus1.GB0000X} % Registrador de Acordes em Sequência
%-----------------------------------------------------------------
\seq{Intro}{ [ X1 X4 X5 X4 ] [ X1 X4 X5 X4 X5 ]}{}
%-----------------------------------------------------------------
%\beginverse* \endverse
%\beginchorus \endchorus
\beginverse
Rece\[X1]bi um novo \[X4]coração do \[X1]pai \[X5V8]
Cora\[X6]ção regene\[X3]rado, cora\[X4]ção transfor\[X1V8]mado
Cora\[X2]ção que é inspi\[X1]rado por Je\[X5V8]sus \[X5V4]\[X5]
\endverse
\beginverse
Como ^fruto deste ^novo cora^ção, ^
Eu de^claro a paz de ^Cristo 
Te aben^çôo meu ir^mão, preci^osa
É a ^nossa comu\[X5]nhão \[X5V4]
\endverse
\beginchorus
Somos \[X1]corpo, e \[X4V8]assim bem ajus\[X5V8]tado,
To\[X3]talmente li\[X6]gado
U\[X4]nido, vi\[X2]vendo em \[X5V4]amor,
Um\[X5]a fa\[X1]mília, sem \[X4]qualquer \[X6]falsi\[X5V8]dade
Vi\[X3]vendo a ver\[X6]dade, expre\[X4]ssando
A \[X2]glória do Se\[X5V4]nhor
Um\[X5]a fa\[X1]mília vi\[X4]vendo o 
\[X6]Compro\[X5V8]misso no \[X3]grande amor de \[X6]Cristo:
Eu pre\[X4]ciso de \[X2]ti que\[X5V4]rido \[X5]ir\[X1]mão
\endchorus
\seq{Riff 1}{X1 X4 X5}{2x}
\act{Retomar}{Verso 2}{1x}
\beginverse
...,Preci\[X6]oso és para \[X4]mim que\[X5V4]rido \[X5]ir\[X1]mão
Eu pre\[X6]ciso de \[X4]ti que\[X5V4]rido \[X5]ir\[X1]mão,
Preci\[X6]oso és para \[X4]mim que\[X5V4]rido \[X5]ir\[X1]mão
\endverse
%-----------------------------------------------------------------
\vspace{4em} % Regulador de Espaçamento
%-----------------------------------------------------------------
\begin{comment}
\lstset{basicstyle=\scriptsize\bf} % Parâmetros da TAB
%-----------------------------------------------------------------
\tab{Solo 1}
\begin{lstlisting}
E|-----------------------------------------------------|
B|-----------------------------------------------------|
G|-----------------------------------------------------|
D|-----------------------------------------------------|
A|-----------------------------------------------------|
E|-----------------------------------------------------|
\end{lstlisting}
%-----------------------------------------------------------------
\end{comment}
%=================================================================
 
%-----------------------------------------------------------------
\color{drawChord}\gtab{\color{nameChord} X1}{}%
\color{drawChord}\gtab{\color{nameChord} X1V8}{}%
\color{drawChord}\gtab{\color{nameChord} X2}{}%
\color{drawChord}\gtab{\color{nameChord} X3}{}%
\color{drawChord}\gtab{\color{nameChord} X4}{}%
\color{drawChord}\gtab{\color{nameChord} X4V8}{}\\%
\color{drawChord}\gtab{\color{nameChord} X5}{}% 
\color{drawChord}\gtab{\color{nameChord} X5V4}{}%
\color{drawChord}\gtab{\color{nameChord} X5V8}{}%
\color{drawChord}\gtab{\color{nameChord} X6}{}%
%-----------------------------------------------------------------
% PADRÃO: [TonalidadeMaior+NOTAX+Variações] .Ex:[X50] [X57V1V7]
% OBS: Variações são alterações do acorde em relação ao campo harmônico.
%-----------------------------------------------------------------
% Tipos de Variações de Acordes:
% V0 - Variação Diversa
% V1 - Menor (m)
% V2 - Maior (M)
% V3 - Meio Tom Abaixo (Bemol)
% V4 - Com Quarta (ex:C4)
% V5 - Com Quinta (ex:C5)
% V6 - Com Sexta (ex:C6)
% V7 - Com Sétima Menor (ex:C7)
% V8 - Com baixo dois Tons Acima (ex:D/F#)
% V9 - Com Nona (ex:C9)
% V10 - Meio Tom Acima (Sustenido)
% V11 - Com Sétima Maior (ex:C7M)
% V12 - Suspenso (Sus)
% V13 - Com baixo dois Tons e Meio Acima (ex:A/E)
% V14 - Com baixo um Tom e Meio Acima (ex:D9/F) 
% V15 - Meio-Diminuto (m7b5)
% N15 - NÃO Meio-Diminuto
% V16 - Diminuto (º)
% N16 - NÃO Diminuto
%=================================================================
\endsong
%=================================================================
\begin{comment}

\end{comment}