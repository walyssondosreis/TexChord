%=================================================================
\songcolumns{2}
\beginsong
{O Poder do Criador %TÍTULO
}[by={Goiano e Paranaense %ARTISTA
},album={@walyssondosreis},
id={\href{https://music.youtube.com/watch?v=jOADeghWRhA&feature=share
}{ SB0132 %COD.ID.: XXNNNN
}},rev={4}, %REVISÃO
qr={https://music.youtube.com/watch?v=jOADeghWRhA&feature=share %LINK
}]
%-----------------------------------------------------------------
\tom{}{Bb}

\act{Solo}{Parte 1}{1x}
\act{Solo}{Parte 2}{1x}

%-----------------------------------------------------------------
\beginverse 
 Hora triste foi aquela
Que Jesus Cristo falou
 Mãe está chegando a hora
A senhora fica e eu vou
 Com certeza mãe e filho
Neste momento chorou
 Hora triste dolorida 
Porque a dor da despedida
Só conhece quem passou
\endverse
\act{Solo: Parte 2}{Repetir}{1x}
\beginverse
 Maria disse meu filho 
Faz tudo que o Pai mandou
 Pra salvar a humanidade
Ele lhe determinou
 Com as lágrimas caindo
O seu rosto ele beijou
 Pra cumprir a profecia 
Naquele instante o Messias
Todo pecado abraçou
\endverse
\act{Solo: Parte 1 | Parte 2}{Repetir}{1x}
\beginverse
 Nas margens do Rio Jordão
Jesus Cristo caminhou
 Para encontrar João
Aquele que testemunhou
 O encontro foi tão lindo
Que o povo se emocionou
 Também foi nessa visita
Que nas mãos de João Batista
Jesus Cristo batizou
\endverse
\act{Solo: Parte 2}{Repetir}{1x}
\beginverse
 Na mesa da Santa Ceia
Jesus Cristo ordenou
 Ensine os meus mandamentos
Que onde está meu Pai eu vou
 Se o mundo lhes odiar
Também já me odiou
 Faça o bem sem ver a quem 
A sua recompensa vem
É o que Jesus profetizou
\endverse

%-----------------------------------------------------------------
\vspace{4em} % Regulador de Espaçamento
%=================================================================

%=================================================================
% PADRÃO: [TonalidadeMaior+NOTAX+Variações] .Ex:[X50] [X57V1V7]
% OBS: Variações são alterações do acorde em relação ao campo harmônico.
%-----------------------------------------------------------------
% Tipos de Variações de Acordes:
% V0 - Variação Diversa
% V1 - Menor (m)
% V2 - Maior (M)
% V3 - Meio Tom Abaixo (Bemol)
% V4 - Com Quarta (ex:C4)
% V5 - Com Quinta (ex:C5)
% V6 - Com Sexta (ex:C6)
% V7 - Com Sétima Menor (ex:C7)
% V8 - Com baixo dois Tons Acima (ex:D/F#)
% V9 - Com Nona (ex:C9)
% V10 - Meio Tom Acima (Sustenido)
% V11 - Com Sétima Maior (ex:C7M)
% V12 - Suspenso (Sus)
% V13 - Com baixo dois Tons e Meio Acima (ex:A/E)
% V14 - Com baixo um Tom e Meio Acima (ex:D9/F) 
% V15 - Meio-Diminuto (m7b5)
% N15 - NÃO Meio-Diminuto
% V16 - Diminuto (º)
% N16 - NÃO Diminuto
% V17 - Com baixo um Tom Acima (ex: C/D)
% V18 - Com baixo um Tom Abaixo (ex: Em/D)
% V19 - Com baixo dois Tons e meio Abaixo (ex: G/D)
%=================================================================
%\vspace{4em} % Regulador de Espaçamento
%=================================================================
\endsong
%=================================================================
