%=================================================================
\songcolumns{2}
\beginsong
{Toma o Teu Lugar %TÍTULO
}[by={Diante do Trono %ARTISTA
},album={@walyssondosreis},
id={GB0121 %COD.ID.: GB0000
}] 
%-----------------------------------------------------------------
\tom{X1}
%=================================================================
%\newchords{verse0.GB0000} % Registrador de Acordes em Sequência
%-----------------------------------------------------------------
\seq{Intro 1}{X1 X5V8 X6 X4}{3x}
\seq{Intro 2}{X1}{}
%-----------------------------------------------------------------
\begin{verse}
\[X1]Estamos reu\[X4]nidos para ado\[X1]rar
Ao \[X1]único que é \[X4]digno exal\[X1]tar
\end{verse}
%-----------------------------------------------------------------
\begin{verse}
\[X6]Entronizado nos louvores do Teu \[X5V9V8]povo
\[X4]Reina, \[X5V9V8]reina!
\end{verse}
%-----------------------------------------------------------------
\begin{chorus}
Toma o teu lu\[X1]gar na minha \[X5V9V8]vida
Toma o teu lu\[X6]gar na minha fa\[X4]mília
Toma o teu lu\[X2]gar, rei da \[X4]glória, rei\[X6]na! \[X5V8]
Toma o teu lu\[X1]gar nesta ci\[X5V9V8]dade
Toma o teu lu\[X6]gar nesta na\[X4]ção
Toma o teu lu\[X2]gar, rei da \[X4]glória, rei\[X6]na! \[X5V8]
\end{chorus}
%-----------------------------------------------------------------
\seq{Riff Intro 1}{X1 X5V8 X6 X4}{2x}
%-----------------------------------------------------------------
\act{Repetir}{Verso 2}
%-----------------------------------------------------------------
\act{Repetir}{Refrão}
%-----------------------------------------------------------------
\begin{verse}
Levan\[X4]tai, ó portas, \[X6]vossas cabeças
Levan\[X4]tai-vos, ó portais e\[X6]ternos
\end{verse}
%-----------------------------------------------------------------
\begin{verse}
\[X4]Para que entre o rei da glória
\[X2]Quem é esse rei da glória?
\[X4]O Senhor dos exércitos, Je\[X6]sus!
\[X4]Para que entre o rei da glória
\[X2]Quem é esse rei da glória?
\[X4]O Senhor dos exércitos, Je\[X1]sus!\[X5V8]
\end{verse}
%-----------------------------------------------------------------
\act{Repetir}{Refrão}
\act{Finalizar}{Verso 2}
%-----------------------------------------------------------------
\begin{comment}
\lstset{basicstyle=\scriptsize\bf} % Parâmetros da TAB
%-----------------------------------------------------------------
\tab{Solo 1}
\begin{lstlisting}
E|-----------------------------------------------------|
B|-----------------------------------------------------|
G|-----------------------------------------------------|
D|-----------------------------------------------------|
A|-----------------------------------------------------|
E|-----------------------------------------------------|
\end{lstlisting}
%-----------------------------------------------------------------
\end{comment}
%=================================================================
\vspace{2em}
%-----------------------------------------------------------------
\gtab{\color{black} X1}{}% G [X1]
\gtab{\color{black} X2}{}% Am [X2]
\gtab{\color{black} X4}{}% C [X4]
\gtab{\color{black} X5V8}{}% D/F# [X5V8]
\gtab{\color{black} X5V9V8}{}% D9/F# [X5V9V8]
\gtab{\color{black} X6}{}% Em [X6]
%-----------------------------------------------------------------
% PADRÃO [TonalidadeMaiorNOTAX.Variação] .Ex:[X50] [X50V1]
% PADRÃO [TonalidadeMenorNOTAX.Variação] .Ex:[mX50] [mX50V1]
% OBS: Variações são alterações do acorde em relação ao campo harmônico.
%-----------------------------------------------------------------
% TIPOS DE VARIAÇÂO DOS ACORDES:
% V1 - ACORDE MENOR (m)
% V2 - ACORDE MAIOR (M)
% V3 - ACORDE MEIO TOM ABAIXO (Bemois)
% V4 - ACORDE COM QUARTA (C4)
% V5 - ACORDE COM QUINTA (C5)
% V6 - ACORDE COM SEXTA (C6)
% V7 - ACORDE COM SÉTIMA MENOR (C7)
% V8 - ACODE COM BAIXO DOIS TONS ACIMA (D/F#)
% V9 - ACORDE COM NONA (C9)
% V10 - ACORDE MEIO TOM ACIMA (Sustenidos)
% V11 - ACORDE COM SÉTIMA MAIOR (C7M)
% V12 - ACORDE SUSPENSO (Sus)
% V13 - ACORDE COM VARIAÇÃO DIVERSA
%=================================================================
\endsong
%=================================================================

\begin{comment}

\end{comment}





