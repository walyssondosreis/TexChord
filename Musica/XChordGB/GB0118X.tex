%=================================================================
\songcolumns{1}
\beginsong
{Teu Santo Nome %TÍTULO
}[by={Gabriela Rocha  %ARTISTA
},album={@walyssondosreis},
id={GB0118 %COD.ID.: GB0000
},rev={3}, %REVISÃO
qr={https://drive.google.com/open?id=1M5PtisFixjQiuvUoJQgB_4WflzlRh4ku %LINK
}]
%-----------------------------------------------------------------
\tom{X1}{G}
%=================================================================
%\newchords{verse1.GB0000X} % Registrador de Acordes em Sequência
%\newchords{chorus1.GB0000X} % Registrador de Acordes em Sequência
%-----------------------------------------------------------------
\seq{Intro}{X4V9 X1}{}
%-----------------------------------------------------------------
%\beginverse \endverse
%\beginchorus \endchorus

\beginverse
Todo ser que \[X2V7]vive 
Louve o nome \[X6V7]do Senhor
Toda cria\[X4V9]tura se derrame aos se\[X1]us pés \[X5V9]
\endverse
\beginverse
Ao som da Sua ^voz 
O universo ^se desfaz
Não há outro ^nome comparado ao grande eu^ sou ^
\endverse

\beginverse
\[X2V7]E mesmo sendo \[X4V9]pó
Com tudo que \[X1]há em mim
Con\[X5V9]fessarei
\[X2V7]Que céus e terra \[X4V9]passarão
Mas \[X1]o Teu nome \[X5V9]é eterno
\endverse

\beginchorus
\[X6V7]Todo joelho dobra\[X4V9]rá
Ao ouvir Teu \[X1]nome
Teu Santo \[X5V9]nome
\[X6V7]Todo ser confessa\[X4V8]rá
Louvado seja o Teu \[X1V13]nome
Teu Santo \[X5V9]nome
\endchorus
\act{Retomar}{Verso 3}{1x}
\act{Repetir}{Refrão}{+2x}
%-----------------------------------------------------------------
\vspace{2em} % Regulador de Espaçamento
%-----------------------------------------------------------------
\begin{comment}
\lstset{basicstyle=\scriptsize\bf} % Parâmetros da TAB
%-----------------------------------------------------------------
\tab{Solo 1}
\begin{lstlisting}
E|-----------------------------------------------------|
B|-----------------------------------------------------|
G|-----------------------------------------------------|
D|-----------------------------------------------------|
A|-----------------------------------------------------|
E|-----------------------------------------------------|
\end{lstlisting}
%-----------------------------------------------------------------
\end{comment}
%=================================================================

%-----------------------------------------------------------------
\color{drawChord}\gtab{\color{nameChord} X1}{}% G [X1]
\color{drawChord}\gtab{\color{nameChord} X1V13}{}% G/D [X1V13]
\color{drawChord}\gtab{\color{nameChord} X2V7}{}% Am7 [X2V7]
\color{drawChord}\gtab{\color{nameChord} X4V8}{}% C/E [X4V8]
\color{drawChord}\gtab{\color{nameChord} X4V9}{}% C9 [X4V9]
\color{drawChord}\gtab{\color{nameChord} X5V9}{}% D9 [X5V9]
\color{drawChord}\gtab{\color{nameChord} X6V7}{}% Em7 [X6V7]
%-----------------------------------------------------------------
% PADRÃO [TonalidadeMaiorNOTAX.Variação] .Ex:[X50] [X50V1]
% PADRÃO [TonalidadeMenorNOTAX.Variação] .Ex:[mX50] [mX50V1]
% OBS: Variações são alterações do acorde em relação ao campo harmônico.
%-----------------------------------------------------------------
% TIPOS DE VARIAÇÂO DOS ACORDES:
% V0 - ACORDE COM VARIAÇÃO DIVERSA
% V1 - ACORDE MENOR (m)
% V2 - ACORDE MAIOR (M)
% V3 - ACORDE MEIO TOM ABAIXO (Bemois)
% V4 - ACORDE COM QUARTA (C4)
% V5 - ACORDE COM QUINTA (C5)
% V6 - ACORDE COM SEXTA (C6)
% V7 - ACORDE COM SÉTIMA MENOR (C7)
% V8 - ACORDE COM BAIXO DOIS TONS ACIMA (D/F#)
% V9 - ACORDE COM NONA (C9)
% V10 - ACORDE MEIO TOM ACIMA (Sustenidos)
% V11 - ACORDE COM SÉTIMA MAIOR (C7M)
% V12 - ACORDE SUSPENSO (Sus)
% V13 - ACORDE COM BAIXO DOIS TONS E MEIO ACIMA (A/E)
% V14 - ACORDE UM TOM E MEIO ACIMA (D9/F)
%=================================================================
\endsong
%=================================================================
\begin{comment}

\end{comment}