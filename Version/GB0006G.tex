%=================================================================
\songcolumns{2}
\beginsong
{Aclame Ao Senhor %TÍTULO
}[by={Diante do Trono %ARTISTA
},album={@walyssondosreis},
id={GB0006 %COD.ID.: GB0000
(Rev.1) %REVISÃO.: 0...N
}]
%-----------------------------------------------------------------
\tom{G}
%=================================================================
%\newchords{verse1.GB0000X} % Registrador de Acordes em Sequência
%\newchords{chorus1.GB0000X} % Registrador de Acordes em Sequência
%-----------------------------------------------------------------
\seq{Intro}{G Em}{2x}
%-----------------------------------------------------------------
%\beginverse \endverse
%\beginchorus \endchorus
\beginverse
\[G] Meu Jesus,\[D] salvador
\[Em]Outro i\[D]gual não \[C]há
Todos os \[G/B]dias \[C]quero lou\[G]var
As \[Em]maravilhas \[F]de \[C]teu a\[D4]mor \[D]
\endverse
\beginverse
^ Consolo,^ abrigo
^Força e re^fúgio é o Se^nhor
Com todo o meu ^ser
Com ^tudo o que ^sou
^Sempre Te a^do^ra^rei ^
\endverse
\beginchorus
A\[G]clame ao Se\[Em]nhor toda a \[C]terra e can\[D4]te\[D]mos
Po\[G]der, majes\[Em]tade e lou\[C]vores ao \[D]Rei \[D4]
Mon\[Em]tanhas se \[D]prostrem e \[C]rujam os mares
Ao \[D]som \[C]de teu \[D/F\#]nome
A\[G]legre Te \[Em]louvo por \[C]Teus grandes \[D4]fei\[D]tos
Fir\[G]mado esta\[Em]rei, sempre \[C]te ama\[D]rei \[D4]
\[Em]Incompa\[D]ráveis são \[C]tuas pro\[D]messas 
Pra \[G]mim
\endchorus
\act{Riff Intro}{G Em}{1x}
\act{Retomar}{Verso 1}{1x}
\act{Repetir:Ao final da penúltima linha}{Refrão}{2x}
\act{Executar:Ao final da penúltima linha}{Verso 3}{}
\beginverse
...\[Em]Incompa\[D]ráveis são \[C]Tuas pro\[D]messas
\endverse
\act{Repetir}{Verso 3}{4x}
\beginverse
...Pra \[G]mim
\endverse


%-----------------------------------------------------------------
\begin{comment}
\lstset{basicstyle=\scriptsize\bf} % Parâmetros da TAB
%-----------------------------------------------------------------
\tab{Solo 1}
\begin{lstlisting}
E|-----------------------------------------------------|
B|-----------------------------------------------------|
G|-----------------------------------------------------|
D|-----------------------------------------------------|
A|-----------------------------------------------------|
E|-----------------------------------------------------|
\end{lstlisting}
%-----------------------------------------------------------------
\end{comment}
%=================================================================
\vspace{2em} 
%-----------------------------------------------------------------
\gtab{\color{black} G}{~:320003}%
\gtab{\color{black} G/B}{}%
\gtab{\color{black} G}{~:320003}% 
\gtab{\color{black} C}{~:X32010}%
\gtab{\color{black} C}{~:X32010}%
\gtab{\color{black} D}{~:XX0232}\\%
\gtab{\color{black} D4}{}%
\gtab{\color{black} D/F\#}{}%
\gtab{\color{black} Em}{}% 
\gtab{\color{black} F}{1:022100}%

%-----------------------------------------------------------------
% PADRÃO: [TonalidadeMaior+NOTAX+Variações] .Ex:[X50] [X57V1V7]
% OBS: Variações são alterações do acorde em relação ao campo harmônico.
%-----------------------------------------------------------------
% Tipos de Variações de Acordes:
% V0 - Variação Diversa
% V1 - Menor (m)
% V2 - Maior (M)
% V3 - Meio Tom Abaixo (Bemol)
% V4 - Com Quarta (ex:C4)
% V5 - Com Quinta (ex:C5)
% V6 - Com Sexta (ex:C6)
% V7 - Com Sétima Menor (ex:C7)
% V8 - Com baixo dois Tons Acima (ex:D/F#)
% V9 - Com Nona (ex:C9)
% V10 - Meio Tom Acima (Sustenido)
% V11 - Com Sétima Maior (ex:C7M)
% V12 - Suspenso (Sus)
% V13 - Com baixo dois Tons e Meio Acima (ex:A/E)
% V14 - Com baixo um Tom e Meio Acima (ex:D9/F) 
% V15 - Meio-Diminuto (m7b5)
% N15 - NÃO Meio-Diminuto
% V16 - Diminuto (º)
% N16 - NÃO Diminuto
%=================================================================
\endsong
%=================================================================
\begin{comment}

\end{comment}