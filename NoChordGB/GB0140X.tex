%=================================================================
\songcolumns{1}
\beginsong
{Faço o Melhor %TÍTULO
}[by={Filhos do Homem %ARTISTA
},album={@walyssondosreis},
id={GB0140 %COD.ID.: GB0000
},rev={0}, %REVISÃO
qr={https://drive.google.com/open?id=151Xl8HxdyGp191NhyrPg56DEIVGpejSJ %LINK
}]
%-----------------------------------------------------------------
\tom{X1}{C}
%=================================================================
%\newchords{verse1.GB0000X} % Registrador de Acordes em Sequência
%\newchords{chorus1.GB0000X} % Registrador de Acordes em Sequência
%-----------------------------------------------------------------
\seq{Intro}{X4 X5 X3 X6V7 X4 X5 X1 XV7}{4x}
%-----------------------------------------------------------------
%\beginverse* \endverse
%\beginchorus \endchorus
\beginverse*
\[X1]Cante, \[X5V8]cante, 
\[X6]Cante ao Senhor pra \[X4]sem\[X5]pre
\[X]Cante, \[X]cante, 
\[X6]Cante ao Senhor e \[X4]faça o seu me\[X5]lhor
\endverse
\beginchorus
\[X4] Faço o me\[X5]lhor, \[X3] faço o me\[X6]lhor ôôô
\[X4] Faço o me\[X5]lhor pra \[X1]Deus \[X1V7]
\endchorus
\beginverse*
Aplauda, Aplauda, Aplauda ao Senhor pra sempre
Aplauda, Aplauda, Aplauda ao Senhor e faça o seu melhor
\endverse
\beginverse*
Pule, Pule, Pule ao Senhor pra sempre
Pule, Pule, Pule ao Senhor e faça o seu melhor
\endverse
\beginverse*
Eu amo, eu amo, eu amo meu Senhor pra sempre
Te amo, Te amo, Te amo meu Senhor pra sempre
Te amo, Te amo, Te amo meu irmão pra sempre
\endverse
%-----------------------------------------------------------------
\vspace{4em} % Regulador de Espaçamento
%-----------------------------------------------------------------
\begin{comment}
\lstset{basicstyle=\scriptsize\bf} % Parâmetros da TAB
%-----------------------------------------------------------------
\tab{Solo 1}
\begin{lstlisting}
E|-----------------------------------------------------|
B|-----------------------------------------------------|
G|-----------------------------------------------------|
D|-----------------------------------------------------|
A|-----------------------------------------------------|
E|-----------------------------------------------------|
\end{lstlisting}
%-----------------------------------------------------------------
\end{comment}
%=================================================================
 
%-----------------------------------------------------------------
\color{drawChord}\gtab{\color{nameChord} X}{}% 
\color{drawChord}\gtab{\color{nameChord} X}{}% 
\color{drawChord}\gtab{\color{nameChord} X}{}% 
\color{drawChord}\gtab{\color{nameChord} X}{}% 
%-----------------------------------------------------------------
% PADRÃO: [TonalidadeMaior+NOTAX+Variações] .Ex:[X50] [X57V1V7]
% OBS: Variações são alterações do acorde em relação ao campo harmônico.
%-----------------------------------------------------------------
% Tipos de Variações de Acordes:
% V0 - Variação Diversa
% V1 - Menor (m)
% V2 - Maior (M)
% V3 - Meio Tom Abaixo (Bemol)
% V4 - Com Quarta (ex:C4)
% V5 - Com Quinta (ex:C5)
% V6 - Com Sexta (ex:C6)
% V7 - Com Sétima Menor (ex:C7)
% V8 - Com baixo dois Tons Acima (ex:D/F#)
% V9 - Com Nona (ex:C9)
% V10 - Meio Tom Acima (Sustenido)
% V11 - Com Sétima Maior (ex:C7M)
% V12 - Suspenso (Sus)
% V13 - Com baixo dois Tons e Meio Acima (ex:A/E)
% V14 - Com baixo um Tom e Meio Acima (ex:D9/F) 
% V15 - Meio-Diminuto (m7b5)
% N15 - NÃO Meio-Diminuto
% V16 - Diminuto (º)
% N16 - NÃO Diminuto
%=================================================================
\endsong
%=================================================================
\begin{comment}

\end{comment}