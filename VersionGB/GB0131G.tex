%=================================================================
\songcolumns{2}
\beginsong
{Preciso de ti %TÍTULO
}[by={Diante do Trono %ARTISTA
},album={@walyssondosreis},
id={GB0131 %COD.ID.: GB0000
},rev={3}, %REVISÃO
qr={https://drive.google.com/open?id=13brifVeXpHOmR09SjlK1eNk0HRDOWg2A %LINK
}]
%-----------------------------------------------------------------
\tom{G}{A}
%=================================================================
%\newchords{verse1.GB0000X} % Registrador de Acordes em Sequência
%\newchords{chorus1.GB0000X} % Registrador de Acordes em Sequência
%-----------------------------------------------------------------
\seq{Intro}{F C G D}{2x}
%-----------------------------------------------------------------
%\beginverse \endverse
%\beginchorus \endchorus
\beginverse
Preciso de \[G]ti
Preciso do \[Em]teu perdão
Preciso de \[G]ti
Quebranta meu \[Em]coração
Como a \[C]corça anseia por \[Am]águas, assim tenho \[Em]sede \[D]
Como terra \[C]seca, assim é a minh'\[Am]alma
Preciso de \[D4]ti \[D]
\endverse
\beginchorus
Distante de ti, Se\[C]nhor, não posso vi\[Am]ver
Não vale a \[G]pena existir \[D]
Escuta o meu cla\[C]mor
Mais que o ar que eu res\[Am]piro
Preciso de \[D/F\#]ti \[Am]\[G]\[D]
Distante de ti, Se\[C]nhor, não posso vi\[Am]ver
Não vale a \[G]pena existir \[D]
Escuta o meu cla\[C]mor
Mais que o ar que eu res\[Am]piro
Preciso de \[F]ti \[C]\[G]\[D]
\endchorus
\beginverse
Não posso esque\[Em]cer
O que fi\[C]zeste por mim
Como alto é o \[Bm]céu
Tua miseri\[Am]córdia é sem fim
Como um \[C]pai se compadece dos \[Am]filhos, assim Tu me \[Em]amas \[D]
Afasta as \[C]minhas transgre\[Am]ssões
Preciso de \[D4]ti \[D]
\endverse
\act{Repetir}{Refrão}{1x}
\beginverse
E as \[Bm]lutas vêm tentando me afas\[C]tar de ti
Fri\[Em]eza, escuri\[Bm]dão procuram \[C]me cegar
Mas eu não \[Am]vou desis\[Bm]tir
A\[Em]juda-\[D]me, Se\[C]nhor
Eu quero \[Am]permane\[Bm]cer Contigo a\[F]té o fim
\endverse
\seq{Riff 1}{G F G Eb F D4 D}{}
\act{Repetir}{Refrão}{1x}
\beginverse
...Preciso de \[F]ti \[C]
Preciso de \[Eb]ti \[F]
Preciso de \[G]ti \[D]
Preciso de \[Eb]ti \[F] \[G]
\endverse
%-----------------------------------------------------------------
\begin{comment}
\lstset{basicstyle=\scriptsize\bf} % Parâmetros da TAB
%-----------------------------------------------------------------
\tab{Solo 1}
\begin{lstlisting}
E|-----------------------------------------------------|
B|-----------------------------------------------------|
G|-----------------------------------------------------|
D|-----------------------------------------------------|
A|-----------------------------------------------------|
E|-----------------------------------------------------|
\end{lstlisting}
%-----------------------------------------------------------------
\end{comment}
%=================================================================
\vspace{2em} 
%-----------------------------------------------------------------
\color{drawChord}\gtab{\color{nameChord} G}{~:320003}% 
\color{drawChord}\gtab{\color{nameChord} Am}{~:X02210}%
\color{drawChord}\gtab{\color{nameChord} Bm}{2:X02210}%
\color{drawChord}\gtab{\color{nameChord} C}{~:X32010}%
\color{drawChord}\gtab{\color{nameChord} D}{~:XX0232}\\%
\color{drawChord}\gtab{\color{nameChord} D4}{}%
\color{drawChord}\gtab{\color{nameChord} D/F\#}{~:200232}%
\color{drawChord}\gtab{\color{nameChord} Em}{~:022000}%
\color{drawChord}\gtab{\color{nameChord} Eb}{}%
\color{drawChord}\gtab{\color{nameChord} F}{1:022100}%

%-----------------------------------------------------------------
% PADRÃO: [TonalidadeMaior+NOTAX+Variações] .Ex:[X50] [X57V1V7]
% OBS: Variações são alterações do acorde em relação ao campo harmônico.
%-----------------------------------------------------------------
% Tipos de Variações de Acordes:
% V0 - Variação Diversa
% V1 - Menor (m)
% V2 - Maior (M)
% V3 - Meio Tom Abaixo (Bemol)
% V4 - Com Quarta (ex:C4)
% V5 - Com Quinta (ex:C5)
% V6 - Com Sexta (ex:C6)
% V7 - Com Sétima Menor (ex:C7)
% V8 - Com baixo dois Tons Acima (ex:D/F#)
% V9 - Com Nona (ex:C9)
% V10 - Meio Tom Acima (Sustenido)
% V11 - Com Sétima Maior (ex:C7M)
% V12 - Suspenso (Sus)
% V13 - Com baixo dois Tons e Meio Acima (ex:A/E)
% V14 - Com baixo um Tom e Meio Acima (ex:D9/F) 
% V15 - Meio-Diminuto (m7b5)
% N15 - NÃO Meio-Diminuto
% V16 - Diminuto (º)
% N16 - NÃO Diminuto
%=================================================================
\endsong
%=================================================================
\begin{comment}

\end{comment}