%=================================================================
\songcolumns{1}
\beginsong
{Galileu %TÍTULO
}[by={Fernandinho %ARTISTA
},album={@walyssondosreis},
id={GB0049 %COD.ID.: GB0000
(Rev.0) %REVISÃO.: 0...N
}]
%-----------------------------------------------------------------
\tom{X1}
%=================================================================
%\newchords{verse1.GB0000X} % Registrador de Acordes em Sequência
%\newchords{chorus1.GB0000X} % Registrador de Acordes em Sequência
%-----------------------------------------------------------------
%\seq{Intro}{}{}
%-----------------------------------------------------------------
%\beginverse* \endverse
%\beginchorus \endchorus
Deixou Sua glória
Foi por amor, foi por amor
E o seu sangue derramou
Que grande amor

Naquela via dolorosa, se entregou
Eu não mereço, mas Sua graça me alcançou

Eu me rendo ao seu amor
Eu me rendo ao seu amor
Eu me rendo ao seu amor
Eu me rendo, eu me rendo, eu me rendo

Deus Emanuel
Estrela da Manhã
Cordeiro de Deus
Pão da Vida
Príncipe da paz
Grande El Shaddai
Santo de Israel
Luz do mundo

Galileu, Jesus, Jesus
Galileu, Jesus, Jesus
Galileu, Jesus, Jesus
Galileu, Jesus, Jesus

Tende em vós o mesmo sentimento que houve também em Cristo Jesus
Pois Ele subsistindo em forma de Deus
Não julgou como usurpação ser igual a Deus
Antes, a si mesmo esvaziou, assumindo a forma de servo
Tornando-se em semelhança de homens
E reconhecido em figura humana

E a si mesmo se humilhou, tornando-se obediente até a morte de cruz
Pelo que também Deus o exaltou sobremaneira
E lhe deu ao nome que está acima de todo nome
Para que ao nome de Jesus se dobre todo joelho nos céu
Na Terra e debaixo da terra
E toda língua confesse: Jesus Cristo
Jesus Cristo é o Senhor
Para a glória de Deus Pai

Galileu, Jesus, Jesus
Galileu, Jesus, Jesus
Galileu, Jesus, Jesus
Galileu, Jesus, Jesus


% Verso de preenchimento
\beginverse*
.
.
.
\endverse
%-----------------------------------------------------------------
\begin{comment}
\lstset{basicstyle=\scriptsize\bf} % Parâmetros da TAB
%-----------------------------------------------------------------
\tab{Solo 1}
\begin{lstlisting}
E|-----------------------------------------------------|
B|-----------------------------------------------------|
G|-----------------------------------------------------|
D|-----------------------------------------------------|
A|-----------------------------------------------------|
E|-----------------------------------------------------|
\end{lstlisting}
%-----------------------------------------------------------------
\end{comment}
%=================================================================
\vspace{2em} 
%-----------------------------------------------------------------
\gtab{\color{black} X}{}% 
\gtab{\color{black} X}{}% 
\gtab{\color{black} X}{}% 
\gtab{\color{black} X}{}% 
%-----------------------------------------------------------------
% PADRÃO: [TonalidadeMaior+NOTAX+Variações] .Ex:[X50] [X57V1V7]
% OBS: Variações são alterações do acorde em relação ao campo harmônico.
%-----------------------------------------------------------------
% Tipos de Variações de Acordes:
% V0 - Variação Diversa
% V1 - Menor (m)
% V2 - Maior (M)
% V3 - Meio Tom Abaixo (Bemol)
% V4 - Com Quarta (ex:C4)
% V5 - Com Quinta (ex:C5)
% V6 - Com Sexta (ex:C6)
% V7 - Com Sétima Menor (ex:C7)
% V8 - Com baixo dois Tons Acima (ex:D/F#)
% V9 - Com Nona (ex:C9)
% V10 - Meio Tom Acima (Sustenido)
% V11 - Com Sétima Maior (ex:C7M)
% V12 - Suspenso (Sus)
% V13 - Com baixo dois Tons e Meio Acima (ex:A/E)
% V14 - Com baixo um Tom e Meio Acima (ex:D9/F) 
% V15 - Meio-Diminuto (m7b5)
% N15 - NÃO Meio-Diminuto
% V16 - Diminuto (º)
% N16 - NÃO Diminuto
%=================================================================
\endsong
%=================================================================
\begin{comment}

\end{comment}