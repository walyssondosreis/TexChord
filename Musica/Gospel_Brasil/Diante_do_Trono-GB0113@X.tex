%=================================================================
\songcolumns{2}
\beginsong
{Te Agradeço %TÍTULO
}[by={Diante do Trono %ARTISTA
},album={@walyssondosreis},
id={GB0113 %COD.ID.: GB0000
},rev={3}, %REVISÃO
qr={https://drive.google.com/open?id=1vjRs1EjL8D0ia2wCxz0Vm4kmcbh9BtzT %LINK
}]
%-----------------------------------------------------------------
\tom{X1}{Eb}
%=================================================================
%\newchords{verse1.GB0000X} % Registrador de Acordes em Sequência
%\newchords{chorus1.GB0000X} % Registrador de Acordes em Sequência
%-----------------------------------------------------------------
\seq{Intro}{X4 X1 X4 X1}{2x}
%-----------------------------------------------------------------
%\beginverse* \endverse
%\beginchorus \endchorus
\beginverse
Por \[X1]tudo o que tens \[X4]feito
Por \[X1]tudo o que vais fa\[X5]zer
Por \[X1]tuas pro\[X1V7]messas e \[X4V8]tudo o que és
Eu \[X1]quero te agrade\[X5]cer
Com todo o meu \[X4]ser \[X5]
\endverse
\beginverse
Te agra\[X4]deço, meu Se\[(X1)]nhor
Te agra\[X1]deço, meu Se\[(X4)]nhor
Te agra\[X4]deço, meu Se\[(X1)]nhor
Te agra\[X1]deço, meu Se\[(X4)]nhor
\endverse
\beginchorus
Te agra\[X1]deço por me libertar e sal\[X1V7]var
\[X4V8]Por ter morrido em meu lugar, te agra\[X1V8]deç\[X4]o
Jesus, te agra\[X1V8]deç\[X4]o
Eu te agra\[X1V8]deç\[X4]o
Te agra\[X2]deç\[X1]o
\endchorus
\act{Retomar}{Verso 1}{1x}
\act{Repetir}{Verso 2}{}
\act{Repetir}{Refrão}{2x}
\beginverse
...Te agra\[X5]deç\[X1]o
Te agra\[X5]deç\[X1]o
\endverse
%-----------------------------------------------------------------
\vspace{4em} % Regulador de Espaçamento
%-----------------------------------------------------------------
\begin{comment}
\lstset{basicstyle=\scriptsize\bf} % Parâmetros da TAB
%-----------------------------------------------------------------
\tab{Solo 1}
\begin{lstlisting}
E|-----------------------------------------------------|
B|-----------------------------------------------------|
G|-----------------------------------------------------|
D|-----------------------------------------------------|
A|-----------------------------------------------------|
E|-----------------------------------------------------|
\end{lstlisting}
%-----------------------------------------------------------------
\end{comment}
%=================================================================
 
%-----------------------------------------------------------------
\color{drawChord}\gtab{\color{nameChord} X1}{}% 
\color{drawChord}\gtab{\color{nameChord} X1V7}{}% 
\color{drawChord}\gtab{\color{nameChord} X1V8}{}% 
\color{drawChord}\gtab{\color{nameChord} X2}{}% 
\color{drawChord}\gtab{\color{nameChord} X4}{}%
\color{drawChord}\gtab{\color{nameChord} X4V8}{}\\%
\color{drawChord}\gtab{\color{nameChord} X5}{}%
%-----------------------------------------------------------------
% PADRÃO: [TonalidadeMaior+NOTAX+Variações] .Ex:[X50] [X57V1V7]
% OBS: Variações são alterações do acorde em relação ao campo harmônico.
%-----------------------------------------------------------------
% Tipos de Variações de Acordes:
% V0 - Variação Diversa
% V1 - Menor (m)
% V2 - Maior (M)
% V3 - Meio Tom Abaixo (Bemol)
% V4 - Com Quarta (ex:C4)
% V5 - Com Quinta (ex:C5)
% V6 - Com Sexta (ex:C6)
% V7 - Com Sétima Menor (ex:C7)
% V8 - Com baixo dois Tons Acima (ex:D/F#)
% V9 - Com Nona (ex:C9)
% V10 - Meio Tom Acima (Sustenido)
% V11 - Com Sétima Maior (ex:C7M)
% V12 - Suspenso (Sus)
% V13 - Com baixo dois Tons e Meio Acima (ex:A/E)
% V14 - Com baixo um Tom e Meio Acima (ex:D9/F) 
% V15 - Meio-Diminuto (m7b5)
% N15 - NÃO Meio-Diminuto
% V16 - Diminuto (º)
% N16 - NÃO Diminuto
%=================================================================
\endsong
%=================================================================
\begin{comment}

\end{comment}