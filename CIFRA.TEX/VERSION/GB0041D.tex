%=================================================================
\songcolumns{2}
\beginsong
{Eu Vou (África) %TÍTULO
}[by={Fernanda Brum %ARTISTA
},album={@walyssondosreis},
id={GB0041 %COD.ID.: GB0000
}] 
%-----------------------------------------------------------------
\tom{D}
%=================================================================
\newchords{verse1.GB0041D} % Registrador de Acordes em Sequência
%\newchords{chorus1.GB0000X} % Registrador de Acordes em Sequência
%-----------------------------------------------------------------
\seq{Intro}{A}{}
%-----------------------------------------------------------------
%\beginverse* \endverse
%\beginchorus \endchorus
\beginverse*\memorize[verse1.GB0041D]
\[A]Alguém da Ásia me \[D]disse "Vem me aju\[A]dar" \[D]
\[A]Posso ouvir a \[D]África pedir por so\[E4]corro \[E]
Ru\[F\#m]anda, Somália, Ni\[D]géria
Clamando por um \[F\#m]pouco de amor \[D]
\[F\#m]Vou fazer tudo que eu \[D]posso fazer, eu \[Bm]vou \[B4]\[Bm]
Tenho \[D]muito pra dar, eu \[Bm]vou \[B4]\[Bm]
O evan\[D]gelho pregar
\endverse
\beginchorus
\[A]Como ouvi\[D]rão se não há quem \[E4]pregue \[E]
\[A]Como pregar se nin\[D]guém se dispõe a \[G]ir \[F\#m]\[E]
\[A]Como crerão na\[D]queles
De quem nada ou\[E4]viram \[E]
Eis-me a\[Bm]qui\[D], eu \[F\#m]vou\[E]
Eis-me a\[Bm]qui\[D], Se\[F\#m]nhor\[E]
Eis-me \[Bm]aqui \[A/C\#]\[D]
\{ Eu \[A]vou, eu vou
Eu \[G]vou, eu vou
Eu \[D]vou, eu vou
Eu \[F]vou \}
\endchorus
\beginverse*\replay[verse1.GB0041D]
^Vi um menino na ^Rússia olhar para o ^céu ^
^El Salvador tá cho^rando por salva^ção ^
Ro^mênia, Arábia Sau^dita, o Iraque
Es^peram por nós ^
Co^lômbia, Indonésia, Al^bânia, pra China
Eu ^vou ^ ^
Pelo ^chão do Brasil, eu ^vou ^ ^
Deus me ^quer pras nações
\endverse


%-----------------------------------------------------------------
\begin{comment}
\lstset{basicstyle=\scriptsize\bf} % Parâmetros da TAB
%-----------------------------------------------------------------
\tab{Solo 1}
\begin{lstlisting}
E|-----------------------------------------------------|
B|-----------------------------------------------------|
G|-----------------------------------------------------|
D|-----------------------------------------------------|
A|-----------------------------------------------------|
E|-----------------------------------------------------|
\end{lstlisting}
%-----------------------------------------------------------------
\end{comment}
%=================================================================
\vspace{2em} 
%-----------------------------------------------------------------
\gtab{\color{black} D}{}% D [X1H] 
\gtab{\color{black} E}{}% E [X2VM]
\gtab{\color{black} E4}{}% E4 [X2VM4]
\gtab{\color{black} F\#m}{}% F#m [X3H]
\gtab{\color{black} F}{}% F [X3VMb]
\gtab{\color{black} G}{}\\% G [X4H]
\gtab{\color{black} A}{}% A [X5H]
\gtab{\color{black} A/C\#}{}% A/C# [A/C\#]
\gtab{\color{black} Bm}{}% Bm [X6H]
\gtab{\color{black} B4}{}% B4 [X6VM4]
%-----------------------------------------------------------------
% PADRÃO [TonalidadeMaiorNOTAX.Variação] .Ex:[X2H] [X3V1]
% PADRÃO [TonalidadeMenorNOTAX.Variação] .Ex:[mX2H] [mX3V1]
% OBS: Variação são alterações do acorde em relação ao campo harmonico.
%=================================================================
\endsong
%=================================================================
\begin{comment}

\end{comment}