%=================================================================
\songcolumns{2}
\beginsong
{Galileu %TÍTULO
}[by={Fernandinho %ARTISTA
},album={@walyssondosreis},
id={GB0049 %COD.ID.: GB0000
},rev={3}, %REVISÃO
qr={https://drive.google.com/open?id=1ov6LvP15MO4lwp6a15f2y_2kzwNBijJp %LINK
}]
%-----------------------------------------------------------------
\tom{X1}{Db}
%=================================================================
%\newchords{verse1.GB0000X} % Registrador de Acordes em Sequência
%\newchords{chorus1.GB0000X} % Registrador de Acordes em Sequência
%-----------------------------------------------------------------
\seq{Intro 1}{X1}{}
\seq{Intro 2}{X6 X4V11 X6 X4V11}{}
%-----------------------------------------------------------------
%\beginverse* \endverse
%\beginchorus \endchorus
\beginverse
\[X1] Deixou Sua glória foi por amor, foi por amor
\[X6] E o seu \[X4V11]sangue derra\[X6]mou
Que grande a\[X4V11]mor
\endverse
\beginverse
\[X2V7] Naquela \[X4V11]via dolo\[X1]rosa, se entre\[X5V4]gou
\[X2V7] Eu não me\[X4V11]reço, mas Sua \[X1]graça me alcan\[X5V4]çou
\endverse
\beginverse
Eu me \[X2V7]rendo ao seu a\[X4V11]mor
Eu me \[X1]rendo ao seu a\[X5V4]mor
Eu me \[X2V7]rendo ao seu a\[X4V11]mor
Eu me \[X6]rendo, eu me \[X5V4]rendo
\endverse
\beginverse
\[X6]Deus Emanuel
Es\[X4V11]trela da Manhã
\[X1]Cordeiro de Deus
\[X5V4]Pão da Vida
\[X6]Príncipe da paz
\[X4V11]Grande El Shaddai
\[X1]Santo de Israel
\[X5V4]Luz do mundo
\endverse
\beginchorus
\[X2V7]Ga...li\[X4V11]leu, Je\[X1]sus, Je\[X5V4]sus
\[X2V7]Ga...li\[X4V11]leu, Je\[X1]sus, Je\[X5V4]sus
\[X2V7]Ga...li\[X4V11]leu, Je\[X1]sus, Je\[X5V4]sus
\[X2V7]Ga...li\[X4V11]leu, Je\[X1]sus, Je\[X5V4]sus
\endchorus
\seq{Riff Intro}{X6 X4V11 X6 X4V11}{}
\act{Retomar}{Verso 3}{1x}
\beginverse
\[X4V11] Oh, oh, oh, oh
\[X5V4] Oh, oh, oh, oh
\[X6] Oh, oh, oh, oh \[X1]
\endverse
\seq{Riff 2}{X4V11 X5V4 X6 X1}{}
\act{Repetir}{Refrão}{1x}
%-----------------------------------------------------------------
\vspace{4em} % Regulador de Espaçamento
%-----------------------------------------------------------------
\begin{comment}
\lstset{basicstyle=\scriptsize\bf} % Parâmetros da TAB
%-----------------------------------------------------------------
\tab{Solo 1}
\begin{lstlisting}
E|-----------------------------------------------------|
B|-----------------------------------------------------|
G|-----------------------------------------------------|
D|-----------------------------------------------------|
A|-----------------------------------------------------|
E|-----------------------------------------------------|
\end{lstlisting}
%-----------------------------------------------------------------
\end{comment}
%=================================================================
 
%-----------------------------------------------------------------
\color{drawChord}\gtab{\color{nameChord} X1}{}% 
\color{drawChord}\gtab{\color{nameChord} X2V7}{}% 
\color{drawChord}\gtab{\color{nameChord} X4V11}{}% 
\color{drawChord}\gtab{\color{nameChord} X5V4}{}% 
\color{drawChord}\gtab{\color{nameChord} X6}{}%
%-----------------------------------------------------------------
% PADRÃO: [TonalidadeMaior+NOTAX+Variações] .Ex:[X50] [X57V1V7]
% OBS: Variações são alterações do acorde em relação ao campo harmônico.
%-----------------------------------------------------------------
% Tipos de Variações de Acordes:
% V0 - Variação Diversa
% V1 - Menor (m)
% V2 - Maior (M)
% V3 - Meio Tom Abaixo (Bemol)
% V4 - Com Quarta (ex:C4)
% V5 - Com Quinta (ex:C5)
% V6 - Com Sexta (ex:C6)
% V7 - Com Sétima Menor (ex:C7)
% V8 - Com baixo dois Tons Acima (ex:D/F#)
% V9 - Com Nona (ex:C9)
% V10 - Meio Tom Acima (Sustenido)
% V11 - Com Sétima Maior (ex:C7M)
% V12 - Suspenso (Sus)
% V13 - Com baixo dois Tons e Meio Acima (ex:A/E)
% V14 - Com baixo um Tom e Meio Acima (ex:D9/F) 
% V15 - Meio-Diminuto (m7b5)
% N15 - NÃO Meio-Diminuto
% V16 - Diminuto (º)
% N16 - NÃO Diminuto
%=================================================================
\endsong
%=================================================================
\begin{comment}

\end{comment}