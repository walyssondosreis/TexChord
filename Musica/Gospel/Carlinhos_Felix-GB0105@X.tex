%=================================================================
\songcolumns{2}
\beginsong
{Santo Nome %TÍTULO
}[by={Carlinhos Félix %ARTISTA
},album={@walyssondosreis},
id={GB0105 %COD.ID.: GB0000
},rev={3}, %REVISÃO
qr={https://drive.google.com/open?id=1i32w00vt_f5K-xyG3EYQPRIcyv0T6R3r %LINK
}]
%-----------------------------------------------------------------
\tom{X1}{A}
%=================================================================
%\newchords{verse0.GB0000} % Registrador de Acordes em Sequência
%-----------------------------------------------------------------
\seq{Intro 1}{X1 X1}{}

\beginverse
\[X1]Seja o que for, eu não \[X5]temerei
\[X2]O Senhor está co\[X4]migo
\endverse

\seq{Intro 2}{X1 X5 X2 X4}{2x}

\beginverse
\[X1]Seja o que for, eu não \[X5]temerei
\[X2]O Senhor está co\[X4]migo
\[X1]E, da batalha, eu não \[X5]fugirei
\[X2]O Senhor é meu a\[X4]brigo
\endverse

\act{Repetir}{Intro 2}{}
\act{Repetir}{Verso 2}{1x}

\beginverse
Con\[X2]fio na palavra de \[X3]quem me criou
\[X4]Em nada ponho a \[X3]minha fé
\[X2]Senão na graça de \[X3]Cristo Jesus
\[X2]San\[X2]to, \[X2]san\[X2]to, \[X5]san\[X5]to \[X5]no\[X5]me!
\endverse

\beginchorus
\[X1]Teu nome é \[X5]santo
\[X2]Maravilhoso és \[X4]Tu, meu pai!
\[X1]Em Ti des\[X5]canso
\[X2]Por toda vida, hei de \[X4]Te louvar
\endchorus
\beginverse*\color{white}
.
\endverse
\act{Repetir}{Refrão}{+1x}
\act{Repetir}{Intro 2}{}
\act{Retomar}{Verso 3}{1x}
\act{Executar}{Solo 1}{}
\act{Repetir}{Refrão}{2x}
\act{Repetir}{Verso 1}{1x}
\act{Repetir}{Intro 1}{}
%-----------------------------------------------------------------
\vspace{4em} % Regulador de Espaçamento
%-----------------------------------------------------------------

\begin{comment}
\lstset{basicstyle=\scriptsize\bf} % Parâmetros da TAB
%-----------------------------------------------------------------
\tab{Solo 1}
\begin{lstlisting}
E|-----------------------------------------------------|
B|-----------------------------------------------------|
G|-----------------------------------------------------|
D|-----------------------------------------------------|
A|-----------------------------------------------------|
E|-----------------------------------------------------|
\end{lstlisting}
%-----------------------------------------------------------------
\end{comment}
%=================================================================

%-----------------------------------------------------------------
\color{drawChord}\gtab{\color{nameChord} X1}{}% A [X1]
\color{drawChord}\gtab{\color{nameChord} X2}{}% Bm [X2]
\color{drawChord}\gtab{\color{nameChord} X3}{}% C#m [X3]
\color{drawChord}\gtab{\color{nameChord} X4}{}% D [X4]
\color{drawChord}\gtab{\color{nameChord} X5}{}% E [X5]
%-----------------------------------------------------------------
% PADRÃO [TonalidadeMaiorNOTAX.Variação] .Ex:[X50] [X50V1]
% PADRÃO [TonalidadeMenorNOTAX.Variação] .Ex:[mX50] [mX50V1]
% OBS: Variações são alterações do acorde em relação ao campo harmônico.
%-----------------------------------------------------------------
% TIPOS DE VARIAÇÂO DOS ACORDES:
% V0 - ACORDE COM VARIAÇÃO DIVERSA
% V1 - ACORDE MENOR (m)
% V2 - ACORDE MAIOR (M)
% V3 - ACORDE MEIO TOM ABAIXO (Bemois)
% V4 - ACORDE COM QUARTA (C4)
% V5 - ACORDE COM QUINTA (C5)
% V6 - ACORDE COM SEXTA (C6)
% V7 - ACORDE COM SÉTIMA MENOR (C7)
% V8 - ACORDE COM BAIXO DOIS TONS ACIMA (D/F#)
% V9 - ACORDE COM NONA (C9)
% V10 - ACORDE MEIO TOM ACIMA (Sustenidos)
% V11 - ACORDE COM SÉTIMA MAIOR (C7M)
% V12 - ACORDE SUSPENSO (Sus)
% V13 - ACORDE COM BAIXO DOIS TONS E MEIO ACIMA (A/E)
% V14 - ACORDE UM TOM E MEIO ACIMA (D9/F)
%=================================================================
\endsong
%=================================================================







