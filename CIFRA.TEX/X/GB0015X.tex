%=================================================================
\songcolumns{2}
\beginsong
{Canção do Apocalipse %TÍTULO
}[by={Diante do Trono  %ARTISTA
},album={@walyssondosreis},
id={GB0015 %COD.ID.: GB0000
}] 
%-----------------------------------------------------------------
\tom{X1}
%=================================================================
%\newchords{verse0.GB0000} % Registrador de Acordes em Sequência
%-----------------------------------------------------------------
\seq{Intro}{X5V9}{}
%-----------------------------------------------------------------
%\beginverse* \endverse
%\beginchorus \endchorus

\beginverse*
\[X5V9] Digno é o Cordeiro \[X2]
Que foi morto \[X4V9]
Santo, Santo Ele \[X1]é
\[X5V9] Um novo cântico \[X2]
Ao que se assenta \[X4V9]
Sobre o Trono do \[X1]Céu
\endverse

\beginchorus
^ Santo, santo, santo ^
Deus todo poderoso ^
Que era, e é, e há de ^vir
^ Com a criação eu canto ^
Louvores ao Rei dos Reis ^
És tudo para mim
E ^eu te adorarei \[X5V9 X2 X4V9 X1]
\endchorus

\beginverse*
^ Está vestido ^ 
Do arco-íris ^
Sons de trovão, luzes, re^lâmpagos
^ Louvores, honra e glória ^
Força e poder pra sempre ^
Ao único Rei eterna^mente
\endverse

\beginverse*
^ Maravilhado, ^
Extasiado ^
Eu fico ao ouvir Teu ^nome
^ Maravilhado, ^
Extasiado ^
Eu fico ao ouvir Teu ^nome
\[X5V9] Jesus, Teu nome é forç\[X2]a
É fôlego de vid\[X4V9]a
Misteriosa Água \[X1]Viva
\endverse
%-----------------------------------------------------------------
\begin{comment}
\lstset{basicstyle=\scriptsize\bf} % Parâmetros da TAB
%-----------------------------------------------------------------
\tab{Solo 1}
\begin{lstlisting}
E|-----------------------------------------------------|
B|-----------------------------------------------------|
G|-----------------------------------------------------|
D|-----------------------------------------------------|
A|-----------------------------------------------------|
E|-----------------------------------------------------|
\end{lstlisting}
%-----------------------------------------------------------------
\end{comment}
%=================================================================
\vspace{2em}
%-----------------------------------------------------------------
\gtab{\color{black} X1}{}% G [X1]
\gtab{\color{black} X2}{}% Am7 [X2V7]
\gtab{\color{black} X4V9}{}% C9 [X4V9]
\gtab{\color{black} X5V9}{}% D9 [X5V9]
%-----------------------------------------------------------------
% PADRÃO [TonalidadeMaiorNOTAX.Variação] .Ex:[X50] [X50V1]
% PADRÃO [TonalidadeMenorNOTAX.Variação] .Ex:[mX50] [mX50V1]
% OBS: Variações são alterações do acorde em relação ao campo harmônico.
%-----------------------------------------------------------------
% TIPOS DE VARIAÇÂO DOS ACORDES:
% V0 - ACORDE COM VARIAÇÃO DIVERSA
% V1 - ACORDE MENOR (m)
% V2 - ACORDE MAIOR (M)
% V3 - ACORDE MEIO TOM ABAIXO (Bemois)
% V4 - ACORDE COM QUARTA (C4)
% V5 - ACORDE COM QUINTA (C5)
% V6 - ACORDE COM SEXTA (C6)
% V7 - ACORDE COM SÉTIMA MENOR (C7)
% V8 - ACORDE COM BAIXO DOIS TONS ACIMA (D/F#)
% V9 - ACORDE COM NONA (C9)
% V10 - ACORDE MEIO TOM ACIMA (Sustenidos)
% V11 - ACORDE COM SÉTIMA MAIOR (C7M)
% V12 - ACORDE SUSPENSO (Sus)
% V13 - ACORDE COM BAIXO DOIS TONS E MEIO ACIMA (A/E)
% V14 - ACORDE UM TOM E MEIO ACIMA (D9/F)
%=================================================================
\endsong
%=================================================================
\begin{comment}

\end{comment}