%=================================================================
\songcolumns{1}
\beginsong
{O Nosso General É Cristo %TÍTULO
}[by={Adhemar de Campos %ARTISTA
},album={@walyssondosreis},
id={GB0077 %COD.ID.: GB0000
(Rev.0) %REVISÃO.: 0...N
}]
%-----------------------------------------------------------------
\tom{X1}
%=================================================================
%\newchords{verse1.GB0077} % Registrador de Acordes em Sequência
%-----------------------------------------------------------------
\seq{Intro 1}{X6 X5 X6 X4}{2x}
\seq{Intro 2}{X4 X5}{}
%-----------------------------------------------------------------
\begin{verse*}
Pelo Se\[X6]\[X5]nhor mar\[X6]chamos \[X4]sim, o seu e\[X6]\[X5]xército pode\[X6]roso \[X4]é
Sua \[X4]Glória será \[X5]vista em toda a \[X6]terra! \[(X5)]
Vamos can\[X6]\[X5]tar o canto \[X6]da vi\[X4]tória:  Glória a \[X6]\[X5]Deus! vencemos \[X6]a ba\[X4]talha
Toda \[X4]arma contra \[X5]nós perece\[X6]rá!
\end{verse*}
%-----------------------------------------------------------------
\begin{chorus}
O \[X4]nosso general é \[X1]Cristo! Se\[X4]guimos os seus \[X1]passos!
Ne\[X4]nhum inimigo \[X5]nos resisti\[X6]rá! 
\end{chorus}
%-----------------------------------------------------------------
\begin{verse*}
Com o se\[X6]\[X5]nhor, mar\[X6]chamos \[X4]sim, em suas \[X6]\[X5]mãos a chave \[X6]da vi\[X4]tória,
Que nos \[X4]leva a possuir a \[X5]terra prome\[X6]tida! \[(X5)]
Vamos can\[X6]\[X5]tar o canto \[X6]da vi\[X4]tória:  Glória a \[X6]\[X5]Deus! vencemos \[X6]a ba\[X4]talha
Toda \[X4]arma contra \[X5]nós perece\[X6]rá!
\end{verse*}
%-----------------------------------------------------------------
\begin{chorus}
O \[X4]nosso general é \[X1]Cristo! Se\[X4]guimos os seus \[X1]passos!
Ne\[X4]nhum inimigo \[X5]nos resisti\[X6]rá! 
\end{chorus}
\beginverse*
.
.
.
.
.
.
.
\endverse
%-----------------------------------------------------------------
\begin{comment}
\lstset{basicstyle=\scriptsize\bf} % Parâmetros da TAB
%-----------------------------------------------------------------
\tab{Solo 1}
\begin{lstlisting}
E|-----------------------------------------------------|
B|-----------------------------------------------------|
G|-----------------------------------------------------|
D|-----------------------------------------------------|
A|-----------------------------------------------------|
E|-----------------------------------------------------|
\end{lstlisting}
%-----------------------------------------------------------------
\end{comment}
%=================================================================
\vspace{2em}
%-----------------------------------------------------------------
\gtab{\color{black} X1}{}% E [X1]
\gtab{\color{black} X4}{}% A [X4]
\gtab{\color{black} X5}{}% B [X5]
\gtab{\color{black} X6}{}% C#m [X6]
%-----------------------------------------------------------------
% PADRÃO [TonalidadeMaiorNOTAX.Variação] .Ex:[X50] [X50V1]
% PADRÃO [TonalidadeMenorNOTAX.Variação] .Ex:[mX50] [mX50V1]
% OBS: Variações são alterações do acorde em relação ao campo harmônico.
%-----------------------------------------------------------------
% TIPOS DE VARIAÇÂO DOS ACORDES:
% V0 - ACORDE COM VARIAÇÃO DIVERSA
% V1 - ACORDE MENOR (m)
% V2 - ACORDE MAIOR (M)
% V3 - ACORDE MEIO TOM ABAIXO (Bemois)
% V4 - ACORDE COM QUARTA (C4)
% V5 - ACORDE COM QUINTA (C5)
% V6 - ACORDE COM SEXTA (C6)
% V7 - ACORDE COM SÉTIMA MENOR (C7)
% V8 - ACORDE COM BAIXO DOIS TONS ACIMA (D/F#)
% V9 - ACORDE COM NONA (C9)
% V10 - ACORDE MEIO TOM ACIMA (Sustenidos)
% V11 - ACORDE COM SÉTIMA MAIOR (C7M)
% V12 - ACORDE SUSPENSO (Sus)
% V13 - ACORDE COM BAIXO DOIS TONS E MEIO ACIMA (A/E)
% V14 - ACORDE UM TOM E MEIO ACIMA (D9/F)
%=================================================================
\endsong
%=================================================================









