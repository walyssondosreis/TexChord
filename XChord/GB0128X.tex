%=================================================================
\songcolumns{1}
\beginsong
{Vim Para Adorar-Te %TÍTULO
}[by={Adoração e Adoradores %ARTISTA
},album={@walyssondosreis},
id={GB0128 %COD.ID.: GB0000
},rev={3}, %REVISÃO
qr={https://drive.google.com/open?id=1A6LGZwYtNPjKIdJYYya2S_DOG10Q90R8 %LINK
}]
%-----------------------------------------------------------------
\tom{X1}{E}
%=================================================================
%\newchords{verse1.GB0000X} % Registrador de Acordes em Sequência
%\newchords{chorus1.GB0000X} % Registrador de Acordes em Sequência
%-----------------------------------------------------------------
\seq{Intro}{X1 X5 X6 X4 X1 X5 X4}{}
%-----------------------------------------------------------------
%\beginverse* \endverse
%\beginchorus \endchorus
\beginverse
\[X1]Luz do \[X5]mundo vi\[X2]este a \[X4]terra
\[X1]Pra que eu pu\[X5]desse te \[X4]ver
\[X1]Tua be\[X5]leza me \[X2]leva a ado\[X4]rar-te
\[X1]Quero con\[X5]tigo vi\[X4]ver
\endverse
\beginchorus
Vim para ado\[X1]rar-te, vim para pros\[X5]trar-me
Vim para di\[X1V8]zer que és meu \[X4]Deus
És totalmente a\[X1]mável, totalmente \[X5]digno
Tão maravi\[X1V8]lhoso para \[X4]mim
\endchorus
\beginverse
^Eterno ^rei exal^tado nas al^turas, ^glori^oso nos ^céus
Hu^milde vi^este a ^Terra que cri^aste
^Por ^amor pobre se ^fez
\endverse
\act{Repetir}{Refrão}{2x}
\beginverse
Eu \[X5]nunca \[X1V8]sabe\[X4]rei o preço
Dos \[X5]meus pe\[X1V8]cados \[X4]lá na cruz
\endverse
\act{Repetir}{Verso 3}{+5x}
\act{Repetir}{Refrão}{5x}


% Verso de preenchimento
\beginverse*
.
.
.
\endverse
%-----------------------------------------------------------------
\begin{comment}
\lstset{basicstyle=\scriptsize\bf} % Parâmetros da TAB
%-----------------------------------------------------------------
\tab{Solo 1}
\begin{lstlisting}
E|-----------------------------------------------------|
B|-----------------------------------------------------|
G|-----------------------------------------------------|
D|-----------------------------------------------------|
A|-----------------------------------------------------|
E|-----------------------------------------------------|
\end{lstlisting}
%-----------------------------------------------------------------
\end{comment}
%=================================================================
\vspace{2em} 
%-----------------------------------------------------------------
\gtab{\color{black} X1}{}% 
\gtab{\color{black} X1V8}{}% 
\gtab{\color{black} X2}{}% 
\gtab{\color{black} X4}{}% 
\gtab{\color{black} X5}{}%
\gtab{\color{black} X6}{}%
%-----------------------------------------------------------------
% PADRÃO: [TonalidadeMaior+NOTAX+Variações] .Ex:[X50] [X57V1V7]
% OBS: Variações são alterações do acorde em relação ao campo harmônico.
%-----------------------------------------------------------------
% Tipos de Variações de Acordes:
% V0 - Variação Diversa
% V1 - Menor (m)
% V2 - Maior (M)
% V3 - Meio Tom Abaixo (Bemol)
% V4 - Com Quarta (ex:C4)
% V5 - Com Quinta (ex:C5)
% V6 - Com Sexta (ex:C6)
% V7 - Com Sétima Menor (ex:C7)
% V8 - Com baixo dois Tons Acima (ex:D/F#)
% V9 - Com Nona (ex:C9)
% V10 - Meio Tom Acima (Sustenido)
% V11 - Com Sétima Maior (ex:C7M)
% V12 - Suspenso (Sus)
% V13 - Com baixo dois Tons e Meio Acima (ex:A/E)
% V14 - Com baixo um Tom e Meio Acima (ex:D9/F) 
% V15 - Meio-Diminuto (m7b5)
% N15 - NÃO Meio-Diminuto
% V16 - Diminuto (º)
% N16 - NÃO Diminuto
%=================================================================
\endsong
%=================================================================
\begin{comment}

\end{comment}