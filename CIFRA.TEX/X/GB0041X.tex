%=================================================================
\songcolumns{2}
\beginsong
{Eu Vou (África) %TÍTULO
}[by={Fernanda Brum %ARTISTA
},album={@walyssondosreis},
id={GB0041 %COD.ID.: GB0000
}] 
%-----------------------------------------------------------------
\tom{X1}
%=================================================================
\newchords{verse1.GB0041X} % Registrador de Acordes em Sequência
%\newchords{chorus1.GB0000X} % Registrador de Acordes em Sequência
%-----------------------------------------------------------------
\seq{Intro}{X1}{}
%-----------------------------------------------------------------
%\beginverse* \endverse
%\beginchorus \endchorus
\beginverse*\memorize[verse1.GB0041X]
\[X1]Alguém da Ásia me \[X4]disse "Vem me aju\[X1]dar" \[X4]
\[X1]Posso ouvir a \[X4]África pedir por so\[X5V4]corro \[X5]
Ru\[X6]anda, Somália, Ni\[X4]géria
Clamando por um \[X6]pouco de amor \[X4]
\[X6]Vou fazer tudo que eu \[X4]posso fazer, eu \[X2V2V4]vou \[X2V2]
Tenho \[X4]muito pra dar, eu \[X2V2V4]vou \[X2V2]
O evan\[X4]gelho pregar
\endverse
\beginchorus
\[X1]Como ouvi\[X4]rão se não há quem \[X5V4]pregue \[X5]
\[X1]Como pregar se nin\[X4]guém se dispõe a \[X7V2]ir \[X6]\[X5]
\[X1]Como crerão na\[X4]queles
De quem nada ou\[X5V4]viram \[X5]
Eis-me a\[X2]qui\[X4], eu \[X6]vou\[X5]
Eis-me a\[X2]qui\[X4], Se\[X6]nhor\[X5]
Eis-me \[X2]aqui \[X1V8]\[X4]
\{ Eu \[X1]vou, eu vou
Eu \[X7V2]vou, eu vou
Eu \[X4]vou, eu vou
Eu \[X6V2V3]vou \}
\endchorus
\beginverse*\replay[verse1.GB0041X]
^Vi um menino na ^Rússia olhar para o ^céu ^
^El Salvador tá cho^rando por salva^ção ^
Ro^mênia, Arábia Sau^dita, o Iraque
Es^peram por nós ^
Co^lômbia, Indonésia, Al^bânia, pra China
Eu ^vou ^
Pelo ^chão do Brasil, eu ^vou ^
Deus me ^quer pras nações
\endverse


%-----------------------------------------------------------------
\begin{comment}
\lstset{basicstyle=\scriptsize\bf} % Parâmetros da TAB
%-----------------------------------------------------------------
\tab{Solo 1}
\begin{lstlisting}
E|-----------------------------------------------------|
B|-----------------------------------------------------|
G|-----------------------------------------------------|
D|-----------------------------------------------------|
A|-----------------------------------------------------|
E|-----------------------------------------------------|
\end{lstlisting}
%-----------------------------------------------------------------
\end{comment}
%=================================================================
\vspace{2em} 
%-----------------------------------------------------------------
\gtab{\color{black} X1}{}% D [X1]
\gtab{\color{black} X1V8}{}% D/F# [X1V8]
\gtab{\color{black} X2}{}% Em [X2]
\gtab{\color{black} X2V2}{}% E [X2V2]
\gtab{\color{black} X2V2V4}{}% E4 [X2V2V4]
\gtab{\color{black} X4}{}\\% G [X4] 
\gtab{\color{black} X5}{}% A [X5]
\gtab{\color{black} X5V4}{}% A4 [X5V4]
\gtab{\color{black} X6}{}% Bm [X6]
\gtab{\color{black} X6V2V3}{}% Bb [X6V2V3]
\gtab{\color{black} X7V2}{}% C [X7V2]
%-----------------------------------------------------------------
% PADRÃO [TonalidadeMaiorNOTAX.Variação] .Ex:[X50H] [X50V1]
% PADRÃO [TonalidadeMenorNOTAX.Variação] .Ex:[mX50H] [mX50V1]
% OBS: Variações são alterações do acorde em relação ao campo harmônico.
% OBS: 'H' Denomina acordes que são naturais do campo harmônico.
%-----------------------------------------------------------------
% TIPOS DE VARIAÇÂO DOS ACORDES:
% V0 - ACORDE COM VARIAÇÃO DIVERSA
% V1 - ACORDE MENOR (m)
% V2 - ACORDE MAIOR (M)
% V3 - ACORDE MEIO TOM ABAIXO (Bemois)
% V4 - ACORDE COM QUARTA (C4)
% V5 - ACORDE COM QUINTA (C5)
% V6 - ACORDE COM SEXTA (C6)
% V7 - ACORDE COM SÉTIMA MENOR (C7)
% V8 - ACORDE COM BAIXO DOIS TONS ACIMA (D/F#)
% V9 - ACORDE COM NONA (C9)
% V10 - ACORDE MEIO TOM ACIMA (Sustenidos)
% V11 - ACORDE COM SÉTIMA MAIOR (C7M)
% V12 - ACORDE SUSPENSO (Sus)
% V13 - ACORDE COM BAIXO DOIS TONS E MEIO ACIMA (A/E)
% V14 - ACORDE UM TOM E MEIO ACIMA (D9/F)
%=================================================================
\endsong
%=================================================================
\begin{comment}

\end{comment}