%=================================================================
\songcolumns{1}
\beginsong
{Grande é o Senhor %TÍTULO
}[by={Adhemar de Campos %ARTISTA
},album={@walyssondosreis},
id={GB0164 %COD.ID.: XXNNNN
},rev={3}, %REVISÃO
qr={https://drive.google.com/open?id=1UUMt9vQPoAf6jk7K6eMGilZ6iY7IbMq3 %LINK
}]
%-----------------------------------------------------------------
\tom{X1}{A}
%=================================================================
%\newchords{verse1.XX0000X} % Registrador de Acordes em Sequência
%\newchords{chorus1.XX0000X} % Registrador de Acordes em Sequência
%-----------------------------------------------------------------
\seq{Intro}{X1 X4 X5}{2x} 
%\act{}{}{}
%-----------------------------------------------------------------
%\beginverse \endverse
%\beginchorus \endchorus
\beginverse
\[X1]Grande é o Se\[X4]nhor e mui \[X5]digno de lou\[X1]vor
Na ci\[X4]dade do nosso \[X5]Deus seu santo \[X6V7]monte
\[X1]Alegria de toda \[X2V7]terra \[X3V7]\[X4]\[X5]
\endverse
\beginverse
^Grande é o Se^nhor em quem nós ^temos a vi^tória
Que ^nos ajuda ^contra o ini^migo
Por ^isso diante dele nos pros^tra...^mo...^oo...^os
\endverse
\beginchorus
Que\[X1]remos o teu nome engrande\[X3V7]cer
\[X4]E agrade\[X1V8]cer-te por tua \[X2]obra em nossas \[X5]vidas
Confi\[X1]amos em teu infinito a\[X3V7]mor
Pois \[X4]só tu és o \[X1V8]Deus eterno
( \[X2]Sobre toda \[X5]terra 
E \[X1]céus )
\endchorus
\act{Retomar}{Verso 1}{1x}
\act{Repetir}{Refrão}{+1x}
\beginverse
...\[X4]Só tu és o \[X1V8]Deus eterno
\[X2]Sobre toda \[X5]terra 
E \[X1]céus
\endverse
%-----------------------------------------------------------------
\vspace{4em} % Regulador de Espaçamento
%-----------------------------------------------------------------
\begin{comment}
\lstset{basicstyle=\scriptsize\bf} % Parâmetros da TAB
%-----------------------------------------------------------------
\tab{Solo 1}
\begin{lstlisting}
E|-----------------------------------------------------|
B|-----------------------------------------------------|
G|-----------------------------------------------------|
D|-----------------------------------------------------|
A|-----------------------------------------------------|
E|-----------------------------------------------------|
\end{lstlisting}
%-----------------------------------------------------------------
\end{comment}
%=================================================================
 
%-----------------------------------------------------------------
\color{drawChord}\gtab{\color{nameChord} X1}{}% 
\color{drawChord}\gtab{\color{nameChord} X1V8}{}% 
\color{drawChord}\gtab{\color{nameChord} X2}{}% 
\color{drawChord}\gtab{\color{nameChord} X2V7}{}%
\color{drawChord}\gtab{\color{nameChord} X3V7}{}%
\color{drawChord}\gtab{\color{nameChord} X4}{}%
\color{drawChord}\gtab{\color{nameChord} X5}{}%
\color{drawChord}\gtab{\color{nameChord} X6V7}{}% 
%-----------------------------------------------------------------
% PADRÃO: [TonalidadeMaior+NOTAX+Variações] .Ex:[X50] [X57V1V7]
% OBS: Variações são alterações do acorde em relação ao campo harmônico.
%-----------------------------------------------------------------
% Tipos de Variações de Acordes:
% V0 - Variação Diversa
% V1 - Menor (m)
% V2 - Maior (M)
% V3 - Meio Tom Abaixo (Bemol)
% V4 - Com Quarta (ex:C4)
% V5 - Com Quinta (ex:C5)
% V6 - Com Sexta (ex:C6)
% V7 - Com Sétima Menor (ex:C7)
% V8 - Com baixo dois Tons Acima (ex:D/F#)
% V9 - Com Nona (ex:C9)
% V10 - Meio Tom Acima (Sustenido)
% V11 - Com Sétima Maior (ex:C7M)
% V12 - Suspenso (Sus)
% V13 - Com baixo dois Tons e Meio Acima (ex:A/E)
% V14 - Com baixo um Tom e Meio Acima (ex:D9/F) 
% V15 - Meio-Diminuto (m7b5)
% N15 - NÃO Meio-Diminuto
% V16 - Diminuto (º)
% N16 - NÃO Diminuto
% V17 - Com baixo um Tom Acima (ex: C/D)
% V18 - Com baixo um Tom Abaixo (ex: Em/D)
% V19 - Com baixo dois Tons e meio Abaixo (ex: G/D)
%=================================================================
\endsong
%=================================================================
\begin{comment}

\end{comment}