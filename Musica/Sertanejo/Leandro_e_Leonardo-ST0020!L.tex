%=================================================================
\songcolumns{1}
\beginsong
{Fui dando Porrada %TÍTULO
}[by={Leandro e Leonardo %ARTISTA
},album={@walyssondosreis},
id={\href{https://music.youtube.com/watch?v=h4-aUYk5QWA&feature=share %LINK
}{ST0020 %COD.ID.: XXNNNN
}},rev={0}, %REVISÃO
qr={https://music.youtube.com/watch?v=h4-aUYk5QWA&feature=share %LINK
}]
%-----------------------------------------------------------------
\tom{}{A}
%=================================================================
%\newchords{verse1.XX0000X} % Registrador de Acordes em Sequência
%\newchords{chorus1.XX0000X} % Registrador de Acordes em Sequência
%-----------------------------------------------------------------
\seq{Intro}{...}{}
%\act{}{}{}
%-----------------------------------------------------------------
%\beginverse \endverse
%\beginchorus \endchorus
\beginverse 
Amor, eu já tentei mas não te esqueço
Com outra sou um homem infeliz
Eu já estou pagando um alto preço
Pra desfazer o mal que eu te fiz
\endverse

\beginverse 
Por isso não desligue o telefone
Mê dê mais uma chance de explicar
Te vendo conversar com outro homem
Senti tanta vontade de brigar
\endverse

\beginchorus
Fui dando porrada, quebrando tudo
Te rebaixando, ele defendendo, o povo gritando
Mas eu fiz tudo foi por amor
Pois quem ama tanto, ciúme muito
Transforma em pranto, eu te confesso
Que não sou santo, mas me perdoe, por favor
Vai, me chame de volta
Me dê uma chance de provar que me arrependi
Vai, diz que não importa
Que já esqueceu e ainda quer viver só para mim
\endchorus

%-----------------------------------------------------------------
\vspace{4em} % Regulador de Espaçamento
%-----------------------------------------------------------------
\begin{comment}
\lstset{basicstyle=\scriptsize\bf} % Parâmetros da TAB
%-----------------------------------------------------------------
\tab{Solo 1}
\begin{lstlisting}
E|-----------------------------------------------------|
B|-----------------------------------------------------|
G|-----------------------------------------------------|
D|-----------------------------------------------------|
A|-----------------------------------------------------|
E|-----------------------------------------------------|
\end{lstlisting}
%-----------------------------------------------------------------
\end{comment}
%=================================================================
\begin{comment}

\color{drawChord}\gtab{\color{nameChord} X}{}% 
\color{drawChord}\gtab{\color{nameChord} X}{}% 
\color{drawChord}\gtab{\color{nameChord} X}{}% 
\color{drawChord}\gtab{\color{nameChord} X}{}% 

\end{comment}
%=================================================================
% PADRÃO: [TonalidadeMaior+NOTAX+Variações] .Ex:[X50] [X57V1V7]
% OBS: Variações são alterações do acorde em relação ao campo harmônico.
%-----------------------------------------------------------------
% Tipos de Variações de Acordes:
% V0 - Variação Diversa
% V1 - Menor (m)
% V2 - Maior (M)
% V3 - Meio Tom Abaixo (Bemol)
% V4 - Com Quarta (ex:C4)
% V5 - Com Quinta (ex:C5)
% V6 - Com Sexta (ex:C6)
% V7 - Com Sétima Menor (ex:C7)
% V8 - Com baixo dois Tons Acima (ex:D/F#)
% V9 - Com Nona (ex:C9)
% V10 - Meio Tom Acima (Sustenido)
% V11 - Com Sétima Maior (ex:C7M)
% V12 - Suspenso (Sus)
% V13 - Com baixo dois Tons e Meio Acima (ex:A/E)
% V14 - Com baixo um Tom e Meio Acima (ex:D9/F) 
% V15 - Meio-Diminuto (m7b5)
% N15 - NÃO Meio-Diminuto
% V16 - Diminuto (º)
% N16 - NÃO Diminuto
% V17 - Com baixo um Tom Acima (ex: C/D)
% V18 - Com baixo um Tom Abaixo (ex: Em/D)
% V19 - Com baixo dois Tons e meio Abaixo (ex: G/D)
%=================================================================
%\vspace{4em} % Regulador de Espaçamento
%=================================================================
\endsong
%=================================================================
