%=================================================================
\songcolumns{2}
\beginsong
{Talismã %TÍTULO
}[by={Leandro e Leonardo %ARTISTA
},album={@walyssondosreis},
id={\href{https://music.youtube.com/watch?v=2xrppJrllv8&feature=share %LINK
}{ST0006 %COD.ID.: XXNNNN
}},rev={5}, %REVISÃO
qr={https://music.youtube.com/watch?v=2xrppJrllv8&feature=share %LINK
}]
%-----------------------------------------------------------------
\tom{}{D}
%=================================================================
%\newchords{verse1.XX0000X} % Registrador de Acordes em Sequência
%\newchords{chorus1.XX0000X} % Registrador de Acordes em Sequência
%-----------------------------------------------------------------
\seq{Intro}{...}{1x}
%\act{}{}{}
%-----------------------------------------------------------------
%\beginverse \endverse
%\beginchorus \endchorus
\beginverse
Sabe, quanto tempo não te vejo?
Cada vez você distante
Mas eu gosto de você
Por quê?
\endverse

\beginverse
Sabe, eu pensei que fosse fácil
Esquecer seu jeito frágil
De se dar sem receber
Só você
\endverse

\beginverse
Só você que me ilumina
Meu pequeno talismã
Como é doce essa rotina
De te amar toda manhã
\endverse

\beginverse
Nos momentos mais difíceis
Você é o meu divã
Nosso amor não tem segredos
Sabe tudo de nós dois
E joga fora nossos medos
\endverse

\beginchorus
Vai, saudade, diz pra ela
Diz pra ela aparecer
Vai, saudade, vê se troca
A minha solidão por ela
Pra valer o meu viver
\endchorus
\act{Riff Intro}{Repetir}{1x}
\act{Verso 3}{Retomar}{1x}
\act{Refrão}{Repetir}{+1x}
%-----------------------------------------------------------------
\vspace{4em} % Regulador de Espaçamento
%-----------------------------------------------------------------
\begin{comment}
\lstset{basicstyle=\scriptsize\bf} % Parâmetros da TAB
%-----------------------------------------------------------------
\tab{Solo 1}
\begin{lstlisting}
E|-----------------------------------------------------|
B|-----------------------------------------------------|
G|-----------------------------------------------------|
D|-----------------------------------------------------|
A|-----------------------------------------------------|
E|-----------------------------------------------------|
\end{lstlisting}
%-----------------------------------------------------------------
\end{comment}
%=================================================================
\begin{comment}

\color{drawChord}\gtab{\color{nameChord} X}{}% 
\color{drawChord}\gtab{\color{nameChord} X}{}% 
\color{drawChord}\gtab{\color{nameChord} X}{}% 
\color{drawChord}\gtab{\color{nameChord} X}{}% 

\end{comment}
%=================================================================
% PADRÃO: [TonalidadeMaior+NOTAX+Variações] .Ex:[X50] [X57V1V7]
% OBS: Variações são alterações do acorde em relação ao campo harmônico.
%-----------------------------------------------------------------
% Tipos de Variações de Acordes:
% V0 - Variação Diversa
% V1 - Menor (m)
% V2 - Maior (M)
% V3 - Meio Tom Abaixo (Bemol)
% V4 - Com Quarta (ex:C4)
% V5 - Com Quinta (ex:C5)
% V6 - Com Sexta (ex:C6)
% V7 - Com Sétima Menor (ex:C7)
% V8 - Com baixo dois Tons Acima (ex:D/F#)
% V9 - Com Nona (ex:C9)
% V10 - Meio Tom Acima (Sustenido)
% V11 - Com Sétima Maior (ex:C7M)
% V12 - Suspenso (Sus)
% V13 - Com baixo dois Tons e Meio Acima (ex:A/E)
% V14 - Com baixo um Tom e Meio Acima (ex:D9/F) 
% V15 - Meio-Diminuto (m7b5)
% N15 - NÃO Meio-Diminuto
% V16 - Diminuto (º)
% N16 - NÃO Diminuto
% V17 - Com baixo um Tom Acima (ex: C/D)
% V18 - Com baixo um Tom Abaixo (ex: Em/D)
% V19 - Com baixo dois Tons e meio Abaixo (ex: G/D)
%=================================================================
%\vspace{4em} % Regulador de Espaçamento
%=================================================================
\endsong
%=================================================================
