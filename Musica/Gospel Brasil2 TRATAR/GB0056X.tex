%=================================================================
\songcolumns{2}
\beginsong
{Jesus Em Tua Presença %TÍTULO
}[by={Quatro Por Um %ARTISTA
},album={@walyssondosreis},
id={GB0056 %COD.ID.: GB0000
},rev={3}, %REVISÃO
qr={https://drive.google.com/open?id=1mi7pPvf8z0wlSok4nB6ZDHVL_0zDr3gt %LINK
}]
%-----------------------------------------------------------------
\tom{X1}{G}
%=================================================================
%\newchords{verse0.GB0000} % Registrador de Acordes em Sequência
%-----------------------------------------------------------------
%-----------------------------------------------------------------
%\beginverse* \endverse
%\beginchorus \endchorus

\beginverse 
Je\[X1]sus em tua presen\[X4V9]ça 
Reu\[X5]nimo-nos aqui\[X1]
Contem\[X1]plamos tua fa\[X4V9]ce
E ren\[X5]demo-nos a ti\[X1]
Pois um \[X6V7]dia tua mor\[X3V7]te 
Trouxe \[X4V9]vida a todos nós\[X1]
E nos \[X1]deu completo aces\[X4V9]so 
Ao \[X5]coração do Pa\[X1]i
\endverse

\seq{Intro}{X1 X4V9 X1 X5}{2x}
\act{Repetir}{Verso 1}{1x}
\beginverse 
O \[X6V7]véu que separa\[X3V7]va, 
Já \[X4V9]não se\[X5]para \[X6V7]mais
A \[X6V7]luz que outrora \[X3V7]apagada
A\[X4V9]gora brilha e \[X5]cada dia brilha \[X6V2V4]mais\[X6V2]
\endverse

\beginchorus
\[X4V9]Só pra te \[X5]ado\[X1]rar \[X5]\[X6V7]\[X5]
E fa\[X4V9]zer teu \[X5]nome \[X6V7]grande \[X5]
\[X4V9]E te \[X5]dar o lou\[X1]vor \[X5]que é de\[X2V2]vido
Es\[X4V9]tamos \[X5]nós a\[X1]qui

\endchorus
\act{Executar}{Solo 1: Intro}{}
\act{Retomar}{Verso 1}{1x}
\act{Retomar}{Verso 2}{1x}
\beginverse
...\[X1]Estamos nós aqui 
Estamos nós a\[X4V9]qui
Estamos nós a\[X1]qui
\endverse
\act{Repetir}{Verso 3}{+1x}
%-----------------------------------------------------------------
\vspace{4em} % Regulador de Espaçamento
%-----------------------------------------------------------------
\begin{comment}
\lstset{basicstyle=\scriptsize\bf} % Parâmetros da TAB
%-----------------------------------------------------------------
\tab{Solo 1}
\begin{lstlisting}
E|-----------------------------------------------------|
B|-----------------------------------------------------|
G|-----------------------------------------------------|
D|-----------------------------------------------------|
A|-----------------------------------------------------|
E|-----------------------------------------------------|
\end{lstlisting}
%-----------------------------------------------------------------
\end{comment}
%=================================================================

%-----------------------------------------------------------------
\color{drawChord}\gtab{\color{nameChord} X1}{}% E [X1]
\color{drawChord}\gtab{\color{nameChord} X2V2}{}% F# [X2V2]
\color{drawChord}\gtab{\color{nameChord} X3V7}{}% G#m7 [X3V7]
\color{drawChord}\gtab{\color{nameChord} X4V9}{}% A9 [X4V9]
\color{drawChord}\gtab{\color{nameChord} X5}{}% B [X5]
\color{drawChord}\gtab{\color{nameChord} X6V7}{}\\% C#m7 [X6V7]
\color{drawChord}\gtab{\color{nameChord} X6V2}{}% C# [X6V2]
\color{drawChord}\gtab{\color{nameChord} X6V2V4}{}% C#4 [X6V2V4]
%-----------------------------------------------------------------
% PADRÃO [TonalidadeMaiorNOTAX.Variação] .Ex:[X50] [X50V1]
% PADRÃO [TonalidadeMenorNOTAX.Variação] .Ex:[mX50] [mX50V1]
% OBS: Variações são alterações do acorde em relação ao campo harmônico.
%-----------------------------------------------------------------
% TIPOS DE VARIAÇÂO DOS ACORDES:
% V0 - ACORDE COM VARIAÇÃO DIVERSA
% V1 - ACORDE MENOR (m)
% V2 - ACORDE MAIOR (M)
% V3 - ACORDE MEIO TOM ABAIXO (Bemois)
% V4 - ACORDE COM QUARTA (C4)
% V5 - ACORDE COM QUINTA (C5)
% V6 - ACORDE COM SEXTA (C6)
% V7 - ACORDE COM SÉTIMA MENOR (C7)
% V8 - ACORDE COM BAIXO DOIS TONS ACIMA (D/F#)
% V9 - ACORDE COM NONA (C9)
% V10 - ACORDE MEIO TOM ACIMA (Sustenidos)
% V11 - ACORDE COM SÉTIMA MAIOR (C7M)
% V12 - ACORDE SUSPENSO (Sus)
% V13 - ACORDE COM BAIXO DOIS TONS E MEIO ACIMA (A/E)
% V14 - ACORDE UM TOM E MEIO ACIMA (D9/F)
%=================================================================
\endsong
%=================================================================