%=================================================================
\songcolumns{1}
\beginsong
{Reina Em Mim %TÍTULO
}[by={Vineyard  %ARTISTA
},album={@walyssondosreis},
id={GB0098 %COD.ID.: GB0000
(Rev.0) %REVISÃO.: 0...N
}]
%-----------------------------------------------------------------
\tom{X1}
%=================================================================
%\newchords{verse1.GB0000X} % Registrador de Acordes em Sequência
%\newchords{chorus1.GB0000X} % Registrador de Acordes em Sequência
%-----------------------------------------------------------------
\seq{Intro}{X1 X5 X4 X5 X1}{2x}
%-----------------------------------------------------------------
%\beginverse* \endverse
%\beginchorus \endchorus
\beginverse*

\[X1] Sobre \[X5]toda a \[X4]terra Tu \[X5]és o \[X1]Rei
\[X1] Sobre \[X5]as mon\[X4]tanhas e o \[X5]pôr-do-\[X1]Sol
\[X1] Uma \[X5]coisa \[X4]só meu de\[X5]sejo \[X2]é
Vem rei\[X4]nar de \[X5]novo em \[X1]mim
\endverse
\beginchorus
^ Rei^na em ^mim com ^Teu po^der
^ Sobre a es^curi^dão
Sobre os ^sonhos ^meus
^ Tu és ^o Se^nhor de tu^do o que ^sou
Vem rei^nar em ^mim, Se^nhor
\endchorus

\seq{Riff Intro}{X1 X5 X4 X5 X1}{2x}

\beginverse*
^ Sobre o ^meu pen^sar, tudo ^que eu fa^lar
^ Faz-me ^refle^tir a beleza que ^há em ^ti
^Tu és ^para ^mim mais que ^tudo a^qui
Vem rei^nar de ^novo em ^mim
\endverse
\beginverse*
.
.
.
.
.
.
.
\endverse
%-----------------------------------------------------------------
\begin{comment}
\lstset{basicstyle=\scriptsize\bf} % Parâmetros da TAB
%-----------------------------------------------------------------
\tab{Solo 1}
\begin{lstlisting}
E|-----------------------------------------------------|
B|-----------------------------------------------------|
G|-----------------------------------------------------|
D|-----------------------------------------------------|
A|-----------------------------------------------------|
E|-----------------------------------------------------|
\end{lstlisting}
%-----------------------------------------------------------------
\end{comment}
%=================================================================
\vspace{2em}
%-----------------------------------------------------------------
\gtab{\color{black} X1}{}% C [X1]
\gtab{\color{black} X2}{}% Dm [X2]
\gtab{\color{black} X4}{}% F [X4]
\gtab{\color{black} X5}{}% G [X5]
\gtab{\color{black} X6}{}% Am [X6]
%-----------------------------------------------------------------
% PADRÃO [TonalidadeMaiorNOTAX.Variação] .Ex:[X50] [X50V1]
% PADRÃO [TonalidadeMenorNOTAX.Variação] .Ex:[mX50] [mX50V1]
% OBS: Variações são alterações do acorde em relação ao campo harmônico.
%-----------------------------------------------------------------
% TIPOS DE VARIAÇÂO DOS ACORDES:
% V0 - ACORDE COM VARIAÇÃO DIVERSA
% V1 - ACORDE MENOR (m)
% V2 - ACORDE MAIOR (M)
% V3 - ACORDE MEIO TOM ABAIXO (Bemois)
% V4 - ACORDE COM QUARTA (C4)
% V5 - ACORDE COM QUINTA (C5)
% V6 - ACORDE COM SEXTA (C6)
% V7 - ACORDE COM SÉTIMA MENOR (C7)
% V8 - ACORDE COM BAIXO DOIS TONS ACIMA (D/F#)
% V9 - ACORDE COM NONA (C9)
% V10 - ACORDE MEIO TOM ACIMA (Sustenidos)
% V11 - ACORDE COM SÉTIMA MAIOR (C7M)
% V12 - ACORDE SUSPENSO (Sus)
% V13 - ACORDE COM BAIXO DOIS TONS E MEIO ACIMA (A/E)
% V14 - ACORDE UM TOM E MEIO ACIMA (D9/F)
%=================================================================
\endsong
%=================================================================
\begin{comment}

\end{comment}