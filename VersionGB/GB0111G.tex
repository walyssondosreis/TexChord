%=================================================================
\songcolumns{1}
\beginsong
{Só tu és Santo %TÍTULO
}[by={Ministério Morada %ARTISTA
},album={@walyssondosreis},
id={GB0111 %COD.ID.: GB0000
},rev={3}, %REVISÃO
qr={https://drive.google.com/open?id=1Ad-5XJ7wezHsKlHXPDlPCc04VB36Mmhx %LINK
}]
%-----------------------------------------------------------------
\tom{G}{A}
%=================================================================
%\newchords{verse1.GB0000X} % Registrador de Acordes em Sequência
%\newchords{chorus1.GB0000X} % Registrador de Acordes em Sequência
%-----------------------------------------------------------------
%\seq{Intro}{}{}
%-----------------------------------------------------------------
%\beginverse* \endverse
%\beginchorus \endchorus
\beginverse
\[G]Tudo está preparado aqui
\[C]A casa e o meu coração também
\[G/B]Tu És o único motivo que me \[C]fez \[D]che\[C]gar
\endverse
\beginverse
^Os filhos já estão chegando aqui
^Agora, somos dois ou três ou mais
^Encontre o meu coração disposto a quei^mar ^por ^Ti
\endverse
\beginverse
Todos os \[G/B]versos e canções que eu \[C]conseguir cantar
Todas as \[G/B]vezes quebrantado, só \[C]quero te falar
Teu é o \[Em]Reino e a \[D/F\#]Glória pra \[C]sempre
Teu é o do\[Em]mínio e o \[D/F\#]poder, a\[C]mém, amém
Teu é o \[Em]Reino e a \[D/F\#]Glória pra \[C]sempre
Teu é o do\[Em]mínio e o \[D/F\#]poder, a\[C]mém, amém
\endverse
\seq{Riff 1}{G C G/B Am G/B C}{}
\beginchorus
Só Tu És \[G]Santo, Só Tu És \[C]Santo
Não há outro como \[G/B]Tu
Não há outro como \[Am]Tu
Não há outro como \[G/B]Tu
Não há outro como \[C]Jesus
\endchorus
\act{Repetir}{Refrão}{+5x}
%-----------------------------------------------------------------
\vspace{2em} % Regulador de Espaçamento
%-----------------------------------------------------------------
\begin{comment}
\lstset{basicstyle=\scriptsize\bf} % Parâmetros da TAB
%-----------------------------------------------------------------
\tab{Solo 1}
\begin{lstlisting}
E|-----------------------------------------------------|
B|-----------------------------------------------------|
G|-----------------------------------------------------|
D|-----------------------------------------------------|
A|-----------------------------------------------------|
E|-----------------------------------------------------|
\end{lstlisting}
%-----------------------------------------------------------------
\end{comment}
%=================================================================
 
%-----------------------------------------------------------------
\color{drawChord}\gtab{\color{nameChord} G}{}% 
\color{drawChord}\gtab{\color{nameChord} G/B}{}% 
\color{drawChord}\gtab{\color{nameChord} Am}{}% 
\color{drawChord}\gtab{\color{nameChord} C}{}%
\color{drawChord}\gtab{\color{nameChord} D}{}%
\color{drawChord}\gtab{\color{nameChord} D/F\#}{}%
\color{drawChord}\gtab{\color{nameChord} Em}{}%
%-----------------------------------------------------------------
% PADRÃO: [TonalidadeMaior+NOTAX+Variações] .Ex:[X50] [X57V1V7]
% OBS: Variações são alterações do acorde em relação ao campo harmônico.
%-----------------------------------------------------------------
% Tipos de Variações de Acordes:
% V0 - Variação Diversa
% V1 - Menor (m)
% V2 - Maior (M)
% V3 - Meio Tom Abaixo (Bemol)
% V4 - Com Quarta (ex:C4)
% V5 - Com Quinta (ex:C5)
% V6 - Com Sexta (ex:C6)
% V7 - Com Sétima Menor (ex:C7)
% V8 - Com baixo dois Tons Acima (ex:D/F#)
% V9 - Com Nona (ex:C9)
% V10 - Meio Tom Acima (Sustenido)
% V11 - Com Sétima Maior (ex:C7M)
% V12 - Suspenso (Sus)
% V13 - Com baixo dois Tons e Meio Acima (ex:A/E)
% V14 - Com baixo um Tom e Meio Acima (ex:D9/F) 
% V15 - Meio-Diminuto (m7b5)
% N15 - NÃO Meio-Diminuto
% V16 - Diminuto (º)
% N16 - NÃO Diminuto
%=================================================================
\endsong
%=================================================================
\begin{comment}

\end{comment}