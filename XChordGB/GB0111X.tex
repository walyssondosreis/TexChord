%=================================================================
\songcolumns{1}
\beginsong
{Só Tu És Santo %TÍTULO
}[by={Ministério Morada %ARTISTA
},album={@walyssondosreis},
id={GB0111 %COD.ID.: GB0000
},rev={3}, %REVISÃO
qr={https://drive.google.com/open?id=1Ad-5XJ7wezHsKlHXPDlPCc04VB36Mmhx %LINK
}]
%-----------------------------------------------------------------
\tom{X1}{A}
%=================================================================
%\newchords{verse1.GB0000X} % Registrador de Acordes em Sequência
%\newchords{chorus1.GB0000X} % Registrador de Acordes em Sequência
%-----------------------------------------------------------------
%\seq{Intro}{}{}
%-----------------------------------------------------------------
%\beginverse* \endverse
%\beginchorus \endchorus
\beginverse
\[X1]Tudo está preparado aqui
\[X4]A casa e o meu coração também
\[X1V8]Tu És o único motivo que me \[X4]fez \[X5]che\[X4]gar
\endverse
\beginverse
^Os filhos já estão chegando aqui
^Agora, somos dois ou três ou mais
^Encontre o meu coração disposto a quei^mar ^por ^Ti
\endverse
\beginverse
Todos os \[X1V8]versos e canções que eu \[X4]conseguir cantar
Todas as \[X1V8]vezes quebrantado, só \[X4]quero te falar
Teu é o \[X6]Reino e a \[X5V8]Glória pra \[X4]sempre
Teu é o do\[X6]mínio e o \[X5V8]poder, a\[X4]mém, amém
Teu é o \[X6]Reino e a \[X5V8]Glória pra \[X4]sempre
Teu é o do\[X6]mínio e o \[X5V8]poder, a\[X4]mém, amém
\endverse
\seq{Riff 1}{X1 X4 X1V8 X2 X1V8 X4}{}
\beginchorus
Só Tu És \[X1]Santo, Só Tu És \[X4]Santo
Não há outro como \[X1V8]Tu
Não há outro como \[X2]Tu
Não há outro como \[X1V8]Tu
Não há outro como \[X4]Jesus
\endchorus
\act{Repetir}{Refrão}{+5x}

%-----------------------------------------------------------------
\begin{comment}
\lstset{basicstyle=\scriptsize\bf} % Parâmetros da TAB
%-----------------------------------------------------------------
\tab{Solo 1}
\begin{lstlisting}
E|-----------------------------------------------------|
B|-----------------------------------------------------|
G|-----------------------------------------------------|
D|-----------------------------------------------------|
A|-----------------------------------------------------|
E|-----------------------------------------------------|
\end{lstlisting}
%-----------------------------------------------------------------
\end{comment}
%=================================================================
 
%-----------------------------------------------------------------
\color{drawChord}\gtab{\color{nameChord} X1}{}% 
\color{drawChord}\gtab{\color{nameChord} X1V8}{}% 
\color{drawChord}\gtab{\color{nameChord} X2}{}% 
\color{drawChord}\gtab{\color{nameChord} X4}{}%
\color{drawChord}\gtab{\color{nameChord} X5}{}%
\color{drawChord}\gtab{\color{nameChord} X5V8}{}%
\color{drawChord}\gtab{\color{nameChord} X6}{}%
%-----------------------------------------------------------------
% PADRÃO: [TonalidadeMaior+NOTAX+Variações] .Ex:[X50] [X57V1V7]
% OBS: Variações são alterações do acorde em relação ao campo harmônico.
%-----------------------------------------------------------------
% Tipos de Variações de Acordes:
% V0 - Variação Diversa
% V1 - Menor (m)
% V2 - Maior (M)
% V3 - Meio Tom Abaixo (Bemol)
% V4 - Com Quarta (ex:C4)
% V5 - Com Quinta (ex:C5)
% V6 - Com Sexta (ex:C6)
% V7 - Com Sétima Menor (ex:C7)
% V8 - Com baixo dois Tons Acima (ex:D/F#)
% V9 - Com Nona (ex:C9)
% V10 - Meio Tom Acima (Sustenido)
% V11 - Com Sétima Maior (ex:C7M)
% V12 - Suspenso (Sus)
% V13 - Com baixo dois Tons e Meio Acima (ex:A/E)
% V14 - Com baixo um Tom e Meio Acima (ex:D9/F) 
% V15 - Meio-Diminuto (m7b5)
% N15 - NÃO Meio-Diminuto
% V16 - Diminuto (º)
% N16 - NÃO Diminuto
%=================================================================
\endsong
%=================================================================
\begin{comment}

\end{comment}