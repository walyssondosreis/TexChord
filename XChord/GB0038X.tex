%=================================================================
\songcolumns{1}
\beginsong
{Eu Navegarei %TÍTULO
}[by={Ana Nóbrega %ARTISTA
},album={@walyssondosreis},
id={GB0038 %COD.ID.: GB0000
},rev={3}, %REVISÃO
qr={https://drive.google.com/open?id=1PrxG6tJrVfHaBPz_nqpbFtbI-guHIoBh %LINK
}]
%-----------------------------------------------------------------
\tom{X1}{A}
%=================================================================
%\newchords{verse1.GB0000X} % Registrador de Acordes em Sequência
%\newchords{chorus1.GB0000X} % Registrador de Acordes em Sequência
%-----------------------------------------------------------------
\seq{Intro}{X6 X5 X2 X4}{2x}
%-----------------------------------------------------------------
%\beginverse* \endverse
%\beginchorus \endchorus
\beginverse
Eu navega\[X6]rei \[X1]
No oceano do es\[X5]pírito \[X2]
E ali adora\[X4]rei \[X2]
Ao Deus do meu a\[X3V2]mor
\endverse
\act{Repetir}{Verso 1}{+1x}
\beginchorus
Espírito, esp^írito ^
Que desce como ^fogo ^
Vem como em pente^costes ^
E enche-me de ^novo
\endchorus
\act{Repetir}{Refrão}{+1x}
\seq{Intro}{X6 X5 X2 X4}{1x}
\beginverse
Eu adora^rei ^
Ao Deus da minha ^vida ^
Que me compreen^deu ^
Sem nenhuma explica^ção
\endverse
\act{Repetir}{Verso 2}{+1x}
\act{Rpetir}{Refrão}{2x}
\act{Executar}{Solo 1}{}
\seq{Intro}{X6 X5 X2 X4}{1x}
% Verso de preenchimento
\beginverse*
.
\endverse
%-----------------------------------------------------------------
\begin{comment}
\lstset{basicstyle=\scriptsize\bf} % Parâmetros da TAB
%-----------------------------------------------------------------
\tab{Solo 1}
\begin{lstlisting}
E|-----------------------------------------------------|
B|-----------------------------------------------------|
G|-----------------------------------------------------|
D|-----------------------------------------------------|
A|-----------------------------------------------------|
E|-----------------------------------------------------|
\end{lstlisting}
%-----------------------------------------------------------------
\end{comment}
%=================================================================
\vspace{2em} 
%-----------------------------------------------------------------
\gtab{\color{black} X1}{}% 
\gtab{\color{black} X2}{}% 
\gtab{\color{black} X3V2}{}%
\gtab{\color{black} X4}{}% 
\gtab{\color{black} X5}{}% 
\gtab{\color{black} X6}{}%
%-----------------------------------------------------------------
% PADRÃO: [TonalidadeMaior+NOTAX+Variações] .Ex:[X50] [X57V1V7]
% OBS: Variações são alterações do acorde em relação ao campo harmônico.
%-----------------------------------------------------------------
% Tipos de Variações de Acordes:
% V0 - Variação Diversa
% V1 - Menor (m)
% V2 - Maior (M)
% V3 - Meio Tom Abaixo (Bemol)
% V4 - Com Quarta (ex:C4)
% V5 - Com Quinta (ex:C5)
% V6 - Com Sexta (ex:C6)
% V7 - Com Sétima Menor (ex:C7)
% V8 - Com baixo dois Tons Acima (ex:D/F#)
% V9 - Com Nona (ex:C9)
% V10 - Meio Tom Acima (Sustenido)
% V11 - Com Sétima Maior (ex:C7M)
% V12 - Suspenso (Sus)
% V13 - Com baixo dois Tons e Meio Acima (ex:A/E)
% V14 - Com baixo um Tom e Meio Acima (ex:D9/F) 
% V15 - Meio-Diminuto (m7b5)
% N15 - NÃO Meio-Diminuto
% V16 - Diminuto (º)
% N16 - NÃO Diminuto
%=================================================================
\endsong
%=================================================================
\begin{comment}

\end{comment}