%=================================================================
\songcolumns{2}
\beginsong
{Vestido de Seda %TÍTULO
}[by={Teodoro e Sampaio %ARTISTA
},album={@walyssondosreis},
id={\href{https://music.youtube.com/watch?v=OHWkCMVNoFk&feature=share %LINK
}{ST0146 %COD.ID.: XXNNNN
}},rev={5}, %REVISÃO
qr={https://music.youtube.com/watch?v=OHWkCMVNoFk&feature=share %LINK
}]
%-----------------------------------------------------------------
\tom{}{G}
%=================================================================
%\newchords{verse1.XX0000X} % Registrador de Acordes em Sequência
%\newchords{chorus1.XX0000X} % Registrador de Acordes em Sequência
%-----------------------------------------------------------------
\seq{Intro}{...}{}
%\act{}{}{}
%-----------------------------------------------------------------
%\beginverse \endverse
%\beginchorus \endchorus
\beginverse 
Meu bem, eu queria que você voltasse
Ao menos pra buscar
Alguns objetos, que na despedida
Você não levou
Um batom usado caído
No canto da penteadeira
Um vestido velho cheio de poeira
Jogado no quarto com marcas de amor
\endverse

\beginchorus
Vestido de seda
O seu manequim também te deixou
Aí no cantinho não tem mais valor
Se não tem aquela que tanto te usou
Eu também não passo de um trapo humano
Sem minha querida
Usado e jogado num canto da vida
Não sei o que faço sem meu grande amor
\endchorus
\act{Riff Intro}{Repetir}{1x}
\beginverse 
Eu já nem acendo a luz do meu quarto
quando vou deitar
Porque no escuro não vejo no espelho
Meus olhos chorando
Não vou na cozinha
Pra não ver dois copos vazios na mesa
Fazendo lembrar com tanta tristeza
Da última noite que nós nos amamos
\endverse
\act{Refrão}{Repetir}{1x}

%-----------------------------------------------------------------
\vspace{4em} % Regulador de Espaçamento
%-----------------------------------------------------------------
\begin{comment}
\lstset{basicstyle=\scriptsize\bf} % Parâmetros da TAB
%-----------------------------------------------------------------
\tab{Solo 1}
\begin{lstlisting}
E|-----------------------------------------------------|
B|-----------------------------------------------------|
G|-----------------------------------------------------|
D|-----------------------------------------------------|
A|-----------------------------------------------------|
E|-----------------------------------------------------|
\end{lstlisting}
%-----------------------------------------------------------------
\end{comment}
%=================================================================
\begin{comment}

\color{drawChord}\gtab{\color{nameChord} X}{}% 
\color{drawChord}\gtab{\color{nameChord} X}{}% 
\color{drawChord}\gtab{\color{nameChord} X}{}% 
\color{drawChord}\gtab{\color{nameChord} X}{}% 

\end{comment}
%=================================================================
% PADRÃO: [TonalidadeMaior+NOTAX+Variações] .Ex:[X50] [X57V1V7]
% OBS: Variações são alterações do acorde em relação ao campo harmônico.
%-----------------------------------------------------------------
% Tipos de Variações de Acordes:
% V0 - Variação Diversa
% V1 - Menor (m)
% V2 - Maior (M)
% V3 - Meio Tom Abaixo (Bemol)
% V4 - Com Quarta (ex:C4)
% V5 - Com Quinta (ex:C5)
% V6 - Com Sexta (ex:C6)
% V7 - Com Sétima Menor (ex:C7)
% V8 - Com baixo dois Tons Acima (ex:D/F#)
% V9 - Com Nona (ex:C9)
% V10 - Meio Tom Acima (Sustenido)
% V11 - Com Sétima Maior (ex:C7M)
% V12 - Suspenso (Sus)
% V13 - Com baixo dois Tons e Meio Acima (ex:A/E)
% V14 - Com baixo um Tom e Meio Acima (ex:D9/F) 
% V15 - Meio-Diminuto (m7b5)
% N15 - NÃO Meio-Diminuto
% V16 - Diminuto (º)
% N16 - NÃO Diminuto
% V17 - Com baixo um Tom Acima (ex: C/D)
% V18 - Com baixo um Tom Abaixo (ex: Em/D)
% V19 - Com baixo dois Tons e meio Abaixo (ex: G/D)
%=================================================================
%\vspace{4em} % Regulador de Espaçamento
%=================================================================
\endsong
%=================================================================
