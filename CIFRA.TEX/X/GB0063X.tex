%=================================================================
\songcolumns{1}
\beginsong
{Meu Alvo %TÍTULO
}[by={Kleber Lucas %ARTISTA
},album={@walyssondosreis},
id={GB0063 %COD.ID.: GB0000
}] 
%-----------------------------------------------------------------
\tom{X1}
%=================================================================
%\newchords{verse0.GB0000} % Registrador de Acordes em Sequência
%-----------------------------------------------------------------
\seq{Intro}{X2 X6 X2 X6 X2 X6 X4}{2x}
%-----------------------------------------------------------------
%\beginverse* \endverse
%\beginchorus \endchorus

\beginverse*
Estou su\[X1]bindo pra um lugar mais\[X6] alto
Eu já queimei as \[X1]pontes com o passado
E em meus \[X6]olhos 
Vejo o fu\[X4]turo
Tudo novo se \[X4V1V7]fez, tudo novo se \[X1]faz
E dessa es\[X5]trada \[X5V4]\[X5] eu não me des\[X6]vio nunca mais
Estou \[X5V8]firme, eu não me des\[X4]vio nunca mais
\endverse

\beginchorus
\[X1]Vou avançar, eu vou cres\[X5V8]cer
Ninguém vai me deter
\[X6]Meu alvo é Cristo, \[X4]meu alvo é Cristo
\[X1]Vou avançar, eu vou cres\[X5V8]cer
Ninguém vai me deter
\[X6]Meu alvo é Cristo, \[X4]meu alvo é Cristo
\endchorus
%-----------------------------------------------------------------
\begin{comment}
\lstset{basicstyle=\scriptsize\bf} % Parâmetros da TAB
%-----------------------------------------------------------------
\tab{Solo 1}
\begin{lstlisting}
E|-----------------------------------------------------|
B|-----------------------------------------------------|
G|-----------------------------------------------------|
D|-----------------------------------------------------|
A|-----------------------------------------------------|
E|-----------------------------------------------------|
\end{lstlisting}
%-----------------------------------------------------------------
\end{comment}
%=================================================================
\vspace{2em}
%-----------------------------------------------------------------
\gtab{\color{black} X1}{}% A [X1]
\gtab{\color{black} X2}{}% Bm [X2]
\gtab{\color{black} X4}{}% D [X4]
\gtab{\color{black} X4V1V7}{}% Dm7 [X4V1V7]
\gtab{\color{black} X5}{}% E [X5]
\gtab{\color{black} X5V8}{}% E/G# [X5V8]
\gtab{\color{black} X5V4}{}% E4 [X5V4]
\gtab{\color{black} X6}{}% F#m [X6]
%-----------------------------------------------------------------
% PADRÃO [TonalidadeMaiorNOTAX.Variação] .Ex:[X50] [X50V1]
% PADRÃO [TonalidadeMenorNOTAX.Variação] .Ex:[mX50] [mX50V1]
% OBS: Variações são alterações do acorde em relação ao campo harmônico.
%-----------------------------------------------------------------
% TIPOS DE VARIAÇÂO DOS ACORDES:
% V1 - ACORDE MENOR (m)
% V2 - ACORDE MAIOR (M)
% V3 - ACORDE MEIO TOM ABAIXO (Bemois)
% V4 - ACORDE COM QUARTA (C4)
% V5 - ACORDE COM QUINTA (C5)
% V6 - ACORDE COM SEXTA (C6)
% V7 - ACORDE COM SÉTIMA MENOR (C7)
% V8 - ACODE COM BAIXO DOIS TONS ACIMA (D/F#)
% V9 - ACORDE COM NONA (C9)
% V10 - ACORDE MEIO TOM ACIMA (Sustenidos)
% V11 - ACORDE COM SÉTIMA MAIOR (C7M)
% V12 - ACORDE SUSPENSO (Sus)
% V13 - ACORDE COM VARIAÇÃO DIVERSA
%=================================================================
\endsong
%=================================================================
\begin{comment}

\end{comment}