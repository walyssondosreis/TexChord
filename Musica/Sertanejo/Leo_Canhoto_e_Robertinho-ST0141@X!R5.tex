%=================================================================
\songcolumns{1}
\beginsong
{Meu Velho Pai %TÍTULO
}[by={Léo Canhoto e Robertinho %ARTISTA
},album={@walyssondosreis},
id={\href{https://music.youtube.com/watch?v=0UIv8dv9O5I&feature=share %LINK
}{ST0141 %COD.ID.: XXNNNN
}},rev={5}, %REVISÃO
qr={https://music.youtube.com/watch?v=0UIv8dv9O5I&feature=share %LINK
}]
%-----------------------------------------------------------------
\tom{X1}{A}
%=================================================================
%\newchords{verse1.XX0000X} % Registrador de Acordes em Sequência
%\newchords{chorus1.XX0000X} % Registrador de Acordes em Sequência
%-----------------------------------------------------------------
\seq{Intro}{X5 X4 X1 X4 X1 X5 X1}{1x}
%\act{}{}{}
%-----------------------------------------------------------------
%\beginverse \endverse
%\beginchorus \endchorus
\beginverse
\[X1] Meu velho pai, preste aten\[X4]ção no que lhe \[X1]digo
Meu pobre papai querido enxugue as \[X4]lágrimas do \[X5]rosto
Porque, pa\[X2]pai, que você \[X4]chora tão so\[X5]zinho
Me con\[X4]ta, meu papai\[X5]zinho
O que \[X4]lhe causa des\[X1]gosto
\endverse
\beginverse
\[X1] Estou notando que vo\[X4]cê está can\[X1]sado
Meu pobre velho adorado, é seu fi\[X1V7]lho que está fa\[X4]lando
Quero sa\[X2]ber qual é a \[X5]tristeza que e\[X1]xiste
Não quero ver você \[X5]triste
Por que é que está cho\[X1]rando?
\endverse
\act{Riff Intro}{Repetir}{1x}
\beginverse
\[X1] Quando lhe vejo, tão tris\[X4]tonho desse \[X1]jeito
Sinto estremecer meu peito ao pul\[X4]sar meu cora\[X5]ção
Meu pobre \[X2]pai, você so\[X4]freu pra me cri\[X5]ar
Agora \[X4]eu vou lhe cui\[X5]dar
Esta é \[X4]minha obriga\[X1]ção
\endverse
\beginverse
\[X1] Não tenha medo, meu ve\[X4]lhinho adorado
Estarei sempre ao seu \[X1V7]lado, não lhe deixarei ja\[X4]mais
Eu sou o \[X2]sangue do seu \[X5]sangue, papai\[X1]zinho
Não vou lhe deixar so\[X5]zinho, não tenha medo, meu \[X1]pai
\endverse
\act{Riff Intro}{Repetir}{1x}
\beginverse
\[X1] Você sofreu quando eu \[X4]era ainda cri\[X1]ança
A sua grande esperança era me \[X4]ver homem for\[X5]mado
Eu fiquei \[X2]grande, estou se\[X4]guindo o meu ca\[X5]minho
E vo\[X4]cê ficou ve\[X5]lhinho, mas es\[X4]tou sempre ao seu \[X1]lado
\endverse
\beginverse
\[X1] Meu pobre pai, seus passos \[X4]longos silenci\[X1]aram
Seus cabelos branquiaram, seu o\[X1V7]lhar se escure\[X4]ceu
A sua \[X2]voz quase que \[X5]não se ouve \[X1]mais
Não tenha medo, meu \[X5]pai, quem cuida de você sou \[X1]eu
\endverse
\beginchorus
\[X1] Meu papai\[X5]zinho, não pre\[X4]cisa mais cho\[X1]rar
Saiba que não vou dei\[X5]xar você so\[X4]zinho, abando\[X1]nado \[X1V7]
Eu sou seu \[X4]guia, sou seu tempo, sou seus \[X1]passos
Sou sua luz e sou seus \[X5]braços
Sou seu filho idola\[X1]trado
\endchorus
\act{Riff Intro}{Repetir}{1x}

%-----------------------------------------------------------------
\vspace{4em} % Regulador de Espaçamento
%-----------------------------------------------------------------
\begin{comment}
\lstset{basicstyle=\scriptsize\bf} % Parâmetros da TAB
%-----------------------------------------------------------------
\tab{Solo 1}
\begin{lstlisting}
E|-----------------------------------------------------|
B|-----------------------------------------------------|
G|-----------------------------------------------------|
D|-----------------------------------------------------|
A|-----------------------------------------------------|
E|-----------------------------------------------------|
\end{lstlisting}
%-----------------------------------------------------------------
\end{comment}
%=================================================================


\color{drawChord}\gtab{\color{nameChord} X1}{}% 
\color{drawChord}\gtab{\color{nameChord} X1V7}{}% 
\color{drawChord}\gtab{\color{nameChord} X2}{}% 
\color{drawChord}\gtab{\color{nameChord} X4}{}%
\color{drawChord}\gtab{\color{nameChord} X5}{}% 


%=================================================================
% PADRÃO: [TonalidadeMaior+NOTAX+Variações] .Ex:[X50] [X57V1V7]
% OBS: Variações são alterações do acorde em relação ao campo harmônico.
%-----------------------------------------------------------------
% Tipos de Variações de Acordes:
% V0 - Variação Diversa
% V1 - Menor (m)
% V2 - Maior (M)
% V3 - Meio Tom Abaixo (Bemol)
% V4 - Com Quarta (ex:C4)
% V5 - Com Quinta (ex:C5)
% V6 - Com Sexta (ex:C6)
% V7 - Com Sétima Menor (ex:C7)
% V8 - Com baixo dois Tons Acima (ex:D/F#)
% V9 - Com Nona (ex:C9)
% V10 - Meio Tom Acima (Sustenido)
% V11 - Com Sétima Maior (ex:C7M)
% V12 - Suspenso (Sus)
% V13 - Com baixo dois Tons e Meio Acima (ex:A/E)
% V14 - Com baixo um Tom e Meio Acima (ex:D9/F) 
% V15 - Meio-Diminuto (m7b5)
% N15 - NÃO Meio-Diminuto
% V16 - Diminuto (º)
% N16 - NÃO Diminuto
% V17 - Com baixo um Tom Acima (ex: C/D)
% V18 - Com baixo um Tom Abaixo (ex: Em/D)
% V19 - Com baixo dois Tons e meio Abaixo (ex: G/D)
%=================================================================
%\vspace{4em} % Regulador de Espaçamento
%=================================================================
\endsong
%=================================================================
