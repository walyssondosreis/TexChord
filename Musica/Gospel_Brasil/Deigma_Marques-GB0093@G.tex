%=================================================================
\songcolumns{1}
\beginsong
{Que Ele Cresça (Humildade) %TÍTULO
}[by={Deigma Marques %ARTISTA
},album={@walyssondosreis},
id={GB0093 %COD.ID.: GB0000
},rev={3}, %REVISÃO
qr={https://drive.google.com/open?id=1JofO6fiOwXKEbZFdlZeXFEcASAl8F2X7 %LINK
}]
%-----------------------------------------------------------------
\tom{G}{D}
%=================================================================
%\newchords{verse1.GB0000X} % Registrador de Acordes em Sequência
%\newchords{chorus1.GB0000X} % Registrador de Acordes em Sequência
%-----------------------------------------------------------------
\seq{Intro}{G  C}{2x}
%-----------------------------------------------------------------
%\beginverse* \endverse
%\beginchorus \endchorus
\beginverse
\[G]Mais de \[C]Ti
\[G]Mais de \[C]Ti
E menos de \[Em7]mim \[D]
E menos de \[C]mim
E menos de \[Em7]mim \[D]
E menos de \[C]mim
\endverse
\act{Retomar}{Verso 1}{1x}
\beginchorus
Que Ele \[Am]cresça e eu dimi\[C]nua
Que Ele apa\[Em7]reça e eu me cons\[D]tranja
Com a Sua \[Am]glória e todo o \[G/B]seu \[C]amor
Infinita humil\[Em7]dade, servo de \[D]todos os irmãos
\endchorus
\act{Retomar}{Verso 1}{1x}
\act{Repetir}{Refrão}{2x}
%-----------------------------------------------------------------
\vspace{4em} % Regulador de Espaçamento
%-----------------------------------------------------------------
\begin{comment}
\lstset{basicstyle=\scriptsize\bf} % Parâmetros da TAB
%-----------------------------------------------------------------
\tab{Solo 1}
\begin{lstlisting}
E|-----------------------------------------------------|
B|-----------------------------------------------------|
G|-----------------------------------------------------|
D|-----------------------------------------------------|
A|-----------------------------------------------------|
E|-----------------------------------------------------|
\end{lstlisting}
%-----------------------------------------------------------------
\end{comment}
%=================================================================
 
%-----------------------------------------------------------------
\color{drawChord}\gtab{\color{nameChord} G}{~:320003}% 
\color{drawChord}\gtab{\color{nameChord} C}{~:X32010}% 
\color{drawChord}\gtab{\color{nameChord} G/B}{~:X20033}%
\color{drawChord}\gtab{\color{nameChord} Am}{~:X02210}% 
\color{drawChord}\gtab{\color{nameChord} C}{~:X32010}% 
\color{drawChord}\gtab{\color{nameChord} D}{~:XX0232}% 
\color{drawChord}\gtab{\color{nameChord} Em7}{~:022030}% 
%-----------------------------------------------------------------
% PADRÃO: [TonalidadeMaior+NOTAX+Variações] .Ex:[X50] [X57V1V7]
% OBS: Variações são alterações do acorde em relação ao campo harmônico.
%-----------------------------------------------------------------
% Tipos de Variações de Acordes:
% V0 - Variação Diversa
% V1 - Menor (m)
% V2 - Maior (M)
% V3 - Meio Tom Abaixo (Bemol)
% V4 - Com Quarta (ex:C4)
% V5 - Com Quinta (ex:C5)
% V6 - Com Sexta (ex:C6)
% V7 - Com Sétima Menor (ex:C7)
% V8 - Com baixo dois Tons Acima (ex:D/F#)
% V9 - Com Nona (ex:C9)
% V10 - Meio Tom Acima (Sustenido)
% V11 - Com Sétima Maior (ex:C7M)
% V12 - Suspenso (Sus)
% V13 - Com baixo dois Tons e Meio Acima (ex:A/E)
% V14 - Com baixo um Tom e Meio Acima (ex:D9/F) 
% V15 - Meio-Diminuto (m7b5)
% N15 - NÃO Meio-Diminuto
% V16 - Diminuto (º)
% N16 - NÃO Diminuto
%=================================================================
\endsong
%=================================================================
\begin{comment}

\end{comment}