%=================================================================
\songcolumns{1}
\beginsong
{Toda Sorte de Bençãos %TÍTULO
}[by={Davi Sacer %ARTISTA
},album={@walyssondosreis},
id={GB0120 %COD.ID.: GB0000
(Rev.0) %REVISÃO.: 0...N
}]
%-----------------------------------------------------------------
\tom{X1}
%=================================================================
%\newchords{verse1.GB0000X} % Registrador de Acordes em Sequência
%\newchords{chorus1.GB0000X} % Registrador de Acordes em Sequência
%-----------------------------------------------------------------
%\seq{Intro}{}{}
%-----------------------------------------------------------------
%\beginverse* \endverse
%\beginchorus \endchorus
Por onde eu for a tua bênção me seguirá
Onde eu colocar as minhas mãos prosperará
A minha entrada e a minha saída bendita será
Pois sobre mim há uma promessa
Prosperarei, transbordarei

Os meus celeiros fartamente se encherão
A minha casa terá sempre tua provisão
Onde eu puser a planta dos meus pés, possuirei
Pois sobre mim há uma promessa
Prosperarei, transbordarei

Para direita, para esquerda
A minha frente
E para trás

Por todo lado, uooo
Sou abençoado, iééé
Em tudo o que eu faço, uooo
Sou abençoado, iééé

Toda sorte de bençãos
O senhor preparou para mim
E em todas as coisas
Eu sou mais do que vencedor

Eu sou livre (Eu sou livre)
Eu posso cantar (Eu posso cantar)

Eu sou livre (Eu sou livre)
Eu posso dançar (Eu posso dançar)

Eu sou livre (Eu sou livre)
Eu posso gritar (Eu posso gritar)

Eu sou livre
Eu posso pular
Pular, pular, pular

Pisa na cabeça do inimigo e declara:
sou mais que vencedor!

Prosperarás,
Transbordarás

Prosperarei
Transbordarei

Para direita, para esquerda
A minha frente
E para trás

Por todo lado, uooo
Sou abençoado, iééé
Em tudo o que eu faço, uooo
Sou abençoado, iééé


% Verso de preenchimento
\beginverse*
.
.
.
\endverse
%-----------------------------------------------------------------
\begin{comment}
\lstset{basicstyle=\scriptsize\bf} % Parâmetros da TAB
%-----------------------------------------------------------------
\tab{Solo 1}
\begin{lstlisting}
E|-----------------------------------------------------|
B|-----------------------------------------------------|
G|-----------------------------------------------------|
D|-----------------------------------------------------|
A|-----------------------------------------------------|
E|-----------------------------------------------------|
\end{lstlisting}
%-----------------------------------------------------------------
\end{comment}
%=================================================================
\vspace{2em} 
%-----------------------------------------------------------------
\gtab{\color{black} X}{}% 
\gtab{\color{black} X}{}% 
\gtab{\color{black} X}{}% 
\gtab{\color{black} X}{}% 
%-----------------------------------------------------------------
% PADRÃO: [TonalidadeMaior+NOTAX+Variações] .Ex:[X50] [X57V1V7]
% OBS: Variações são alterações do acorde em relação ao campo harmônico.
%-----------------------------------------------------------------
% Tipos de Variações de Acordes:
% V0 - Variação Diversa
% V1 - Menor (m)
% V2 - Maior (M)
% V3 - Meio Tom Abaixo (Bemol)
% V4 - Com Quarta (ex:C4)
% V5 - Com Quinta (ex:C5)
% V6 - Com Sexta (ex:C6)
% V7 - Com Sétima Menor (ex:C7)
% V8 - Com baixo dois Tons Acima (ex:D/F#)
% V9 - Com Nona (ex:C9)
% V10 - Meio Tom Acima (Sustenido)
% V11 - Com Sétima Maior (ex:C7M)
% V12 - Suspenso (Sus)
% V13 - Com baixo dois Tons e Meio Acima (ex:A/E)
% V14 - Com baixo um Tom e Meio Acima (ex:D9/F) 
% V15 - Meio-Diminuto (m7b5)
% N15 - NÃO Meio-Diminuto
% V16 - Diminuto (º)
% N16 - NÃO Diminuto
%=================================================================
\endsong
%=================================================================
\begin{comment}

\end{comment}