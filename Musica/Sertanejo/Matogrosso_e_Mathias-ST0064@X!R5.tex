%=================================================================
\songcolumns{2}
\beginsong
{De Igual Pra Igual %TÍTULO
}[by={Matogrosso e Mathias %ARTISTA
},album={@walyssondosreis},
id={\href{https://music.youtube.com/watch?v=lD1V2B9h3Ko&feature=share %LINK
}{ST0064 %COD.ID.: XXNNNN
}},rev={5}, %REVISÃO
qr={https://music.youtube.com/watch?v=lD1V2B9h3Ko&feature=share %LINK
}]
%-----------------------------------------------------------------
\tom{X1}{F}
%=================================================================
%\newchords{verse1.XX0000X} % Registrador de Acordes em Sequência
%\newchords{chorus1.XX0000X} % Registrador de Acordes em Sequência
%-----------------------------------------------------------------
\seq{Intro}{X4 X1V11 X2 X5 X1 X5V7 }{2x}
%\act{}{}{}
%-----------------------------------------------------------------
%\beginverse \endverse
%\beginchorus \endchorus
\beginverse
Vo\[X1]cê mentiu\[X1V11]
Quando ju\[X1]rava \[X1V11]para mim fideli\[X1]dade\[X1V11]
Fui a\[X1]penas \[X1V11]um escravo da mal\[X1]dade\[X1V11]
Você \[X1]quis, você \[X1V11]lutou e conse\[X5]guiu\[X2]\[X5]
\endverse
\beginverse
Vo\[X2]cê feriu
Os senti\[X5]mentos que a ti eu dedi\[X2]quei
Quantas \[X5]vezes o seu pranto, eu enxu\[X2]guei
Por pen\[X5V4]sar que era por mim que cho\[X1]rava \[X5V7]
\endverse
\beginverse
Vo\[X1]cê fingiu\[X1V11]
Você brin\[X1]cou com a minha\[X1V11] sensibili\[X1]dade\[X1V11]
É o \[X1]fim do nosso\[X1V11] caso, na ver\[X1]dade \[X5V1]\[X1V7]
Só nos restam re\[X4]cordações
\endverse
\beginchorus
Não \[X4]toque em mim
Hoje descobri que você não é \[X1]nada
Não podemos seguir juntos nesta es\[X2]trada \[X5]
É o \[X2]fim do a\[X5]mor sincero que sen\[X5V1]tii\[X1V7]i
Mas \[X4]aprendi
Fazer amor pra te ferir sem sentir \[X1]nada
Enquanto eu amava você me enga\[X2]nava \[X5]
De igual pra i\[X2]gual
Quem \[X5]sabe a gente pode \[X1]ser feliz \[(X1V7)]
\endchorus
\act{Riff Intro}{Repetir}{1x}
\act{Verso 3}{Retomar}{1x}
\beginverse
... \[X2]Ser fe\[X5]liz, \[X1]ser feliz
\[X2]Ser fe\[X5]liz, \[X1]ser feliz ...
\endverse
%-----------------------------------------------------------------
\vspace{4em} % Regulador de Espaçamento
%-----------------------------------------------------------------
\begin{comment}
\lstset{basicstyle=\scriptsize\bf} % Parâmetros da TAB
%-----------------------------------------------------------------
\tab{Solo 1}
\begin{lstlisting}
E|-----------------------------------------------------|
B|-----------------------------------------------------|
G|-----------------------------------------------------|
D|-----------------------------------------------------|
A|-----------------------------------------------------|
E|-----------------------------------------------------|
\end{lstlisting}
%-----------------------------------------------------------------
\end{comment}
%=================================================================


\color{drawChord}\gtab{\color{nameChord} X1}{}% 
\color{drawChord}\gtab{\color{nameChord} X1V7}{}% 
\color{drawChord}\gtab{\color{nameChord} X1V11}{}% 
\color{drawChord}\gtab{\color{nameChord} X2}{}%
\color{drawChord}\gtab{\color{nameChord} X4}{}% 
\color{drawChord}\gtab{\color{nameChord} X5}{}\\% 
\color{drawChord}\gtab{\color{nameChord} X5V4}{}% 
\color{drawChord}\gtab{\color{nameChord} X5V7}{}% 

%=================================================================
% PADRÃO: [TonalidadeMaior+NOTAX+Variações] .Ex:[X50] [X57V1V7]
% OBS: Variações são alterações do acorde em relação ao campo harmônico.
%-----------------------------------------------------------------
% Tipos de Variações de Acordes:
% V0 - Variação Diversa
% V1 - Menor (m)
% V2 - Maior (M)
% V3 - Meio Tom Abaixo (Bemol)
% V4 - Com Quarta (ex:C4)
% V5 - Com Quinta (ex:C5)
% V6 - Com Sexta (ex:C6)
% V7 - Com Sétima Menor (ex:C7)
% V8 - Com baixo dois Tons Acima (ex:D/F#)
% V9 - Com Nona (ex:C9)
% V10 - Meio Tom Acima (Sustenido)
% V11 - Com Sétima Maior (ex:C7M)
% V12 - Suspenso (Sus)
% V13 - Com baixo dois Tons e Meio Acima (ex:A/E)
% V14 - Com baixo um Tom e Meio Acima (ex:D9/F) 
% V15 - Meio-Diminuto (m7b5)
% N15 - NÃO Meio-Diminuto
% V16 - Diminuto (º)
% N16 - NÃO Diminuto
% V17 - Com baixo um Tom Acima (ex: C/D)
% V18 - Com baixo um Tom Abaixo (ex: Em/D)
% V19 - Com baixo dois Tons e meio Abaixo (ex: G/D)
%=================================================================
%\vspace{4em} % Regulador de Espaçamento
%=================================================================
\endsong
%=================================================================
