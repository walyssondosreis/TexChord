%=================================================================
\songcolumns{2}
\beginsong
{Ninguém Explica Deus %TÍTULO
}[by={Preto No Branco %ARTISTA
},album={@walyssondosreis},
id={GB0169 %COD.ID.: XXNNNN
},rev={0}, %REVISÃO
qr={https://drive.google.com/open?id=1n-uwsW-62AWNqSQxrhr-fh6GQ-OcwFI6 %LINK
}]
%-----------------------------------------------------------------
\tom{X1}{G}
%=================================================================
%\newchords{verse1.XX0000X} % Registrador de Acordes em Sequência
%\newchords{chorus1.XX0000X} % Registrador de Acordes em Sequência
%-----------------------------------------------------------------
\seq{Intro}{X1 X6 X4 X1V8 X2}{}
%\act{}{}{}
%-----------------------------------------------------------------
%\beginverse \endverse
%\beginchorus \endchorus
\beginverse
\[X1]Nada é igual ao Seu redor
\[X6]Tudo se faz no Seu olhar
\[X4]Todo o universo se for\[X1V8]mou no Seu fa\[X2]lar
\[X1]Teologia pra explicar
\[X6]Ou Big Bang pra disfarçar
\[X4]Pode alguém até duvid\[X1V8]ar
\[X2]Sei que há um Deus a \[X5]me guardar
\endverse
\beginverse
E \[X4]eu, tão pe\[X5]queno e frágil
Que\[X6]rendo Sua atenção
\[X2] No si\[X1]lêncio encontro 
Res\[X7V3N15]posta certa, en\[X5]tão
\endverse
\beginchorus
\[X1] Dono de toda ciência
\[X6] Sabedoria e poder
Oh, \[X4]dá-me de be\[X1V8]ber 
\[X2]Da água da \[X5]fonte da vida
\[X1] Antes que o haja houvesse
\[X6] Ele já era Deus
Se \[X4]revelou ao s\[X1V8]eus
Do \[X2]crente ao a\[X5]teu
Ninguém explica \[X1]Deus
\endchorus
\seq{Riff 1}{X1 X6 X4 X1V8 X2 X5}{}
\beginverse
\[X1] Ninguém explica
Ninguém explica Deus
\[X6] Ninguém explica
Ninguém explica Deus
\[X4] E se duvida, \[X1V8] ou se acredita
\[X2] Ninguém explica
\[X3V2V7] Ninguém explica Deus
\[X6] Ninguém explica
Ninguém explica Deus
\[X5V1] Ninguém explica
\[X1] Ninguém explica Deus
\[X4] E se duvida, \[X1V8] ou se acredita
\[X2] Ninguém explica
\[X5] Ninguém explica Deus
\endverse
\beginverse
... Ninguém explica \[X1]Deus
\endverse

%-----------------------------------------------------------------
\begin{comment}
\lstset{basicstyle=\scriptsize\bf} % Parâmetros da TAB
%-----------------------------------------------------------------
\tab{Solo 1}
\begin{lstlisting}
E|-----------------------------------------------------|
B|-----------------------------------------------------|
G|-----------------------------------------------------|
D|-----------------------------------------------------|
A|-----------------------------------------------------|
E|-----------------------------------------------------|
\end{lstlisting}
%-----------------------------------------------------------------
\end{comment}
%=================================================================
 
%-----------------------------------------------------------------
\color{drawChord}\gtab{\color{nameChord} X1}{}% 
\color{drawChord}\gtab{\color{nameChord} X1V8}{}% 
\color{drawChord}\gtab{\color{nameChord} X2}{}% 
\color{drawChord}\gtab{\color{nameChord} X3V2V7}{}% 
\color{drawChord}\gtab{\color{nameChord} X4}{}% 
\color{drawChord}\gtab{\color{nameChord} X5}{}\\% 
\color{drawChord}\gtab{\color{nameChord} X5V1}{}% 
\color{drawChord}\gtab{\color{nameChord} X6}{}% 
\color{drawChord}\gtab{\color{nameChord} X7V3N15}{}% 
%-----------------------------------------------------------------
% PADRÃO: [TonalidadeMaior+NOTAX+Variações] .Ex:[X50] [X57V1V7]
% OBS: Variações são alterações do acorde em relação ao campo harmônico.
%-----------------------------------------------------------------
% Tipos de Variações de Acordes:
% V0 - Variação Diversa
% V1 - Menor (m)
% V2 - Maior (M)
% V3 - Meio Tom Abaixo (Bemol)
% V4 - Com Quarta (ex:C4)
% V5 - Com Quinta (ex:C5)
% V6 - Com Sexta (ex:C6)
% V7 - Com Sétima Menor (ex:C7)
% V8 - Com baixo dois Tons Acima (ex:D/F#)
% V9 - Com Nona (ex:C9)
% V10 - Meio Tom Acima (Sustenido)
% V11 - Com Sétima Maior (ex:C7M)
% V12 - Suspenso (Sus)
% V13 - Com baixo dois Tons e Meio Acima (ex:A/E)
% V14 - Com baixo um Tom e Meio Acima (ex:D9/F) 
% V15 - Meio-Diminuto (m7b5)
% N15 - NÃO Meio-Diminuto
% V16 - Diminuto (º)
% N16 - NÃO Diminuto
% V17 - Com baixo um Tom Acima (ex: C/D)
% V18 - Com baixo um Tom Abaixo (ex: Em/D)
% V19 - Com baixo dois Tons e meio Abaixo (ex: G/D)
%=================================================================
\endsong
%=================================================================
\begin{comment}

\end{comment}