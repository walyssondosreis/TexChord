%=================================================================
\songcolumns{2}
\beginsong
{Se Não For Pra Te Adorar %TÍTULO
}[by={Fernandinho %ARTISTA
},album={@walyssondosreis},
id={GB0107 %COD.ID.: GB0000
}] 
%-----------------------------------------------------------------
\tom{X1}
%=================================================================
%\newchords{verse0.GB0000} % Registrador de Acordes em Sequência
%-----------------------------------------------------------------
%\seq{Intro}{}{}
%-----------------------------------------------------------------
%\beginverse* \endverse
%\beginchorus \endchorus

\beginchorus 
\[X1]Se não for pra te adorar, para que eu nas\[X4]ci?
Se não for pra te ser\[X6V7]vir, porque eu estou a\[X5V9]qui?
Sim eu quero te ado\[X4]rar, te ado\[X5V9]rar
Senhor, estou a\[X1]qui, \[X5V9]
Senhor, estou a\[X6]qui, \[X5]
Senhor, estou a\[X4]qui, \[X5]
Senhor, estou a\[X1]qui
\endchorus

\seq{Riff}{X1 X4 X6V7 X5V9}{2x}

\beginverse* 
Di\[X1]ante do trono, Se\[X4]nhor 
Quero le\[X6V7]var minha oferta de a\[X5V9]mor
Di\[X6V7]ante do trono, Se\[X3]nhor 
Quero le\[X4]var meu sacrifício de lou\[X5V9]vor
\[X1]As minhas mãos levan\[X4]tar
Tua be\[X6V7]leza então contem\[X5V9]plar
\[X6V7]Com meus lábios decla\[X3]rar 
Toda a \[X4]minha a adora\[X5V9]ção
\endverse

\beginchorus 
Se não for pra te ado\[X1]rar, para que eu nas\[X4]ci?
Se não for pra te ser\[X6V7]vir, porque eu estou a\[X5V9]qui?
Sim eu quero te ado\[X4]rar, te ado\[X5V9]rar
Senhor, estou a\[X1]qui
\endchorus

\seq{Riff}{X1 X4 X6V7 X5V9}{}

\beginverse* 
\[X1]Êô, êô, \[X4]êôô
\[X6V7]Êô, êô, \[X5V9]êôô
\[X1]Êô, êô, \[X4]êôô
\[X6V7]Êô, êô, \[X5V9]êôô
\endverse
%-----------------------------------------------------------------
\begin{comment}
\lstset{basicstyle=\scriptsize\bf} % Parâmetros da TAB
%-----------------------------------------------------------------
\tab{Solo 1}
\begin{lstlisting}
E|-----------------------------------------------------|
B|-----------------------------------------------------|
G|-----------------------------------------------------|
D|-----------------------------------------------------|
A|-----------------------------------------------------|
E|-----------------------------------------------------|
\end{lstlisting}
%-----------------------------------------------------------------
\end{comment}
%=================================================================
\vspace{2em}
%-----------------------------------------------------------------
\gtab{\color{black} X1}{}% D [X1]
\gtab{\color{black} X3}{}% F#m [X3]
\gtab{\color{black} X4}{}% G [X4]
\gtab{\color{black} X5}{}% A [X5]
\gtab{\color{black} X5V9}{}% A9 [X5V9]
\gtab{\color{black} X6}{}\\% Bm [X6]
\gtab{\color{black} X6V7}{}% Bm7 [X6V7]
%-----------------------------------------------------------------
% PADRÃO [TonalidadeMaiorNOTAX.Variação] .Ex:[X50] [X50V1]
% PADRÃO [TonalidadeMenorNOTAX.Variação] .Ex:[mX50] [mX50V1]
% OBS: Variações são alterações do acorde em relação ao campo harmônico.
%-----------------------------------------------------------------
% TIPOS DE VARIAÇÂO DOS ACORDES:
% V0 - ACORDE COM VARIAÇÃO DIVERSA
% V1 - ACORDE MENOR (m)
% V2 - ACORDE MAIOR (M)
% V3 - ACORDE MEIO TOM ABAIXO (Bemois)
% V4 - ACORDE COM QUARTA (C4)
% V5 - ACORDE COM QUINTA (C5)
% V6 - ACORDE COM SEXTA (C6)
% V7 - ACORDE COM SÉTIMA MENOR (C7)
% V8 - ACORDE COM BAIXO DOIS TONS ACIMA (D/F#)
% V9 - ACORDE COM NONA (C9)
% V10 - ACORDE MEIO TOM ACIMA (Sustenidos)
% V11 - ACORDE COM SÉTIMA MAIOR (C7M)
% V12 - ACORDE SUSPENSO (Sus)
% V13 - ACORDE COM BAIXO DOIS TONS E MEIO ACIMA (A/E)
% V14 - ACORDE UM TOM E MEIO ACIMA (D9/F)
%=================================================================
\endsong
%=================================================================







