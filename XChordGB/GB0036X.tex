%=================================================================
\songcolumns{2}
\beginsong
{Eu Jamais Serei o Mesmo %TÍTULO
}[by={Fernandinho %ARTISTA
},album={@walyssondosreis},
id={GB0036 %COD.ID.: GB0000
},rev={3}, %REVISÃO
qr={https://drive.google.com/open?id=1UsDLzNKj5rSDUxAa9-0t_EF3VIogOsie %LINK
}]
%-----------------------------------------------------------------
\tom{X1}{Bb}
%=================================================================
%\newchords{verse1.GB0000X} % Registrador de Acordes em Sequência
%\newchords{chorus1.GB0000X} % Registrador de Acordes em Sequência
%-----------------------------------------------------------------
\seq{Intro}{X1 X1V9 X1 X4}{2x}
%-----------------------------------------------------------------
%\beginverse* \endverse
%\beginchorus \endchorus
\beginverse
Se o Se\[X1]nhor tocar meus \[X4]olhos
Eu ve\[X6]rei a \[X4]Tua \[X1]face
Se o Se\[X1]nhor tocar meus \[X4]lábios
Eu se\[X6]rei pu\[X5V8]rifi\[X1]cado
Se o Se\[X2]nhor to\[X1V8]car meus \[X2]pés
Eles \[X6]corre\[X5V8]rão pra \[X1]Ti
Se o Se\[X2]nhor to\[X1V8]car meu \[X2]coração
Eu se\[X4]rei totalmente \[X6]teu
Totalmente \[X4]teu
\endverse
\act{Repetir}{Verso 1}{+1x}
\beginchorus
Eu ja\[X1]mais serei o \[X2]mesmo
Eu ja\[X1]mais serei o \[X2]mesmo
Eu ja\[X1]mais serei o \[X5V8]mesmo
Quando o Teu a\[X6]mor to\[X5]car em \[X4]mim
Eu ja\[X1]mais serei o \[X2]mesmo
Eu ja\[X1]mais serei o \[X2]mesmo
Eu ja\[X1]mais serei o \[X5V8]mesmo
Toca-\[X6]me, toca-\[X4]me
\endchorus
\beginverse
\[X1]Sonda o meu coração
\[X5V8]Vê se há em mim algum ca\[X6]minho mal \[X4]
\[X1]Guia-me por Teus caminhos
\[X5V8]Tua Palavra é lâmpada para \[X6]os meus pés \[X4]
\endverse
\act{Repetir}{Verso 2}{+1x}
\act{Repetir}{Refrão}{1x}
\beginverse
...Toca-\[X6]me Senhor, toca-\[X4]me
\endverse
%-----------------------------------------------------------------
\vspace{4em} % Regulador de Espaçamento
%-----------------------------------------------------------------
\begin{comment}
\lstset{basicstyle=\scriptsize\bf} % Parâmetros da TAB
%-----------------------------------------------------------------
\tab{Solo 1}
\begin{lstlisting}
E|-----------------------------------------------------|
B|-----------------------------------------------------|
G|-----------------------------------------------------|
D|-----------------------------------------------------|
A|-----------------------------------------------------|
E|-----------------------------------------------------|
\end{lstlisting}
%-----------------------------------------------------------------
\end{comment}
%=================================================================
 
%-----------------------------------------------------------------
\color{drawChord}\gtab{\color{nameChord} X1}{}%  
\color{drawChord}\gtab{\color{nameChord} X1V8}{}% 
\color{drawChord}\gtab{\color{nameChord} X1V9}{}%  
\color{drawChord}\gtab{\color{nameChord} X2}{}% 
\color{drawChord}\gtab{\color{nameChord} X4}{}%
\color{drawChord}\gtab{\color{nameChord} X5}{}\\%
\color{drawChord}\gtab{\color{nameChord} X5V8}{}%
\color{drawChord}\gtab{\color{nameChord} X6}{}%
%-----------------------------------------------------------------
% PADRÃO: [TonalidadeMaior+NOTAX+Variações] .Ex:[X50] [X57V1V7]
% OBS: Variações são alterações do acorde em relação ao campo harmônico.
%-----------------------------------------------------------------
% Tipos de Variações de Acordes:
% V0 - Variação Diversa
% V1 - Menor (m)
% V2 - Maior (M)
% V3 - Meio Tom Abaixo (Bemol)
% V4 - Com Quarta (ex:C4)
% V5 - Com Quinta (ex:C5)
% V6 - Com Sexta (ex:C6)
% V7 - Com Sétima Menor (ex:C7)
% V8 - Com baixo dois Tons Acima (ex:D/F#)
% V9 - Com Nona (ex:C9)
% V10 - Meio Tom Acima (Sustenido)
% V11 - Com Sétima Maior (ex:C7M)
% V12 - Suspenso (Sus)
% V13 - Com baixo dois Tons e Meio Acima (ex:A/E)
% V14 - Com baixo um Tom e Meio Acima (ex:D9/F) 
% V15 - Meio-Diminuto (m7b5)
% N15 - NÃO Meio-Diminuto
% V16 - Diminuto (º)
% N16 - NÃO Diminuto
%=================================================================
\endsong
%=================================================================
\begin{comment}

\end{comment}