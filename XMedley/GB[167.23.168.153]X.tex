%=================================================================
\songcolumns{2}
\beginsong
{Medley %TÍTULO
}[by={Vários Artistas %ARTISTA
},album={@walyssondosreis},
id={GB[167.23.168.153] %COD.ID.: XXNNNN
},rev={0}, %REVISÃO
qr={ %LINK
}]
%-----------------------------------------------------------------
\tom{X1}{D}
%=================================================================
%\newchords{verse1.XX0000X} % Registrador de Acordes em Sequência
%\newchords{chorus1.XX0000X} % Registrador de Acordes em Sequência
%-----------------------------------------------------------------
\act{\bf\color{red} ATRAI O MEU CORAÇÃO - Filhos do Homem\color{black}}{}{Música Inteira}
\seq{Intro}{X1V9 X1V11V9 X1V4V9 X4V8}{2x}
%\act{}{}{}
%-----------------------------------------------------------------
%\beginverse \endverse
%\beginchorus \endchorus
\beginverse
\[X1] Tu és minha \[X4]vida, Je\[X5V8]sus
És meu a\[X1]migo \[(X6)]
E a tua von\[X4]tade, Doce Es\[X5V8]pírito
Meu ali\[X1]mento
\endverse
\beginverse
Sem \[X4]ti não há valor em \[X1V8]mim, sou como um vaso de \[X2]barro
Pronto a ser que\[X4]brado para ser o que \[X5V4]queres de \[X5]mim
\endverse
\beginverse
A tua pre\[X4]sença é tudo que eu pre\[X1V8]ciso
A tua pre\[X4]sença é o meu maior va\[X1V8]lor
\endverse
\beginchorus
Atrai o meu \[X4]coração
Atrai o meu \[X2]coração
És tudo que eu \[X1]quero
Atrai o meu \[X4]coração
Atrai o meu \[X2]coração
Eu posso te to\[X1]car
\endchorus
\seq{Interlúdio}{X4 X5V8 X4V8 X5V4V7 X1}{1x}
\act{Retomar}{Verso 1}{1x}
\act{Repetir}{Refrão}{+1x}
\seq{Interlúdio}{X4 X5V8 X4V8 X5V4V7 X1}{1x}
\seq{\bf\color{red} DEIXA QUEIMAR - Alessandro Vilas Boas \color{black}}{X1 X5 X6V7 X4}{ Música Inteira}

\act{\bf\color{red} QUÃO GRANDE É O MEU DEUS - Soraya Moraes\color{black}}{}{Apenas Refrão}
\beginchorus
Quão \[X1]grande é o meu Deus
Cantarei quão \[X6]grande é o meu Deus
E todos hão de \[X4]ver
Quão \[X5]grande é o meu \[X1]Deus \[X5]
\endchorus

\act{\bf\color{red} CONSAGRAÇÃO - Aline Barros\color{black}}{}{Apenas Refrão}
\beginchorus
A \[X1]honra, a \[X5V8]glória, a \[X6]força
E \[X3]o poder ao rei Je\[X4]sus \[X1V8]
E o lou\[X2]vor \[X1] ao \[X7V3N15]rei \[X4V8] Je\[X5]sus
\endchorus
%-----------------------------------------------------------------
\begin{comment}
\lstset{basicstyle=\scriptsize\bf} % Parâmetros da TAB
%-----------------------------------------------------------------
\tab{Solo 1}
\begin{lstlisting}
E|-----------------------------------------------------|
B|-----------------------------------------------------|
G|-----------------------------------------------------|
D|-----------------------------------------------------|
A|-----------------------------------------------------|
E|-----------------------------------------------------|
\end{lstlisting}
%-----------------------------------------------------------------
\end{comment}
%=================================================================
\begin{comment}
\vspace{2em} 
%-----------------------------------------------------------------
\color{drawChord}\gtab{\color{nameChord} X1}{}% 
\color{drawChord}\gtab{\color{nameChord} X1V4V9}{}%
\color{drawChord}\gtab{\color{nameChord} X1V8}{}% 
\color{drawChord}\gtab{\color{nameChord} X1V9}{}% 
\color{drawChord}\gtab{\color{nameChord} X1V11V9}{}% 
\color{drawChord}\gtab{\color{nameChord} X2}{}% 
\color{drawChord}\gtab{\color{nameChord} X3}{}%
\color{drawChord}\gtab{\color{nameChord} X4}{}% 
\color{drawChord}\gtab{\color{nameChord} X4V8}{}% 
\color{drawChord}\gtab{\color{nameChord} X5}{}% 
\color{drawChord}\gtab{\color{nameChord} X5V4}{}%
\color{drawChord}\gtab{\color{nameChord} X5V4V7}{}% 
\color{drawChord}\gtab{\color{nameChord} X5V8}{}% 
\color{drawChord}\gtab{\color{nameChord} X6}{}% 
\color{drawChord}\gtab{\color{nameChord} X6V7}{}%
\color{drawChord}\gtab{\color{nameChord} X7V3N15}{}%
\end{comment}
%-----------------------------------------------------------------
% PADRÃO: [TonalidadeMaior+NOTAX+Variações] .Ex:[X50] [X57V1V7]
% OBS: Variações são alterações do acorde em relação ao campo harmônico.
%-----------------------------------------------------------------
% Tipos de Variações de Acordes:
% V0 - Variação Diversa
% V1 - Menor (m)
% V2 - Maior (M)
% V3 - Meio Tom Abaixo (Bemol)
% V4 - Com Quarta (ex:C4)
% V5 - Com Quinta (ex:C5)
% V6 - Com Sexta (ex:C6)
% V7 - Com Sétima Menor (ex:C7)
% V8 - Com baixo dois Tons Acima (ex:D/F#)
% V9 - Com Nona (ex:C9)
% V10 - Meio Tom Acima (Sustenido)
% V11 - Com Sétima Maior (ex:C7M)
% V12 - Suspenso (Sus)
% V13 - Com baixo dois Tons e Meio Acima (ex:A/E)
% V14 - Com baixo um Tom e Meio Acima (ex:D9/F) 
% V15 - Meio-Diminuto (m7b5)
% N15 - NÃO Meio-Diminuto
% V16 - Diminuto (º)
% N16 - NÃO Diminuto
% V17 - Com baixo um Tom Acima (ex: C/D)
% V18 - Com baixo um Tom Abaixo (ex: Em/D)
% V19 - Com baixo dois Tons e meio Abaixo (ex: G/D)
%=================================================================
\endsong
%=================================================================
\begin{comment}

\end{comment}