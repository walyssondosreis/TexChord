%=================================================================
\songcolumns{1}
\beginsong
{Ainda Que a Figueira %TÍTULO
}[by={Fernandinho %ARTISTA
},album={@walyssondosreis},
id={GB0159 %COD.ID.: GB0000
},rev={1}, %REVISÃO
qr={ %LINK
}]
%-----------------------------------------------------------------
\tom{X1}{X1}
%=================================================================
%\newchords{verse1.GB0000X} % Registrador de Acordes em Sequência
%\newchords{chorus1.GB0000X} % Registrador de Acordes em Sequência
%-----------------------------------------------------------------
\seq{Intro}{X6 X1 X5 X2}{2x}
%\act{}{}{}
%-----------------------------------------------------------------
%\beginverse \endverse
%\beginchorus \endchorus
\beginverse
\[X6] Tu és a minh\[X5]a porção
\[X2V7] Tu és a \[X4]minha herança
\[X6] Tu és o m\[X5]eu socorro
Nos \[X2V7]dias de tribula\[X4]ção
\endverse
\beginverse
\[X1] Mesmo que meus \[X5]pais me deixem
\[X2V7] Mesmo que a\[X4]migos me traiam
\[X1] Eu sei que em seus \[X5]braços
Eu en\[X2V7]contro salva\[X4]ção
\endverse
\beginchorus
A\[X1]inda que a figueira \[X5]não floresça
A\[X2V7]inda que a videira não \[X4]dê o seu fruto
\[X1]Mesmo que não haja ali\[X5]mento nos campos
Eu me a\[X7V3N15]legrarei em \[X4]ti
\endchorus
\act{Repetir}{Refrão}{+1x}
\act{Executar}{Riff Intro}{}
\act{Retomar}{Verso 1}{1x}
\seq{Base Solo}{X2V7 X1 X5 (X4)}{4x}
\act{Repetir}{Refrão}{2x}
% Verso de preenchimento
\beginverse*
.
.
.
\endverse
%-----------------------------------------------------------------
\begin{comment}
\lstset{basicstyle=\scriptsize\bf} % Parâmetros da TAB
%-----------------------------------------------------------------
\tab{Solo 1}
\begin{lstlisting}
E|-----------------------------------------------------|
B|-----------------------------------------------------|
G|-----------------------------------------------------|
D|-----------------------------------------------------|
A|-----------------------------------------------------|
E|-----------------------------------------------------|
\end{lstlisting}
%-----------------------------------------------------------------
\end{comment}
%=================================================================
\vspace{2em} 
%-----------------------------------------------------------------
\gtab{\color{black} X1}{}% 
\gtab{\color{black} X2}{}% 
\gtab{\color{black} X2V7}{}%
\gtab{\color{black} X4}{}%
\gtab{\color{black} X5}{}% 
\gtab{\color{black} X6}{}% 
\gtab{\color{black} X7V3N15}{}%
%-----------------------------------------------------------------
% PADRÃO: [TonalidadeMaior+NOTAX+Variações] .Ex:[X50] [X57V1V7]
% OBS: Variações são alterações do acorde em relação ao campo harmônico.
%-----------------------------------------------------------------
% Tipos de Variações de Acordes:
% V0 - Variação Diversa
% V1 - Menor (m)
% V2 - Maior (M)
% V3 - Meio Tom Abaixo (Bemol)
% V4 - Com Quarta (ex:C4)
% V5 - Com Quinta (ex:C5)
% V6 - Com Sexta (ex:C6)
% V7 - Com Sétima Menor (ex:C7)
% V8 - Com baixo dois Tons Acima (ex:D/F#)
% V9 - Com Nona (ex:C9)
% V10 - Meio Tom Acima (Sustenido)
% V11 - Com Sétima Maior (ex:C7M)
% V12 - Suspenso (Sus)
% V13 - Com baixo dois Tons e Meio Acima (ex:A/E)
% V14 - Com baixo um Tom e Meio Acima (ex:D9/F) 
% V15 - Meio-Diminuto (m7b5)
% N15 - NÃO Meio-Diminuto
% V16 - Diminuto (º)
% N16 - NÃO Diminuto
% V17 - Com baixo um Tom Acima (ex: C/D)
%=================================================================
\endsong
%=================================================================
\begin{comment}

\end{comment}