%=================================================================
\songcolumns{1}
\beginsong
{ %TÍTULO
}[by={ %ARTISTA
},album={@walyssondosreis},
id={ %COD.ID.: GB0000
}] 
%-----------------------------------------------------------------
\tom{X1}
%=================================================================
%\newchords{verse1.GB0000X} % Registrador de Acordes em Sequência
%\newchords{chorus1.GB0000X} % Registrador de Acordes em Sequência
%-----------------------------------------------------------------
%\seq{Intro}{}{}
%-----------------------------------------------------------------
%\beginverse* \endverse
%\beginchorus \endchorus



%-----------------------------------------------------------------
\begin{comment}
\lstset{basicstyle=\scriptsize\bf} % Parâmetros da TAB
%-----------------------------------------------------------------
\tab{Solo 1}
\begin{lstlisting}
E|-----------------------------------------------------|
B|-----------------------------------------------------|
G|-----------------------------------------------------|
D|-----------------------------------------------------|
A|-----------------------------------------------------|
E|-----------------------------------------------------|
\end{lstlisting}
%-----------------------------------------------------------------
\end{comment}
%=================================================================
\vspace{2em} 
%-----------------------------------------------------------------
\gtab{\color{black} X1}{}%  [X1]
\gtab{\color{black} X2}{}%  [X2]
\gtab{\color{black} X3}{}%  [X3]
\gtab{\color{black} X4}{}%  [X4]
%-----------------------------------------------------------------
% PADRÃO [TonalidadeMaiorNOTAX.Variação] .Ex:[X50] [X50V1]
% PADRÃO [TonalidadeMenorNOTAX.Variação] .Ex:[mX50] [mX50V1]
% OBS: Variações são alterações do acorde em relação ao campo harmônico.
%-----------------------------------------------------------------
% TIPOS DE VARIAÇÂO DOS ACORDES:
% V0 - VARIAÇÃO DIVERSA
% V1 - MENOR (m)
% V2 - MAIOR (M)
% V3 - MEIO TOM ABAIXO (Bemol)
% V4 - COM QUARTA (C4)
% V5 - COM QUINTA (C5)
% V6 - COM SEXTA (C6)
% V7 - COM SÉTIMA MENOR (C7)
% V8 - COM BAIXO DOIS TONS ACIMA (D/F#)
% V9 - COM NONA (C9)
% V10 - MEIO TOM ACIMA (Sustenido)
% V11 - COM SÉTIMA MAIOR (C7M)
% V12 - SUSPENSO (Sus)
% V13 - COM BAIXO DOIS TONS E MEIO ACIMA (A/E)
% V14 - COM BAIXO UM TOM E MEIO ACIMA (D9/F)
% V15 - MEIO-DIMINUTO 
% V16 - DIMINUTO
% N17 - NÂO MEIO-DIMINUTO
% N18 - NÃO DIMINUTO
%=================================================================
\endsong
%=================================================================
\begin{comment}

\end{comment}