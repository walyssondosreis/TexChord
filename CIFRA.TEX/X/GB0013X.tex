%=================================================================
\songcolumns{1}
\beginsong
{Bom Estarmos Aqui %TÍTULO
}[by={Renascer Praise  %ARTISTA
},album={@walyssondosreis},
id={GB0013 %COD.ID.: GB0000
}] 
%-----------------------------------------------------------------
\tom{X1}
%=================================================================
%\newchords{verse0.GB0000} % Registrador de Acordes em Sequência
%-----------------------------------------------------------------
%\seq{Intro}{}{}
%-----------------------------------------------------------------
%\beginverse* \endverse
%\beginchorus \endchorus

\beginverse* 
\[X1]Bom estarmos a\[X2]qui louvando a \[X3]Deus \[X2]
\[X1]Podendo exal\[X2]tar seu Santo \[X3]nome \[X2]
\endverse

\beginverse* 
\[X4]Tempo para \[X5]isso
\[X3]Tempo para lou\[X6]varmos a Deus
\[X4]Num só amor,\[X5] num só Es\[X1]pírito
\[X4]Tempo para \[X5]isso
\[X3]Tempo para lou\[X6]varmos a Deus
\[X4]Num só amor,\[X5] num só Es\[X1]pírito
\endverse

\beginchorus 
\[X1]Deus, venha nos abenço\[X2]ar
E que esta uni\[X3]ão
\[X4]Nunca falte \[X5]para \[X1]nós
\endchorus

%-----------------------------------------------------------------
\begin{comment}
\lstset{basicstyle=\scriptsize\bf} % Parâmetros da TAB
%-----------------------------------------------------------------
\tab{Solo 1}
\begin{lstlisting}
E|-----------------------------------------------------|
B|-----------------------------------------------------|
G|-----------------------------------------------------|
D|-----------------------------------------------------|
A|-----------------------------------------------------|
E|-----------------------------------------------------|
\end{lstlisting}
%-----------------------------------------------------------------
\end{comment}
%=================================================================
\vspace{2em}
%-----------------------------------------------------------------
\gtab{\color{black} X1}{}% E [X1]
\gtab{\color{black} X2}{}% F#m [X2]
\gtab{\color{black} X3}{}% G#m [X3]
\gtab{\color{black} X4}{}% A [X4]
\gtab{\color{black} X5}{}% B [X5]
\gtab{\color{black} X6}{}% C#m [X6]
%-----------------------------------------------------------------
% PADRÃO [TonalidadeMaiorNOTAX.Variação] .Ex:[X50] [X50V1]
% PADRÃO [TonalidadeMenorNOTAX.Variação] .Ex:[mX50] [mX50V1]
% OBS: Variações são alterações do acorde em relação ao campo harmônico.
%-----------------------------------------------------------------
% TIPOS DE VARIAÇÂO DOS ACORDES:
% V1 - ACORDE MENOR (m)
% V2 - ACORDE MAIOR (M)
% V3 - ACORDE MEIO TOM ABAIXO (Bemois)
% V4 - ACORDE COM QUARTA (C4)
% V5 - ACORDE COM QUINTA (C5)
% V6 - ACORDE COM SEXTA (C6)
% V7 - ACORDE COM SÉTIMA MENOR (C7)
% V8 - ACODE COM BAIXO DOIS TONS ACIMA (D/F#)
% V9 - ACORDE COM NONA (C9)
% V10 - ACORDE MEIO TOM ACIMA (Sustenidos)
% V11 - ACORDE COM SÉTIMA MAIOR (C7M)
% V12 - ACORDE SUSPENSO (Sus)
% V13 - ACORDE COM VARIAÇÃO DIVERSA
%=================================================================
\endsong
%=================================================================
\begin{comment}

\end{comment}