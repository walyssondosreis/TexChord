%=================================================================
\songcolumns{1}
\beginsong
{Deus É Deus %TÍTULO
}[by={Delino Marçal %ARTISTA
},album={@walyssondosreis},
id={GB0155 %COD.ID.: GB0000
}] 
%-----------------------------------------------------------------
\tom{X1}
%=================================================================
%\newchords{verse0.GB0000} % Registrador de Acordes em Sequência
%-----------------------------------------------------------------
\seq{Intro}{X1 X5 X6V7 X5 X4 X1 X2V7 X5V4 X5}{}
%-----------------------------------------------------------------
%\beginverse* \endverse
%\beginchorus \endchorus

\beginverse*
Minha \[X1]fé não está fir\[X5V4]mada
Nas coisas \[X6V7]que podes fa\[X5]zer
Eu apren\[X1]di a Te ado\[X4]rar pelo que \[X5V4]és \[X5]
Dele \[X2V7]vêm o sim e o a\[X5V4]mém
Somente \[X1]dele e \[X5]mais nin\[X6V7]guém
\[X5]A \[X4]Deus seja \[X5]o lou\[X4]vor \[X5V4]
\endverse

\beginchorus
Se Deus fi\[X1]zer, Ele é \[X5]Deus
Se não fi\[X6V7]zer, Ele é \[X5]Deus
Se a porta a\[X4]brir, Ele é \[X1]Deus
Mas se fe\[X2V7]char, continua sendo \[X5V4]Deus
Se a doença vi\[X1]er, Ele é \[X5]Deus
Se curado eu \[X6V7]for, Ele é \[X5]Deus
Se tudo der \[X4]certo, Ele é \[X1]Deus
Mas se não \[X2V7]der, continua sendo \[X5V4]Deus \[(X5)]
\endchorus

\beginverse*
Eu não o a\[X4]doro pelo que Ele \[X1]faz
Eu o a\[X2V7]doro pelo que Ele \[X6V7]é
\[X5]Haja o que hou\[X4]ver, \[X5]sempre será \[X4]Deus \[X5]
\endverse

\beginverse*
\[X4]Deus é Deus
\[X6V7]Deus é Deus \[(X5)]
\endverse
%-----------------------------------------------------------------
\begin{comment}
\lstset{basicstyle=\scriptsize\bf} % Parâmetros da TAB
%-----------------------------------------------------------------
\tab{Solo 1}
\begin{lstlisting}
E|-----------------------------------------------------|
B|-----------------------------------------------------|
G|-----------------------------------------------------|
D|-----------------------------------------------------|
A|-----------------------------------------------------|
E|-----------------------------------------------------|
\end{lstlisting}
%-----------------------------------------------------------------
\end{comment}
%=================================================================
\vspace{2em}
%-----------------------------------------------------------------
\gtab{\color{black} X1}{}% C [X1]
\gtab{\color{black} X2V7}{}% Dm7 [X2V7]
\gtab{\color{black} X4}{}% F [X4]
\gtab{\color{black} X5}{}% G [X5]
\gtab{\color{black} X5V4}{}% G4 [X5V4]
\gtab{\color{black} X6V7}{}% Am7 [X6V7]
%-----------------------------------------------------------------
% PADRÃO [TonalidadeMaiorNOTAX.Variação] .Ex:[X50] [X50V1]
% PADRÃO [TonalidadeMenorNOTAX.Variação] .Ex:[mX50] [mX50V1]
% OBS: Variações são alterações do acorde em relação ao campo harmônico.
%-----------------------------------------------------------------
% TIPOS DE VARIAÇÂO DOS ACORDES:
% V0 - ACORDE COM VARIAÇÃO DIVERSA
% V1 - ACORDE MENOR (m)
% V2 - ACORDE MAIOR (M)
% V3 - ACORDE MEIO TOM ABAIXO (Bemois)
% V4 - ACORDE COM QUARTA (C4)
% V5 - ACORDE COM QUINTA (C5)
% V6 - ACORDE COM SEXTA (C6)
% V7 - ACORDE COM SÉTIMA MENOR (C7)
% V8 - ACORDE COM BAIXO DOIS TONS ACIMA (D/F#)
% V9 - ACORDE COM NONA (C9)
% V10 - ACORDE MEIO TOM ACIMA (Sustenidos)
% V11 - ACORDE COM SÉTIMA MAIOR (C7M)
% V12 - ACORDE SUSPENSO (Sus)
% V13 - ACORDE COM BAIXO DOIS TONS E MEIO ACIMA (A/E)
% V14 - ACORDE UM TOM E MEIO ACIMA (D9/F)
%=================================================================
\endsong
%=================================================================
\begin{comment}

\end{comment}