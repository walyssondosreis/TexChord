%=================================================================
\songcolumns{2}
\beginsong
{A Alegria do Senhor %TÍTULO
}[by={Fernandinho %ARTISTA
},album={@walyssondosreis},
id={GB0002 %COD.ID.: GB0000
},rev={3}, %REVISÃO
qr={https://drive.google.com/open?id=1iVNukfQ34hJRmYMf-DZfe-Ft_X0zvUea %LINK
}]
%-----------------------------------------------------------------
\tom{X1}{Bb}
%=================================================================
%\newchords{verse0.GB0000} % Registrador de Acordes em Sequência
%-----------------------------------------------------------------
\seq{Intro}{X2 X1 X6}{2x}
%-----------------------------------------------------------------
\begin{verse}
\[X6]Vento sopra forte
\chordsoff Tuas águas não podem me afogar
\chordson \[X6]Vento sopra forte
\chordsoff E em suas mãos vou segurar
\chordson\end{verse}
\seq{Riff Intro}{X2 X1 X6}{2x}
\act{Repetir}{Verso 1}{1x}
\begin{verse}
\[X2V7]E não me guio pelo que vejo
\[X6V7]Mas eu sigo pelo que creio
\[X2V7]Eu não olho as circunstâncias
\[X4]Eu vejo o teu a\[X3V2]mor \[X4 X3V2 X4 X3V2]
\end{verse}

\begin{chorus}
\[X6]A alegria do Se\[X1]nhor é a nossa \[X4]força\[X2]
\[X6]A alegria do Se\[X1]nhor é a nossa \[X4]força\[X2]
\[X6]A alegria do Se\[X1]nhor é a nossa \[X4]força\[X2]
\[X6]A alegria do Se\[X1]nhor é a nossa \[X4]força\[X2]
\end{chorus}
\seq{Riff Intro}{X2 X1 X6}{2x}
\act{Executar}{Solo 1}{}
\act{Retomar}{Verso 1}{1x}
\begin{verse}
Essa ale\[X1]gria não vai mais sa\[X4]ir
Essa ale\[X2]gria não vai mais sa\[X6]ir
Essa ale\[X1]gria não vai mais sa\[X4]ir
De \[X5]dentro do meu cora\[X1]ção
\end{verse}
\act{Repetir}{Refrão}{2x}
\begin{chorus}
\[X6]A alegria do Se\[X1]nhor é a nossa \[X4]força\[X2]
\[X6]A alegria do Se\[X1]nhor é a nossa \[X4]força\[X2]
\[X6]A alegria do Se\[X1]nhor é a nossa \[X4]força\[X2]
\end{chorus}
%-----------------------------------------------------------------
\vspace{4em} % Regulador de Espaçamento
%-----------------------------------------------------------------
\begin{comment}
\lstset{basicstyle=\scriptsize\bf} % Parâmetros da TAB
%-----------------------------------------------------------------
\tab{Solo 1}
\begin{lstlisting}
E|-----------------------------------------------------|
B|-----------------------------------------------------|
G|-----------------------------------------------------|
D|-----------------------------------------------------|
A|-----------------------------------------------------|
E|-----------------------------------------------------|
\end{lstlisting}
%-----------------------------------------------------------------
\end{comment}
%=================================================================

%-----------------------------------------------------------------
\color{drawChord}\gtab{\color{nameChord} X1}{}% A [X1]
\color{drawChord}\gtab{\color{nameChord} X2}{}% Bm [X2]
\color{drawChord}\gtab{\color{nameChord} X2V7}{}% Bm7 [X2V7]
\color{drawChord}\gtab{\color{nameChord} X3V2}{}% C# [X3V2]
\color{drawChord}\gtab{\color{nameChord} X4}{}% D [X4]
\color{drawChord}\gtab{\color{nameChord} X5}{}\\% E [X5]
\color{drawChord}\gtab{\color{nameChord} X6}{}% F#m [X6]
\color{drawChord}\gtab{\color{nameChord} X6V7}{}% F#m7 [X6V7]
%-----------------------------------------------------------------
% PADRÃO [TonalidadeMaiorNOTAX.Variação] .Ex:[X50] [X50V1]
% PADRÃO [TonalidadeMenorNOTAX.Variação] .Ex:[mX50] [mX50V1]
% OBS: Variações são alterações do acorde em relação ao campo harmônico.
%-----------------------------------------------------------------
% TIPOS DE VARIAÇÂO DOS ACORDES:
% V0 - ACORDE COM VARIAÇÃO DIVERSA
% V1 - ACORDE MENOR (m)
% V2 - ACORDE MAIOR (M)
% V3 - ACORDE MEIO TOM ABAIXO (Bemois)
% V4 - ACORDE COM QUARTA (C4)
% V5 - ACORDE COM QUINTA (C5)
% V6 - ACORDE COM SEXTA (C6)
% V7 - ACORDE COM SÉTIMA MENOR (C7)
% V8 - ACORDE COM BAIXO DOIS TONS ACIMA (D/F#)
% V9 - ACORDE COM NONA (C9)
% V10 - ACORDE MEIO TOM ACIMA (Sustenidos)
% V11 - ACORDE COM SÉTIMA MAIOR (C7M)
% V12 - ACORDE SUSPENSO (Sus)
% V13 - ACORDE COM BAIXO DOIS TONS E MEIO ACIMA (A/E)
% V14 - ACORDE UM TOM E MEIO ACIMA (D9/F)
%=================================================================
\endsong
%=================================================================
