%=================================================================
\songcolumns{2}
\beginsong
{O Chamado %TÍTULO
}[by={Pr. Lucas %ARTISTA
},album={@walyssondosreis},
id={GB0073 %COD.ID.: GB0000
},rev={3}, %REVISÃO
qr={https://drive.google.com/open?id=1eGMbUhGaSQtR91fKzSbYguWXdeVfwu0F %LINK
}]
%-----------------------------------------------------------------
\tom{X1}{F}
%=================================================================
%\newchords{verse1.GB0000X} % Registrador de Acordes em Sequência
\newchords{chorus1.GB0073X} % Registrador de Acordes em Sequência
%-----------------------------------------------------------------
\seq{Intro}{}{}
%-----------------------------------------------------------------
%\beginverse* \endverse
%\beginchorus \endchorus
\beginchorus\memorize[chorus1.GB0073X]
\[X6]Ouô, \[X4] \[X1]pode me cha\[X5]mar, Senhor
\[X6]Ouô, \[X4] \[X1]eu estou a\[X5]qui
\[X6]Ouô, \[X4] \[X1]estou ao seu dis\[X5]por
\[X6]Quando fizer algo nessa \[X4]geração
\[X5]Não faça sem mim
\endchorus
\act{Executar}{Riff 1}{}
\act{Executar}{Solo 1}{}
\beginverse\replay[chorus1.GB0073X]
Quan^do curar, ^ me ^usa pra o^rar
Quan^do falar, ^ ^deixe-me ser um pro^feta
Quan^do mover, ^ ^enche-me de ^Ti
Quan^do fizer algo nessa ^geração
^Não faça sem mim
\endverse
\act{Executar}{Refrão}{1x}
\act{Executar}{Riff 1}{}
\beginverse\replay[chorus1.GB0073X]
Se ^quer usar ^ ^a minha can^ção
Pra ^despertar ^
Teu ^povo nessa na^ção
Pra ^combater ^
Todo ^mal, toda corrup^ção
^Quando fizer algo nessa ^geração
^Não faça sem mim
\endverse
\act{Repetir}{Refrão}{1x}
\act{Executar}{Solo 2}{}
\beginverse
Me \[X4]usa pra libertar os cativos
Me \[X2V2]usa pra alimentar os famintos
Me \[X4]usa pra ser a força do fraco
Eu es\[X3V2]tou a seu dispor
Eu a\[X3V2]ceito o seu chamado
Me \[X4]usa Espírito Santo
E \[X2V2]ungi aquilo que eu canto
E \[X4]enche o meu coração
De co\[X3V2]ragem pra mudar a minha geração
\endverse
\act{Repetir}{Refrão}{1x}
\act{Executar}{Riff 2}{}

%-----------------------------------------------------------------
\begin{comment}
\lstset{basicstyle=\scriptsize\bf} % Parâmetros da TAB
%-----------------------------------------------------------------
\tab{Solo 1}
\begin{lstlisting}
E|-----------------------------------------------------|
B|-----------------------------------------------------|
G|-----------------------------------------------------|
D|-----------------------------------------------------|
A|-----------------------------------------------------|
E|-----------------------------------------------------|
\end{lstlisting}
%-----------------------------------------------------------------
\end{comment}
%=================================================================
\vspace{2em} 
%-----------------------------------------------------------------
\color{drawChord}\gtab{\color{nameChord} X1}{}% 
\color{drawChord}\gtab{\color{nameChord} X2V2}{}% 
\color{drawChord}\gtab{\color{nameChord} X3V2}{}% 
\color{drawChord}\gtab{\color{nameChord} X4}{}%
\color{drawChord}\gtab{\color{nameChord} X5}{}%
\color{drawChord}\gtab{\color{nameChord} X6}{}% 
%-----------------------------------------------------------------
% PADRÃO: [TonalidadeMaior+NOTAX+Variações] .Ex:[X50] [X57V1V7]
% OBS: Variações são alterações do acorde em relação ao campo harmônico.
%-----------------------------------------------------------------
% Tipos de Variações de Acordes:
% V0 - Variação Diversa
% V1 - Menor (m)
% V2 - Maior (M)
% V3 - Meio Tom Abaixo (Bemol)
% V4 - Com Quarta (ex:C4)
% V5 - Com Quinta (ex:C5)
% V6 - Com Sexta (ex:C6)
% V7 - Com Sétima Menor (ex:C7)
% V8 - Com baixo dois Tons Acima (ex:D/F#)
% V9 - Com Nona (ex:C9)
% V10 - Meio Tom Acima (Sustenido)
% V11 - Com Sétima Maior (ex:C7M)
% V12 - Suspenso (Sus)
% V13 - Com baixo dois Tons e Meio Acima (ex:A/E)
% V14 - Com baixo um Tom e Meio Acima (ex:D9/F) 
% V15 - Meio-Diminuto (m7b5)
% N15 - NÃO Meio-Diminuto
% V16 - Diminuto (º)
% N16 - NÃO Diminuto
%=================================================================
\endsong
%=================================================================
\begin{comment}

\end{comment}