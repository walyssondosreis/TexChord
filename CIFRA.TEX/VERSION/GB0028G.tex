%=================================================================
\songcolumns{2}
\beginsong
{Diante de Ti %TÍTULO
}[by={Quatro Por Um %ARTISTA
},album={@walyssondosreis},
id={GB0028 %COD.ID.: GB0000
}] 
%-----------------------------------------------------------------
\tom{G}
%=================================================================
%\newchords{verse1.GB0000X} % Registrador de Acordes em Sequência
%\newchords{chorus1.GB0000X} % Registrador de Acordes em Sequência
%-----------------------------------------------------------------
\seq{Intro}{G C}{2x}
%-----------------------------------------------------------------
%\beginverse* \endverse
%\beginchorus \endchorus
\beginverse*
\[G]Vem, Se\[C]nhor, encher este lu\[G]gar \[C]
\[G]Vem, Se\[C]nhor, encher este lu\[G]gar \[C]
Com tua \[D]gló...ria, com tua \[C]gló...\[G]ria
Com tua \[D]gló...ria, \[C]com tua \[G]glória \[G]\[C]
\endverse
\beginverse*
^Fala-^me, eu quero te ^ouvir ^
^Toca-^me, eu quero te ^sentir ^
Vem e a^braça-me, vem e a^braça-^me
Vem e a^braça-me, ^vem e a^braça-me \[(G)]
\endverse
\beginverse*
\[D]Todo dia é \[Em]dia de adorar ao \[C]Se\[G]nhor
\[D]Eu conto os se\[Em]gundos só pra te en\[C]con\[G]trar
Quando es\[Am]tou em \[Bm]tua pre\[C]sença \[(C)]
\endverse
\beginchorus
Dá vontade de pu\[G]la...\[D]ar, dá vontade de dan\[Em]ça...\[C]ar
Dá vontade de gri\[G]ta...\[D]ar, dá vontade de co\[Em]rre...\[C]er
Diante de \[G]ti
Dá vontade de pu\[D]lar, dá vontade de dan\[Em]çar
Dá vontade de gri\[C]tar, dá vontade de co\[G]rrer
Dá vontade de pu\[D]lar, dá vontade de dan\[Em]çar
Hey \[C]yeah
\endchorus
\beginverse*
Na tua pre\[G]sença, Senhor
Dá von\[D]tade de correr, de sal\[Em]tar de alegria
De \[C]te conhecer, Senhor
\[G] De erguer minhas \[D]mãos
E te ado\[Em]rar, e te ado\[C]rar, Senhor
Da von\[G]tade
Dá vontade de pu\[D]lar, dá vontade de dan\[Em]çar
Dá vontade de gri\[C]tar, dá vontade de co\[G]rrer
Dá vontade de pu\[D]lar, dá vontade de dan\[Em]çar \[C]
Diante de \[G]ti
\endverse
%-----------------------------------------------------------------
\begin{comment}
\lstset{basicstyle=\scriptsize\bf} % Parâmetros da TAB
%-----------------------------------------------------------------
\tab{Solo 1}
\begin{lstlisting}
E|-----------------------------------------------------|
B|-----------------------------------------------------|
G|-----------------------------------------------------|
D|-----------------------------------------------------|
A|-----------------------------------------------------|
E|-----------------------------------------------------|
\end{lstlisting}
%-----------------------------------------------------------------
\end{comment}
%=================================================================
\vspace{2em} 
%-----------------------------------------------------------------
\gtab{\color{black} G}{}% G  
\gtab{\color{black} Am}{}% Am
\gtab{\color{black} Bm}{}% Bm
\gtab{\color{black} C}{}% C
\gtab{\color{black} D}{}% D 
\gtab{\color{black} Em}{}% Em
%-----------------------------------------------------------------
% PADRÃO: [TonalidadeMaior+NOTAX+Variações] .Ex:[X50] [X57V1V7]
% OBS: Variações são alterações do acorde em relação ao campo harmônico.
%-----------------------------------------------------------------
% Tipos de Variações de Acordes:
% V0 - Variação Diversa
% V1 - Menor (m)
% V2 - Maior (M)
% V3 - Meio Tom Abaixo (Bemol)
% V4 - Com Quarta (ex:C4)
% V5 - Com Quinta (ex:C5)
% V6 - Com Sexta (ex:C6)
% V7 - Com Sétima Menor (ex:C7)
% V8 - Com baixo dois Tons Acima (ex:D/F#)
% V9 - Com Nona (ex:C9)
% V10 - Meio Tom Acima (Sustenido)
% V11 - Com Sétima Maior (ex:C7M)
% V12 - Suspenso (Sus)
% V13 - Com baixo dois Tons e Meio Acima (ex:A/E)
% V14 - Com baixo um Tom e Meio Acima (ex:D9/F) 
% V15 - Meio-Diminuto (m7b5)
% N15 - NÃO Meio-Diminuto
% V16 - Diminuto (º)
% N16 - NÃO Diminuto
%=================================================================
\endsong
%=================================================================
\begin{comment}

\end{comment}