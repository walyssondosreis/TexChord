%=================================================================
\songcolumns{1}
\beginsong
{Atos 2 %TÍTULO
}[by={Gabriela Rocha %ARTISTA
},album={@walyssondosreis},
id={GB0012 %COD.ID.: GB0000
}] 
%-----------------------------------------------------------------
\tom{X1}
%=================================================================
%\newchords{verse1.GB0000X} % Registrador de Acordes em Sequência
\newchords{chorus1.GB0012X} % Registrador de Acordes em Sequência
%-----------------------------------------------------------------
\seq{Intro}{X6V7 X4V9 X1 X3V7}{2x}
%-----------------------------------------------------------------
%\beginverse* \endverse
%\beginchorus \endchorus
\beginverse*
Nós es\[X6V7]tamos aqui
Tão se\[X4V9]dentos de Ti
Vem oh \[X1]Deus, vem oh \[X3V7]Deus
Enche \[X6V7]este lugar
Meu de\[X4V9]sejo é sentir
Teu po\[X1]der, Teu po\[X3V7]der
\endverse

\seq{Riff Intro}{X6V7 X4V9 X1 X3V7}{}

\beginverse*
Nós es\[X6V7]tamos aqui
Tão se\[X4V9]dentos de Ti
Vem oh \[X1]Deus, vem oh \[X3V7]Deus
Enche \[X6V7]este lugar
Meu de\[X4V9]sejo é sentir
Teu po\[X1]der, Teu po\[X5V9]der
\endverse

\beginchorus\memorize[chorus1.GB0012X]
Então \[X4V9]vem me incendi\[X6V7]ar
Meu cora\[X1]ção é \[X4V9]Teu al\[X5V9]tar
Quero ou\[X4V9]vir o som do \[X6V7]céu
Tua \[X1]Glória \[X4V9]contem\[X5V9]plar
\endchorus
\beginverse*\replay[chorus1.GB0012X]
Te damos ^honra
Te damos ^glória
Teu é o ^poder
Pra ^sempre, a^mém
\endverse

%-----------------------------------------------------------------
\begin{comment}
\lstset{basicstyle=\scriptsize\bf} % Parâmetros da TAB
%-----------------------------------------------------------------
\tab{Solo 1}
\begin{lstlisting}
E|-----------------------------------------------------|
B|-----------------------------------------------------|
G|-----------------------------------------------------|
D|-----------------------------------------------------|
A|-----------------------------------------------------|
E|-----------------------------------------------------|
\end{lstlisting}
%-----------------------------------------------------------------
\end{comment}
%=================================================================
\vspace{2em}
%-----------------------------------------------------------------
\gtab{\color{black} X1}{}% G [X1]
\gtab{\color{black} X3V7}{}% Bm7 [X3V7]
\gtab{\color{black} X4V9}{}% C9 [X4V9]
\gtab{\color{black} X5V9}{}% D9 [X5V9]
\gtab{\color{black} X6V7}{}% Em7 [X6V7]
%-----------------------------------------------------------------
% PADRÃO [TonalidadeMaiorNOTAX.Variação] .Ex:[X50] [X50V1]
% PADRÃO [TonalidadeMenorNOTAX.Variação] .Ex:[mX50] [mX50V1]
% OBS: Variações são alterações do acorde em relação ao campo harmônico.
%-----------------------------------------------------------------
% TIPOS DE VARIAÇÂO DOS ACORDES:
% V0 - ACORDE COM VARIAÇÃO DIVERSA
% V1 - ACORDE MENOR (m)
% V2 - ACORDE MAIOR (M)
% V3 - ACORDE MEIO TOM ABAIXO (Bemois)
% V4 - ACORDE COM QUARTA (C4)
% V5 - ACORDE COM QUINTA (C5)
% V6 - ACORDE COM SEXTA (C6)
% V7 - ACORDE COM SÉTIMA MENOR (C7)
% V8 - ACORDE COM BAIXO DOIS TONS ACIMA (D/F#)
% V9 - ACORDE COM NONA (C9)
% V10 - ACORDE MEIO TOM ACIMA (Sustenidos)
% V11 - ACORDE COM SÉTIMA MAIOR (C7M)
% V12 - ACORDE SUSPENSO (Sus)
% V13 - ACORDE COM BAIXO DOIS TONS E MEIO ACIMA (A/E)
% V14 - ACORDE UM TOM E MEIO ACIMA (D9/F)
%=================================================================
\endsong
%=================================================================
\begin{comment}

\end{comment}