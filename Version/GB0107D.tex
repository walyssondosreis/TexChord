%=================================================================
\songcolumns{2}
\beginsong
{Se Não For Pra Te Adorar %TÍTULO
}[by={Fernandinho %ARTISTA
},album={@walyssondosreis},
id={GB0107 %COD.ID.: GB0000
},rev={1}, %REVISÃO
qr={ %LINK
}]
%-----------------------------------------------------------------
\tom{D}{D}
%=================================================================
%\newchords{verse0.GB0000} % Registrador de Acordes em Sequência
%-----------------------------------------------------------------
%\seq{Intro}{}{}
%-----------------------------------------------------------------
%\beginverse \endverse
%\beginchorus \endchorus

\beginverse
\[D]Se não for pra te adorar 
Para que eu nas\[G]ci?
Se não for pra te ser\[Bm7]vir
Porque eu estou a\[A9]qui?
Sim eu quero te ado\[G]rar, te ado\[A9]rar
Senhor, estou a\[D]qui, \[A9]
Senhor, estou a\[Bm]qui, \[A]
Senhor, estou a\[D]qui
\endverse

\seq{Riff}{D G Bm7 A9}{2x}

\beginverse 
Di\[D]ante do trono, Se\[G]nhor 
Quero le\[Bm7]var minha oferta de a\[A9]mor
Di\[Bm7]ante do trono, Se\[F\#m]nhor 
Quero le\[G]var meu sacrifício de lou\[A9]vor
\[D]As minhas mãos levan\[G]tar
Tua be\[Bm7]leza então contem\[A9]plar
\[Bm7]Com meus lábios decla\[F\#m]rar 
Toda a \[G]minha a adora\[A9]ção
\endverse

\beginchorus 
Se não for pra te ado\[D]rar
Para que eu nas\[G]ci?
Se não for pra te ser\[Bm7]vir
Porque eu estou a\[A9]qui?
Sim eu quero te ado\[G]rar, te ado\[A9]rar
Se não for pra te ado\[D]rar
Para que eu nas\[G]ci?
Se não for pra te ser\[Bm7]vir
Porque eu estou a\[A9]qui?
Sim eu quero te ado\[G]rar, te ado\[A9]rar
Senhor, estou a\[D]qui
\endchorus

\seq{Riff}{D G Bm7 A9}{}
\act{Retomar}{Verso 2}{1x}
\seq{Riff 2}{D G Bm7 A9 }{}
\beginverse 
\[D]Êô, êô, \[G]êôô
\[Bm7]Êô, êô, \[A9]êôô
\[D]Êô, êô, \[G]êôô
\[Bm7]Êô, êô, \[A9]êôô
\endverse
%-----------------------------------------------------------------
\begin{comment}
\lstset{basicstyle=\scriptsize\bf} % Parâmetros da TAB
%-----------------------------------------------------------------
\tab{Solo 1}
\begin{lstlisting}
E|-----------------------------------------------------|
B|-----------------------------------------------------|
G|-----------------------------------------------------|
D|-----------------------------------------------------|
A|-----------------------------------------------------|
E|-----------------------------------------------------|
\end{lstlisting}
%-----------------------------------------------------------------
\end{comment}
%=================================================================
\vspace{2em}
%-----------------------------------------------------------------
\gtab{\color{black} D}{}% D [D]
\gtab{\color{black} F\#m}{}% F#m [F\#m]
\gtab{\color{black} G}{}% G [G]
\gtab{\color{black} A}{}% A [A]
\gtab{\color{black} A9}{}% A9 [A9]
\gtab{\color{black} Bm}{}\\% Bm [Bm]
\gtab{\color{black} Bm7}{}% Bm7 [Bm7]
%-----------------------------------------------------------------
% PADRÃO [TonalidadeMaiorNOTAX.Variação] .Ex:[X50] [X50V1]
% PADRÃO [TonalidadeMenorNOTAX.Variação] .Ex:[mX50] [mX50V1]
% OBS: Variações são alterações do acorde em relação ao campo harmônico.
%-----------------------------------------------------------------
% TIPOS DE VARIAÇÂO DOS ACORDES:
% V0 - ACORDE COM VARIAÇÃO DIVERSA
% V1 - ACORDE MENOR (m)
% V2 - ACORDE MAIOR (M)
% V3 - ACORDE MEIO TOM ABAIXO (Bemois)
% V4 - ACORDE COM QUARTA (C4)
% V5 - ACORDE COM QUINTA (C5)
% V6 - ACORDE COM SEXTA (C6)
% V7 - ACORDE COM SÉTIMA MENOR (C7)
% V8 - ACORDE COM BAIXO DOIS TONS ACIMA (D/F#)
% V9 - ACORDE COM NONA (C9)
% V10 - ACORDE MEIO TOM ACIMA (Sustenidos)
% V11 - ACORDE COM SÉTIMA MAIOR (C7M)
% V12 - ACORDE SUSPENSO (Sus)
% V13 - ACORDE COM BAIXO DOIS TONS E MEIO ACIMA (A/E)
% V14 - ACORDE UM TOM E MEIO ACIMA (D9/F)
%=================================================================
\endsong
%=================================================================







