%=================================================================
\songcolumns{2}
\beginsong
{Nunca Me Deixou %TÍTULO
}[by={Livres Para Adorar %ARTISTA
},album={@walyssondosreis},
id={GB0072 %COD.ID.: GB0000
},rev={0}, %REVISÃO
qr={https://drive.google.com/open?id=1uxawhAEnULrf2xT0_LDE0FHNFGZw6GHh %LINK
}]
%-----------------------------------------------------------------
\tom{X1}{B}
%=================================================================
%\newchords{verse1.GB0000X} % Registrador de Acordes em Sequência
%\newchords{chorus1.GB0000X} % Registrador de Acordes em Sequência
%-----------------------------------------------------------------
%\seq{Intro}{}{}
%-----------------------------------------------------------------
%\beginverse* \endverse
%\beginchorus \endchorus
\beginverse*
Ainda que eu ande
Pelo vale da sombra da morte
O Teu amor lança fora o medo
Ainda que eu me encontre
Bem no meio das tempestades da vida
Não voltarei, pois perto estás
\endverse
\beginverse*
Eu não temerei o mal
Pois o meu Deus comigo está
E se o meu Deus comigo está
A quem eu temerei? A quem eu temerei?
\endverse
\beginchorus
Oh, não! Nunca me deixou
Na tempestade ou na paz
Oh, não! Nunca me deixou
Quando bem ou quando mal
Oh, não! Nunca me deixou
Oh, o Senhor nunca me deixou
\endchorus
\beginverse*
Eu posso ver a luz
Que está vindo ao coração que espera
Incomparável, gloriosa luz
E haverá um fim aos problemas
Mas até este dia
Viverei sabendo que estás aqui
\endverse
\beginverse*
Eu posso ver a luz
Que está vindo ao coração que espera
E haverá um fim aos problemas
Mas até este dia
Te louvarei, Te louvarei
\endverse
%-----------------------------------------------------------------
\vspace{4em} % Regulador de Espaçamento
%-----------------------------------------------------------------
\begin{comment}
\lstset{basicstyle=\scriptsize\bf} % Parâmetros da TAB
%-----------------------------------------------------------------
\tab{Solo 1}
\begin{lstlisting}
E|-----------------------------------------------------|
B|-----------------------------------------------------|
G|-----------------------------------------------------|
D|-----------------------------------------------------|
A|-----------------------------------------------------|
E|-----------------------------------------------------|
\end{lstlisting}
%-----------------------------------------------------------------
\end{comment}
%=================================================================
 
%-----------------------------------------------------------------
\color{drawChord}\gtab{\color{nameChord} X}{}% 
\color{drawChord}\gtab{\color{nameChord} X}{}% 
\color{drawChord}\gtab{\color{nameChord} X}{}% 
\color{drawChord}\gtab{\color{nameChord} X}{}% 
%-----------------------------------------------------------------
% PADRÃO: [TonalidadeMaior+NOTAX+Variações] .Ex:[X50] [X57V1V7]
% OBS: Variações são alterações do acorde em relação ao campo harmônico.
%-----------------------------------------------------------------
% Tipos de Variações de Acordes:
% V0 - Variação Diversa
% V1 - Menor (m)
% V2 - Maior (M)
% V3 - Meio Tom Abaixo (Bemol)
% V4 - Com Quarta (ex:C4)
% V5 - Com Quinta (ex:C5)
% V6 - Com Sexta (ex:C6)
% V7 - Com Sétima Menor (ex:C7)
% V8 - Com baixo dois Tons Acima (ex:D/F#)
% V9 - Com Nona (ex:C9)
% V10 - Meio Tom Acima (Sustenido)
% V11 - Com Sétima Maior (ex:C7M)
% V12 - Suspenso (Sus)
% V13 - Com baixo dois Tons e Meio Acima (ex:A/E)
% V14 - Com baixo um Tom e Meio Acima (ex:D9/F) 
% V15 - Meio-Diminuto (m7b5)
% N15 - NÃO Meio-Diminuto
% V16 - Diminuto (º)
% N16 - NÃO Diminuto
%=================================================================
\endsong
%=================================================================
\begin{comment}

\end{comment}