%=================================================================
\songcolumns{2}
\beginsong
{Nos Bares da Cidade %TÍTULO
}[by={Rick e Renner %ARTISTA
},album={@walyssondosreis},
id={\href{ https://music.youtube.com/watch?v=c6wJ0Mk1ZvY&feature=share
}{ ST0130  %COD.ID.: XXNNNN
}},rev={5}, %REVISÃO
qr={ https://music.youtube.com/watch?v=c6wJ0Mk1ZvY&feature=share %LINK
}]
%-----------------------------------------------------------------
\tom{X1}{D}
%=================================================================
%\newchords{verse1.XX0000X} % Registrador de Acordes em Sequência
%\newchords{chorus1.XX0000X} % Registrador de Acordes em Sequência
%-----------------------------------------------------------------
\seq{Intro}{X1 X1V4 X1V4}{2x}
%\act{}{}{}
%-----------------------------------------------------------------
%\beginverse \endverse
%\beginchorus \endchorus
\beginverse
Gar\[X1]çom, me traga 
Outra ga\[X1V4]rrafa de cer\[X1]veja
Vou ficar so\[X1V4]zinho nessa \[X1]mesa
Eu quero beber e chorar por \[X5]ela \[X5V4]
\endverse
\beginverse
Gar\[X4]çom, a minha vida agora
Tá de ponta ca\[X1]beça
Já tentei mas nada 
Faz com que eu es\[X5]queça
Os olhos e os \[X4]lábios
Daquela \[X1]mulher \[X4]\[X5]\[X4]
\endverse
\beginverse
Gar\[X1]çom, ela saiu de \[X1V4]vez
Da minha \[X1]vida
E agora eu \[X1V4]busco uma sa\[X1]ída
Minha história de a\[X1V7]mor 
Acaba em \[X4]solidão
Garçom, se eu ficar muito chato
E der algum ve\[X1]xame
Pegue toda a minha cerveja e de\[X5]rrame
Faça o que ela \[X4]fez
Com a minha pai\[X1]xão \[X5V7]
\endverse
\beginchorus
De\[X1]rrama cerveja, derrama
Derrama a tris\[X5]teza do meu coração
Que essa angustia é uma be\[X5V4]bida
Misturada, ba\[X1]tida
Com a soli\[X5V7]dão
De\[X1]rrama cerveja, derrama
Enquanto eu de\[X1V7]rramo
Toda essa sau\[X4]dade
Eu sou apenas um qual\[X1]quer
Bebendo por mu\[X5]lher
Nos bares da ci\[X1]dade
\endchorus
\act{Riff Intro}{Repetir}{1x}
\act{Verso 1}{Retomar}{1x}
\act{Refrão}{Repetir}{+1x}
\beginverse
... Eu sou apenas um qual\[X1]quer
Bebendo por mu\[X5]lher
Nos bares da ci\[X1]dade
\endverse
\act{Verso 4}{Repetir}{+1x}
%-----------------------------------------------------------------
\vspace{4em} % Regulador de Espaçamento
%-----------------------------------------------------------------
\begin{comment}
\lstset{basicstyle=\scriptsize\bf} % Parâmetros da TAB
%-----------------------------------------------------------------
\tab{Solo 1}
\begin{lstlisting}
E|-----------------------------------------------------|
B|-----------------------------------------------------|
G|-----------------------------------------------------|
D|-----------------------------------------------------|
A|-----------------------------------------------------|
E|-----------------------------------------------------|
\end{lstlisting}
%-----------------------------------------------------------------
\end{comment}
%=================================================================

\color{drawChord}\gtab{\color{nameChord} X1}{}% 
\color{drawChord}\gtab{\color{nameChord} X1V4}{}% 
\color{drawChord}\gtab{\color{nameChord} X4}{}% 
\color{drawChord}\gtab{\color{nameChord} X5}{}%
\color{drawChord}\gtab{\color{nameChord} X5V4}{}%
\color{drawChord}\gtab{\color{nameChord} X5V7}{}% 

%=================================================================
% PADRÃO: [TonalidadeMaior+NOTAX+Variações] .Ex:[X50] [X57V1V7]
% OBS: Variações são alterações do acorde em relação ao campo harmônico.
%-----------------------------------------------------------------
% Tipos de Variações de Acordes:
% V0 - Variação Diversa
% V1 - Menor (m)
% V2 - Maior (M)
% V3 - Meio Tom Abaixo (Bemol)
% V4 - Com Quarta (ex:C4)
% V5 - Com Quinta (ex:C5)
% V6 - Com Sexta (ex:C6)
% V7 - Com Sétima Menor (ex:C7)
% V8 - Com baixo dois Tons Acima (ex:D/F#)
% V9 - Com Nona (ex:C9)
% V10 - Meio Tom Acima (Sustenido)
% V11 - Com Sétima Maior (ex:C7M)
% V12 - Suspenso (Sus)
% V13 - Com baixo dois Tons e Meio Acima (ex:A/E)
% V14 - Com baixo um Tom e Meio Acima (ex:D9/F) 
% V15 - Meio-Diminuto (m7b5)
% N15 - NÃO Meio-Diminuto
% V16 - Diminuto (º)
% N16 - NÃO Diminuto
% V17 - Com baixo um Tom Acima (ex: C/D)
% V18 - Com baixo um Tom Abaixo (ex: Em/D)
% V19 - Com baixo dois Tons e meio Abaixo (ex: G/D)
%=================================================================
%\vspace{4em} % Regulador de Espaçamento
%=================================================================
\endsong
%=================================================================
