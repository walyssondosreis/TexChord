%=================================================================
\songcolumns{2}
\beginsong
{Em Espírito, Em Verdade\\Meu prazer %TÍTULO
}[by={Ministério Koinonya %ARTISTA
},album={@walyssondosreis},
id={GB0032 %COD.ID.: GB0000
},rev={3}, %REVISÃO
qr={https://drive.google.com/open?id=1Zow-pYsWJrr4sj6kBDEsyMh52slxbxx6 %LINK
}]
%-----------------------------------------------------------------
\tom{E}{A}
%=================================================================
%\newchords{verse0.GB0000} % Registrador de Acordes em Sequência
%-----------------------------------------------------------------
\seq{Intro}{}{2x}
%-----------------------------------------------------------------
%\beginverse* \endverse
%\beginchorus \endchorus

\beginverse
\[E]Em espírito, em ver\[B]dade
Te ado\[A]ramos,  te ado\[E]ramos \[B]
\[E]Em espírito, em ver\[B]dade
Te ado\[A]ramos,  te ado\[E]ramos \[B]
\endverse

\beginverse
Rei dos \[C\#m]reis  e se\[E]nhor 
Te entre\[A]gamos \[B]nosso vi\[E]ver \[B]
Rei dos \[C\#m]reis  e se\[E]nhor 
Te entre\[A]gamos \[F\#m]nosso vi\[B]ver \[B4]
\endverse

\beginchorus 
Pra te ado\[A]rar oh! \[F\#m]rei dos \[B]reis \[B4]
Foi que eu nas\[A]ci oh! \[F\#m]rei Je\[B]sus!
Meu pra\[G\#7]zer é te lou\[C\#m]var
Meu pra\[B]zer é estar\[A]
Nos \[B]átrios do se\[E]nhor
Meu pra\[B]zer é vi\[C\#m]ver
Na \[B]casa de \[A]Deus
Onde \[B]flui o a\[E]mor \[(B)]
\endchorus
\act{Executar}{Solo 1}{}
\act{Retomar}{Verso 1}{1x}
\act{Repetir}{Refrão}{1x}
\beginverse
... Onde \[B]flui o a\[E]mor \[(B)]

\endverse
\act{Repetir}{Verso 3}{+4x}
%-----------------------------------------------------------------
\vspace{4em} % Regulador de Espaçamento
%-----------------------------------------------------------------
\begin{comment}
\lstset{basicstyle=\scriptsize\bf} % Parâmetros da TAB
%-----------------------------------------------------------------
\tab{Solo 1}
\begin{lstlisting}
E|-----------------------------------------------------|
B|-----------------------------------------------------|
B|-----------------------------------------------------|
D|-----------------------------------------------------|
A|-----------------------------------------------------|
E|-----------------------------------------------------|
\end{lstlisting}
%-----------------------------------------------------------------
\end{comment}
%=================================================================

%-----------------------------------------------------------------
\color{drawChord}\gtab{\color{nameChord} E}{}% C [E]
\color{drawChord}\gtab{\color{nameChord} F\#m}{}% Dm [F\#m]
\color{drawChord}\gtab{\color{nameChord} G\#7}{}% E7 [G\#7]
\color{drawChord}\gtab{\color{nameChord} A}{}% F [A]
\color{drawChord}\gtab{\color{nameChord} B}{}% G [B]
\color{drawChord}\gtab{\color{nameChord} B4}{}\\% G4 [B4]
\color{drawChord}\gtab{\color{nameChord} C\#m}{}% Am [C\#m]
%-----------------------------------------------------------------
% PADRÃO [TonalidadeMaiorNOTAX.Variação] .Ex:[X50] [X50V1]
% PADRÃO [TonalidadeMenorNOTAX.Variação] .Ex:[mX50] [mX50V1]
% OBS: Variações são alterações do acorde em relação ao campo harmônico.
%-----------------------------------------------------------------
% TIPOS DE VARIAÇÂO DOS ACORDES:
% V0 - ACORDE COM VARIAÇÃO DIVERSA
% V1 - ACORDE MENOR (m)
% V2 - ACORDE MAIOR (M)
% V3 - ACORDE MEIO TOM ABAIXO (Bemois)
% V4 - ACORDE COM QUARTA (C4)
% V5 - ACORDE COM QUINTA (C5)
% V6 - ACORDE COM SEXTA (C6)
% V7 - ACORDE COM SÉTIMA MENOR (C7)
% V8 - ACORDE COM BAIXO DOIS TONS ACIMA (D/F#)
% V9 - ACORDE COM NONA (C9)
% V10 - ACORDE MEIO TOM ACIMA (Sustenidos)
% V11 - ACORDE COM SÉTIMA MAIOR (C7M)
% V12 - ACORDE SUSPENSO (Sus)
% V13 - ACORDE COM BAIXO DOIS TONS E MEIO ACIMA (A/E)
% V14 - ACORDE UM TOM E MEIO ACIMA (D9/F)
%=================================================================
\endsong
%=================================================================

\begin{comment}

\end{comment}