%=================================================================
\songcolumns{1}
\beginsong
{Vem, Está É a Hora %TÍTULO
}[by={Vineyard Brasil %ARTISTA
},album={@walyssondosreis},
id={GB0125 %COD.ID.: GB0000
},rev={3}, %REVISÃO
qr={https://drive.google.com/open?id=1gTwKDfVqIa5NdLAEQpuTGMvYVRbtxHw- %LINK
}]
%-----------------------------------------------------------------
\tom{X1}{D}
%=================================================================
%\newchords{verse1.GB0000X} % Registrador de Acordes em Sequência
%\newchords{chorus1.GB0000X} % Registrador de Acordes em Sequência
%-----------------------------------------------------------------
\seq{Intro}{}{}
%-----------------------------------------------------------------
%\beginverse* \endverse
%\beginchorus \endchorus
\beginverse
\[X1]Vem, \[X1V9]esta é a hora da a\[X1V4]dora\[X1]ção
\[X5]Vem, \[X5V9]dar a Ele teu \[X2]co\[X1V8]ra\[X4]ção
\[X1]Vem, a\[X1V9]ssim como estás para \[X1V4]ado\[X1]rar
\[X5]Vem, \[X5V9]assim como estás dian\[X2]te \[X1V8]do \[X4]Pai
\[X1]Vem
\endverse
\beginchorus
\[X4]Toda língua confessar\[X1]á ao Senhor
\[X4]Todo joelho se dobra\[X1]rá
\[X4]Mas aquele que a Ti \[X6]escolher
O te\[X2]souro maior te\[X5V4]rá \[X5]
\endchorus
\act{Retomar}{Verso 1}{1x}
\act{Executar}{Solo 1}{}
\act{Repetir}{Refrão}{2x}
\beginverse
\[X1]Vem, \[X1V9]esta é a hora da a\[X1V4]dora\[X1]ção
\[X5]Vem, \[X5V9]dar a Ele teu \[X2]co\[X1V8]ra\[X4]ção
\[X1]Vem, a\[X1V9]ssim como estás
Vem
Para adorar
Vem
Vem
Vem
\endverse
%-----------------------------------------------------------------
\begin{comment}
\lstset{basicstyle=\scriptsize\bf} % Parâmetros da TAB
%-----------------------------------------------------------------
\tab{Solo 1}
\begin{lstlisting}
E|-----------------------------------------------------|
B|-----------------------------------------------------|
G|-----------------------------------------------------|
D|-----------------------------------------------------|
A|-----------------------------------------------------|
E|-----------------------------------------------------|
\end{lstlisting}
%-----------------------------------------------------------------
\end{comment}
%=================================================================
\vspace{2em} 
%-----------------------------------------------------------------
\gtab{\color{black} X1}{}% 
\gtab{\color{black} X1V4}{}%
\gtab{\color{black} X1V8}{}% 
\gtab{\color{black} X1V9}{}% 
\gtab{\color{black} X2}{}% 
\gtab{\color{black} X4}{}%
\gtab{\color{black} X5}{}% 
\gtab{\color{black} X5V4}{}% 
\gtab{\color{black} X5V9}{}%
\gtab{\color{black} X6}{}% 
%-----------------------------------------------------------------
% PADRÃO: [TonalidadeMaior+NOTAX+Variações] .Ex:[X50] [X57V1V7]
% OBS: Variações são alterações do acorde em relação ao campo harmônico.
%-----------------------------------------------------------------
% Tipos de Variações de Acordes:
% V0 - Variação Diversa
% V1 - Menor (m)
% V2 - Maior (M)
% V3 - Meio Tom Abaixo (Bemol)
% V4 - Com Quarta (ex:C4)
% V5 - Com Quinta (ex:C5)
% V6 - Com Sexta (ex:C6)
% V7 - Com Sétima Menor (ex:C7)
% V8 - Com baixo dois Tons Acima (ex:D/F#)
% V9 - Com Nona (ex:C9)
% V10 - Meio Tom Acima (Sustenido)
% V11 - Com Sétima Maior (ex:C7M)
% V12 - Suspenso (Sus)
% V13 - Com baixo dois Tons e Meio Acima (ex:A/E)
% V14 - Com baixo um Tom e Meio Acima (ex:D9/F) 
% V15 - Meio-Diminuto (m7b5)
% N15 - NÃO Meio-Diminuto
% V16 - Diminuto (º)
% N16 - NÃO Diminuto
%=================================================================
\endsong
%=================================================================
\begin{comment}

\end{comment}