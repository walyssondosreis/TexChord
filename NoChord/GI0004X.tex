%=================================================================
\songcolumns{2}
\beginsong
{Oceans\\Where Feet May Fail %TÍTULO
}[by={Hillsong United  %ARTISTA
},album={@walyssondosreis},
id={GI0004 %COD.ID.: XX0000
},rev={0}, %REVISÃO
qr={ %LINK
}]
%-----------------------------------------------------------------
\tom{X1}{X1}
%=================================================================
%\newchords{verse1.XX0000X} % Registrador de Acordes em Sequência
%\newchords{chorus1.XX0000X} % Registrador de Acordes em Sequência
%-----------------------------------------------------------------
%\seq{Intro}{}{}
%\act{}{}{}
%-----------------------------------------------------------------
%\beginverse \endverse
%\beginchorus \endchorus
\beginverse
You call me out upon the waters
The great unknown where feet may fail
And there I find You in the mystery
In oceans deep, my faith will stand
\endverse
\beginchorus
And I will call upon Your name
And keep my eyes above the waves
When oceans rise, my soul will rest in Your embrace
I am Yours and You are mine
\endchorus
\beginverse
Your grace abounds in deepest waters
Your sovereign hand
Will be my guide
Where feet may fail and fear surrounds me
You've never failed and You won't start now
\endverse
\beginchorus
So I will call upon Your name
And keep my eyes above the waves
When oceans rise, my soul will rest in Your embrace
For I am Yours and You are mine, oh
And You are mine, oh
\endchorus
\beginverse
Spirit lead me where my trust is without borders
Let me walk upon the waters
Wherever You would call me
Take me deeper than my feet could ever wander
And my faith will be made stronger
In the presence of my Savior
\endverse
\beginchorus
I will call upon Your name
Keep my eyes above the waves
My soul will rest in Your embrace
I am Yours and You are mine
\endchorus

%-----------------------------------------------------------------
\begin{comment}
\lstset{basicstyle=\scriptsize\bf} % Parâmetros da TAB
%-----------------------------------------------------------------
\tab{Solo 1}
\begin{lstlisting}
E|-----------------------------------------------------|
B|-----------------------------------------------------|
G|-----------------------------------------------------|
D|-----------------------------------------------------|
A|-----------------------------------------------------|
E|-----------------------------------------------------|
\end{lstlisting}
%-----------------------------------------------------------------
\end{comment}
%=================================================================
\vspace{2em} 
%-----------------------------------------------------------------
\color{drawChord}\gtab{\color{nameChord} X}{}% 
\color{drawChord}\gtab{\color{nameChord} X}{}% 
\color{drawChord}\gtab{\color{nameChord} X}{}% 
\color{drawChord}\gtab{\color{nameChord} X}{}% 
%-----------------------------------------------------------------
% PADRÃO: [TonalidadeMaior+NOTAX+Variações] .Ex:[X50] [X57V1V7]
% OBS: Variações são alterações do acorde em relação ao campo harmônico.
%-----------------------------------------------------------------
% Tipos de Variações de Acordes:
% V0 - Variação Diversa
% V1 - Menor (m)
% V2 - Maior (M)
% V3 - Meio Tom Abaixo (Bemol)
% V4 - Com Quarta (ex:C4)
% V5 - Com Quinta (ex:C5)
% V6 - Com Sexta (ex:C6)
% V7 - Com Sétima Menor (ex:C7)
% V8 - Com baixo dois Tons Acima (ex:D/F#)
% V9 - Com Nona (ex:C9)
% V10 - Meio Tom Acima (Sustenido)
% V11 - Com Sétima Maior (ex:C7M)
% V12 - Suspenso (Sus)
% V13 - Com baixo dois Tons e Meio Acima (ex:A/E)
% V14 - Com baixo um Tom e Meio Acima (ex:D9/F) 
% V15 - Meio-Diminuto (m7b5)
% N15 - NÃO Meio-Diminuto
% V16 - Diminuto (º)
% N16 - NÃO Diminuto
% V17 - Com baixo um Tom Acima (ex: C/D)
%=================================================================
\endsong
%=================================================================
\begin{comment}

\end{comment}