%=================================================================
\songcolumns{2}
\beginsong
{Águas Purificadoras %TÍTULO
}[by={Diante do Trono %ARTISTA
},album={@walyssondosreis},
id={GB0008 %COD.ID.: GB0000
},rev={3}, %REVISÃO
qr={https://drive.google.com/open?id=1U9G5IfcmhRFIP38JJ_urXNEM9r_2IVON %LINK
}]
%-----------------------------------------------------------------
\tom{X1}{D}
%=================================================================
%\newchords{verse1.GB0000X} % Registrador de Acordes em Sequência
%\newchords{chorus1.GB0000X} % Registrador de Acordes em Sequência
%-----------------------------------------------------------------
\seq{Intro}{X1 X2V18}{}
%-----------------------------------------------------------------
%\beginverse* \endverse
%\beginchorus \endchorus
\beginverse
\[X1]Existe um rio, Se\[X2V18]nhor
\[X1]Que flui do Teu grande \[X2V18]amor \[X1V8]
\[X4]Águas que correm do \[X1V8]Trono
\[X4]Águas que c\[X2]uram, que \[X5V4]lim\[X5]pam \[X4V17]
\endverse
\beginverse
\[X1]Por onde o rio pas\[X2V18]sar
\[X6V7]Tudo vai transfor\[X5V4]mar \[X5]
Pois \[X4]leva a vida do \[X1V8]próprio Deus
E esse \[X4]rio es\[X2V7]tá \[X5V4]neste lu\[X5]gar \[X4V8]\[X5V8]
\endverse
\beginchorus
\[X1]Quero beber do Teu \[X4V19]rio, Se\[X1]nhor \[X1V8]
Sa\[X4]cia a minha \[X1V8]sede, lava o \[X2V7]meu interi\[X5]or
Eu \[X6V7]quero fluir em Tuas \[X4]águas \[X5]
Eu \[X6V7]quero be\[X3V7]ber da Tua \[X4]fonte
Fonte de Águas \[X5V4]Vivas \[X5]
Tu és a \[X4V17]fonte, Se\[X1]nhor \[(X5)]
\endchorus
\act{Executar}{Solo 1}{}
\act{Retomar}{Verso 1}{1x}
\act{Executar}{Solo 2}{}
\act{Repetir}{Refrão}{1x}
\beginverse
...Tu és o rio, Se\[X1]nhor \[X4V19]
Tu és a fonte, Se\[X6]nhor \[X5]
Tu és o rio, Se\[X4]nhor \[X5]
Tu és a fonte, Se\[X1]nhor \[X1V4]\[X1]
\endverse
%-----------------------------------------------------------------
\vspace{4em} % Regulador de Espaçamento
%-----------------------------------------------------------------
\begin{comment}
\lstset{basicstyle=\scriptsize\bf} % Parâmetros da TAB
%-----------------------------------------------------------------
\tab{Solo 1}
\begin{lstlisting}
E|-----------------------------------------------------|
B|-----------------------------------------------------|
G|-----------------------------------------------------|
D|-----------------------------------------------------|
A|-----------------------------------------------------|
E|-----------------------------------------------------|
\end{lstlisting}
%-----------------------------------------------------------------
\end{comment}
%=================================================================
 
%-----------------------------------------------------------------
\color{drawChord}\gtab{\color{nameChord} X1}{}% 
\color{drawChord}\gtab{\color{nameChord} X1V8}{}% 
\color{drawChord}\gtab{\color{nameChord} X2}{}%
\color{drawChord}\gtab{\color{nameChord} X2V7}{}% 
\color{drawChord}\gtab{\color{nameChord} X2V18}{}%
\color{drawChord}\gtab{\color{nameChord} X3V7}{}\\%
\color{drawChord}\gtab{\color{nameChord} X4}{}% 
\color{drawChord}\gtab{\color{nameChord} X4V8}{}% 
\color{drawChord}\gtab{\color{nameChord} X4V17}{}% 
\color{drawChord}\gtab{\color{nameChord} X4V19}{}% 
\color{drawChord}\gtab{\color{nameChord} X5}{}% 
\color{drawChord}\gtab{\color{nameChord} X5V4}{}\\% 
\color{drawChord}\gtab{\color{nameChord} X5V8}{}%
\color{drawChord}\gtab{\color{nameChord} X6}{}% 
\color{drawChord}\gtab{\color{nameChord} X6V7}{}% 
%-----------------------------------------------------------------
% PADRÃO: [TonalidadeMaior+NOTAX+Variações] .Ex:[X50] [X57V1V7]
% OBS: Variações são alterações do acorde em relação ao campo harmônico.
%-----------------------------------------------------------------
% Tipos de Variações de Acordes:
% V0 - Variação Diversa
% V1 - Menor (m)
% V2 - Maior (M)
% V3 - Meio Tom Abaixo (Bemol)
% V4 - Com Quarta (ex:C4)
% V5 - Com Quinta (ex:C5)
% V6 - Com Sexta (ex:C6)
% V7 - Com Sétima Menor (ex:C7)
% V8 - Com baixo dois Tons Acima (ex:D/F#)
% V9 - Com Nona (ex:C9)
% V10 - Meio Tom Acima (Sustenido)
% V11 - Com Sétima Maior (ex:C7M)
% V12 - Suspenso (Sus)
% V13 - Com baixo dois Tons e Meio Acima (ex:A/E)
% V14 - Com baixo um Tom e Meio Acima (ex:D9/F) 
% V15 - Meio-Diminuto (m7b5)
% N15 - NÃO Meio-Diminuto
% V16 - Diminuto (º)
% N16 - NÃO Diminuto
% V17 - Com baixo um Tom Acima (ex: C/D)
% V18 - Com baixo um Tom Abaixo (ex: Em/D)
% V19 - Com baixo dois Tons e meio Abaixo. (ex: G/D)
%=================================================================
\endsong
%=================================================================
\begin{comment}

\end{comment}