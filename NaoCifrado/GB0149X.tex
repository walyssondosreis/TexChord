%=================================================================
\songcolumns{2}
\beginsong
{Aquieta Minh'alma %TÍTULO
}[by={Ministério Zoe %ARTISTA
},album={@walyssondosreis},
id={GB0149 %COD.ID.: GB0000
(Rev.0) %REVISÃO.: 0...N
}]
%-----------------------------------------------------------------
\tom{X1}
%=================================================================
%\newchords{verse1.GB0000X} % Registrador de Acordes em Sequência
%\newchords{chorus1.GB0000X} % Registrador de Acordes em Sequência
%-----------------------------------------------------------------
%\seq{Intro}{}{}
%-----------------------------------------------------------------
%\beginverse* \endverse
%\beginchorus \endchorus
\beginchorus
Aquieta minh'alma
Faz meu coração ouvir Tua voz
Me chama pra perto
Só assim eu não me sinto só
\endchorus
\beginverse*
Porque, na verdade, eu descobri que tudo o que eu preciso está em Ti
Mas meu coração é teimoso demais pra admitir
Sei que depender é como viver perigosamente
Mas eu preciso acreditar e confiar no que Você me diz
\endverse
\beginverse*
Eu sei que, mesmo sem entender, Você está no controle
Então, me esconda no Teu coração
Me amarre a Ti pra eu não desistir
\endverse
\beginverse*
Eu não quero mais fugir da Tua vontade pra mim
Eu sei que vai ser difícil
Mas Você estará sempre comigo
\endverse
\beginverse*
E mesmo que minh'alma grite e tente me fazer voltar atrás
Eu vou confiar, eu vou descansar
Me lançar no Teu amor
No Teu amor, Senhor
\endverse
\beginverse*
O tempo não pode apagar, as muitas águas nunca levarão o amor
Que Você sente por mim, eu sei que tudo vai se cumprir
O tempo não pode apagar, as muitas águas nunca levarão o amor
Que Você sente por mim, eu sei que tudo vai se cumprir
\endverse
\beginverse*
Vai ser difícil, eu sei, largar tudo por Você
Mas eu sei que quando eu pensar em desistir
Você estará ao meu lado
Me segurando, me assegurando de que tudo vai ficar bem
Tudo vai ficar bem
\endverse
\beginverse*
E se eu cair, a Tua mão me levantará
E se eu chorar, toda lágrima Você enxugará
E se eu cair, a Tua mão me levantará
E se eu chorar, toda lágrima Você enxugará
Então vem!
\endverse

%-----------------------------------------------------------------
\begin{comment}
\lstset{basicstyle=\scriptsize\bf} % Parâmetros da TAB
%-----------------------------------------------------------------
\tab{Solo 1}
\begin{lstlisting}
E|-----------------------------------------------------|
B|-----------------------------------------------------|
G|-----------------------------------------------------|
D|-----------------------------------------------------|
A|-----------------------------------------------------|
E|-----------------------------------------------------|
\end{lstlisting}
%-----------------------------------------------------------------
\end{comment}
%=================================================================
\vspace{2em} 
%-----------------------------------------------------------------
\gtab{\color{black} X}{}% 
\gtab{\color{black} X}{}% 
\gtab{\color{black} X}{}% 
\gtab{\color{black} X}{}% 
%-----------------------------------------------------------------
% PADRÃO: [TonalidadeMaior+NOTAX+Variações] .Ex:[X50] [X57V1V7]
% OBS: Variações são alterações do acorde em relação ao campo harmônico.
%-----------------------------------------------------------------
% Tipos de Variações de Acordes:
% V0 - Variação Diversa
% V1 - Menor (m)
% V2 - Maior (M)
% V3 - Meio Tom Abaixo (Bemol)
% V4 - Com Quarta (ex:C4)
% V5 - Com Quinta (ex:C5)
% V6 - Com Sexta (ex:C6)
% V7 - Com Sétima Menor (ex:C7)
% V8 - Com baixo dois Tons Acima (ex:D/F#)
% V9 - Com Nona (ex:C9)
% V10 - Meio Tom Acima (Sustenido)
% V11 - Com Sétima Maior (ex:C7M)
% V12 - Suspenso (Sus)
% V13 - Com baixo dois Tons e Meio Acima (ex:A/E)
% V14 - Com baixo um Tom e Meio Acima (ex:D9/F) 
% V15 - Meio-Diminuto (m7b5)
% N15 - NÃO Meio-Diminuto
% V16 - Diminuto (º)
% N16 - NÃO Diminuto
%=================================================================
\endsong
%=================================================================
\begin{comment}

\end{comment}