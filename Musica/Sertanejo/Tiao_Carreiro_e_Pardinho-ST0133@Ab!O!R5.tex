%=================================================================
\songcolumns{1}
\beginsong
{O Prato do Dia %TÍTULO
}[by={Tião Carreiro e Pardinho %ARTISTA
},album={@walyssondosreis},
id={\href{https://music.youtube.com/watch?v=Jt-H8QO2HFo&feature=share %LINK
}{ST0133 %COD.ID.: XXNNNN
}},rev={5}, %REVISÃO
qr={https://music.youtube.com/watch?v=Jt-H8QO2HFo&feature=share %LINK
}]
%-----------------------------------------------------------------
\tom{G\#}{G\#}
%=================================================================
%\newchords{verse1.XX0000X} % Registrador de Acordes em Sequência
%\newchords{chorus1.XX0000X} % Registrador de Acordes em Sequência
%-----------------------------------------------------------------
\seq{Intro}{G\# D\# G\# D\# G\#}{1x}
\act{Solo}{Parte 1 | Parte 2}{1x}
%\act{}{}{}
%-----------------------------------------------------------------
%\beginverse \endverse
%\beginchorus \endchorus
\beginverse
Sobre às \[G\#]margens de uma estrada
Uma \[D\#7]simples pensão exis\[G\#]tia
A co\[G\#7]mida era tipo ca\[C\#]seira e o frango cai\[G\#7]pira era o prato do \[C\#]dia \[G\#]\[C\#]
Proprie\[G\#7]tário, homem de res\[C\#]peito, ali traba\[G\#]lhava com sua família
Cozinheira era sua es\[D\#]posa e a garço\[D\#7]nete era uma das \[G\#]filhas
\endverse
\act{Solo: Parte 1 | Parte 2}{Repetir}{1x}
\beginverse
Foi che^gando naquela pensão, um via^jante já fora de ^hora
Foi di^zendo para a garço^nete: Me traga um ^frango, vou jantar a^gora ^^
Eu es^tou bastante atra^sado, termi^nando eu já vou embora
Ela então respondeu num so^rriso: Mamãe tá de ^pé, pode crer, não de^mora
\endverse
\act{Solo: Parte 1 | Parte 2}{Repetir}{1x}
\beginverse
Quando ^ela foi servir a mesa, deli^cada e com muito bom ^jeito
Me des^culpe, mas trouxe uma ^franga, talvez não es^teja cozida di^reito ^^
O via^jante foi lhe respon^dendo: Pra mim franga ^crua talvez eu aceito
Sendo uma igual a vo^cê, seja à qualquer ^hora também não en^jeito
\endverse
\act{Solo: Parte 1 | Parte 2}{Repetir}{1x}
\beginverse
Foi sa^indo de cabeça baixa, pra quei^xar ao seu pai a mo^cinha
Minha ^filha, mate outra ^franga, pode tempe^rar, porém não co^zinha ^^
Vou le^var esta franga na ^mesa, se bem que comigo a conversa é cur^tinha
É a coisa que mais eu de^testo, ver homem bar^bado fazendo gra^cinha
\endverse
\act{Solo: Parte 1 | Parte 2}{Repetir}{1x}
\beginverse
Foi che^gando o velho e dizendo
Vim tra^zer o pedido que ^fez
Quando o ^cara tentou recu^sar já se viu na ^mira de um Schimidt in^glês ^^
O ne^gócio foi limpar o ^prato quando o proprie^tário lhe disse cortês
Nós estamos de portas a^bertas pra servir à ^moda que pede o fre^guês
\endverse
\act{Solo: Parte 2}{Repetir}{1x}
%-----------------------------------------------------------------
\vspace{4em} % Regulador de Espaçamento
%-----------------------------------------------------------------
\begin{comment}
\lstset{basicstyle=\scriptsize\bf} % Parâmetros da TAB
%-----------------------------------------------------------------
\tab{Solo 1}
\begin{lstlisting}
E|-----------------------------------------------------|
B|-----------------------------------------------------|
G|-----------------------------------------------------|
D|-----------------------------------------------------|
A|-----------------------------------------------------|
E|-----------------------------------------------------|
\end{lstlisting}
%-----------------------------------------------------------------
\end{comment}
%=================================================================

\color{drawChord}\gtab{\color{nameChord} G\#}{4:022100}% 
\color{drawChord}\gtab{\color{nameChord} G\#7}{4:020100}% 
\color{drawChord}\gtab{\color{nameChord} C\#}{4:X02220}% 
\color{drawChord}\gtab{\color{nameChord} D\#}{6:X02220}% 
\color{drawChord}\gtab{\color{nameChord} D\#7}{6:X02020}% 

%=================================================================
% PADRÃO: [TonalidadeMaior+NOTAX+Variações] .Ex:[D\#0] [D\#7V1V7]
% OBS: Variações são alterações do acorde em relação ao campo harmônico.
%-----------------------------------------------------------------
% Tipos de Variações de Acordes:
% V0 - Variação Diversa
% V1 - Menor (m)
% V2 - Maior (M)
% V3 - Meio Tom Abaixo (Bemol)
% V4 - Com Quarta (ex:C4)
% V5 - Com Quinta (ex:C5)
% V6 - Com Sexta (ex:C6)
% V7 - Com Sétima Menor (ex:C7)
% V8 - Com baixo dois Tons Acima (ex:D/F#)
% V9 - Com Nona (ex:C9)
% V10 - Meio Tom Acima (Sustenido)
% V11 - Com Sétima Maior (ex:C7M)
% V12 - Suspenso (Sus)
% V13 - Com baixo dois Tons e Meio Acima (ex:A/E)
% V14 - Com baixo um Tom e Meio Acima (ex:D9/F) 
% V15 - Meio-Diminuto (m7b5)
% N15 - NÃO Meio-Diminuto
% V16 - Diminuto (º)
% N16 - NÃO Diminuto
% V17 - Com baixo um Tom Acima (ex: C/D)
% V18 - Com baixo um Tom Abaixo (ex: Em/D)
% V19 - Com baixo dois Tons e meio Abaixo (ex: G/D)
%=================================================================
%\vspace{4em} % Regulador de Espaçamento
%=================================================================
\endsong
%=================================================================
