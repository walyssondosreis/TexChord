%=================================================================
\songcolumns{1}
\beginsong
{O Poder do Criador %TÍTULO
}[by={Goiano e Paranaense %ARTISTA
},album={@walyssondosreis},
id={ %COD.ID.: XXNNNN
},rev={1}, %REVISÃO
qr={https://music.youtube.com/watch?v=jOADeghWRhA&feature=share %LINK
}]
%-----------------------------------------------------------------
\tom{Bb}{Bb}
%=================================================================
%\newchords{verse1.XX0000X} % Registrador de Acordes em Sequência
%\newchords{chorus1.XX0000X} % Registrador de Acordes em Sequência
%-----------------------------------------------------------------
\seq{Intro}{Bb  F  Bb  Bb7  Eb | Bb  F  Bb  F  Bb}{1x}
%\act{}{}{}
%-----------------------------------------------------------------
%\begin{comment}
\lstset{basicstyle=\scriptsize\bf} % Parâmetros da TAB
%-----------------------------------------------------------------
\tab{Solo}
\begin{lstlisting}
E|--13/--10/-------------------------------------------------------|
B|--15/--11/--6--6h8--6--------------------------------------------|
G|------------7--7----7--7--7h8--7--5------------------------------|
D|-----------------------8--8----8--7------------------------------|
A|-----------------------------------------------------------------|
E|-----------------------------------------------------------------|

E|--15/--11/--8----------------------------------------------------|
B|-------------------1------3--1---------6--3--3--3--3--3----4-----|
G|--12/--8/---5----2---2----------3--3/--7--3--3--2--1-------------|
D|---------------3-------3--3--1--0--3------------------3/5--5-----|
A|-----------------------------------------------------------------|
E|-----------------------------------------------------------------|
                                            (F) (Bb)   (F) (Bb)
E|-----------------------------------------------------------------|
B|-------------------------------------6-8-------------------------|
G|------3----------3-----------5-----7-----------------------------|
D|----5---5------3-----------7---7-8-------------------------------|
A|--6-------5--5-----5--3--8---------------------------------------|
E|-----------------------------------------------------------------|
\end{lstlisting}
%-----------------------------------------------------------------
%\end{comment}
%-----------------------------------------------------------------
\beginverse 
\[Bb] Hora triste foi aquela, Que Jesus Cristo fa\[F]lou
\[F] Mãe está chegando a hora, A senhora fica e eu \[Bb]vou
\[Bb] Com certeza mãe e filho, Neste momento cho\[F]rou
\[F] Hora triste dolo\[Bb]rida Porque a dor da despe\[F]dida
Só co\[Eb]nhece \[F]quem pa\[Bb]ssou
\endverse

\beginverse
\[Bb] Maria disse meu filho, Faz tudo que o Pai man\[F]dou
\[F] Pra salvar a humanidade, Ele lhe determi\[Bb]nou
\[Bb] Com as lágrimas caindo, O seu rosto ele bei\[F]jou
\[F] Pra cumprir a profe\[Bb]cia Naquele instante o Me\[F]ssias
Todo \[Eb]pecado \[F]abra\[Bb]çou
\endverse

\beginverse
\[Bb] Nas margens do Rio Jordão, Jesus Cristo cami\[F]nhou
\[F] Para encontrar João, Aquele que testemu\[Bb]nhou
\[Bb] O encontro foi tão lindo, Que o povo se emocio\[F]nou
\[F] Também foi nessa vi\[Bb]sita Que nas mãos de João Ba\[F]tista
Jesus \[Eb]Cristo \[F]bati\[Bb]zou
\endverse

\beginverse
\[Bb] Na mesa da Santa Ceia, Jesus Cristo orde\[F]nou
\[F] Ensine os meus mandamentos, Que onde está meu Pai eu \[Bb]vou
\[Bb] Se o mundo lhes odiar, Também já me odi\[F]ou
\[F] Faça o bem sem ver a \[Bb]quem A sua recompensa \[F]vem
É o que Je\[Eb]sus p\[F]rofeti\[Bb]zou
\endverse

%-----------------------------------------------------------------
\vspace{4em} % Regulador de Espaçamento
%=================================================================
\color{drawChord}\gtab{\color{black} F}{1:022100}
\color{drawChord}\gtab{\color{black} Bb}{1:002220}
\color{drawChord}\gtab{\color{black} Eb}{3:X32010}

%=================================================================
% PADRÃO: [TonalidadeMaior+NOTAX+Variações] .Ex:[X50] [X57V1V7]
% OBS: Variações são alterações do acorde em relação ao campo harmônico.
%-----------------------------------------------------------------
% Tipos de Variações de Acordes:
% V0 - Variação Diversa
% V1 - Menor (m)
% V2 - Maior (M)
% V3 - Meio Tom Abaixo (Bemol)
% V4 - Com Quarta (ex:C4)
% V5 - Com Quinta (ex:C5)
% V6 - Com Sexta (ex:C6)
% V7 - Com Sétima Menor (ex:C7)
% V8 - Com baixo dois Tons Acima (ex:D/F#)
% V9 - Com Nona (ex:C9)
% V10 - Meio Tom Acima (Sustenido)
% V11 - Com Sétima Maior (ex:C7M)
% V12 - Suspenso (Sus)
% V13 - Com baixo dois Tons e Meio Acima (ex:A/E)
% V14 - Com baixo um Tom e Meio Acima (ex:D9/F) 
% V15 - Meio-Diminuto (m7b5)
% N15 - NÃO Meio-Diminuto
% V16 - Diminuto (º)
% N16 - NÃO Diminuto
% V17 - Com baixo um Tom Acima (ex: C/D)
% V18 - Com baixo um Tom Abaixo (ex: Em/D)
% V19 - Com baixo dois Tons e meio Abaixo (ex: G/D)
%=================================================================
%\vspace{4em} % Regulador de Espaçamento
%=================================================================
\endsong
%=================================================================
