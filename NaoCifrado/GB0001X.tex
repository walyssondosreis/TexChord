%=================================================================
\songcolumns{2}
\beginsong
{1000 Graus %TÍTULO
}[by={Renascer Praise %ARTISTA
},album={@walyssondosreis},
id={GB0001 %COD.ID.: GB0000
(Rev.0) %REVISÃO.: 0...N
}]
%-----------------------------------------------------------------
\tom{X1}
%=================================================================
%\newchords{verse1.GB0000X} % Registrador de Acordes em Sequência
%\newchords{chorus1.GB0000X} % Registrador de Acordes em Sequência
%-----------------------------------------------------------------
%\seq{Intro}{}{}
%-----------------------------------------------------------------
%\beginverse* \endverse
%\beginchorus \endchorus
\beginverse*
Na presença dos homens
Na presença dos anjos sempre
Eu te louvarei
Te louvarei
\endverse
\beginverse*
Mesmo estando em guerra
Vou celebrando minha vitória
Eu te louvarei
Te louvarei
\endverse
\beginverse*
Sobre toda terra, novo som se ouvirá
Tua alegria, força pra continuar
Deus de maravilhas, que maravilha
Te louvar, te louvar, te louvar!
\endverse
\beginchorus
Eu entro na tua presença, pra receber o seu poder
E quanto mais o tempo passa, mais quero Deus
O fogo cai a igreja canta, o inimigo vai ao chão
Fogo e glória nas cabeças, mil graus de unção
Mil graus de unção!
\endchorus
\beginverse*
Som da festa vai subir
E a sua glória descerá
Se ouvirá um novo som
De aleluia!
Aleluia!
\endverse
% Verso de preenchimento
\beginverse*
.
.
.
\endverse
%-----------------------------------------------------------------
\begin{comment}
\lstset{basicstyle=\scriptsize\bf} % Parâmetros da TAB
%-----------------------------------------------------------------
\tab{Solo 1}
\begin{lstlisting}
E|-----------------------------------------------------|
B|-----------------------------------------------------|
G|-----------------------------------------------------|
D|-----------------------------------------------------|
A|-----------------------------------------------------|
E|-----------------------------------------------------|
\end{lstlisting}
%-----------------------------------------------------------------
\end{comment}
%=================================================================
\vspace{2em} 
%-----------------------------------------------------------------
\gtab{\color{black} X}{}% 
\gtab{\color{black} X}{}% 
\gtab{\color{black} X}{}% 
\gtab{\color{black} X}{}% 
%-----------------------------------------------------------------
% PADRÃO: [TonalidadeMaior+NOTAX+Variações] .Ex:[X50] [X57V1V7]
% OBS: Variações são alterações do acorde em relação ao campo harmônico.
%-----------------------------------------------------------------
% Tipos de Variações de Acordes:
% V0 - Variação Diversa
% V1 - Menor (m)
% V2 - Maior (M)
% V3 - Meio Tom Abaixo (Bemol)
% V4 - Com Quarta (ex:C4)
% V5 - Com Quinta (ex:C5)
% V6 - Com Sexta (ex:C6)
% V7 - Com Sétima Menor (ex:C7)
% V8 - Com baixo dois Tons Acima (ex:D/F#)
% V9 - Com Nona (ex:C9)
% V10 - Meio Tom Acima (Sustenido)
% V11 - Com Sétima Maior (ex:C7M)
% V12 - Suspenso (Sus)
% V13 - Com baixo dois Tons e Meio Acima (ex:A/E)
% V14 - Com baixo um Tom e Meio Acima (ex:D9/F) 
% V15 - Meio-Diminuto (m7b5)
% N15 - NÃO Meio-Diminuto
% V16 - Diminuto (º)
% N16 - NÃO Diminuto
%=================================================================
\endsong
%=================================================================
\begin{comment}

\end{comment}