%=================================================================
\songcolumns{2}
\beginsong
{Amor de Primavera %TÍTULO
}[by={Di Paullo e Paulino %ARTISTA
},album={@walyssondosreis},
id={\href{https://music.youtube.com/watch?v=jB5m0_9ti2k&feature=share %LINK
}{ ST0144 %COD.ID.: XXNNNN
}},rev={0}, %REVISÃO
qr={https://music.youtube.com/watch?v=jB5m0_9ti2k&feature=share %LINK
}]
%-----------------------------------------------------------------
\tom{X1}{E}
%=================================================================
%\newchords{verse1.XX0000X} % Registrador de Acordes em Sequência
%\newchords{chorus1.XX0000X} % Registrador de Acordes em Sequência
%-----------------------------------------------------------------
\seq{Intro}{X1 X4 X1 X5V7 X1 X4 X1}{}
%\act{}{}{}
%-----------------------------------------------------------------
%\beginverse \endverse
%\beginchorus \endchorus
\beginverse
\[X1] O amor de prima\[X4]vera
Não termina no ve\[X1]rão
No outono ele flo\[X5V7]resce
No inverno é só pai\[X1]xão \[X1V7]
Eu pensei que fosse \[X4]fácil
Esquecer aquele a\[X1]mor
Mas quando veio a sau\[X5V7]dade
Foi demais a minha \[X1]dor
Quando veio\[X5V7] a saudade
Foi demais a minha \[X1]dor \[X1V7]
\endverse
\beginchorus 
Uuuuuuu uuu \[X4]uu oh \[X5V7]oh, ah \[X1]ah \[X1V7]
Uuuuuuu uuu \[X4]uu oh \[X5V7]oh, ah \[X1]ah \[X1V4] \[X1]
\endchorus
\beginverse
^ Coração que sai ven^cido
Quase sempre tem ra^zão
E a razão que sempre ^vence
Nunca teve cora^ção ^
O amor é como um ^dia
É a luz na escuri^dão
Traz de volta a ale^gria
Onde existe soli^dão
Traz de volta^ a alegria
Onde existe soli^dão ^
\endverse

%-----------------------------------------------------------------
\vspace{4em} % Regulador de Espaçamento
%-----------------------------------------------------------------
\begin{comment}
\lstset{basicstyle=\scriptsize\bf} % Parâmetros da TAB
%-----------------------------------------------------------------
\tab{Solo 1}
\begin{lstlisting}
E|-----------------------------------------------------|
B|-----------------------------------------------------|
G|-----------------------------------------------------|
D|-----------------------------------------------------|
A|-----------------------------------------------------|
E|-----------------------------------------------------|
\end{lstlisting}
%-----------------------------------------------------------------
\end{comment}
%=================================================================

\color{drawChord}\gtab{\color{nameChord} X1}{}% 
\color{drawChord}\gtab{\color{nameChord} X1V4}{}% 
\color{drawChord}\gtab{\color{nameChord} X1V7}{}% 
\color{drawChord}\gtab{\color{nameChord} X4}{}%
\color{drawChord}\gtab{\color{nameChord} X5V7}{}% 

%=================================================================
% PADRÃO: [TonalidadeMaior+NOTAX+Variações] .Ex:[X50] [X57V1V7]
% OBS: Variações são alterações do acorde em relação ao campo harmônico.
%-----------------------------------------------------------------
% Tipos de Variações de Acordes:
% V0 - Variação Diversa
% V1 - Menor (m)
% V2 - Maior (M)
% V3 - Meio Tom Abaixo (Bemol)
% V4 - Com Quarta (ex:C4)
% V5 - Com Quinta (ex:C5)
% V6 - Com Sexta (ex:C6)
% V7 - Com Sétima Menor (ex:C7)
% V8 - Com baixo dois Tons Acima (ex:D/F#)
% V9 - Com Nona (ex:C9)
% V10 - Meio Tom Acima (Sustenido)
% V11 - Com Sétima Maior (ex:C7M)
% V12 - Suspenso (Sus)
% V13 - Com baixo dois Tons e Meio Acima (ex:A/E)
% V14 - Com baixo um Tom e Meio Acima (ex:D9/F) 
% V15 - Meio-Diminuto (m7b5)
% N15 - NÃO Meio-Diminuto
% V16 - Diminuto (º)
% N16 - NÃO Diminuto
% V17 - Com baixo um Tom Acima (ex: C/D)
% V18 - Com baixo um Tom Abaixo (ex: Em/D)
% V19 - Com baixo dois Tons e meio Abaixo (ex: G/D)
%=================================================================
%\vspace{4em} % Regulador de Espaçamento
%=================================================================
\endsong
%=================================================================
