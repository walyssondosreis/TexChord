%=================================================================
\songcolumns{2}
\beginsong
{Santo Espírito %TÍTULO
}[by={Laura Souguellis %ARTISTA
},album={@walyssondosreis},
id={GB0104 %COD.ID.: GB0000
}] 
%-----------------------------------------------------------------
\tom{X1}
%=================================================================
%\newchords{verse0.GB0000} % Registrador de Acordes em Sequência
%-----------------------------------------------------------------
\seq{Intro}{X5 X1}{}
%-----------------------------------------------------------------
%\beginverse* \endverse
%\beginchorus \endchorus

\beginchorus
\[X5]Santo Espírito, és bem-vindo aqui
Vem \[X1]inundar, encher e\[X6V7]sse lugar
É \[X5]o desejo do meu coração
Sermos \[X1]inundados por Tua \[X6V7]glória, Senh\[X5]or
\{\[X5] Tua glória, Se\[X1]nhor \[X6V7]
Tua glória \}
\endchorus

\beginverse*
\[X5] Não há nada igual
Não há nada melhor\[X1]
A que se compara\[X6V7] à esperança viv\[X5]a
Tua presen\[X1]ça \[X6V7]
\endverse

\beginverse*
\[X5] Eu provei e vi
O mais doce am\[X1]or
Que liberta o meu s\[X6V7]er
E a vergonha d\[X5]esfaz
Tua presen\[X1]ça \[X6V7]
\endverse

\seq{Riff}{X5 X1 X6V7}{}

\beginverse*
\[X1] Vamos pro\[X5]var quão re\[X6V7]al é Tua pre\[X5]senç\[X1]a
\[X1] Vamos pro\[X5]var a Tua \[X6V7]glória e bon\[X5]dad\[X1]e
\[X1] Vamos pro\[X5]var quão re\[X6V7]al é Tua pre\[X5]senç\[X1]a
\[X1] Vamos pro\[X5]var a Tua \[X6V7]glória e bon\[X5]dad\[X1]e 
\{, Se\[X1]nhor\}
\endverse

%-----------------------------------------------------------------
\begin{comment}
\lstset{basicstyle=\scriptsize\bf} % Parâmetros da TAB
%-----------------------------------------------------------------
\tab{Solo 1}
\begin{lstlisting}
E|-----------------------------------------------------|
B|-----------------------------------------------------|
G|-----------------------------------------------------|
D|-----------------------------------------------------|
A|-----------------------------------------------------|
E|-----------------------------------------------------|
\end{lstlisting}
%-----------------------------------------------------------------
\end{comment}
%=================================================================
\vspace{2em}
%-----------------------------------------------------------------
\gtab{\color{black} X1}{}% A [X1]
\gtab{\color{black} X5}{}% E [X5]
\gtab{\color{black} X6V7}{}% F#m7 [X6V7]
%-----------------------------------------------------------------
% PADRÃO [TonalidadeMaiorNOTAX.Variação] .Ex:[X50] [X50V1]
% PADRÃO [TonalidadeMenorNOTAX.Variação] .Ex:[mX50] [mX50V1]
% OBS: Variações são alterações do acorde em relação ao campo harmônico.
%-----------------------------------------------------------------
% TIPOS DE VARIAÇÂO DOS ACORDES:
% V0 - ACORDE COM VARIAÇÃO DIVERSA
% V1 - ACORDE MENOR (m)
% V2 - ACORDE MAIOR (M)
% V3 - ACORDE MEIO TOM ABAIXO (Bemois)
% V4 - ACORDE COM QUARTA (C4)
% V5 - ACORDE COM QUINTA (C5)
% V6 - ACORDE COM SEXTA (C6)
% V7 - ACORDE COM SÉTIMA MENOR (C7)
% V8 - ACORDE COM BAIXO DOIS TONS ACIMA (D/F#)
% V9 - ACORDE COM NONA (C9)
% V10 - ACORDE MEIO TOM ACIMA (Sustenidos)
% V11 - ACORDE COM SÉTIMA MAIOR (C7M)
% V12 - ACORDE SUSPENSO (Sus)
% V13 - ACORDE COM BAIXO DOIS TONS E MEIO ACIMA (A/E)
% V14 - ACORDE UM TOM E MEIO ACIMA (D9/F)
%=================================================================
\endsong
%=================================================================
\begin{comment}

\end{comment}