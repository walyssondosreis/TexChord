%=================================================================
\songcolumns{2}
\beginsong
{1000 Graus %TÍTULO
}[by={Renascer Praise %ARTISTA
},album={@walyssondosreis},
id={GB0001 %COD.ID.: GB0000
},rev={3}, %REVISÃO
qr={https://drive.google.com/open?id=14zoYmJDGY2Ej9pLyUMQN7iVfNRAgwBDk %LINK
}]
%-----------------------------------------------------------------
\tom{X1}{C}
%=================================================================
%\newchords{verse1.GB0000X} % Registrador de Acordes em Sequência
%\newchords{chorus1.GB0000X} % Registrador de Acordes em Sequência
%-----------------------------------------------------------------
\seq{Intro}{X4 X1 X3V7 X6V7 X5V8 X4 X1 X5}{}
%-----------------------------------------------------------------
%\beginverse \endverse
%\beginchorus \endchorus
\beginverse
Ohhohh\[X4]ohh oh\[X1]ohh
Ohhohh\[X3V7]ohh oh\[X6V7]ohh \[X5V8]
Ohhohh\[X4]ohh oh\[X1]ohh
Ale\[X5]lúia (Alelúia!)
\endverse
\beginverse
\[X1] 
\chordsoff Na presença dos homens
Na presença dos anjos sempre
\chordson Eu te \[X4]lou\[X3]va\[X6V7]rei
Te \[X2V7]lou\[X3]va\[X4]rei
\endverse
\beginverse
\[X1]
\chordsoff Mesmo estando em guerra
Vou celebrando minha vitória
\chordson Eu te \[X4]lou\[X3]va\[X6V7]rei
Te \[X2V7]lou\[X3]va\[X4]rei
\endverse
\beginverse
Sobre \[X5]toda t\[X1]erra, novo \[X4]som se \[X5]ouvi\[X1]rá
Tua \[X5]ale\[X1]gria, força \[X4]pra con\[X5]ti\[X1]nuar
Deus de \[X5]mara\[X1]vilhas, que \[X6V2V8]mara\[X2V7]vilha
Te lou\[X7V3N15]var, te lou\[X2]var, te lou\[X3V2]vaaa\[X6]ar!
\endverse
\beginchorus
Eu entro \[X2V7]na tua pre\[X6V7]sença
Pra rece\[X3V7]ber o seu po\[X6V7]der
E quanto \[X2V7]mais o tempo \[X6V7]passa
Mais quero \[X3V7]Deee\[X6V7]eus
O fogo \[X2V7]cai a igreja \[X6V7]canta
O ini\[X3V7]migo vai ao \[X6V7]chão
Fogo e \[X2V7]glória nas ca\[X6V7]beças
Mil graus de un\[X3V7]çããã\[X6V7]ão
Mil graus de un\[X3V7]çããã\[X6V7]ão!
\endchorus
\beginverse
Som da \[X4]festa vai su\[X1]bir
Sua \[X3]glória desce\[X6]rá
Se ouvi\[X4]rá um novo \[X1]som
De ale\[X5]luia!

\endverse
\act{Repetir}{Verso 5}{+1x}
\beginverse
...Aleluia\[X1]!
\endverse
\act{Retomar}{Verso 1}{1x}
\act{Repetir}{Verso 5}{+4x}
%-----------------------------------------------------------------
\begin{comment}
\lstset{basicstyle=\scriptsize\bf} % Parâmetros da TAB
%-----------------------------------------------------------------
\tab{Solo 1}
\begin{lstlisting}
E|-----------------------------------------------------|
B|-----------------------------------------------------|
G|-----------------------------------------------------|
D|-----------------------------------------------------|
A|-----------------------------------------------------|
E|-----------------------------------------------------|
\end{lstlisting}
%-----------------------------------------------------------------
\end{comment}
%=================================================================
\vspace{2em} 
%-----------------------------------------------------------------
\color{drawChord}\gtab{\color{nameChord} X1}{}% 
\color{drawChord}\gtab{\color{nameChord} X2}{}% 
\color{drawChord}\gtab{\color{nameChord} X2V7}{}% 
\color{drawChord}\gtab{\color{nameChord} X3}{}%
\color{drawChord}\gtab{\color{nameChord} X3V2}{}%
\color{drawChord}\gtab{\color{nameChord} X3V7}{}\\%
\color{drawChord}\gtab{\color{nameChord} X4}{}%
\color{drawChord}\gtab{\color{nameChord} X5}{}%
\color{drawChord}\gtab{\color{nameChord} X6V7}{}%
\color{drawChord}\gtab{\color{nameChord} X7V3N15}{}%
%-----------------------------------------------------------------
% PADRÃO: [TonalidadeMaior+NOTAX+Variações] .Ex:[X50] [X57V1V7]
% OBS: Variações são alterações do acorde em relação ao campo harmônico.
%-----------------------------------------------------------------
% Tipos de Variações de Acordes:
% V0 - Variação Diversa
% V1 - Menor (m)
% V2 - Maior (M)
% V3 - Meio Tom Abaixo (Bemol)
% V4 - Com Quarta (ex:C4)
% V5 - Com Quinta (ex:C5)
% V6 - Com Sexta (ex:C6)
% V7 - Com Sétima Menor (ex:C7)
% V8 - Com baixo dois Tons Acima (ex:D/F#)
% V9 - Com Nona (ex:C9)
% V10 - Meio Tom Acima (Sustenido)
% V11 - Com Sétima Maior (ex:C7M)
% V12 - Suspenso (Sus)
% V13 - Com baixo dois Tons e Meio Acima (ex:A/E)
% V14 - Com baixo um Tom e Meio Acima (ex:D9/F) 
% V15 - Meio-Diminuto (m7b5)
% N15 - NÃO Meio-Diminuto
% V16 - Diminuto (º)
% N16 - NÃO Diminuto
%=================================================================
\endsong
%=================================================================
\begin{comment}

\end{comment}