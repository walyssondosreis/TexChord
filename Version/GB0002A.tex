%=================================================================
\songcolumns{2}
\beginsong
{A Alegria do Senhor %TÍTULO
}[by={Fernandinho %ARTISTA
},album={@walyssondosreis},
id={GB0002 %COD.ID.: GB0000
},rev={3}, %REVISÃO
qr={https://drive.google.com/open?id=1iVNukfQ34hJRmYMf-DZfe-Ft_X0zvUea %LINK
}]
%-----------------------------------------------------------------
\tom{A}{Bb}
%=================================================================
%\newchords{verse0.GB0000} % Registrador de Acordes em Sequência
%-----------------------------------------------------------------
\seq{Intro}{Bm A F\#m}{2x}
%-----------------------------------------------------------------
\begin{verse}
\[F\#m]Vento sopra forte
\chordsoff Tuas águas não podem me afogar
\chordson \[F\#m]Vento sopra forte
\chordsoff E em suas mãos vou segurar
\chordson\end{verse}
\seq{Riff Intro}{Bm A F\#m}{2x}
\act{Repetir}{Verso 1}{1x}
\begin{verse}
\[Bm7]E não me guio pelo que vejo
\[F\#m7]Mas eu sigo pelo que creio
\[Bm7]Eu não olho as circunstâncias
\[D]Eu vejo o teu a\[C\#]mor \[D C\# D C\#]
\end{verse}

\begin{chorus}
\[F\#m]A alegria do Se\[A]nhor é a nossa \[D]força\[Bm]
\[F\#m]A alegria do Se\[A]nhor é a nossa \[D]força\[Bm]
\[F\#m]A alegria do Se\[A]nhor é a nossa \[D]força\[Bm]
\[F\#m]A alegria do Se\[A]nhor é a nossa \[D]força\[Bm]
\end{chorus}
\seq{Riff Intro}{Bm A F\#m}{2x}
\act{Executar}{Solo 1}{}
\act{Retomar}{Verso 1}{1x}
\begin{verse}
Essa ale\[A]gria não vai mais sa\[D]ir
Essa ale\[Bm]gria não vai mais sa\[F\#m]ir
Essa ale\[A]gria não vai mais sa\[D]ir
De \[E]dentro do meu cora\[A]ção
\end{verse}
\act{Repetir}{Refrão}{2x}
\begin{chorus}
\[F\#m]A alegria do Se\[A]nhor é a nossa \[D]força\[Bm]
\[F\#m]A alegria do Se\[A]nhor é a nossa \[D]força\[Bm]
\[F\#m]A alegria do Se\[A]nhor é a nossa \[D]força\[Bm]
\end{chorus}

%-----------------------------------------------------------------
\begin{comment}
\lstset{basicstyle=\scriptsize\bf} % Parâmetros da TAB
%-----------------------------------------------------------------
\tab{Solo 1}
\begin{lstlisting}
E|-----------------------------------------------------|
B|-----------------------------------------------------|
G|-----------------------------------------------------|
D|-----------------------------------------------------|
A|-----------------------------------------------------|
E|-----------------------------------------------------|
\end{lstlisting}
%-----------------------------------------------------------------
\end{comment}
%=================================================================
\vspace{2em}
%-----------------------------------------------------------------
\color{drawChord}\gtab{\color{nameChord} A}{}% A [A]
\color{drawChord}\gtab{\color{nameChord} Bm}{}% Bm [Bm]
\color{drawChord}\gtab{\color{nameChord} Bm7}{}% Bm7 [Bm7]
\color{drawChord}\gtab{\color{nameChord} C\#}{}% C\# [C\#]
\color{drawChord}\gtab{\color{nameChord} D}{}% D [D]
\color{drawChord}\gtab{\color{nameChord} E}{}\\% E [E]
\color{drawChord}\gtab{\color{nameChord} F\#m}{}% F#m [F\#m]
\color{drawChord}\gtab{\color{nameChord} F\#m7}{}% F#m7 [F\#m7]
%-----------------------------------------------------------------
% PADRÃO [TonalidadeMaiorNOTAX.Variação] .Ex:[X50] [X50V1]
% PADRÃO [TonalidadeMenorNOTAX.Variação] .Ex:[mX50] [mX50V1]
% OBS: Variações são alterações do acorde em relação ao campo harmônico.
%-----------------------------------------------------------------
% TIPOS DE VARIAÇÂO DOS ACORDES:
% V0 - ACORDE COM VARIAÇÃO DIVERSA
% V1 - ACORDE MENOR (m)
% V2 - ACORDE MAIOR (M)
% V3 - ACORDE MEIO TOM ABAIXO (Bemois)
% V4 - ACORDE COM QUARTA (C4)
% V5 - ACORDE COM QUINTA (C5)
% V6 - ACORDE COM SEXTA (C6)
% V7 - ACORDE COM SÉTIMA MENOR (C7)
% V8 - ACORDE COM BAIXO DOIS TONS ACIMA (D/F#)
% V9 - ACORDE COM NONA (C9)
% V10 - ACORDE MEIO TOM ACIMA (Sustenidos)
% V11 - ACORDE COM SÉTIMA MAIOR (C7M)
% V12 - ACORDE SUSPENSO (Sus)
% V13 - ACORDE COM BAIXO DOIS TONS E MEIO ACIMA (A/E)
% V14 - ACORDE UM TOM E MEIO ACIMA (D9/F)
%=================================================================
\endsong
%=================================================================
