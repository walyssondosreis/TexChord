%=================================================================
\songcolumns{2}
\beginsong
{Venha O Teu Reino %TÍTULO
}[by={Davi Sacer %ARTISTA
},album={@walyssondosreis},
id={GB0127 %COD.ID.: GB0000
},rev={3}, %REVISÃO
qr={https://drive.google.com/open?id=1LTxwKkEKFQiDP8I69lg5gI2OrxNGBbz4 %LINK
}]
%-----------------------------------------------------------------
\tom{X1}{F}
%=================================================================
%\newchords{verse1.GB0000X} % Registrador de Acordes em Sequência
%\newchords{chorus1.GB0000X} % Registrador de Acordes em Sequência
%-----------------------------------------------------------------
\seq{Intro}{X4 X1 X6 X5}{2x}
%-----------------------------------------------------------------
%\beginverse* \endverse
%\beginchorus \endchorus
\beginverse
\[X4] Nosso \[X1]Pai
Que o teu \[X6]nome seja man\[X5]tido san\[X4]to
Venha \[X1]o teu rei\[X5]no
\[X4] Dá-nos \[X1]hoje o ali\[X6]mento nece\[X5]ssário pra vi\[X4]ver
O pão de \[X1]cada di\[X5]a
\endverse
\beginverse
Per\[X4]doa os \[X1]erros que \[X6]temos come\[X5]tido
Como \[X4]perdo\[X1]amos quem nos \[X6]ofendeu \[X5]
Não nos \[X4]deixe \[X1]cair em \[X6]provações se\[X5]veras
Mas \[X4]livra-\[X1]nos do \[X5]mal
\endverse
\beginchorus
Venha o teu \[X4]reino
Faça a tua von\[X1]tade
Seja assim na \[X3]terra
Como é no \[X2]céu
\endchorus
\act{Repetir}{Refrão}{+1x}
\seq{Riff 1}{X4 X5 X1 X6 X4 X5 X1}{}
\act{Retomar}{Verso 1}{1x}
\seq{Solo 1}{X2 X1V8 X4 X5 X6 X5V8 X1 X5}{}
\beginverse
\[X4] Porque o reino, o po\[X1]der e a glória
\[X6] São Teus para \[X5]sempre
\endverse
\act{Repetir}{Verso 3}{+1x}
\beginverse
...A\[X4]\[X1]m\[X5]ém
A\[X4]\[X1]m\[X5]ém
A\[X4]\[X1]m\[X5]ém
\endverse
%-----------------------------------------------------------------
\vspace{4em} % Regulador de Espaçamento
%-----------------------------------------------------------------
\begin{comment}
\lstset{basicstyle=\scriptsize\bf} % Parâmetros da TAB
%-----------------------------------------------------------------
\tab{Solo 1}
\begin{lstlisting}
E|-----------------------------------------------------|
B|-----------------------------------------------------|
G|-----------------------------------------------------|
D|-----------------------------------------------------|
A|-----------------------------------------------------|
E|-----------------------------------------------------|
\end{lstlisting}
%-----------------------------------------------------------------
\end{comment}
%=================================================================
 
%-----------------------------------------------------------------
\color{drawChord}\gtab{\color{nameChord} X1}{}% 
\color{drawChord}\gtab{\color{nameChord} X1V8}{}% 
\color{drawChord}\gtab{\color{nameChord} X2}{}% 
\color{drawChord}\gtab{\color{nameChord} X3}{}%
\color{drawChord}\gtab{\color{nameChord} X4}{}%
\color{drawChord}\gtab{\color{nameChord} X5}{}\\%
\color{drawChord}\gtab{\color{nameChord} X5V8}{}%
%-----------------------------------------------------------------
% PADRÃO: [TonalidadeMaior+NOTAX+Variações] .Ex:[X50] [X57V1V7]
% OBS: Variações são alterações do acorde em relação ao campo harmônico.
%-----------------------------------------------------------------
% Tipos de Variações de Acordes:
% V0 - Variação Diversa
% V1 - Menor (m)
% V2 - Maior (M)
% V3 - Meio Tom Abaixo (Bemol)
% V4 - Com Quarta (ex:C4)
% V5 - Com Quinta (ex:C5)
% V6 - Com Sexta (ex:C6)
% V7 - Com Sétima Menor (ex:C7)
% V8 - Com baixo dois Tons Acima (ex:D/F#)
% V9 - Com Nona (ex:C9)
% V10 - Meio Tom Acima (Sustenido)
% V11 - Com Sétima Maior (ex:C7M)
% V12 - Suspenso (Sus)
% V13 - Com baixo dois Tons e Meio Acima (ex:A/E)
% V14 - Com baixo um Tom e Meio Acima (ex:D9/F) 
% V15 - Meio-Diminuto (m7b5)
% N15 - NÃO Meio-Diminuto
% V16 - Diminuto (º)
% N16 - NÃO Diminuto
%=================================================================
\endsong
%=================================================================
\begin{comment}

\end{comment}