%=================================================================
\songcolumns{1}
\beginsong
{Uma Coisa %TÍTULO
}[by={Ministério Morada %ARTISTA
},album={@walyssondosreis},
id={GB0163 %COD.ID.: XX0000
},rev={0}, %REVISÃO
qr={https://drive.google.com/open?id=1CnREQfa6l7d_KlciJ8iPMzwNs6x27HrX %LINK
}]
%-----------------------------------------------------------------
\tom{X1}{A}
%=================================================================
%\newchords{verse1.XX0000X} % Registrador de Acordes em Sequência
%\newchords{chorus1.XX0000X} % Registrador de Acordes em Sequência
%-----------------------------------------------------------------
%\seq{Intro}{}{}
%\act{}{}{}
%-----------------------------------------------------------------
%\beginverse \endverse
%\beginchorus \endchorus
\beginchorus
Uma coisa vou pedir
Deixa eu ficar neste lugar
Todos os dias da minha vida
Uma coisa vou pedir
Deixa eu ficar neste lugar
Todos os dias da minha vida
\endchorus
\beginverse
Nós adoramos Aquele que faz
Vento aos Teus anjos
Nós adoramos Aquele que faz
Fogo aos Teus ministros
\endverse
\beginverse
Nós somos Seus ministros
Não negue o Seu fogo
Nós somos Seus ministros
Queremos queimar! Queremos queimar!
\endverse
% Verso de preenchimento
\beginverse*\color{white}
.
.
.
\endverse
%-----------------------------------------------------------------
\begin{comment}
\lstset{basicstyle=\scriptsize\bf} % Parâmetros da TAB
%-----------------------------------------------------------------
\tab{Solo 1}
\begin{lstlisting}
E|-----------------------------------------------------|
B|-----------------------------------------------------|
G|-----------------------------------------------------|
D|-----------------------------------------------------|
A|-----------------------------------------------------|
E|-----------------------------------------------------|
\end{lstlisting}
%-----------------------------------------------------------------
\end{comment}
%=================================================================
\vspace{2em} 
%-----------------------------------------------------------------
\color{drawChord}\gtab{\color{nameChord} X}{}% 
\color{drawChord}\gtab{\color{nameChord} X}{}% 
\color{drawChord}\gtab{\color{nameChord} X}{}% 
\color{drawChord}\gtab{\color{nameChord} X}{}% 
%-----------------------------------------------------------------
% PADRÃO: [TonalidadeMaior+NOTAX+Variações] .Ex:[X50] [X57V1V7]
% OBS: Variações são alterações do acorde em relação ao campo harmônico.
%-----------------------------------------------------------------
% Tipos de Variações de Acordes:
% V0 - Variação Diversa
% V1 - Menor (m)
% V2 - Maior (M)
% V3 - Meio Tom Abaixo (Bemol)
% V4 - Com Quarta (ex:C4)
% V5 - Com Quinta (ex:C5)
% V6 - Com Sexta (ex:C6)
% V7 - Com Sétima Menor (ex:C7)
% V8 - Com baixo dois Tons Acima (ex:D/F#)
% V9 - Com Nona (ex:C9)
% V10 - Meio Tom Acima (Sustenido)
% V11 - Com Sétima Maior (ex:C7M)
% V12 - Suspenso (Sus)
% V13 - Com baixo dois Tons e Meio Acima (ex:A/E)
% V14 - Com baixo um Tom e Meio Acima (ex:D9/F) 
% V15 - Meio-Diminuto (m7b5)
% N15 - NÃO Meio-Diminuto
% V16 - Diminuto (º)
% N16 - NÃO Diminuto
% V17 - Com baixo um Tom Acima (ex: C/D)
%=================================================================
\endsong
%=================================================================
\begin{comment}

\end{comment}