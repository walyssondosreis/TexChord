%=================================================================
\songcolumns{2}
\beginsong
{Liberdade Provisória %TÍTULO
}[by={Henrique e Juliano %ARTISTA
},album={@walyssondosreis},
id={\href{https://music.youtube.com/watch?v=rBU17BVDwuo&feature=share %LINK
}{ST0114 %COD.ID.: XXNNNN
}},rev={0}, %REVISÃO
qr={https://music.youtube.com/watch?v=rBU17BVDwuo&feature=share %LINK
}]
%-----------------------------------------------------------------
\tom{}{Gb}
%=================================================================
%\newchords{verse1.XX0000X} % Registrador de Acordes em Sequência
%\newchords{chorus1.XX0000X} % Registrador de Acordes em Sequência
%-----------------------------------------------------------------
%\seq{Intro}{}{}
%\act{}{}{}
%-----------------------------------------------------------------
%\beginverse \endverse
%\beginchorus \endchorus
\beginverse
No início foi assim
Terminou tá terminado
Cada um pro seu lado
Não precisa ligar mais
\endverse
\beginverse
Só que foi eu quem terminou
E quem foi largado não espera
Eu segui minha vida
Até ela começar seguir a dela
\endverse
\beginverse
E do meio pro final
Eu só ia pra onde ela tava
Cada beijo no rosto que outra boca dava
Eu morria de raiva
\endverse
\beginverse
E ela tava mais linda
Cada vez que eu olhava
O ciúme não tava batendo
Tava dando porrada
\endverse
\beginchorus
E eu implorei pra você voltar
E ela me matou na unha
Disse que eu tava solteiro
Eu tava solteiro porra nenhuma
Implorei pra voltar
Não me manda embora
Sou preso na sua vida
Era só liberdade provisória
Vai ter que me aceitar de volta
Ah, ah, ah, ah, ah, ah
\endchorus
%-----------------------------------------------------------------
\vspace{4em} % Regulador de Espaçamento
%-----------------------------------------------------------------
\begin{comment}
\lstset{basicstyle=\scriptsize\bf} % Parâmetros da TAB
%-----------------------------------------------------------------
\tab{Solo 1}
\begin{lstlisting}
E|-----------------------------------------------------|
B|-----------------------------------------------------|
G|-----------------------------------------------------|
D|-----------------------------------------------------|
A|-----------------------------------------------------|
E|-----------------------------------------------------|
\end{lstlisting}
%-----------------------------------------------------------------
\end{comment}
%=================================================================
\begin{comment}

\color{drawChord}\gtab{\color{nameChord} X}{}% 
\color{drawChord}\gtab{\color{nameChord} X}{}% 
\color{drawChord}\gtab{\color{nameChord} X}{}% 
\color{drawChord}\gtab{\color{nameChord} X}{}% 

\end{comment}
%=================================================================
% PADRÃO: [TonalidadeMaior+NOTAX+Variações] .Ex:[X50] [X57V1V7]
% OBS: Variações são alterações do acorde em relação ao campo harmônico.
%-----------------------------------------------------------------
% Tipos de Variações de Acordes:
% V0 - Variação Diversa
% V1 - Menor (m)
% V2 - Maior (M)
% V3 - Meio Tom Abaixo (Bemol)
% V4 - Com Quarta (ex:C4)
% V5 - Com Quinta (ex:C5)
% V6 - Com Sexta (ex:C6)
% V7 - Com Sétima Menor (ex:C7)
% V8 - Com baixo dois Tons Acima (ex:D/F#)
% V9 - Com Nona (ex:C9)
% V10 - Meio Tom Acima (Sustenido)
% V11 - Com Sétima Maior (ex:C7M)
% V12 - Suspenso (Sus)
% V13 - Com baixo dois Tons e Meio Acima (ex:A/E)
% V14 - Com baixo um Tom e Meio Acima (ex:D9/F) 
% V15 - Meio-Diminuto (m7b5)
% N15 - NÃO Meio-Diminuto
% V16 - Diminuto (º)
% N16 - NÃO Diminuto
% V17 - Com baixo um Tom Acima (ex: C/D)
% V18 - Com baixo um Tom Abaixo (ex: Em/D)
% V19 - Com baixo dois Tons e meio Abaixo (ex: G/D)
%=================================================================
%\vspace{4em} % Regulador de Espaçamento
%=================================================================
\endsong
%=================================================================
