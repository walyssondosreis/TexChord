%=================================================================
\songcolumns{1}
\beginsong
{Te Louvarei \\ Perto Quero Estar %TÍTULO
}[by={Ministério Apascentar de Louvor %ARTISTA
},album={@walyssondosreis},
id={GB0114 %COD.ID.: GB0000
},rev={3}, %REVISÃO
qr={https://drive.google.com/open?id=1s6ZZwJ2TEX40bgkbjXewQHE0ceKYIFnG %LINK
}]
%-----------------------------------------------------------------
\tom{X1}{Bb}
%=================================================================
%\newchords{verse1.GB0000X} % Registrador de Acordes em Sequência
%\newchords{chorus1.GB0000X} % Registrador de Acordes em Sequência
%-----------------------------------------------------------------
\seq{Intro}{X5 X1 X5 X2 X6 X5 X7V3N15}{}
%-----------------------------------------------------------------
%\beginverse* \endverse
%\beginchorus \endchorus
\beginverse
\[X1] Perto quero es\[X4]tar \[X5V8]
Junto aos teus \[X1]pés \[X5V8]
Pois prazer mai\[X4]or não há \[X6]
Que \[X5]me render e te \[X4]adorar
\[X1] Tudo que há em \[X4]mim \[X5V8]
Quero te ofer\[X1]tar \[X5V8]
Mas, ainda é \[X4]pouco eu sei \[X6]
Se \[X5]comparado ao \[X4]que ganhei
\[X1] Não sou apenas \[X4]servo
Teu a\[X5V8]migo me tor\[X4]nei \[X5V8]
\endverse
\beginchorus
\[X1] Te \[X5]louva\[X4]rei
\[X1] Não im\[X5]portam as \[X4]circuns\[X5]tâncias \[X1]
A\[X5]dora\[X4]rei \[X6]
Somente a \[X5]ti Je\[X1]sus \[(X4 X5V8)]
\endchorus
\act{Retomar}{Verso 1}{1x}
\act{Repetir}{Refrão}{2x}
\beginverse
...Somente a \[X5]ti Je\[X1]sus \[X6]
Somente a \[X5]ti Je\[X1]sus \[X6]
Somente a \[X5]ti Je\[X1]sus 
\endverse
%-----------------------------------------------------------------
\vspace{2em} % Regulador de Espaçamento
%-----------------------------------------------------------------
\begin{comment}
\lstset{basicstyle=\scriptsize\bf} % Parâmetros da TAB
%-----------------------------------------------------------------
\tab{Solo 1}
\begin{lstlisting}
E|-----------------------------------------------------|
B|-----------------------------------------------------|
G|-----------------------------------------------------|
D|-----------------------------------------------------|
A|-----------------------------------------------------|
E|-----------------------------------------------------|
\end{lstlisting}
%-----------------------------------------------------------------
\end{comment}
%=================================================================
 
%-----------------------------------------------------------------
\color{drawChord}\gtab{\color{nameChord} X1}{}% 
\color{drawChord}\gtab{\color{nameChord} X4}{}% 
\color{drawChord}\gtab{\color{nameChord} X4V17}{}% 
\color{drawChord}\gtab{\color{nameChord} X4V8}{}%
\color{drawChord}\gtab{\color{nameChord} X5}{}%
\color{drawChord}\gtab{\color{nameChord} X5V8}{}%
\color{drawChord}\gtab{\color{nameChord} X6}{}%
\color{drawChord}\gtab{\color{nameChord} X7V3N15}{}%
%-----------------------------------------------------------------
% PADRÃO: [TonalidadeMaior+NOTAX+Variações] .Ex:[X50] [X57V1V7]
% OBS: Variações são alterações do acorde em relação ao campo harmônico.
%-----------------------------------------------------------------
% Tipos de Variações de Acordes:
% V0 - Variação Diversa
% V1 - Menor (m)
% V2 - Maior (M)
% V3 - Meio Tom Abaixo (Bemol)
% V4 - Com Quarta (ex:C4)
% V5 - Com Quinta (ex:C5)
% V6 - Com Sexta (ex:C6)
% V7 - Com Sétima Menor (ex:C7)
% V8 - Com baixo dois Tons Acima (ex:D/F#)
% V9 - Com Nona (ex:C9)
% V10 - Meio Tom Acima (Sustenido)
% V11 - Com Sétima Maior (ex:C7M)
% V12 - Suspenso (Sus)
% V13 - Com baixo dois Tons e Meio Acima (ex:A/E)
% V14 - Com baixo um Tom e Meio Acima (ex:D9/F) 
% V15 - Meio-Diminuto (m7b5)
% N15 - NÃO Meio-Diminuto
% V16 - Diminuto (º)
% N16 - NÃO Diminuto
% V17 - Com baixo um Tom Acima (ex: C/D)
%=================================================================
\endsong
%=================================================================
\begin{comment}

\end{comment}