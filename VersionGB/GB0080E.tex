%=================================================================
\songcolumns{2}
\beginsong
{Oh, Quão Lindo Esse Nome É %TÍTULO
}[by={Ana Nóbrega  %ARTISTA
},album={@walyssondosreis},
id={GB0080 %COD.ID.: GB0000
},rev={3}, %REVISÃO
qr={https://drive.google.com/open?id=1xVSlN55VrROWvW7_7CF4OTrxUcdwQ3Jg %LINK
}]
%-----------------------------------------------------------------
\tom{E}{D}
%=================================================================
\newchords{verse1.GB0080E} % Registrador de Acordes em Sequência
\newchords{verse2.GB0080E}
\newchords{chorus.GB0080E}
%-----------------------------------------------------------------
\seq{Intro}{E A}{}
%-----------------------------------------------------------------
%\beginverse* \endverse
%\beginchorus \endchorus

\beginverse\memorize[verse1.GB0080E]
\[E] No início eras a palavra
Um com \[A]Deus o Al\[C\#m7]tíssi\[B4]mo
\[C\#m7] O mis\[B/D\#]tério de Tua \[E]glória
Cristo em \[A]Ti se \[C\#m7]reve\[B4]lou
\endverse

\beginchorus\memorize[chorus.GB0080E]
Ó, quão lindo esse nome \[E]é
Ó, quão lindo esse nome \[B]é
O nome \[C\#m7]de Je\[B]sus, meu \[A]Rei
Ó, quão lindo esse nome \[E/G\#]é
Maior que tudo Ele \[B]é
Ó, quão lindo esse nome \[C\#m7]é,
O no\[B]me de \[A]Jesus
\endchorus

\beginverse\replay[verse1.GB0080E]
^ Deixou o céu para buscar-nos
Veio ^pra nos ^resga^tar
^ Amor maior que o ^meu pe^cado
Nada ^vai nos ^sepa^rar
\endverse

\beginchorus\replay[chorus.GB0080E]
Ó, quão maravilhoso ^é
Ó, quão maravilhoso ^é
O nome ^de Je^sus, meu ^Rei
Ó, quão maravilhoso ^é
Maior que tudo Ele ^é
Ó, quão maravilhoso ^é o no^me de ^Jesus
Ó, quão maravilhoso \[C\#m7]é o no\[B]me de \[A]Jesus
\endchorus

\seq{Riff}{A B C\#m7 G\#m7}{5x}

\beginverse\memorize[verse2.GB0080E]
A morte ven\[A]ceste
O véu Tu rom\[B]peste
A tumba va\[C\#m7]zia agora es\[G\#m7]tá
O céu Te a\[A]dora
Proclama Tua \[B]glória
Pois ressusci\[C\#m7]taste e vivo es\[B]tás
\endverse

\beginverse\replay[verse2.GB0080E]
És inven^cível
Inigua^lável
Hoje e pra ^sempre reina^rás
Teu é o ^reino
Tua é a ^glória
E acima de ^todo nome es^tás
\endverse

\beginchorus\replay[chorus.GB0080E]
Poderoso esse nome ^é
Poderoso esse nome ^é
O nome ^de Je^sus, meu ^Rei
Poderoso esse nome ^é
Mais forte que tudo ^é
Poderoso esse nome ^é, o no^me de ^Jesus
Poderoso esse nome \[C\#m7]é, o no\[B]me de \[A]Jesus
\endchorus
\act{Retomar}{Verso 4}{1x}
\beginverse
...Poderoso esse nome \[C\#m7]é, o no\[B]me de \[A]Jesus
\endverse

%-----------------------------------------------------------------
\begin{comment}
\lstset{basicstyle=\scriptsize\bf} % Parâmetros da TAB
%-----------------------------------------------------------------
\tab{Solo 1}
\begin{lstlisting}
E|-----------------------------------------------------|
B|-----------------------------------------------------|
G|-----------------------------------------------------|
D|-----------------------------------------------------|
A|-----------------------------------------------------|
E|-----------------------------------------------------|
\end{lstlisting}
%-----------------------------------------------------------------
\end{comment}
%=================================================================
\vspace{2em}
%-----------------------------------------------------------------
\color{drawChord}\gtab{\color{nameChord} E}{}% E [E] 
\color{drawChord}\gtab{\color{nameChord} E/G\#}{}% E/G# [E/G\#]
\color{drawChord}\gtab{\color{nameChord} G\#m7}{}% G#m7 [G\#m7]
\color{drawChord}\gtab{\color{nameChord} A}{}% A [A]
\color{drawChord}\gtab{\color{nameChord} B}{}% B [B]
\color{drawChord}\gtab{\color{nameChord} B/D\#}{}\\% B/D# [B/D\#]
\color{drawChord}\gtab{\color{nameChord} B4}{}% B4 [B4]
\color{drawChord}\gtab{\color{nameChord} C\#m7}{}% C#m7 [C\#m7]
%-----------------------------------------------------------------
% PADRÃO [TonalidadeMaiorNOTAX.Variação] .Ex:[X50] [X50V1]
% PADRÃO [TonalidadeMenorNOTAX.Variação] .Ex:[mX50] [mX50V1]
% OBS: Variações são alterações do acorde em relação ao campo harmônico.
%-----------------------------------------------------------------
% TIPOS DE VARIAÇÂO DOS ACORDES:
% V0 - ACORDE COM VARIAÇÃO DIVERSA
% V1 - ACORDE MENOR (m)
% V2 - ACORDE MAIOR (M)
% V3 - ACORDE MEIO TOM ABAIXO (Bemois)
% V4 - ACORDE COM QUARTA (C4)
% V5 - ACORDE COM QUINTA (C5)
% V6 - ACORDE COM SEXTA (C6)
% V7 - ACORDE COM SÉTIMA MENOR (C7)
% V8 - ACORDE COM BAIXO DOIS TONS ACIMA (D/F#)
% V9 - ACORDE COM NONA (C9)
% V10 - ACORDE MEIO TOM ACIMA (Sustenidos)
% V11 - ACORDE COM SÉTIMA MAIOR (C7M)
% V12 - ACORDE SUSPENSO (Sus)
% V13 - ACORDE COM BAIXO DOIS TONS E MEIO ACIMA (A/E)
% V14 - ACORDE UM TOM E MEIO ACIMA (D9/F)
%=================================================================
\endsong
%=================================================================
\begin{comment}

\end{comment}