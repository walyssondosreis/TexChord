%=================================================================
\songcolumns{1}
\beginsong
{A Alegria do Senhor %TÍTULO
}[by={Fernandinho %ARTISTA
},album={@walyssondosreis},
id={GB0002 %COD.ID.: GB0000
}] 
%-----------------------------------------------------------------
\tom{mX1}
%=================================================================
%\newchords{verse0.GB0000} % Registrador de Acordes em Sequência
%-----------------------------------------------------------------
\seq{Intro}{mX4V2V5 mX3V5 mX1V2V5}{2x}
%-----------------------------------------------------------------
\begin{verse*}
\[mX1]Vento sopra forte
\chordsoff Tuas águas não podem me afogar
\chordson \[mX1]Vento sopra forte
\chordsoff E em suas mãos vou segurar
\chordson\end{verse*}
%-----------------------------------------------------------------
\seq{Riff Intro}{mX4V2V5 mX3V5 mX1V2V5}{2x}
%-----------------------------------------------------------------
\begin{verse*}
\[mX4V7]E não me guio pelo que vejo
\[mX1V7]Mas eu sigo pelo que creio
\[mX4V7]Eu não olho as circunstâncias
\[mX6V5]Eu vejo o teu a\[mX5V5]mor \[mX6V5 mX5V5 mX6V5 mX5V5]
\end{verse*}
%-----------------------------------------------------------------
\begin{chorus}
\[mX1V2V5]A alegria do Se\[mX3V5]nhor é a nossa \[mX6V5]força\[mX4V2V5]
\[mX1V2V5]A alegria do Se\[mX3V5]nhor é a nossa \[mX6V5]força\[mX4V2V5]
\[mX1V2V5]A alegria do Se\[mX3V5]nhor é a nossa \[mX6V5]força\[mX4V2V5]
\[mX1V2V5]A alegria do Se\[mX3V5]nhor é a nossa \[mX6V5]força\[mX4V2V5]
\end{chorus}
%-----------------------------------------------------------------
\begin{verse*}
Essa ale\[mX3V5]gria não vai mais sa\[mX6V5]ir
Essa ale\[mX4V2V5]gria não vai mais sa\[mX1V2V5]ir
Essa ale\[mX3V5]gria não vai mais sa\[mX6V5]ir
De \[mX7V5]dentro do meu cora\[mX3V5]ção
\end{verse*}
%-----------------------------------------------------------------
\begin{comment}
\lstset{basicstyle=\scriptsize\bf} % Parâmetros da TAB
%-----------------------------------------------------------------
\tab{Solo 1}
\begin{lstlisting}
E|-----------------------------------------------------|
B|-----------------------------------------------------|
G|-----------------------------------------------------|
D|-----------------------------------------------------|
A|-----------------------------------------------------|
E|-----------------------------------------------------|
\end{lstlisting}
%-----------------------------------------------------------------
\end{comment}
%=================================================================
\vspace{2em}
%-----------------------------------------------------------------
\gtab{\color{black} mX1}{}% Gm [mX1]
\gtab{\color{black} mX1V7}{}% Gm7 [mX1V7]
\gtab{\color{black} mX1V2V5}{}% G5 [mX1V2V5]
\gtab{\color{black} mX3V5}{}% Bb5 [mX3V5]
\gtab{\color{black} mX4V2V5}{}% C5 [mX4V2V5]
\gtab{\color{black} mX4V7}{}% Cm7 [mX4V7]
\gtab{\color{black} mX5V5}{}% D5 [mX5V5]
\gtab{\color{black} mX6V5}{}% Eb5 [mX6V5]
\gtab{\color{black} mX7V5}{}% F5 [mX7V5]
%-----------------------------------------------------------------
% PADRÃO [TonalidadeMaiorNOTAX.Variação] .Ex:[X50] [X50V1]
% PADRÃO [TonalidadeMenorNOTAX.Variação] .Ex:[mX50] [mX50V1]
% OBS: Variações são alterações do acorde em relação ao campo harmônico.
%-----------------------------------------------------------------
% TIPOS DE VARIAÇÂO DOS ACORDES:
% V1 - ACORDE MENOR (m)
% V2 - ACORDE MAIOR (M)
% V3 - ACORDE MEIO TOM ABAIXO (Bemois)
% V4 - ACORDE COM QUARTA (C4)
% V5 - ACORDE COM QUINTA (C5)
% V6 - ACORDE COM SEXTA (C6)
% V7 - ACORDE COM SÉTIMA MENOR (C7)
% V8 - ACODE COM BAIXO DOIS TONS ACIMA (D/F#)
% V9 - ACORDE COM NONA (C9)
% V10 - ACORDE MEIO TOM ACIMA (Sustenidos)
% V11 - ACORDE COM SÉTIMA MAIOR (C7M)
% V12 - ACORDE SUSPENSO (Sus)
% V13 - ACORDE COM VARIAÇÃO DIVERSA
%=================================================================
\endsong
%=================================================================
