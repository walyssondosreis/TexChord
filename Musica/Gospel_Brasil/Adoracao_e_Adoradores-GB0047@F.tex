%=================================================================
\songcolumns{2}
\beginsong
{Fogo de Deus %TÍTULO
}[by={Adoração e Ado. %ARTISTA
},album={@walyssondosreis},
id={GB0047 %COD.ID.: GB0000
},rev={0}, %REVISÃO
qr={https://drive.google.com/open?id=1FGnwJHCHdwPguWSScXKibUjt_BEL1pOp %LINK
}]
%-----------------------------------------------------------------
\tom{F}{G}
%=================================================================
%\newchords{verse1.GB0000X} % Registrador de Acordes em Sequência
%\newchords{chorus1.GB0000X} % Registrador de Acordes em Sequência
%-----------------------------------------------------------------
\seq{Intro}{Dm Bb/D F}{2x}
%-----------------------------------------------------------------
%\beginverse* \endverse
%\beginchorus \endchorus
\beginverse*
\[Dm]Há muito \[C/E]mais que \[F]isto
Es\[Bb]pirito \[Dm]santo \[C/E]sopra em \[Bb]nós
\[Dm]Há muito \[C/E]mais que \[F]isto
Es\[Bb]pirito \[Dm]santo espe\[C/E]ramos por \[Bb]ti
\[Gm]Enche-nos \[F/A]outra \[C/E]vez
\[Gm]Enche-nos \[F/A]outra \[C/E]vez
\endverse
\beginchorus
Fogo de \[F]Deus
Acende \[C/E]em nós
Pai\[Dm]xão pelo teu \[Bb]nome
Espirito \[F]de Deus
Derrama \[C/E]aqui
O teu po\[Dm]der
O teu a\[C]mor
Em \[Bb]nós
\endchorus
\beginverse*
^Vento im^petu^oso
^Enche-nos ^com o ^teu po^der
^Vem liber^tar os ca^tivos
Que^remos vi^ver para o ^teu lou^vor
^Derrama a ^tua ^glória
^Derrama a ^tua ^glória
\endverse
\seq{Riff 1}{Bb C Dm C}{}
\beginverse*
\[Bb] Vem acende em nós, Se\[C/E]nhor!
Vem acende em nós, Se\[Dm]nhor!
Vem acende em nós pai\[C]xão pelo teu nome!
\endverse
%-----------------------------------------------------------------
\vspace{4em} % Regulador de Espaçamento
%-----------------------------------------------------------------
\begin{comment}
\lstset{basicstyle=\scriptsize\bf} % Parâmetros da TAB
%-----------------------------------------------------------------
\tab{Solo 1}
\begin{lstlisting}
E|-----------------------------------------------------|
B|-----------------------------------------------------|
G|-----------------------------------------------------|
D|-----------------------------------------------------|
A|-----------------------------------------------------|
E|-----------------------------------------------------|
\end{lstlisting}
%-----------------------------------------------------------------
\end{comment}
%=================================================================
 
%-----------------------------------------------------------------
\color{drawChord}\gtab{\color{nameChord} F}{}% 
\color{drawChord}\gtab{\color{nameChord} F/A}{}% 
\color{drawChord}\gtab{\color{nameChord} Gm}{}% 
\color{drawChord}\gtab{\color{nameChord} Bb}{}%
\color{drawChord}\gtab{\color{nameChord} Bb/D}{}%
\color{drawChord}\gtab{\color{nameChord} C}{}\\%
\color{drawChord}\gtab{\color{nameChord} C/E}{}%
\color{drawChord}\gtab{\color{nameChord} Dm}{}%
%-----------------------------------------------------------------
% PADRÃO: [TonalidadeMaior+NOTAX+Variações] .Ex:[X50] [X57V1V7]
% OBS: Variações são alterações do acorde em relação ao campo harmônico.
%-----------------------------------------------------------------
% Tipos de Variações de Acordes:
% V0 - Variação Diversa
% V1 - Menor (m)
% V2 - Maior (M)
% V3 - Meio Tom Abaixo (Bemol)
% V4 - Com Quarta (ex:C4)
% V5 - Com Quinta (ex:C5)
% V6 - Com Sexta (ex:C6)
% V7 - Com Sétima Menor (ex:C7)
% V8 - Com baixo dois Tons Acima (ex:D/F#)
% V9 - Com Nona (ex:C9)
% V10 - Meio Tom Acima (Sustenido)
% V11 - Com Sétima Maior (ex:C7M)
% V12 - Suspenso (Sus)
% V13 - Com baixo dois Tons e Meio Acima (ex:A/E)
% V14 - Com baixo um Tom e Meio Acima (ex:D9/F) 
% V15 - Meio-Diminuto (m7b5)
% N15 - NÃO Meio-Diminuto
% V16 - Diminuto (º)
% N16 - NÃO Diminuto
%=================================================================
\endsong
%=================================================================
\begin{comment}

\end{comment}