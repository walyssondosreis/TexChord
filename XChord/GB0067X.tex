%=================================================================
\songcolumns{2}
\beginsong
{Nada Além do Sangue %TÍTULO
}[by={Fernandinho  %ARTISTA
},album={@walyssondosreis},
id={GB0067 %COD.ID.: GB0000
(Rev.1) %REVISÃO.: 0...N
}]
%-----------------------------------------------------------------
\tom{X1}
%=================================================================
\newchords{verse1.GB0067X}
\newchords{chorus.GB0067X}
%-----------------------------------------------------------------
\seq{Intro}{X1 X6}{2x}
%-----------------------------------------------------------------
%\beginverse* \endverse
%\beginchorus \endchorus

\beginverse \memorize[verse1.GB0067X]
Teu \[X1]sangue
Leva-me a\[X6]lém, a todas as al\[X1]turas
Onde ouço a Tua \[X6]voz
Fala de Tua jus\[X5]tiça pela minha \[X4]vida
Jesus este é o Teu \[X1]sangue \[X6]\[X5]
\endverse

\beginverse \replay[verse1.GB0067X]
Tua ^cruz
Mostra Tua ^graça, fala do Amor do ^Pai
Que prepara para ^nós um caminho para ^Ele
Onde posso me ache^gar
Somente pelo ^sangue ^^
\endverse

\beginchorus \memorize[chorus.GB0067X]
\[X1]Que nos lava dos pecados
\[X6]Que nos traz restauração
\[X5]Nada além do sangue
\[X4]Nada além do sangue de \[X1]Jesus \[X5]\[X4]
\[X1]Que nos faz brancos como a neve
A\[X6]ceitos como Amigos de Deus
\[X5]Nada além do sangue
\[X4]Nada além do sangue de \[X1]Jesus \[X5]\[X4]
\endchorus

\act{Retomar}{Verso 2}{1x}

\beginverse \replay[chorus.GB0067X]
^Eu sou livre!
^Eu sou livre!
^Nada além do sangue
^Nada além do sangue de ^Jesus ^^
\endverse
\act{Repetir}{Verso 3}{+1x}
\beginverse 
\[X1]Alvo mais que a \[X5]neve
\[X2]Alvo mais que a \[X1]neve
\[X1]Sim, neste sangue la\[X4]vado
\[X4]Mais \[X1]alvo que a \[X5]neve se\[X1]rei \[X5]
\endverse
\act{Repetir}{Verso 4}{+2x}
\act{Repetir}{Verso 3}{2x}
\beginverse
 ... de \[X1]Jesus \[X5]\[X4]
\endverse
\act{Repetir}{Verso 5}{+2x}

%-----------------------------------------------------------------
\begin{comment}
\lstset{basicstyle=\scriptsize\bf} % Parâmetros da TAB
%-----------------------------------------------------------------
\tab{Solo 1}
\begin{lstlisting}
E|-----------------------------------------------------|
B|-----------------------------------------------------|
G|-----------------------------------------------------|
D|-----------------------------------------------------|
A|-----------------------------------------------------|
E|-----------------------------------------------------|
\end{lstlisting}
%-----------------------------------------------------------------
\end{comment}
%=================================================================
\vspace{2em}
%-----------------------------------------------------------------
\gtab{\color{black} X1}{}% A [X1]
\gtab{\color{black} X2}{}% Bm [X2]
\gtab{\color{black} X4}{}% D [X4]
\gtab{\color{black} X5}{}% E [X5]
\gtab{\color{black} X6}{}% F#m [X6]
%-----------------------------------------------------------------
% PADRÃO [TonalidadeMaiorNOTAX.Variação] .Ex:[X50] [X50V1]
% PADRÃO [TonalidadeMenorNOTAX.Variação] .Ex:[mX50] [mX50V1]
% OBS: Variações são alterações do acorde em relação ao campo harmônico.
%-----------------------------------------------------------------
% TIPOS DE VARIAÇÂO DOS ACORDES:
% V0 - ACORDE COM VARIAÇÃO DIVERSA
% V1 - ACORDE MENOR (m)
% V2 - ACORDE MAIOR (M)
% V3 - ACORDE MEIO TOM ABAIXO (Bemois)
% V4 - ACORDE COM QUARTA (C4)
% V5 - ACORDE COM QUINTA (C5)
% V6 - ACORDE COM SEXTA (C6)
% V7 - ACORDE COM SÉTIMA MENOR (C7)
% V8 - ACORDE COM BAIXO DOIS TONS ACIMA (D/F#)
% V9 - ACORDE COM NONA (C9)
% V10 - ACORDE MEIO TOM ACIMA (Sustenidos)
% V11 - ACORDE COM SÉTIMA MAIOR (C7M)
% V12 - ACORDE SUSPENSO (Sus)
% V13 - ACORDE COM BAIXO DOIS TONS E MEIO ACIMA (A/E)
% V14 - ACORDE UM TOM E MEIO ACIMA (D9/F)
%=================================================================
\endsong
%=================================================================
\begin{comment}

\end{comment}