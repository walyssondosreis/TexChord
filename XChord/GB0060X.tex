%=================================================================
\songcolumns{2}
\beginsong
{Maravilhado %TÍTULO
}[by={Diante do Trono %ARTISTA
},album={@walyssondosreis},
id={GB0060 %COD.ID.: GB0000
},rev={3}, %REVISÃO
qr={https://drive.google.com/open?id=1vbFZ11wp_kQSQfBWJ-gE8TfQgg2UNQVN %LINK
}]
%-----------------------------------------------------------------
\tom{X1}{A}
%=================================================================
%\newchords{verse1.GB0000X} % Registrador de Acordes em Sequência
%\newchords{chorus1.GB0000X} % Registrador de Acordes em Sequência
%-----------------------------------------------------------------
\seq{Intro}{X1 X3 X6 X4V9}{}
%-----------------------------------------------------------------
%\beginverse* \endverse
%\beginchorus \endchorus
\beginverse
Tu \[X6]reinas
No trono dos \[X4V9]céus
A criação se \[X1]prostra
\[X5]Aos Teus pés
\endverse
\beginverse
Tu ^reinas
Vestido de ^glória
Os anjos te a^doram
Aos teus ^pés
\endverse
\beginverse
Pra \[X3]sempre governa\[X4V9]rás
Teu \[X3]reino não passa\[X4V9]rá \[X5]
\endverse
\beginchorus
Ó Santo \[X1]Deus
Fico maravi\[X5V8]lhado
Tu és \[X6]muito mais do que eu possa expre\[X4V9]ssar \[X5]
Ó Santo \[X1]Deus
Quebro o vaso de ala\[X5V8]bastro
Sobre o \[X6]Deus que sabe me maravi\[X4V9]lhar \[(X5)]
\endchorus
\beginverse
Tu ^És
Totalmente a^mável
Tesouro dese^jável
Que ^eu procuro
\endverse
\beginverse
Tu ^és
Um fogo apaixo^nado
Eu sou do meu a^mado
E Tu és ^meu
\endverse
\act{Retomar}{Verso 3}{1x}
\beginverse
\[X1] De\[X5V8]rramo o \[X6]meu lou\[X4V9]vor sobre \[X1]Ti
De\[X5V8]rramo o \[X6]meu a\[X4V9]mor sobre \[X1]Ti
De\[X5V8]rramo o \[X6]meu lou\[X4V9]vor sobre \[X1]Ti
De\[X5V8]rramo o \[X6]meu a\[X4V9]mor \[(X5)]
\endverse
\act{Repetir}{Refrão}{1x}
%-----------------------------------------------------------------
\begin{comment}
\lstset{basicstyle=\scriptsize\bf} % Parâmetros da TAB
%-----------------------------------------------------------------
\tab{Solo 1}
\begin{lstlisting}
E|-----------------------------------------------------|
B|-----------------------------------------------------|
G|-----------------------------------------------------|
D|-----------------------------------------------------|
A|-----------------------------------------------------|
E|-----------------------------------------------------|
\end{lstlisting}
%-----------------------------------------------------------------
\end{comment}
%=================================================================
\vspace{2em} 
%-----------------------------------------------------------------
\color{act-gray}\gtab{\color{act-gray} X1}{}%
\color{act-gray}\gtab{\color{act-gray} X3}{}%   
\color{act-gray}\gtab{\color{act-gray} X4V9}{}% 
\color{act-gray}\gtab{\color{act-gray} X5}{}%
\color{act-gray}\gtab{\color{act-gray} X5V8}{}%
\color{act-gray}\gtab{\color{act-gray} X6}{}%
%-----------------------------------------------------------------
% PADRÃO: [TonalidadeMaior+NOTAX+Variações] .Ex:[X50] [X57V1V7]
% OBS: Variações são alterações do acorde em relação ao campo harmônico.
%-----------------------------------------------------------------
% Tipos de Variações de Acordes:
% V0 - Variação Diversa
% V1 - Menor (m)
% V2 - Maior (M)
% V3 - Meio Tom Abaixo (Bemol)
% V4 - Com Quarta (ex:C4)
% V5 - Com Quinta (ex:C5)
% V6 - Com Sexta (ex:C6)
% V7 - Com Sétima Menor (ex:C7)
% V8 - Com baixo dois Tons Acima (ex:D/F#)
% V9 - Com Nona (ex:C9)
% V10 - Meio Tom Acima (Sustenido)
% V11 - Com Sétima Maior (ex:C7M)
% V12 - Suspenso (Sus)
% V13 - Com baixo dois Tons e Meio Acima (ex:A/E)
% V14 - Com baixo um Tom e Meio Acima (ex:D9/F) 
% V15 - Meio-Diminuto (m7b5)
% N15 - NÃO Meio-Diminuto
% V16 - Diminuto (º)
% N16 - NÃO Diminuto
%=================================================================
\endsong
%=================================================================
\begin{comment}

\end{comment}