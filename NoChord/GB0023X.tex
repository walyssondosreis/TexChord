%=================================================================
\songcolumns{1}
\beginsong
{Deixa Queimar  %TÍTULO
}[by={Alessandro Vilas Boas %ARTISTA
},album={@walyssondosreis},
id={GB0023 %COD.ID.: GB0000
},rev={0}, %REVISÃO
qr={https://drive.google.com/open?id=1GJlMu-B_uaYH9FRP6D8HEE2g-3t7RZZ0 %LINK
}]
%-----------------------------------------------------------------
\tom{X1}{G}
%=================================================================
%\newchords{verse1.GB0000X} % Registrador de Acordes em Sequência
%\newchords{chorus1.GB0000X} % Registrador de Acordes em Sequência
%-----------------------------------------------------------------
%\seq{Intro}{}{}
%-----------------------------------------------------------------
%\beginverse* \endverse
%\beginchorus \endchorus
\beginverse*
Ouvi o meu Senhor bater na porta
E o meu coração queimou
Ouvi o meu Senhor bater na porta
E o meu coração queimou
\endverse
\beginverse*
Eu não vou parar
Gastarei minha vida aos teus pés
Eu não vou parar
Gastarei minha vida aos teus pés
\endverse
\beginverse*
Eu tenho fogo cerrado em meu peito
Que não me dá descanso, eu não tenho descanso
Eu tenho fogo cerrado em meu peito
Que não me dá descanso, eu não tenho descanso
\endverse
\beginchorus
Deixa queimar, deixa queimar, deixa queimar, deixa queimar
Deixa queimar, deixa queimar, deixa queimar, deixa queimar
Deixa queimar, deixa queimar
\endchorus
% Verso de preenchimento
\beginverse*
.
.
.
\endverse

%-----------------------------------------------------------------
\begin{comment}
\lstset{basicstyle=\scriptsize\bf} % Parâmetros da TAB
%-----------------------------------------------------------------
\tab{Solo 1}
\begin{lstlisting}
E|-----------------------------------------------------|
B|-----------------------------------------------------|
G|-----------------------------------------------------|
D|-----------------------------------------------------|
A|-----------------------------------------------------|
E|-----------------------------------------------------|
\end{lstlisting}
%-----------------------------------------------------------------
\end{comment}
%=================================================================
\vspace{2em} 
%-----------------------------------------------------------------
\color{drawChord}\gtab{\color{nameChord} X}{}% 
\color{drawChord}\gtab{\color{nameChord} X}{}% 
\color{drawChord}\gtab{\color{nameChord} X}{}% 
\color{drawChord}\gtab{\color{nameChord} X}{}% 
%-----------------------------------------------------------------
% PADRÃO: [TonalidadeMaior+NOTAX+Variações] .Ex:[X50] [X57V1V7]
% OBS: Variações são alterações do acorde em relação ao campo harmônico.
%-----------------------------------------------------------------
% Tipos de Variações de Acordes:
% V0 - Variação Diversa
% V1 - Menor (m)
% V2 - Maior (M)
% V3 - Meio Tom Abaixo (Bemol)
% V4 - Com Quarta (ex:C4)
% V5 - Com Quinta (ex:C5)
% V6 - Com Sexta (ex:C6)
% V7 - Com Sétima Menor (ex:C7)
% V8 - Com baixo dois Tons Acima (ex:D/F#)
% V9 - Com Nona (ex:C9)
% V10 - Meio Tom Acima (Sustenido)
% V11 - Com Sétima Maior (ex:C7M)
% V12 - Suspenso (Sus)
% V13 - Com baixo dois Tons e Meio Acima (ex:A/E)
% V14 - Com baixo um Tom e Meio Acima (ex:D9/F) 
% V15 - Meio-Diminuto (m7b5)
% N15 - NÃO Meio-Diminuto
% V16 - Diminuto (º)
% N16 - NÃO Diminuto
%=================================================================
\endsong
%=================================================================
\begin{comment}

\end{comment}