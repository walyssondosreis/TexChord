%=================================================================
\songcolumns{1}
\beginsong
{O Boêmio %TÍTULO
}[by={Amado Batista %ARTISTA
},album={@walyssondosreis},
id={\href{https://music.youtube.com/watch?v=4ERKRLQDUjs&feature=share %LINK
}{ST0147 %COD.ID.: XXNNNN
}},rev={0}, %REVISÃO
qr={https://music.youtube.com/watch?v=4ERKRLQDUjs&feature=share %LINK
}]
%-----------------------------------------------------------------
\tom{}{C}
%=================================================================
%\newchords{verse1.XX0000X} % Registrador de Acordes em Sequência
%\newchords{chorus1.XX0000X} % Registrador de Acordes em Sequência
%-----------------------------------------------------------------
%\seq{Intro}{}{}
%\act{}{}{}
%-----------------------------------------------------------------
%\beginverse \endverse
%\beginchorus \endchorus
\beginverse
Eu não sei o que fazer
Da minha vida
Sem você perto de mim
Eu não sei o que fazer
Da minha vida
Sem o seu carinho
\endverse
\beginverse
Eu já fiz tudo
Mas não pude
Conquistar seu coração
Sei que és muito volúvel
Jamais pude compreender
Sua ilusão
\endverse
\beginverse
E quem sou eu
Pra ter você só pra mim
Essa rosa vermelha tem espinhos
Eu estou morrendo por esta flor
\endverse
\beginverse
Eu não sou boêmio
Mas hoje eu vou beber
Quero esquecer a paixão
E a cruel solidão
Que vivo sem você
\endverse
\beginverse
Se eu bebo eu choro
No mesmo instante eu canto
E grito seu nome
Oh! querida amante
\endverse
\beginverse
Grito seu nome
Oh! querida amante
\endverse
%-----------------------------------------------------------------
\vspace{4em} % Regulador de Espaçamento
%-----------------------------------------------------------------
\begin{comment}
\lstset{basicstyle=\scriptsize\bf} % Parâmetros da TAB
%-----------------------------------------------------------------
\tab{Solo 1}
\begin{lstlisting}
E|-----------------------------------------------------|
B|-----------------------------------------------------|
G|-----------------------------------------------------|
D|-----------------------------------------------------|
A|-----------------------------------------------------|
E|-----------------------------------------------------|
\end{lstlisting}
%-----------------------------------------------------------------
\end{comment}
%=================================================================
\begin{comment}

\color{drawChord}\gtab{\color{nameChord} X}{}% 
\color{drawChord}\gtab{\color{nameChord} X}{}% 
\color{drawChord}\gtab{\color{nameChord} X}{}% 
\color{drawChord}\gtab{\color{nameChord} X}{}% 

\end{comment}
%=================================================================
% PADRÃO: [TonalidadeMaior+NOTAX+Variações] .Ex:[X50] [X57V1V7]
% OBS: Variações são alterações do acorde em relação ao campo harmônico.
%-----------------------------------------------------------------
% Tipos de Variações de Acordes:
% V0 - Variação Diversa
% V1 - Menor (m)
% V2 - Maior (M)
% V3 - Meio Tom Abaixo (Bemol)
% V4 - Com Quarta (ex:C4)
% V5 - Com Quinta (ex:C5)
% V6 - Com Sexta (ex:C6)
% V7 - Com Sétima Menor (ex:C7)
% V8 - Com baixo dois Tons Acima (ex:D/F#)
% V9 - Com Nona (ex:C9)
% V10 - Meio Tom Acima (Sustenido)
% V11 - Com Sétima Maior (ex:C7M)
% V12 - Suspenso (Sus)
% V13 - Com baixo dois Tons e Meio Acima (ex:A/E)
% V14 - Com baixo um Tom e Meio Acima (ex:D9/F) 
% V15 - Meio-Diminuto (m7b5)
% N15 - NÃO Meio-Diminuto
% V16 - Diminuto (º)
% N16 - NÃO Diminuto
% V17 - Com baixo um Tom Acima (ex: C/D)
% V18 - Com baixo um Tom Abaixo (ex: Em/D)
% V19 - Com baixo dois Tons e meio Abaixo (ex: G/D)
%=================================================================
%\vspace{4em} % Regulador de Espaçamento
%=================================================================
\newpage
\endsong
%=================================================================
