%=================================================================
\songcolumns{1}
\beginsong
{Não Há Deus Maior %TÍTULO
}[by={Comunidade de Nilópolis %ARTISTA
},album={@walyssondosreis},
id={GB0068 %COD.ID.: GB0000
},rev={0}, %REVISÃO
qr={https://drive.google.com/open?id=1rzonZur6QP-YrQo8Lrm8Vs6razE96CwB %LINK
}]
%-----------------------------------------------------------------
\tom{C}{F}
%=================================================================
%\newchords{verse1.GB0000X} % Registrador de Acordes em Sequência
%\newchords{chorus1.GB0000X} % Registrador de Acordes em Sequência
%-----------------------------------------------------------------
%\seq{Intro}{}{}
%-----------------------------------------------------------------
%\beginverse* \endverse
%\beginchorus \endchorus
\beginchorus
\[C] Não há Deus mai\[Dm7]or \[G]
Não há Deus me\[C]lhor \[Am7]
Não há Deus tão \[Dm7]grande \[G]
Como nosso \[C]Deus \[C4]\[C]
\endchorus
\beginverse*
Criou os \[Bm]céus, cri\[E7]ou a \[Am]terra \[E/G\#]
Criou o Sol e as es\[Am7]trelas
\[Dm7]Tudo ele \[G]fez 
\[Dm7]Tudo cri\[G]ou
\[Dm7]Tudo for\[G]mou \[F]
Para o \[G]Seu lou\[C]vor \[F]
Para o \[G]Seu lou\[C]vor \[F]
Para o \[G]Seu, \[Em7] para o \[Am7]Seu \[Dm7]
Para o \[G]Seu lou\[C]vor
\endverse
%-----------------------------------------------------------------
\vspace{4em} % Regulador de Espaçamento
%-----------------------------------------------------------------
\begin{comment}
\lstset{basicstyle=\scriptsize\bf} % Parâmetros da TAB
%-----------------------------------------------------------------
\tab{Solo 1}
\begin{lstlisting}
E|-----------------------------------------------------|
B|-----------------------------------------------------|
G|-----------------------------------------------------|
D|-----------------------------------------------------|
A|-----------------------------------------------------|
E|-----------------------------------------------------|
\end{lstlisting}
%-----------------------------------------------------------------
\end{comment}
%=================================================================
 
%-----------------------------------------------------------------
\color{drawChord}\gtab{\color{nameChord} C}{}% 
\color{drawChord}\gtab{\color{nameChord} C4}{}% 
\color{drawChord}\gtab{\color{nameChord} Dm7}{}% 
\color{drawChord}\gtab{\color{nameChord} E7}{}%
\color{drawChord}\gtab{\color{nameChord} Em7}{}%
\color{drawChord}\gtab{\color{nameChord} F}{}%
\color{drawChord}\gtab{\color{nameChord} G}{}%
\color{drawChord}\gtab{\color{nameChord} Am7}{}% 
%-----------------------------------------------------------------
% PADRÃO: [TonalidadeMaior+NOTAX+Variações] .Ex:[X50] [X57V1V7]
% OBS: Variações são alterações do acorde em relação ao campo harmônico.
%-----------------------------------------------------------------
% Tipos de Variações de Acordes:
% V0 - Variação Diversa
% V1 - Menor (m)
% V2 - Maior (M)
% V3 - Meio Tom Abaixo (Bemol)
% V4 - Com Quarta (ex:C4)
% V5 - Com Quinta (ex:C5)
% V6 - Com Sexta (ex:C6)
% V7 - Com Sétima Menor (ex:C7)
% V8 - Com baixo dois Tons Acima (ex:D/F#)
% V9 - Com Nona (ex:C9)
% V10 - Meio Tom Acima (Sustenido)
% V11 - Com Sétima Maior (ex:C7M)
% V12 - Suspenso (Sus)
% V13 - Com baixo dois Tons e Meio Acima (ex:A/E)
% V14 - Com baixo um Tom e Meio Acima (ex:D9/F) 
% V15 - Meio-Diminuto (m7b5)
% N15 - NÃO Meio-Diminuto
% V16 - Diminuto (º)
% N16 - NÃO Diminuto
%=================================================================
\endsong
%=================================================================
\begin{comment}

\end{comment}