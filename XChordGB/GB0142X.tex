%=================================================================
\songcolumns{2}
\beginsong
{Eu Vejo a Glória %TÍTULO
}[by={Marcos Góes %ARTISTA
},album={@walyssondosreis},
id={GB0142 %COD.ID.: GB0000
},rev={3}, %REVISÃO
qr={https://drive.google.com/open?id=1iWEFc2GryRh3zp50xU-cEMNSlyczac2g %LINK
}]
%-----------------------------------------------------------------
\tom{X1}{G}
%=================================================================
%\newchords{verse0.GB0000} % Registrador de Acordes em Sequência
%-----------------------------------------------------------------
\seq{Intro}{X1 X4 X6 X5}{2x}
%-----------------------------------------------------------------
\begin{verse}
\[X1]Eu vejo a \[X4]glória do Se\[X6]nhor hoje a\[X5]qui
\[X1]A sua \[X4]mão e o seu po\[X6]der sobre \[X5]mim
\[X1]Os céus a\[X4]bertos hoje eu \[X6]vou contem\[X5]plar
O a\[X2]mor descer \[X6]neste lu\[X5]gar
\end{verse}
\act{Repetir}{Verso 1}{+1x}
%-----------------------------------------------------------------
\begin{chorus}
\[X1]Eu quero \[X4]ver agora \[X6]o teu po\[X5]der
\[X1]A tua \[X4]glória inun\[X6]dando meu \[X5]ser
\[X1]Vou levan\[X4]tar as mãos \[X6]vou rece\[X5]ber
Vou lou\[X2]vando o teu \[X6]nome
Porque \[X4]sinto o Se\[X5]nhor me to\[X1]car
\[X1]Eu quero \[X4]ver agora \[X6]o teu po\[X5]der
\[X1]A tua \[X4]glória inun\[X6]dando meu \[X5]ser
\[X1]Vou levan\[X4]tar as mãos \[X6]vou rece\[X5]ber
Vou lou\[X2]vando o teu \[X6]nome
Porque \[X4]sinto o Se\[X5]nhor me to\[X1]car\[( X1 X4 X6 X5 )]
\end{chorus}
\act{Repetir}{Verso 1}{1x}
\act{Repetir}{Refrão}{1x}
\beginverse
... O Senhor me to\[( X1 X4 X6 X5 )]car
O Senhor me to\[( X1 X4 X6 X5 )]car
O Senhor me to\[( X1 )]car
\endverse
%-----------------------------------------------------------------
\begin{comment}
\lstset{basicstyle=\scriptsize\bf} % Parâmetros da TAB
%-----------------------------------------------------------------
\tab{Solo 1}
\begin{lstlisting}
E|-----------------------------------------------------|
B|-----------------------------------------------------|
G|-----------------------------------------------------|
D|-----------------------------------------------------|
A|-----------------------------------------------------|
E|-----------------------------------------------------|
\end{lstlisting}
%-----------------------------------------------------------------
\end{comment}
%=================================================================
\vspace{2em}
%-----------------------------------------------------------------
\color{drawChord}\gtab{\color{nameChord} X1}{}% A [X1]
\color{drawChord}\gtab{\color{nameChord} X2}{}% Bm [X2]
\color{drawChord}\gtab{\color{nameChord} X4}{}% D [X4]
\color{drawChord}\gtab{\color{nameChord} X5}{}% E [X5]
\color{drawChord}\gtab{\color{nameChord} X6}{}% F#m [X6]
%-----------------------------------------------------------------
% PADRÃO [TonalidadeMaiorNOTAX.Variação] .Ex:[X50] [X50V1]
% PADRÃO [TonalidadeMenorNOTAX.Variação] .Ex:[mX50] [mX50V1]
% OBS: Variações são alterações do acorde em relação ao campo harmônico.
%-----------------------------------------------------------------
% TIPOS DE VARIAÇÂO DOS ACORDES:
% V0 - ACORDE COM VARIAÇÃO DIVERSA
% V1 - ACORDE MENOR (m)
% V2 - ACORDE MAIOR (M)
% V3 - ACORDE MEIO TOM ABAIXO (Bemois)
% V4 - ACORDE COM QUARTA (C4)
% V5 - ACORDE COM QUINTA (C5)
% V6 - ACORDE COM SEXTA (C6)
% V7 - ACORDE COM SÉTIMA MENOR (C7)
% V8 - ACORDE COM BAIXO DOIS TONS ACIMA (D/F#)
% V9 - ACORDE COM NONA (C9)
% V10 - ACORDE MEIO TOM ACIMA (Sustenidos)
% V11 - ACORDE COM SÉTIMA MAIOR (C7M)
% V12 - ACORDE SUSPENSO (Sus)
% V13 - ACORDE COM BAIXO DOIS TONS E MEIO ACIMA (A/E)
% V14 - ACORDE UM TOM E MEIO ACIMA (D9/F)
%=================================================================
\endsong
%=================================================================

\begin{comment}

\end{comment}








