%=================================================================
\songcolumns{2}
\beginsong
{Santo Espírito %TÍTULO
}[by={Laura Souguellis %ARTISTA
},album={@walyssondosreis},
id={GB0104 %COD.ID.: GB0000
},rev={3}, %REVISÃO
qr={https://drive.google.com/open?id=11cUpNv7bUkeIQwNkpzjMRyiKcQXmemAE %LINK
}]
%-----------------------------------------------------------------
\tom{X1}{E}
%=================================================================
%\newchords{verse0.GB0000} % Registrador de Acordes em Sequência
%-----------------------------------------------------------------
\seq{Intro}{X1 X4}{}
%-----------------------------------------------------------------
%\beginverse* \endverse
%\beginchorus \endchorus

\beginverse
\[X1] Não há nada igual
Não há nada melhor\[X4]
A que se compara\[X2V7] à esperança viv\[X1]a
Tua presen\[X4]ça \[X2V7]
\endverse

\beginverse
\[X1] Eu provei e vi
O mais doce am\[X4]or
Que liberta o meu s\[X2V7]er
E a vergonha d\[X1]esfaz
Tua presen\[X4]ça \[X2V7]
\endverse

\beginchorus
\[X1]Santo Espírito, és bem-vindo aqui
Vem \[X4]inundar, encher e\[X2V7]sse lugar
É \[X1]o desejo do meu coração
Sermos \[X4]inundados por Tua \[X2V7]glória, Senh\[X1]or
\endchorus

\seq{Riff}{X1 X4 X2V7}{}
\act{Retormar}{Verso 1}{1x}
\act{Executar}{Solo 1}{}

\beginverse
\[X4] Vamos pro\[X1]var quão re\[X2V7]al é tua pre\[X1]senç\[X4]a
\[X4] Vamos pro\[X1]var a tua \[X2V7]glória e bon\[X1]dad\[X4]e
\[X4] Vamos pro\[X1]var quão re\[X2V7]al é tua pre\[X1]senç\[X4]a
\[X4] Vamos pro\[X1]var a Tua \[X2V7]glória e bon\[X1]dad\[X4]e
\endverse
\act{Repetir}{Verso 3}{+1x}
\beginverse
..., Se\[X4]nhor
\endverse
\act{Repetir}{Refrão}{2x}
\beginverse
...\[X1] Tua glória, Se\[X4]nhor \[X2V7]
Tua glória
\endverse
%-----------------------------------------------------------------
\vspace{4em} % Regulador de Espaçamento
%-----------------------------------------------------------------
\begin{comment}
\lstset{basicstyle=\scriptsize\bf} % Parâmetros da TAB
%-----------------------------------------------------------------
\tab{Solo 1}
\begin{lstlisting}
E|-----------------------------------------------------|
B|-----------------------------------------------------|
G|-----------------------------------------------------|
D|-----------------------------------------------------|
A|-----------------------------------------------------|
E|-----------------------------------------------------|
\end{lstlisting}
%-----------------------------------------------------------------
\end{comment}
%=================================================================

%-----------------------------------------------------------------
\color{drawChord}\gtab{\color{nameChord} X1}{}% E [X4]
\color{drawChord}\gtab{\color{nameChord} X4}{}% A [X4]
\color{drawChord}\gtab{\color{nameChord} X2V7}{}% F#m7 [X2V7]
%-----------------------------------------------------------------
% PADRÃO [TonalidadeMaiorNOTAX.Variação] .Ex:[X50] [X50V1]
% PADRÃO [TonalidadeMenorNOTAX.Variação] .Ex:[mX50] [mX50V1]
% OBS: Variações são alterações do acorde em relação ao campo harmônico.
%-----------------------------------------------------------------
% TIPOS DE VARIAÇÂO DOS ACORDES:
% V0 - ACORDE COM VARIAÇÃO DIVERSA
% V1 - ACORDE MENOR (m)
% V2 - ACORDE MAIOR (M)
% V3 - ACORDE MEIO TOM ABAIXO (Bemois)
% V4 - ACORDE COM QUARTA (C4)
% V5 - ACORDE COM QUINTA (C5)
% V6 - ACORDE COM SEXTA (C6)
% V7 - ACORDE COM SÉTIMA MENOR (C7)
% V8 - ACORDE COM BAIXO DOIS TONS ACIMA (D/F#)
% V9 - ACORDE COM NONA (C9)
% V10 - ACORDE MEIO TOM ACIMA (Sustenidos)
% V11 - ACORDE COM SÉTIMA MAIOR (C7M)
% V12 - ACORDE SUSPENSO (Sus)
% V13 - ACORDE COM BAIXO DOIS TONS E MEIO ACIMA (A/E)
% V14 - ACORDE UM TOM E MEIO ACIMA (D9/F)
%=================================================================
\endsong
%=================================================================
\begin{comment}

\end{comment}