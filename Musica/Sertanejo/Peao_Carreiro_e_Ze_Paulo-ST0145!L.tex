%=================================================================
\songcolumns{1}
\beginsong
{Porta do Mundo %TÍTULO
}[by={Peão Carreiro e Zé Paulo %ARTISTA
},album={@walyssondosreis},
id={\href{https://music.youtube.com/watch?v=doI8N9a49WA&feature=share %LINK
}{ST0145 %COD.ID.: XXNNNN
}},rev={0}, %REVISÃO
qr={https://music.youtube.com/watch?v=doI8N9a49WA&feature=share %LINK
}]
%-----------------------------------------------------------------
\tom{}{D}
%=================================================================
%\newchords{verse1.XX0000X} % Registrador de Acordes em Sequência
%\newchords{chorus1.XX0000X} % Registrador de Acordes em Sequência
%-----------------------------------------------------------------
%\seq{Intro}{}{}
%\act{}{}{}
%-----------------------------------------------------------------
%\beginverse \endverse
%\beginchorus \endchorus
\beginverse
O som da viola bateu
No meu peito doeu, meu irmão
Assim eu me fiz cantador
Sem nenhum professor aprendi a lição
São coisas divinas do mundo
Que vem num segundo a sorte mudar
Trazendo pra dentro da gente
As coisas que a mente vai longe buscar
\endverse
\beginverse
Em versos se fala e canta
O mal se espanta e a gente é feliz
No mundo das rimas e trovas
Eu sempre dei provas das coisas que fiz
Por muitos lugares passei
Mas nunca pisei em falso no chão
Cantando interpreto a poesia
Levando alegria onde há solidão
\endverse
\beginverse
O destino é o meu calendário
O meu dicionário é a inspiração
A porta do mundo é aberta
Minha alma desperta
Buscando a canção
Com minha viola no peito
Meus versos são feitos pro mundo cantar
É a luta de um velho talento
Menino por dentro sem nunca cansar
\endverse
%-----------------------------------------------------------------
\vspace{4em} % Regulador de Espaçamento
%-----------------------------------------------------------------
\begin{comment}
\lstset{basicstyle=\scriptsize\bf} % Parâmetros da TAB
%-----------------------------------------------------------------
\tab{Solo 1}
\begin{lstlisting}
E|-----------------------------------------------------|
B|-----------------------------------------------------|
G|-----------------------------------------------------|
D|-----------------------------------------------------|
A|-----------------------------------------------------|
E|-----------------------------------------------------|
\end{lstlisting}
%-----------------------------------------------------------------
\end{comment}
%=================================================================
\begin{comment}

\color{drawChord}\gtab{\color{nameChord} X}{}% 
\color{drawChord}\gtab{\color{nameChord} X}{}% 
\color{drawChord}\gtab{\color{nameChord} X}{}% 
\color{drawChord}\gtab{\color{nameChord} X}{}% 

\end{comment}
%=================================================================
% PADRÃO: [TonalidadeMaior+NOTAX+Variações] .Ex:[X50] [X57V1V7]
% OBS: Variações são alterações do acorde em relação ao campo harmônico.
%-----------------------------------------------------------------
% Tipos de Variações de Acordes:
% V0 - Variação Diversa
% V1 - Menor (m)
% V2 - Maior (M)
% V3 - Meio Tom Abaixo (Bemol)
% V4 - Com Quarta (ex:C4)
% V5 - Com Quinta (ex:C5)
% V6 - Com Sexta (ex:C6)
% V7 - Com Sétima Menor (ex:C7)
% V8 - Com baixo dois Tons Acima (ex:D/F#)
% V9 - Com Nona (ex:C9)
% V10 - Meio Tom Acima (Sustenido)
% V11 - Com Sétima Maior (ex:C7M)
% V12 - Suspenso (Sus)
% V13 - Com baixo dois Tons e Meio Acima (ex:A/E)
% V14 - Com baixo um Tom e Meio Acima (ex:D9/F) 
% V15 - Meio-Diminuto (m7b5)
% N15 - NÃO Meio-Diminuto
% V16 - Diminuto (º)
% N16 - NÃO Diminuto
% V17 - Com baixo um Tom Acima (ex: C/D)
% V18 - Com baixo um Tom Abaixo (ex: Em/D)
% V19 - Com baixo dois Tons e meio Abaixo (ex: G/D)
%=================================================================
%\vspace{4em} % Regulador de Espaçamento
%=================================================================
\newpage
\endsong
%=================================================================
