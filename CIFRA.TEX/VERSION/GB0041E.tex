%=================================================================
\songcolumns{2}
\beginsong
{Eu Vou (África) %TÍTULO
}[by={Fernanda Brum %ARTISTA
},album={@walyssondosreis},
id={GB0041 %COD.ID.: GB0000
}] 
%-----------------------------------------------------------------
\tom{E}
%=================================================================
\newchords{verse1.GB0041E} % Registrador de Acordes em Sequência
%\newchords{chorus1.GB0000X} % Registrador de Acordes em Sequência
%-----------------------------------------------------------------
\seq{Intro}{B}{}
%-----------------------------------------------------------------
%\beginverse* \endverse
%\beginchorus \endchorus
\beginverse*\memorize[verse1.GB0041E]
\[B]Alguém da Ásia me \[E]disse "Vem me aju\[B]dar" \[E]
\[B]Posso ouvir a \[E]África pedir por so\[F\#4]corro \[F\#]
Ru\[G\#m]anda, Somália, Ni\[E]géria
Clamando por um \[G\#m]pouco de amor \[E]
\[G\#m]Vou fazer tudo que eu \[E]posso fazer, eu \[C\#m]vou \[C\#4]\[C\#m]
Tenho \[E]muito pra dar, eu \[C\#m]vou \[C\#4]\[C\#m]
O evan\[E]gelho pregar
\endverse
\beginchorus
\[B]Como ouvi\[E]rão se não há quem \[F\#4]pregue \[F\#]
\[B]Como pregar se nin\[E]guém se dispõe a \[A]ir \[G\#m]\[F\#]
\[B]Como crerão na\[E]queles
De quem nada ou\[F\#4]viram \[F\#]
Eis-me a\[C\#m]qui\[E], eu \[G\#m]vou\[F\#]
Eis-me a\[C\#m]qui\[E], Se\[G\#m]nhor\[F\#]
Eis-me \[C\#m]aqui \[B/D\#]\[E]
\{ Eu \[B]vou, eu vou
Eu \[A]vou, eu vou
Eu \[E]vou, eu vou
Eu \[G]vou \}
\endchorus
\beginverse*\replay[verse1.GB0041E]
^Vi um menino na ^Rússia olhar para o ^céu ^
^El Salvador tá cho^rando por salva^ção ^
Ro^mênia, Arábia Sau^dita, o Iraque
Es^peram por nós ^
Co^lômbia, Indonésia, Al^bânia, pra China
Eu ^vou ^ ^
Pelo ^chão do Brasil, eu ^vou ^ ^
Deus me ^quer pras nações
\endverse


%-----------------------------------------------------------------
\begin{comment}
\lstset{basicstyle=\scriptsize\bf} % Parâmetros da TAB
%-----------------------------------------------------------------
\tab{Solo 1}
\begin{lstlisting}
E|-----------------------------------------------------|
B|-----------------------------------------------------|
G|-----------------------------------------------------|
D|-----------------------------------------------------|
A|-----------------------------------------------------|
E|-----------------------------------------------------|
\end{lstlisting}
%-----------------------------------------------------------------
\end{comment}
%=================================================================
\vspace{2em} 
%-----------------------------------------------------------------
\gtab{\color{black} E}{}% E [X1H] 
\gtab{\color{black} F\#}{}% F# [X2VM]
\gtab{\color{black} F\#4}{}% F#4 [X2VM4]
\gtab{\color{black} G\#m}{}% G#m [X3H]
\gtab{\color{black} G}{}% G [X3VMb]
\gtab{\color{black} A}{}\\% A [X4H]
\gtab{\color{black} B}{}% B [X5H]
\gtab{\color{black} B/D\#}{}% B/D# [B/D\#]
\gtab{\color{black} C\#m}{}% C#m [X6H]
\gtab{\color{black} C\#4}{}% C#4 [X6VM4]
%-----------------------------------------------------------------
% PADRÃO [TonalidadeMaiorNOTAX.Variação] .Ex:[X2H] [X3V1]
% PADRÃO [TonalidadeMenorNOTAX.Variação] .Ex:[mX2H] [mX3V1]
% OBS: Variação são alterações do acorde em relação ao campo harmonico.
%=================================================================
\endsong
%=================================================================
\begin{comment}

\end{comment}