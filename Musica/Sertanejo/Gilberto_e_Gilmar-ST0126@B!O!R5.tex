%=================================================================
\songcolumns{2}
\beginsong
{Assino Com X %TÍTULO
}[by={Gilberto e Gilmar %ARTISTA
},album={@walyssondosreis},
id={\href{https://music.youtube.com/watch?v=mR6dwTDrtr4&feature=share %LINK
}{ST0126 %COD.ID.: XXNNNN
}},rev={5}, %REVISÃO
qr={https://music.youtube.com/watch?v=mR6dwTDrtr4&feature=share %LINK
}]
%-----------------------------------------------------------------
\tom{B}{B}
%=================================================================
%\newchords{verse1.XX0000X} % Registrador de Acordes em Sequência
%\newchords{chorus1.XX0000X} % Registrador de Acordes em Sequência
%-----------------------------------------------------------------
\seq{Intro}{E B F\# B}{1x}
%\act{}{}{}
%-----------------------------------------------------------------
%\beginverse \endverse
%\beginchorus \endchorus
\beginverse
\[B] Não sei de onde vim, não sei pra onde vou
Perdi a memória, não \[B7]sei quem eu \[E]sou
Estou preso num quarto com grades de \[F\#]aço
Estão dizendo que matei por a\[B]mor
\endverse
\beginverse
\[B] Não lembro o que eu fiz, ninguém eu matei
Meu nome não \[B7]sei, assino com \[E]x
Meu Deus não fiz nada que fosse e\[F\#]rrado
Não sou o culpado, me tirem da\[B]qui
\endverse
\beginchorus
\[B] Eles \[B7]pensam que estou fin\[E]gindo
Mas é a pura ver\[B]dade
Perdi o sentido de \[F\#]tudo
Minha mente está apa\[B]gada
Hoje é o \[B7]dia do meu julga\[E]mento
Estou doente, não lembro o pa\[B]ssado
Não tenho nenhuma res\[F\#]posta
Certamente serei conde\[B]nado
(...Certamente serei condena\[B]do)
\endchorus
\act{Riff Intro}{Repetir}{1x}
\act{Refrão}{Repetir}{1x}

%-----------------------------------------------------------------
\vspace{4em} % Regulador de Espaçamento
%-----------------------------------------------------------------
\begin{comment}
\lstset{basicstyle=\scriptsize\bf} % Parâmetros da TAB
%-----------------------------------------------------------------
\tab{Solo 1}
\begin{lstlisting}
E|-----------------------------------------------------|
B|-----------------------------------------------------|
G|-----------------------------------------------------|
D|-----------------------------------------------------|
A|-----------------------------------------------------|
E|-----------------------------------------------------|
\end{lstlisting}
%-----------------------------------------------------------------
\end{comment}
%=================================================================


\color{drawChord}\gtab{\color{nameChord} B}{2:X02220}% 
\color{drawChord}\gtab{\color{nameChord} B7}{2:X02020}% 
\color{drawChord}\gtab{\color{nameChord} E}{~:022100}% 
\color{drawChord}\gtab{\color{nameChord} F\#}{2:022100}% 


%=================================================================
% PADRÃO: [TonalidadeMaior+NOTAX+Variações] .Ex:[F\#0] [F\#7V1V7]
% OBS: Variações são alterações do acorde em relação ao campo harmônico.
%-----------------------------------------------------------------
% Tipos de Variações de Acordes:
% V0 - Variação Diversa
% V1 - Menor (m)
% V2 - Maior (M)
% V3 - Meio Tom Abaixo (Bemol)
% V4 - Com Quarta (ex:C4)
% V5 - Com Quinta (ex:C5)
% V6 - Com Sexta (ex:C6)
% V7 - Com Sétima Menor (ex:C7)
% V8 - Com baixo dois Tons Acima (ex:D/F#)
% V9 - Com Nona (ex:C9)
% V10 - Meio Tom Acima (Sustenido)
% V11 - Com Sétima Maior (ex:C7M)
% V12 - Suspenso (Sus)
% V13 - Com baixo dois Tons e Meio Acima (ex:A/E)
% V14 - Com baixo um Tom e Meio Acima (ex:D9/F) 
% V15 - Meio-Diminuto (m7b5)
% N15 - NÃO Meio-Diminuto
% V16 - Diminuto (º)
% N16 - NÃO Diminuto
% V17 - Com baixo um Tom Acima (ex: C/D)
% V18 - Com baixo um Tom Abaixo (ex: Em/D)
% V19 - Com baixo dois Tons e meio Abaixo (ex: G/D)
%=================================================================
%\vspace{4em} % Regulador de Espaçamento
%=================================================================
\endsong
%=================================================================
