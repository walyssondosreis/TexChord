%=================================================================
\songcolumns{2}
\beginsong
{Os Sonhos De Deus %TÍTULO
}[by={Ludmila Ferber %ARTISTA
},album={@walyssondosreis},
id={GB0081 %COD.ID.: GB0000
},rev={1}, %REVISÃO
qr={ %LINK
}]
%-----------------------------------------------------------------
\tom{X1}{X1}
%=================================================================
%\newchords{verse1.GB0000X} % Registrador de Acordes em Sequência
%\newchords{chorus1.GB0000X} % Registrador de Acordes em Sequência
%-----------------------------------------------------------------
\seq{Intro}{X1 X5 X4}{2x}
%-----------------------------------------------------------------
%\beginverse* \endverse
%\beginchorus \endchorus
\beginverse
Se ten\[X5V8]taram matar os teus \[X4]sonhos
Sufo\[X5V8]cando o teu cora\[X6]ção
Se lan\[X3]çaram você numa \[X4]cova
E fe\[X2]rido perdeu a vi\[X1]são
\endverse
\beginverse
Se ten\[X5V8]taram matar os teus \[X4]sonhos
Sufo\[X5V8]cando o teu cora\[X6]ção
Se lan\[X3]çaram você numa \[X4]cova
E fe\[X2]rido per\[X1V8]deu a \[X5]visão
\endverse
\beginverse
Não de\[X4]sista, não \[X5]pare de \[X6]crer
Os sonhos de \[X1]Deus jamais vão m\[X5]orrer
Não de\[X4]sista não \[X5]pare de l\[X6]utar
Não pare de ado\[X1V8]rar
Le\[X7V3N15]vanta os teus olhos e \[X4]vê
Deus es\[X2]tá restaurando os teus \[X5]sonhos
E a tua vi\[X2]são \[X6] \[(X6)]
\endverse
\beginchorus
Recebe a \[X1]cura
Recebe a un\[X4]ção
Unção de ousa\[X6]dia
Unção de conqu\[X5]ista
Unção de multi\[X4]plicação
\endchorus
\act{Repetir}{Refrão}{+1x}
\act{Retomar}{Verso 1}{1x}
\act{Repetir}{Refrão}{2x}
% Verso de preenchimento
\beginverse*
.
\endverse
%-----------------------------------------------------------------
\begin{comment}
\lstset{basicstyle=\scriptsize\bf} % Parâmetros da TAB
%-----------------------------------------------------------------
\tab{Solo 1}
\begin{lstlisting}
E|-----------------------------------------------------|
B|-----------------------------------------------------|
G|-----------------------------------------------------|
D|-----------------------------------------------------|
A|-----------------------------------------------------|
E|-----------------------------------------------------|
\end{lstlisting}
%-----------------------------------------------------------------
\end{comment}
%=================================================================
\vspace{2em} 
%-----------------------------------------------------------------
\gtab{\color{black} X1}{}% 
\gtab{\color{black} X1V8}{}% 
\gtab{\color{black} X2}{}% 
\gtab{\color{black} X3}{}% 
\gtab{\color{black} X4}{}% 
\gtab{\color{black} X5}{}\\%
\gtab{\color{black} X5V8}{}%
\gtab{\color{black} X6}{}%
\gtab{\color{black} X7V3N15}{}%
%-----------------------------------------------------------------
% PADRÃO: [TonalidadeMaior+NOTAX+Variações] .Ex:[X50] [X57V1V7]
% OBS: Variações são alterações do acorde em relação ao campo harmônico.
%-----------------------------------------------------------------
% Tipos de Variações de Acordes:
% V0 - Variação Diversa
% V1 - Menor (m)
% V2 - Maior (M)
% V3 - Meio Tom Abaixo (Bemol)
% V4 - Com Quarta (ex:C4)
% V5 - Com Quinta (ex:C5)
% V6 - Com Sexta (ex:C6)
% V7 - Com Sétima Menor (ex:C7)
% V8 - Com baixo dois Tons Acima (ex:D/F#)
% V9 - Com Nona (ex:C9)
% V10 - Meio Tom Acima (Sustenido)
% V11 - Com Sétima Maior (ex:C7M)
% V12 - Suspenso (Sus)
% V13 - Com baixo dois Tons e Meio Acima (ex:A/E)
% V14 - Com baixo um Tom e Meio Acima (ex:D9/F) 
% V15 - Meio-Diminuto (m7b5)
% N15 - NÃO Meio-Diminuto
% V16 - Diminuto (º)
% N16 - NÃO Diminuto
%=================================================================
\endsong
%=================================================================
\begin{comment}

\end{comment}