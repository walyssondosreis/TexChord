%=================================================================
\songcolumns{1}
\beginsong
{Isaías 9 %TÍTULO
}[by={Rodolfo Abrantes  %ARTISTA
},album={@walyssondosreis},
id={GB0054 %COD.ID.: GB0000
(Rev.0) %REVISÃO.: 0...N
}] 
%-----------------------------------------------------------------
\tom{X1}
%=================================================================
%\newchords{verse0.GB0000} % Registrador de Acordes em Sequência
%-----------------------------------------------------------------
\seq{Intro}{X1 X5 X6}{4x}
%-----------------------------------------------------------------
%\beginverse* \endverse
%\beginchorus \endchorus

\beginverse*
\[X1]Um me\[X5]nino nas\[X6]ceu
Como um \[X1]filho \[X5]se nos \[X6]deu
Ele é o próprio \[X1]Deus
E \[X5]vive em \[X6]mim
Debaixo de suas \[X1]asas
Eu \[X5]me escon\[X6]di
\endverse

\beginverse*
\[X4] E o Seu \[X6V7]nome é\[X2] maravi\[X2]lhoso
\[X4] Glória ao \[X6V7]Príncipe da \[X5]paz
\endverse

\beginchorus 
O \[X6V7]céu começa a se a\[X3]brir
Toda a \[X6V7]terra se dobrou a Ti\[X4]
\[X1]Cristo, Rei dos \[X3]reis
Veio nos bus\[X4]car
\[X6V7]Leva-nos em Tuas \[X3]mãos
Pelas \[X6V7]portas da cidade\[X4]
\[X1]Na nova Jerusa\[X3]lém
Tua noiva vai en\[X4]trar
\endchorus

\beginverse*
\[X6V7]Santo, \[X3]Santo, \[X6V7]Santo é o Se\[X4]nhor
\endverse
\beginverse*
.
.
.
\endverse
%-----------------------------------------------------------------
\begin{comment}
\lstset{basicstyle=\scriptsize\bf} % Parâmetros da TAB
%-----------------------------------------------------------------
\tab{Solo 1}
\begin{lstlisting}
E|-----------------------------------------------------|
B|-----------------------------------------------------|
G|-----------------------------------------------------|
D|-----------------------------------------------------|
A|-----------------------------------------------------|
E|-----------------------------------------------------|
\end{lstlisting}
%-----------------------------------------------------------------
\end{comment}
%=================================================================
\vspace{2em}
%-----------------------------------------------------------------
\gtab{\color{black} X1}{}% D [X1]
\gtab{\color{black} X2}{}% Em [X2]
\gtab{\color{black} X3}{}% F#m [X3]
\gtab{\color{black} X4}{}% G [X4]
\gtab{\color{black} X5}{}% A [X5]
\gtab{\color{black} X6}{}% Bm [X6]
\gtab{\color{black} X6V7}{}% Bm7 [X6V7]
%-----------------------------------------------------------------
% PADRÃO [TonalidadeMaiorNOTAX.Variação] .Ex:[X50] [X50V1]
% PADRÃO [TonalidadeMenorNOTAX.Variação] .Ex:[mX50] [mX50V1]
% OBS: Variações são alterações do acorde em relação ao campo harmônico.
%-----------------------------------------------------------------
% TIPOS DE VARIAÇÂO DOS ACORDES:
% V0 - ACORDE COM VARIAÇÃO DIVERSA
% V1 - ACORDE MENOR (m)
% V2 - ACORDE MAIOR (M)
% V3 - ACORDE MEIO TOM ABAIXO (Bemois)
% V4 - ACORDE COM QUARTA (C4)
% V5 - ACORDE COM QUINTA (C5)
% V6 - ACORDE COM SEXTA (C6)
% V7 - ACORDE COM SÉTIMA MENOR (C7)
% V8 - ACORDE COM BAIXO DOIS TONS ACIMA (D/F#)
% V9 - ACORDE COM NONA (C9)
% V10 - ACORDE MEIO TOM ACIMA (Sustenidos)
% V11 - ACORDE COM SÉTIMA MAIOR (C7M)
% V12 - ACORDE SUSPENSO (Sus)
% V13 - ACORDE COM BAIXO DOIS TONS E MEIO ACIMA (A/E)
% V14 - ACORDE UM TOM E MEIO ACIMA (D9/F)
%=================================================================
\endsong
%=================================================================
\begin{comment}

\end{comment}