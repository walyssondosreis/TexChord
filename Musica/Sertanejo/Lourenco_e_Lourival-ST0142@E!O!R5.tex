%=================================================================
\songcolumns{2}
\beginsong
{Velha Porteira %TÍTULO
}[by={Lourenço e Lourival %ARTISTA
},album={@walyssondosreis},
id={\href{https://music.youtube.com/watch?v=USV5l8Kwokw&feature=share %LINK
}{ST0142 %COD.ID.: XXNNNN
}},rev={5}, %REVISÃO
qr={https://music.youtube.com/watch?v=USV5l8Kwokw&feature=share %LINK
}]
%-----------------------------------------------------------------
\tom{E}{E}
%=================================================================
\newchords{verse1.ST0142} % Registrador de Acordes em Sequência
\newchords{verse2.ST0142} % Registrador de Acordes em Sequência
%\newchords{chorus1.XX0000X} % Registrador de Acordes em Sequência
%-----------------------------------------------------------------
\seq{Intro}{E B E A B E}{1x}
\act{Solo}{Parte 1 | Parte 2}{1x}
%-----------------------------------------------------------------
%\beginverse \endverse
%\beginchorus \endchorus

\beginverse
\memorize[verse1.ST0142]
Ao pa\[E]ssar pela velha porteira
Senti minha terra mais perto de \[B7]mim
De emoção eu estava chorando
Porque minha an\[A]gústia che\[B7]gava ao \[E]fim
\endverse


\beginverse
\memorize[verse2.ST0142]
Eu con\[E]fesso que era meu sonho
Rever a fa\[E7]zenda onde me cri\[A]ei
Não via che\[B7]gar o mo\[E]mento 
De abraçar de \[B7]novo
Meu querido povo que um dia eu dei\[E]xei
\endverse

\act{Solo: Parte 1 | Parte 2}{Repetir}{1x}

\beginverse
\replay[verse1.ST0142]
Que sur^presa cruel me aguardava
Ao ver a fazenda como transfor^mou
Quase todos dali se mudaram
E a velha co^lônia de^serta fi^cou
\endverse

\beginverse
\replay[verse2.ST0142]
Os a^migos que ali permanecem
Transformaram ^tanto que nem conhe^ci
E eles não ^me conhe^ceram e nem perce^beram
Que os anos passaram e eu envelhe^ci
\endverse

\act{Solo: Parte 1 | Parte 2}{Repetir}{1x}

\beginverse
\replay[verse1.ST0142]
E vo^cê, minha velha porteira
Também não está como outrora dei^xei
Seus mourões pelo tempo roído
No solo ca^ídos tam^bém encon^trei
\endverse

\beginverse
\replay[verse2.ST0142]
Já não ^ouço as suas batidas
Seu triste ran^gido lembrança me ^traz
Porteira na ^reali^dade, você é a sau^dade
Do tempo da infância que não volta ^mais
\endverse

\act{Solo: Parte 2}{Repetir}{1x}

%-----------------------------------------------------------------
\vspace{4em} % Regulador de Espaçamento
%-----------------------------------------------------------------
\begin{comment}
\lstset{basicstyle=\scriptsize\bf} % Parâmetros da TAB
%-----------------------------------------------------------------
\tab{Solo 1}
\begin{lstlisting}
E|-----------------------------------------------------|
B|-----------------------------------------------------|
G|-----------------------------------------------------|
D|-----------------------------------------------------|
A|-----------------------------------------------------|
E|-----------------------------------------------------|
\end{lstlisting}
%-----------------------------------------------------------------
\end{comment}
%=================================================================


\color{drawChord}\gtab{\color{nameChord} E}{~:022100}% 
\color{drawChord}\gtab{\color{nameChord} E7}{~:020100}% 
\color{drawChord}\gtab{\color{nameChord} A}{~:X02220}% 
\color{drawChord}\gtab{\color{nameChord} B}{2:X02220}% 
\color{drawChord}\gtab{\color{nameChord} B7}{2:X02020}% 


%=================================================================
% PADRÃO: [TonalidadeMaior+NOTAX+Variações] .Ex:[B0] [B7V1V7]
% OBS: Variações são alterações do acorde em relação ao campo harmônico.
%-----------------------------------------------------------------
% Tipos de Variações de Acordes:
% V0 - Variação Diversa
% V1 - Menor (m)
% V2 - Maior (M)
% V3 - Meio Tom Abaixo (Bemol)
% V4 - Com Quarta (ex:C4)
% V5 - Com Quinta (ex:C5)
% V6 - Com Sexta (ex:C6)
% V7 - Com Sétima Menor (ex:C7)
% V8 - Com baixo dois Tons Acima (ex:D/F#)
% V9 - Com Nona (ex:C9)
% V10 - Meio Tom Acima (Sustenido)
% V11 - Com Sétima Maior (ex:C7M)
% V12 - Suspenso (Sus)
% V13 - Com baixo dois Tons e Meio Acima (ex:A/E)
% V14 - Com baixo um Tom e Meio Acima (ex:D9/F) 
% V15 - Meio-Diminuto (m7b5)
% N15 - NÃO Meio-Diminuto
% V16 - Diminuto (º)
% N16 - NÃO Diminuto
% V17 - Com baixo um Tom Acima (ex: C/D)
% V18 - Com baixo um Tom Abaixo (ex: Em/D)
% V19 - Com baixo dois Tons e meio Abaixo (ex: G/D)
%=================================================================
%\vspace{4em} % Regulador de Espaçamento
%=================================================================
\endsong
%=================================================================
