%=================================================================
\songcolumns{1}
\beginsong
{Oferta de Amor %TÍTULO
}[by={Ministério Koinonya %ARTISTA
},album={@walyssondosreis},
id={GB0165 %COD.ID.: XXNNNN
},rev={3}, %REVISÃO
qr={https://drive.google.com/open?id=1X7MMxF3OHwXnJ5_-YYkoo-JZmko85i9G %LINK
}]
%-----------------------------------------------------------------
\tom{X1}{A}
%=================================================================
%\newchords{verse1.XX0000X} % Registrador de Acordes em Sequência
%\newchords{chorus1.XX0000X} % Registrador de Acordes em Sequência
%-----------------------------------------------------------------
\seq{Intro}{X1 X5V8 X6 X4 X2 X5}{}
%\act{}{}{}
%-----------------------------------------------------------------
%\beginverse \endverse
%\beginchorus \endchorus
\beginverse
\[X1]Venho, Senhor, minha \[X5V8]vida ofere\[X6]cer
\[X4]Como oferta \[X1V8]de amor e \[X2]sacrifí\[X5]cio
\[X1]Quero minha \[X5V8]vida a \[X6]Ti entregar
\[X4]Como oferta \[X5V8]viva em Teu al\[X1]tar \[(X1V7)]
\endverse
\act{Retomar}{Verso 1}{1x}
\beginchorus
\[X4]Pois \[X5]pra Te ado\[X3]rar \[X6]
\[X4]Foi \[X5]que eu nas\[X6]ci
\[X3]Cumpre em mim 
O \[X6]Teu querer
\[X4]Faça o que es\[X5]tá em Teu cora\[X1]ção
\[X4]E que a cada \[X5]dia 
Eu \[X3]queira mais e \[X6]mais
\[X4]Estar ao Teu \[X5]lado, Se\[X1]nhor
\endchorus
\act{Repetir}{Refrão}{+1x}
\beginverse
\[X4]E que a cada \[X5]dia 
Eu \[X3]queira mais e \[X6]mais
\[X4]Estar ao Teu \[X5]lado, Se\[X1]nhor
\endverse

% Verso de preenchimento
\beginverse*\color{white}\color{white}
.
\endverse
%-----------------------------------------------------------------
\begin{comment}
\lstset{basicstyle=\scriptsize\bf} % Parâmetros da TAB
%-----------------------------------------------------------------
\tab{Solo 1}
\begin{lstlisting}
E|-----------------------------------------------------|
B|-----------------------------------------------------|
G|-----------------------------------------------------|
D|-----------------------------------------------------|
A|-----------------------------------------------------|
E|-----------------------------------------------------|
\end{lstlisting}
%-----------------------------------------------------------------
\end{comment}
%=================================================================
\vspace{2em} 
%-----------------------------------------------------------------
\color{drawChord}\gtab{\color{nameChord} X1}{}% 
\color{drawChord}\gtab{\color{nameChord} X1V7}{}%
\color{drawChord}\gtab{\color{nameChord} X1V8}{}%
\color{drawChord}\gtab{\color{nameChord} X2}{}% 
\color{drawChord}\gtab{\color{nameChord} X3}{}%
\color{drawChord}\gtab{\color{nameChord} X4}{}%
\color{drawChord}\gtab{\color{nameChord} X5}{}%
\color{drawChord}\gtab{\color{nameChord} X5V8}{}%
\color{drawChord}\gtab{\color{nameChord} X6}{}%
%-----------------------------------------------------------------
% PADRÃO: [TonalidadeMaior+NOTAX+Variações] .Ex:[X50] [X57V1V7]
% OBS: Variações são alterações do acorde em relação ao campo harmônico.
%-----------------------------------------------------------------
% Tipos de Variações de Acordes:
% V0 - Variação Diversa
% V1 - Menor (m)
% V2 - Maior (M)
% V3 - Meio Tom Abaixo (Bemol)
% V4 - Com Quarta (ex:C4)
% V5 - Com Quinta (ex:C5)
% V6 - Com Sexta (ex:C6)
% V7 - Com Sétima Menor (ex:C7)
% V8 - Com baixo dois Tons Acima (ex:D/F#)
% V9 - Com Nona (ex:C9)
% V10 - Meio Tom Acima (Sustenido)
% V11 - Com Sétima Maior (ex:C7M)
% V12 - Suspenso (Sus)
% V13 - Com baixo dois Tons e Meio Acima (ex:A/E)
% V14 - Com baixo um Tom e Meio Acima (ex:D9/F) 
% V15 - Meio-Diminuto (m7b5)
% N15 - NÃO Meio-Diminuto
% V16 - Diminuto (º)
% N16 - NÃO Diminuto
% V17 - Com baixo um Tom Acima (ex: C/D)
% V18 - Com baixo um Tom Abaixo (ex: Em/D)
% V19 - Com baixo dois Tons e meio Abaixo (ex: G/D)
%=================================================================
\endsong
%=================================================================
\begin{comment}

\end{comment}