%=================================================================
\songcolumns{1}
\beginsong
{Amizade %TÍTULO
}[by={Quatro Por Um %ARTISTA
},album={@walyssondosreis},
id={GB0010 %COD.ID.: GB0000
},rev={3}, %REVISÃO
qr={https://drive.google.com/open?id=1FPPj0ps6TuPa631JNSVxWLh59OqA5D1A %LINK
}]
%-----------------------------------------------------------------
\tom{A}{A}
%=================================================================
%\newchords{verse1.GB0000X} % Registrador de Acordes em Sequência
%\newchords{chorus1.GB0000X} % Registrador de Acordes em Sequência
%-----------------------------------------------------------------
\seq{Intro}{A E D F\#m E}{} 
%-----------------------------------------------------------------
%\beginverse* \endverse
%\beginchorus \endchorus
\beginverse
\[A] Que bom te \[E]ter aqui co\[D]migo, \[F\#m]\[E]
\[A] Pra conver\[E]sar e te conhe\[D]cer \[F\#m]\[E]
\[A] Entra na \[E]roda e vem co\[D]migo \[F\#m]\[E]
\[A] Só é fe\[E]liz quem tem a\[D]migos \[F\#m]\[E]
\endverse
\beginverse
\[D9] Aproveitar esse mo\[A]mento lindo
\[Bm] Cantar, sorrir, fa\[E]zer amigos
\[D9] Celebrando a Deus que \[A]nos uniu
\[Bm] Como foi bom te \[E]conhecer
\endverse
\beginchorus
\[A] Que bom te \[A/C\#]conhecer
\[D] Pra mim foi \[E]um prazer
\[A] Viver em \[A/C\#]comunhão
A\[D]migos \[A]mais che\[Bm]gados \[E]que ir\[A]mãos \[(E)]
\endchorus
\act{Repetir}{Refrão}{+1x}
\act{Executar}{Riff Intro}{}
\act{Retomar}{Verso 1}{1x}
\act{Repetir}{Refrão}{+2x}
\beginverse
...\[(F\#m)] A\[D]migos \[A]mais che\[Bm]gados \[E]que ir\[A]mãos 
\[(F\#m)] A\[D]migos \[A]mais che\[Bm]gados \[E]que ir\[A]mãos 
\endverse

%-----------------------------------------------------------------
\begin{comment}
\lstset{basicstyle=\scriptsize\bf} % Parâmetros da TAB
%-----------------------------------------------------------------
\tab{Solo 1}
\begin{lstlisting}
E|-----------------------------------------------------|
B|-----------------------------------------------------|
G|-----------------------------------------------------|
D|-----------------------------------------------------|
A|-----------------------------------------------------|
E|-----------------------------------------------------|
\end{lstlisting}
%-----------------------------------------------------------------
\end{comment}
%=================================================================
\vspace{2em} 
%-----------------------------------------------------------------
\color{drawChord}\gtab{\color{nameChord} X}{}% 
\color{drawChord}\gtab{\color{nameChord} X}{}% 
\color{drawChord}\gtab{\color{nameChord} X}{}% 
\color{drawChord}\gtab{\color{nameChord} X}{}% 
%-----------------------------------------------------------------
% PADRÃO: [TonalidadeMaior+NOTAX+Variações] .Ex:[X50] [X57V1V7]
% OBS: Variações são alterações do acorde em relação ao campo harmônico.
%-----------------------------------------------------------------
% Tipos de Variações de Acordes:
% V0 - Variação Diversa
% V1 - Menor (m)
% V2 - Maior (M)
% V3 - Meio Tom Abaixo (Bemol)
% V4 - Com Quarta (ex:C4)
% V5 - Com Quinta (ex:C5)
% V6 - Com Sexta (ex:C6)
% V7 - Com Sétima Menor (ex:C7)
% V8 - Com baixo dois Tons Acima (ex:D/F#)
% V9 - Com Nona (ex:C9)
% V10 - Meio Tom Acima (Sustenido)
% V11 - Com Sétima Maior (ex:C7M)
% V12 - Suspenso (Sus)
% V13 - Com baixo dois Tons e Meio Acima (ex:A/E)
% V14 - Com baixo um Tom e Meio Acima (ex:D9/F) 
% V15 - Meio-Diminuto (m7b5)
% N15 - NÃO Meio-Diminuto
% V16 - Diminuto (º)
% N16 - NÃO Diminuto
%=================================================================
\endsong
%=================================================================
\begin{comment}

\end{comment}