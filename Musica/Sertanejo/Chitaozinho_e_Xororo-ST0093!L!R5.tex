%=================================================================
\songcolumns{2}
\beginsong
{Fio de Cabelo %TÍTULO
}[by={Chitãozinho e Xororó %ARTISTA
},album={@walyssondosreis},
id={\href{https://music.youtube.com/watch?v=zpp0t3scZYs&feature=share %LINK
}{ST0093 %COD.ID.: XXNNNN
}},rev={5}, %REVISÃO
qr={https://music.youtube.com/watch?v=zpp0t3scZYs&feature=share %LINK
}]
%-----------------------------------------------------------------
\tom{}{F\#}
%=================================================================
%\newchords{verse1.XX0000X} % Registrador de Acordes em Sequência
%\newchords{chorus1.XX0000X} % Registrador de Acordes em Sequência
%-----------------------------------------------------------------
\seq{Intro}{...}{1x}
%\act{}{}{}
%-----------------------------------------------------------------
%\beginverse \endverse
%\beginchorus \endchorus
\beginverse
 Quando a gente ama
Qualquer coisa serve para relembrar
Um vestido velho da mulher amada
Tem muito valor
Aquele restinho do perfume dela que ficou no rasco
Sobre a penteadeira mostrando que o quarto
Já foi o cenário de um grande amor
\endverse

\beginchorus
E hoje o que encontrei me deixou mais triste
Um pedacinho dela que existe
Um fio de cabelo no meu paletó
Lembrei de tudo entre nós, do amor vivido
Aquele fio de cabelo comprido
Já esteve grudado em nosso suor
\endchorus
\act{Riff Intro}{Repetir}{1x}
\beginverse
 Quando a gente ama
E não vive junto da mulher amada
Uma coisa à toa
É um bom motivo pra gente chorar
Apagam-se as luzes ao chegar a hora
De ir para a cama
A gente começa a esperar por quem ama
Na impressão que ela venha se deitar
\endverse
\act{Refrão}{Repetir}{1x}
%-----------------------------------------------------------------
\vspace{4em} % Regulador de Espaçamento
%-----------------------------------------------------------------
\begin{comment}
\lstset{basicstyle=\scriptsize\bf} % Parâmetros da TAB
%-----------------------------------------------------------------
\tab{Solo 1}
\begin{lstlisting}
E|-----------------------------------------------------|
B|-----------------------------------------------------|
G|-----------------------------------------------------|
D|-----------------------------------------------------|
A|-----------------------------------------------------|
E|-----------------------------------------------------|
\end{lstlisting}
%-----------------------------------------------------------------
\end{comment}
%=================================================================

\begin{comment}
\color{drawChord}\gtab{\color{nameChord} F\#}{2:022100}% 
\color{drawChord}\gtab{\color{nameChord} F\#7}{2:020100}% 
\color{drawChord}\gtab{\color{nameChord} B}{2:X02220}% 
\color{drawChord}\gtab{\color{nameChord} C\#}{4:X02220}% 
\color{drawChord}\gtab{\color{nameChord} C\#7}{4:X02020}% 
\end{comment}

%=================================================================
% PADRÃO: [TonalidadeMaior+NOTAX+Variações] .Ex:[C\#0] [C\#7V1V7]
% OBS: Variações são alterações do acorde em relação ao campo harmônico.
%-----------------------------------------------------------------
% Tipos de Variações de Acordes:
% V0 - Variação Diversa
% V1 - Menor (m)
% V2 - Maior (M)
% V3 - Meio Tom Abaixo (Bemol)
% V4 - Com Quarta (ex:C4)
% V5 - Com Quinta (ex:C5)
% V6 - Com Sexta (ex:C6)
% V7 - Com Sétima Menor (ex:C7)
% V8 - Com baixo dois Tons Acima (ex:D/F\#)
% V9 - Com Nona (ex:C9)
% V10 - Meio Tom Acima (Sustenido)
% V11 - Com Sétima Maior (ex:C7M)
% V12 - Suspenso (Sus)
% V13 - Com baixo dois Tons e Meio Acima (ex:A/E)
% V14 - Com baixo um Tom e Meio Acima (ex:D9/F) 
% V15 - Meio-Diminuto (m7b5)
% N15 - NÃO Meio-Diminuto
% V16 - Diminuto (º)
% N16 - NÃO Diminuto
% V17 - Com baixo um Tom Acima (ex: C/D)
% V18 - Com baixo um Tom Abaixo (ex: Em/D)
% V19 - Com baixo dois Tons e meio Abaixo (ex: G/D)
%=================================================================
%\vspace{4em} % Regulador de Espaçamento
%=================================================================
\endsong
%=================================================================
