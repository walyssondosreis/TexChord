%=================================================================
\songcolumns{1}
\beginsong
{E Ele Vem %TÍTULO
}[by={Judson Oliveira %ARTISTA
},album={@walyssondosreis},
id={GB0138 %COD.ID.: GB0000
(Rev.0) %REVISÃO.: 0...N
}]
%-----------------------------------------------------------------
\tom{X1}
%=================================================================
%\newchords{verse0.GB0000} % Registrador de Acordes em Sequência
%-----------------------------------------------------------------
\seq{Intro}{X1 X5 X4 X5}{}
%-----------------------------------------------------------------
\begin{verse*}
O \[X1]tempo de cantar che\[X5]gou
O \[X4]tempo de adorar che\[X5]gou
O \[X1]tempo de dançar che\[X5]gou
O \[X4]tempo de louvar che\[X5]gou \[(X5)]
\end{verse*}
%-----------------------------------------------------------------
\begin{verse*}
E ele \[X1]vem,
Ele \[X5]vem
Saltando pelos \[X4]montes
E ele \[X1]vem,
Ele \[X5]vem
Saltando pelos \[X4]montes
Os seus ca\[X1]belos,
Seus ca\[X5]belos
São brancos como a \[X4]neve
E nos seus \[X1]olhos,
Nos seus \[X5]olhos
Há \[X4]fogo!!! \[(X4)]
\end{verse*}
%-----------------------------------------------------------------
\begin{chorus}
Incen\[X1]deia Senhor a sua \[X5]noiva
Incen\[X4]deia Senhor a sua i\[X5]greja
Incen\[X1]deia Senhor a sua \[X5]casa
Vem me in\[X4]cen\[X1V8]di\[X2V7]ar \[(X2V7)]
\end{chorus}
%-----------------------------------------------------------------
\begin{comment}
\lstset{basicstyle=\scriptsize\bf} % Parâmetros da TAB
%-----------------------------------------------------------------
\tab{Solo 1}
\begin{lstlisting}
E|-----------------------------------------------------|
B|-----------------------------------------------------|
G|-----------------------------------------------------|
D|-----------------------------------------------------|
A|-----------------------------------------------------|
E|-----------------------------------------------------|
\end{lstlisting}
%-----------------------------------------------------------------
\end{comment}
%=================================================================
\vspace{2em}
%-----------------------------------------------------------------
\gtab{\color{black} X1}{}% G [X1]
\gtab{\color{black} X1V8}{}% G/B [X1V8]
\gtab{\color{black} X2V7}{}% Am7 [X2V7]
\gtab{\color{black} X4}{}% C [X4]
\gtab{\color{black} X5}{}% D [X5]
%-----------------------------------------------------------------
% PADRÃO [TonalidadeMaiorNOTAX.Variação] .Ex:[X50] [X50V1]
% PADRÃO [TonalidadeMenorNOTAX.Variação] .Ex:[mX50] [mX50V1]
% OBS: Variações são alterações do acorde em relação ao campo harmônico.
%-----------------------------------------------------------------
% TIPOS DE VARIAÇÂO DOS ACORDES:
% V0 - ACORDE COM VARIAÇÃO DIVERSA
% V1 - ACORDE MENOR (m)
% V2 - ACORDE MAIOR (M)
% V3 - ACORDE MEIO TOM ABAIXO (Bemois)
% V4 - ACORDE COM QUARTA (C4)
% V5 - ACORDE COM QUINTA (C5)
% V6 - ACORDE COM SEXTA (C6)
% V7 - ACORDE COM SÉTIMA MENOR (C7)
% V8 - ACORDE COM BAIXO DOIS TONS ACIMA (D/F#)
% V9 - ACORDE COM NONA (C9)
% V10 - ACORDE MEIO TOM ACIMA (Sustenidos)
% V11 - ACORDE COM SÉTIMA MAIOR (C7M)
% V12 - ACORDE SUSPENSO (Sus)
% V13 - ACORDE COM BAIXO DOIS TONS E MEIO ACIMA (A/E)
% V14 - ACORDE UM TOM E MEIO ACIMA (D9/F)
%=================================================================
\endsong
%=================================================================








