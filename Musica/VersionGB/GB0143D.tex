%=================================================================
\songcolumns{2}
\beginsong
{O Hino %TÍTULO
}[by={Fernandinho %ARTISTA
},album={@walyssondosreis},
id={GB0143 %COD.ID.: GB0000
},rev={3}, %REVISÃO
qr={https://drive.google.com/open?id=1HS8r7AFpIes8zz1tAXks_90yTNAg9mAe %LINK
}]
%-----------------------------------------------------------------
\tom{D}{F}
%=================================================================
%\newchords{verse1.GB0000X} % Registrador de Acordes em Sequência
%\newchords{chorus1.GB0000X} % Registrador de Acordes em Sequência
%-----------------------------------------------------------------
\seq{Intro}{Bm G D A}{2x}
%-----------------------------------------------------------------
%\beginverse* \endverse
%\beginchorus \endchorus
\beginverse
\[Bm]Posso ouvir os passos do meu \[G]Rei
\[D]Posso escutar Seu cora\[A]ção
Da \[Bm]minha escuridão me liber\[G]tou
\[D]Posso ouvir Seu Espírito cla\[A]mar
\endverse
\beginverse
Eu ouço: ^Filho meu, che^gou a hora de bri^lhar
Você nasceu pra ^Mim
Eu ouço: ^Filho meu, che^gou tua hora de bri^lhar
Você nasceu pra ^Mim
( Pra esse tem\[Bm]po )
\endverse
\seq{Riff Intro}{Bm G D A}{}
\beginverse
^Posso ouvir o santo som a^qui
A^nunciando sobre um grande ^Rei
Rom^pendo as barreiras e gri^lhões
Eu ^quero ouvir o Pai a me cha^mar
\endverse
\act{Repetir}{Verso 2}{2x}
\act{Executar}{Solo 1}{}
\beginchorus
^ Esse é o hino dessa geração ^
Usa-nos Deus, aqui estamos ^
Queremos mais de Ti, Senhor
O Teu a^mor nos alcançou
\endchorus
\act{Repetir}{Refrão}{+3x}
\act{Executar}{Solo 2}{}
\act{Excutar}{Riff 2}{}
\beginverse
\[Bm]Ah! Jesus é Rei
\[G]Ah! Me conquistou
\[Bm]Ah! Me libertou
\[G]Pra viver uma nova história
\endverse
\act{Repetir}{Verso 4}{+5x}
\act{Executar}{Riff 3}{}
\beginverse
Vamos mudar o \[Bm]mundo
Vamos mudar o \[G]mundo
Vamos mudar o \[D]mundo
Vamos mudar o \[A]mundo
\endverse
%-----------------------------------------------------------------
\vspace{4em} % Regulador de Espaçamento
%-----------------------------------------------------------------
\begin{comment}
\lstset{basicstyle=\scriptsize\bf} % Parâmetros da TAB
%-----------------------------------------------------------------
\tab{Solo 1}
\begin{lstlisting}
E|-----------------------------------------------------|
B|-----------------------------------------------------|
G|-----------------------------------------------------|
D|-----------------------------------------------------|
A|-----------------------------------------------------|
E|-----------------------------------------------------|
\end{lstlisting}
%-----------------------------------------------------------------
\end{comment}
%=================================================================
 
%-----------------------------------------------------------------
\color{drawChord}\gtab{\color{nameChord} D}{}% 
\color{drawChord}\gtab{\color{nameChord} G}{}% 
\color{drawChord}\gtab{\color{nameChord} A}{}% 
\color{drawChord}\gtab{\color{nameChord} Bm}{}% 
%-----------------------------------------------------------------
% PADRÃO: [TonalidadeMaior+NOTAX+Variações] .Ex:[X50] [X57V1V7]
% OBS: Variações são alterações do acorde em relação ao campo harmônico.
%-----------------------------------------------------------------
% Tipos de Variações de Acordes:
% V0 - Variação Diversa
% V1 - Menor (m)
% V2 - Maior (M)
% V3 - Meio Tom Abaixo (Bemol)
% V4 - Com Quarta (ex:C4)
% V5 - Com Quinta (ex:C5)
% V6 - Com Sexta (ex:C6)
% V7 - Com Sétima Menor (ex:C7)
% V8 - Com baixo dois Tons Acima (ex:D/F#)
% V9 - Com Nona (ex:C9)
% V10 - Meio Tom Acima (Sustenido)
% V11 - Com Sétima Maior (ex:C7M)
% V12 - Suspenso (Sus)
% V13 - Com baixo dois Tons e Meio Acima (ex:A/E)
% V14 - Com baixo um Tom e Meio Acima (ex:D9/F) 
% V15 - Meio-Diminuto (m7b5)
% N15 - NÃO Meio-Diminuto
% V16 - Diminuto (º)
% N16 - NÃO Diminuto
%=================================================================
\endsong
%=================================================================
\begin{comment}

\end{comment}