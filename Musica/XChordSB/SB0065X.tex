%=================================================================
\songcolumns{1}
\beginsong
{Na Hora do Adeus %TÍTULO
}[by={Matogrosso \& Mathias %ARTISTA
},album={@walyssondosreis},
id={SB0065 %COD.ID.: XXNNNN
},rev={0}, %REVISÃO
qr={https://music.youtube.com/watch?v=yQsiGqOUPk0&feature=share %LINK
}]
%-----------------------------------------------------------------
\tom{X1}{C}
%=================================================================
%\newchords{verse1.XX0000X} % Registrador de Acordes em Sequência
%\newchords{chorus1.XX0000X} % Registrador de Acordes em Sequência
%-----------------------------------------------------------------
\seq{Intro}{X6 X3V2V7 X6 X3V2V7 X6}{1x}
%\act{}{}{}
%-----------------------------------------------------------------
%\beginverse \endverse
%\beginchorus \endchorus
\beginverse 
\[X6] Na hora do a\[X3V2V7]deus você olhou pra \[X6]mim
E não acredi\[X3V2V7]tou ao ver chegar o \[X6]fim \[X6V2V7]
Tentou me sedu\[X2]zir \[X5V7], chorando me aga\[X1]rrou
\[X6] Teu corpo ofere\[X2]ceu \[X3V2V7], pediu e supli\[X6]cou \[X6V2V7]
\endverse

\beginverse 
E perguntou, \[X2]por quê? \[X5V7]
Mas eu não respon\[X1]di
\[X6] Só pra não te ofen\[X2]der, disse: A\[X3V2V7]deus, e sa\[X6]í \[X6V2V7]
\endverse

\beginchorus
Saí da tua \[X2]vida, eu só represen\[X5V7]tava
O cheque no fi\[X1]nal do mês
Você não respei\[X6]tou quem te amou \[X2]demais
Só abusou de \[X3V2V7]mim e me passou \[X6]pra trás \[X6V2V7]
Saí da tua \[X2]vida de cabeça er\[X5V7]guida
Coisa que vo\[X1]cê não fez
Eu já chorei de\[X6]mais, agora vem a \[X2]sua vez
Eu acho que vai \[X3V2V7]ser melhor
Melhor pra todos \[X6]três
\endchorus
\act{Riff Intro}{Repetir}{1x}
\beginverse
Na hora do a\[X3V2V7]deus, você olhou pra \[X6]mim
E não acredi\[X3V2V7]tou ao ver chegar no \[X6]fim
\endverse

%-----------------------------------------------------------------
\vspace{4em} % Regulador de Espaçamento
%-----------------------------------------------------------------
\begin{comment}
\lstset{basicstyle=\scriptsize\bf} % Parâmetros da TAB
%-----------------------------------------------------------------
\tab{Solo 1}
\begin{lstlisting}
E|-----------------------------------------------------|
B|-----------------------------------------------------|
G|-----------------------------------------------------|
D|-----------------------------------------------------|
A|-----------------------------------------------------|
E|-----------------------------------------------------|
\end{lstlisting}
%-----------------------------------------------------------------
\end{comment}
%=================================================================
\color{drawChord}\gtab{\color{nameChord} X1}{}% 
\color{drawChord}\gtab{\color{nameChord} X2}{}% 
\color{drawChord}\gtab{\color{nameChord} X3V2V7}{}% 
\color{drawChord}\gtab{\color{nameChord} }{}% 

%=================================================================
% PADRÃO: [TonalidadeMaior+NOTAX+Variações] .Ex:[X50] [X57V1V7]
% OBS: Variações são alterações do acorde em relação ao campo harmônico.
%-----------------------------------------------------------------
% Tipos de Variações de Acordes:
% V0 - Variação Diversa
% V1 - Menor (m)
% V2 - Maior (M)
% V3 - Meio Tom Abaixo (Bemol)
% V4 - Com Quarta (ex:C4)
% V5 - Com Quinta (ex:C5)
% V6 - Com Sexta (ex:C6)
% V7 - Com Sétima Menor (ex:C7)
% V8 - Com baixo dois Tons Acima (ex:D/F#)
% V9 - Com Nona (ex:C9)
% V10 - Meio Tom Acima (Sustenido)
% V11 - Com Sétima Maior (ex:C7M)
% V12 - Suspenso (Sus)
% V13 - Com baixo dois Tons e Meio Acima (ex:A/E)
% V14 - Com baixo um Tom e Meio Acima (ex:D9/F) 
% V15 - Meio-Diminuto (m7b5)
% N15 - NÃO Meio-Diminuto
% V16 - Diminuto (º)
% N16 - NÃO Diminuto
% V17 - Com baixo um Tom Acima (ex: C/D)
% V18 - Com baixo um Tom Abaixo (ex: Em/D)
% V19 - Com baixo dois Tons e meio Abaixo (ex: G/D)
%=================================================================
%\vspace{4em} % Regulador de Espaçamento
%=================================================================
\endsong
%=================================================================
