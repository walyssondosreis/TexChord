%=================================================================
\songcolumns{1}
\beginsong
{Agindo Deus %TÍTULO
}[by={Fernandinho %ARTISTA
},album={@walyssondosreis},
id={GB0007 %COD.ID.: GB0000
},rev={3}, %REVISÃO
qr={https://drive.google.com/open?id=1YDpM8Pgh7TRx5SvUL_OnqkxKUn-rZMDh %LINK
}]
%-----------------------------------------------------------------
\tom{A}{C}
%=================================================================
%\newchords{verse1.GB0000X} % Registrador de Acordes em Sequência
%\newchords{chorus1.GB0000X} % Registrador de Acordes em Sequência
%-----------------------------------------------------------------
\seq{Intro}{F\#m}{}
%-----------------------------------------------------------------
%\beginverse* \endverse
%\beginchorus \endchorus
\beginverse
\[F\#m] As dores, as lutas, angústias e aflições
O choro, o medo, as perdas e solidão \[(F\#m A B) 2x]
\endverse
\act{Repetir}{Verso 1}{1x}
\beginverse
\[F\#m]Somos entregues a morte todo \[E]dia
Por \[D]amor a \[F\#m]ti \[D]\[E]
\[F\#m]Somos a geração que se le\[E]vanta e nunca vai desis\[B]tir
\endverse
\beginchorus
A\[F\#m]gindo Deus, \[Bm]quem impedi\[A]rá? \[E]
A\[D]gindo Deus, \[Bm]quem impedi\[A]rá? \[E]
A\[F\#m]gindo Deus, \[Bm]quem impedi\[A]rá? \[E]
A\[D]gindo Deus, \[Bm]quem impedi\[A]rá? \[E]
\endchorus
\act{Executar}{Riff Intro}{}
\act{Repetir}{Verso 1}{1x}
\act{Repetir}{Verso 2}{2x}
\act{Repetir}{Refrão}{1x}
\act{Executar}{Solo 1}{}
\act{Repetir}{Refrão}{2x}
\act{Executar}{Solo 2}{}
% Verso de preenchimento
\beginverse*\color{white}
.
.
.
\endverse


%-----------------------------------------------------------------
\begin{comment}
\lstset{basicstyle=\scriptsize\bf} % Parâmetros da TAB
%-----------------------------------------------------------------
\tab{Solo 1}
\begin{lstlisting}
E|-----------------------------------------------------|
B|-----------------------------------------------------|
G|-----------------------------------------------------|
D|-----------------------------------------------------|
A|-----------------------------------------------------|
E|-----------------------------------------------------|
\end{lstlisting}
%-----------------------------------------------------------------
\end{comment}
%=================================================================
\vspace{2em} 
%-----------------------------------------------------------------
\color{drawChord}\gtab{\color{nameChord} A}{}% 
\color{drawChord}\gtab{\color{nameChord} Bm}{}% 
\color{drawChord}\gtab{\color{nameChord} B}{}% 
\color{drawChord}\gtab{\color{nameChord} D}{}%
\color{drawChord}\gtab{\color{nameChord} E}{}% 
\color{drawChord}\gtab{\color{nameChord} F\#m}{}% 
%-----------------------------------------------------------------
% PADRÃO: [TonalidadeMaior+NOTAX+Variações] .Ex:[X50] [X57V1V7]
% OBS: Variações são alterações do acorde em relação ao campo harmônico.
%-----------------------------------------------------------------
% Tipos de Variações de Acordes:
% V0 - Variação Diversa
% V1 - Menor (m)
% V2 - Maior (M)
% V3 - Meio Tom Abaixo (Bemol)
% V4 - Com Quarta (ex:C4)
% V5 - Com Quinta (ex:C5)
% V6 - Com Sexta (ex:C6)
% V7 - Com Sétima Menor (ex:C7)
% V8 - Com baixo dois Tons Acima (ex:D/F#)
% V9 - Com Nona (ex:C9)
% V10 - Meio Tom Acima (Sustenido)
% V11 - Com Sétima Maior (ex:C7M)
% V12 - Suspenso (Sus)
% V13 - Com baixo dois Tons e Meio Acima (ex:A/E)
% V14 - Com baixo um Tom e Meio Acima (ex:D9/F) 
% V15 - Meio-Diminuto (m7b5)
% N15 - NÃO Meio-Diminuto
% V16 - Diminuto (º)
% N16 - NÃO Diminuto
%=================================================================
\endsong
%=================================================================
\begin{comment}

\end{comment}