%=================================================================
\songcolumns{1}
\beginsong
{Foi Por Amor %TÍTULO
}[by={Adoração e Adoradores %ARTISTA
},album={@walyssondosreis},
id={GB0048 %COD.ID.: GB0000
},rev={0}, %REVISÃO
qr={https://drive.google.com/open?id=1YRRjajSopcHXBHlShOeD8dZsSZ7qb3lp %LINK
}]
%-----------------------------------------------------------------
\tom{X1}{B}
%=================================================================
%\newchords{verse1.GB0000X} % Registrador de Acordes em Sequência
%\newchords{chorus1.GB0000X} % Registrador de Acordes em Sequência
%-----------------------------------------------------------------
%\seq{Intro}{}{}
%-----------------------------------------------------------------
%\beginverse* \endverse
%\beginchorus \endchorus
\beginverse*
Foi por amor que Jesus se entregou numa cruz
Por nos amar, do pecado ele nos libertou
Não há no mundo nada que vence esse amor
\endverse
\beginverse*
Eu sei, é tão difícil entender a morte de Jesus
A sua dor, o seu sofrer
Tudo o que ele passou por mim e por você
\endverse
\beginverse*
A morte tentou vencer o amor de Jesus
Mas ele venceu e hoje vive
E ele nos ensinou que tudo passará
E o que realmente irá contar, é o amor
\endverse
\beginchorus
Foi por amor, que Jesus se entregou numa cruz
Por nos amar, do pecado ele nos libertou
Não há no mundo, nada que vence esse amor
\endchorus
\beginverse*
Nada pode vencer teu sacrifício na cruz
Nada pode vencer o teu amor meu Jesus
Nada pode vencer teu sacrifício Jesus
\endverse


% Verso de preenchimento
\beginverse*\color{white}
.
.
.
\endverse
%-----------------------------------------------------------------
\begin{comment}
\lstset{basicstyle=\scriptsize\bf} % Parâmetros da TAB
%-----------------------------------------------------------------
\tab{Solo 1}
\begin{lstlisting}
E|-----------------------------------------------------|
B|-----------------------------------------------------|
G|-----------------------------------------------------|
D|-----------------------------------------------------|
A|-----------------------------------------------------|
E|-----------------------------------------------------|
\end{lstlisting}
%-----------------------------------------------------------------
\end{comment}
%=================================================================
 
%-----------------------------------------------------------------
\color{drawChord}\gtab{\color{nameChord} X}{}% 
\color{drawChord}\gtab{\color{nameChord} X}{}% 
\color{drawChord}\gtab{\color{nameChord} X}{}% 
\color{drawChord}\gtab{\color{nameChord} X}{}% 
%-----------------------------------------------------------------
% PADRÃO: [TonalidadeMaior+NOTAX+Variações] .Ex:[X50] [X57V1V7]
% OBS: Variações são alterações do acorde em relação ao campo harmônico.
%-----------------------------------------------------------------
% Tipos de Variações de Acordes:
% V0 - Variação Diversa
% V1 - Menor (m)
% V2 - Maior (M)
% V3 - Meio Tom Abaixo (Bemol)
% V4 - Com Quarta (ex:C4)
% V5 - Com Quinta (ex:C5)
% V6 - Com Sexta (ex:C6)
% V7 - Com Sétima Menor (ex:C7)
% V8 - Com baixo dois Tons Acima (ex:D/F#)
% V9 - Com Nona (ex:C9)
% V10 - Meio Tom Acima (Sustenido)
% V11 - Com Sétima Maior (ex:C7M)
% V12 - Suspenso (Sus)
% V13 - Com baixo dois Tons e Meio Acima (ex:A/E)
% V14 - Com baixo um Tom e Meio Acima (ex:D9/F) 
% V15 - Meio-Diminuto (m7b5)
% N15 - NÃO Meio-Diminuto
% V16 - Diminuto (º)
% N16 - NÃO Diminuto
%=================================================================
\endsong
%=================================================================
\begin{comment}

\end{comment}