%=================================================================
\songcolumns{2}
\beginsong
{Poderoso Deus %TÍTULO
}[by={Santa Geração %ARTISTA
},album={@walyssondosreis},
id={GB0087 %COD.ID.: GB0000
},rev={3}, %REVISÃO
qr={https://drive.google.com/open?id=1nG8MJXDBdR5563z1DAQqhCE6Hh9zplGX %LINK
}]
%-----------------------------------------------------------------
\tom{X1}{D}
%=================================================================
%\newchords{verse1.GB0000X} % Registrador de Acordes em Sequência
%\newchords{chorus1.GB0000X} % Registrador de Acordes em Sequência
%-----------------------------------------------------------------
\seq{Intro}{X1 X6V7 X4 X1V8 X2V7 X5V4 X5}{2x}
%-----------------------------------------------------------------
%\beginverse* \endverse
%\beginchorus \endchorus
\beginverse
Ao que está sen\[X1]tado no \[X6V7]trono
E ao Cor\[X4]deiro\[X1V8]\[X2V7] seja o lou\[X5V4]vor \[X5]
\endverse
\act{Repetir}{Verso 1}{+1x}
\beginverse
Seja a ^honra, seja a ^glória, seja o do^mínio^^
Pelos séculos dos ^séculos ^
\endverse
\act{Repetir}{Verso 2}{+1x}
\beginchorus
Poderoso ^Deus, poderoso ^Deus
Poderoso ^Deus^^, minh'alma anseia por ^Ti ^
\endchorus
\act{Repetir}{Refrão}{+1x}
\act{Executar}{Riff Intro}{}
\act{Retomar}{Verso 1}{1x}
\act{Repetir}{Refrão}{+2x}
\beginverse
Minh'alma anseia por \[X2V7]Ti
Minh'alma \[X1V8]anseia por \[X4]Ti
Minh'alma anseia por \[X5]Ti
Minh'alma \[X5V4]anseia por \[X5]Ti
\endverse
\act{Repetir}{Verso 3}{+2x}
\beginverse
Como a corsa suspira pelas \[X2V7]águas \[X1V8]\[X4]
Minh'alma anseia por \[X5]Ti
Minh'alma \[X5V4]anseia por \[X5]Ti
\endverse
\act{Repetir}{Verso 4}{+2}
\act{Rpetir}{Verso 3}{1x}
\act{Executar}{Solo 1}{}
\beginverse
Abro minha boca e sus\[X2V7]piro \[X1V8]\[X4]
Pois tenho sede de \[X5]Ti \[X5V4]
\endverse
\act{Repetir}{Verso 5}{+1x}
\act{Repetir}{Verso 4}{1x}
\act{Repetir}{Verso 5}{1x}
\beginverse
... Eu tenho sede de \[X2V7]ti \[X1V8]\[X4]
Tenho sede de \[X5]ti \[X5V4]
\endverse
%-----------------------------------------------------------------
\begin{comment}
\lstset{basicstyle=\scriptsize\bf} % Parâmetros da TAB
%-----------------------------------------------------------------
\tab{Solo 1}
\begin{lstlisting}
E|-----------------------------------------------------|
B|-----------------------------------------------------|
G|-----------------------------------------------------|
D|-----------------------------------------------------|
A|-----------------------------------------------------|
E|-----------------------------------------------------|
\end{lstlisting}
%-----------------------------------------------------------------
\end{comment}
%=================================================================
\vspace{2em} 
%-----------------------------------------------------------------
\color{drawChord}\gtab{\color{nameChord} X1}{}% 
\color{drawChord}\gtab{\color{nameChord} X1V8}{}% 
\color{drawChord}\gtab{\color{nameChord} X2V7}{}% 
\color{drawChord}\gtab{\color{nameChord} X4}{}% 
\color{drawChord}\gtab{\color{nameChord} X5}{}% 
\color{drawChord}\gtab{\color{nameChord} X5V4}{}\\% 
\color{drawChord}\gtab{\color{nameChord} X6V7}{}% 

%-----------------------------------------------------------------
% PADRÃO: [TonalidadeMaior+NOTAX+Variações] .Ex:[X50] [X57V1V7]
% OBS: Variações são alterações do acorde em relação ao campo harmônico.
%-----------------------------------------------------------------
% Tipos de Variações de Acordes:
% V0 - Variação Diversa
% V1 - Menor (m)
% V2 - Maior (M)
% V3 - Meio Tom Abaixo (Bemol)
% V4 - Com Quarta (ex:C4)
% V5 - Com Quinta (ex:C5)
% V6 - Com Sexta (ex:C6)
% V7 - Com Sétima Menor (ex:C7)
% V8 - Com baixo dois Tons Acima (ex:D/F#)
% V9 - Com Nona (ex:C9)
% V10 - Meio Tom Acima (Sustenido)
% V11 - Com Sétima Maior (ex:C7M)
% V12 - Suspenso (Sus)
% V13 - Com baixo dois Tons e Meio Acima (ex:A/E)
% V14 - Com baixo um Tom e Meio Acima (ex:D9/F) 
% V15 - Meio-Diminuto (m7b5)
% N15 - NÃO Meio-Diminuto
% V16 - Diminuto (º)
% N16 - NÃO Diminuto
%=================================================================
\endsong
%=================================================================
\begin{comment}

\end{comment}