%=================================================================
\songcolumns{2}
\beginsong
{Rede Ao Mar %TÍTULO
}[by={Adoração e Adoradores %ARTISTA
},album={@walyssondosreis},
id={GB0097 %COD.ID.: GB0000
},rev={3}, %REVISÃO
qr={https://drive.google.com/open?id=10vODSbP2UWRX0JrdK66ypEzU81J2Gg_4 %LINK
}]
%-----------------------------------------------------------------
\tom{X1}{A}
%=================================================================
%\newchords{verse1.GB0000X} % Registrador de Acordes em Sequência
%\newchords{chorus1.GB0000X} % Registrador de Acordes em Sequência
%-----------------------------------------------------------------
\seq{Intro}{X1 X5 X4}{2x}
\seq{Intro}{X6 X5 X4}{2x}
%-----------------------------------------------------------------
%\beginverse* \endverse
%\beginchorus \endchorus
\beginverse
\[X1] Não po\[X2]dia enten\[X6]der \[X4] 
\[X1] Que vo\[X2]cê ia me que\[X6]rer \[X4]
\[X1] Lançou a \[X2]rede ao mar \[X6]
E que\[X4]rendo me pe\[X1]gar \[X2]
Pegou meu \[X6]coração \[X4]
\endverse
\seq{Riff 1}{X1 X5 X4}{2x}
\beginverse
^ Hoje ^eu estou a^qui ^ 
^ Pois vo^cê me esco^lheu ^
^ Agora ^posso enten^der 
E o ^mesmo eu vou fa^zer
Vou lançar ^
A minha ^rede ao mar ^
\endverse
\beginchorus
\[X1] Vou lan\[X5]çar a minha \[X4]rede ao mar
\[X1] Muitas \[X5]almas também \[X4]vou ganhar
\[X6] Tantas \[X5]que eu não pode\[X4]rei contar
\[X6]Almas como as \[X5]areias do \[X4]mar
\chordsoff(Almas como as areias do mar)
\chordson
\endchorus
\seq{Riff 1}{X1 X5 X4}{}
\act{Repetir}{Verso 1}{1x}
\act{Repetir}{Verso 2}{1x}
\act{Repetir}{Refrão}{2x}
\beginverse
\[X6]Ide, \[X5] fazei dis\[X4]cípulos
De \[X6]todas \[X5] as na\[X4]ções \[(X5)]
\endverse
\act{Repetir}{Verso 3}{+4x}
\beginchorus
\act{Variar}{+1 Tom}{}
\[X1.] Vou lan\[X5.]çar a minha \[X4.]rede ao mar
\[X1.] Muitas \[X5.]almas também \[X4.]vou ganhar
\[X6.] Tantas \[X5.]que eu não pode\[X4.]rei contar
\[X6.]Almas como as \[X5.]areias do \[X4.]mar
\chordsoff(Almas como as areias do mar)
\chordson
\endchorus
\act{Repetir}{Refrão}{+1x}
\beginverse
...\[X6.]Almas como as \[X5.]areias do \[X4.]mar
\chordsoff(Almas como as areias do mar)
\chordson \[X6.]Almas como as \[X5.]areias do \[X4.]mar
\chordsoff(Almas como as areias do mar)
\chordson
\endverse
%-----------------------------------------------------------------
\vspace{4em} % Regulador de Espaçamento
%-----------------------------------------------------------------
\begin{comment}
\lstset{basicstyle=\scriptsize\bf} % Parâmetros da TAB
%-----------------------------------------------------------------
\tab{Solo 1}
\begin{lstlisting}
E|-----------------------------------------------------|
B|-----------------------------------------------------|
G|-----------------------------------------------------|
D|-----------------------------------------------------|
A|-----------------------------------------------------|
E|-----------------------------------------------------|
\end{lstlisting}
%-----------------------------------------------------------------
\end{comment}
%=================================================================
 
%-----------------------------------------------------------------
\color{drawChord}\gtab{\color{nameChord} X1}{}% 
\color{drawChord}\gtab{\color{nameChord} X1.}{}% 
\color{drawChord}\gtab{\color{nameChord} X2}{}% 
\color{drawChord}\gtab{\color{nameChord} X4}{}%
\color{drawChord}\gtab{\color{nameChord} X4.}{}% 
\color{drawChord}\gtab{\color{nameChord} X5}{}\\%
\color{drawChord}\gtab{\color{nameChord} X5.}{}% 
\color{drawChord}\gtab{\color{nameChord} X6}{}%
\color{drawChord}\gtab{\color{nameChord} X6.}{}% 
%-----------------------------------------------------------------
% PADRÃO: [TonalidadeMaior+NOTAX+Variações] .Ex:[X50] [X57V1V7]
% OBS: Variações são alterações do acorde em relação ao campo harmônico.
%-----------------------------------------------------------------
% Tipos de Variações de Acordes:
% V0 - Variação Diversa
% V1 - Menor (m)
% V2 - Maior (M)
% V3 - Meio Tom Abaixo (Bemol)
% V4 - Com Quarta (ex:C4)
% V5 - Com Quinta (ex:C5)
% V6 - Com Sexta (ex:C6)
% V7 - Com Sétima Menor (ex:C7)
% V8 - Com baixo dois Tons Acima (ex:D/F#)
% V9 - Com Nona (ex:C9)
% V10 - Meio Tom Acima (Sustenido)
% V11 - Com Sétima Maior (ex:C7M)
% V12 - Suspenso (Sus)
% V13 - Com baixo dois Tons e Meio Acima (ex:A/E)
% V14 - Com baixo um Tom e Meio Acima (ex:D9/F) 
% V15 - Meio-Diminuto (m7b5)
% N15 - NÃO Meio-Diminuto
% V16 - Diminuto (º)
% N16 - NÃO Diminuto
%=================================================================
\endsong
%=================================================================
\begin{comment}

\end{comment}