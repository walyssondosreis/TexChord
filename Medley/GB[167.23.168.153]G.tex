%=================================================================
\songcolumns{2}
\beginsong
{Medley %TÍTULO
}[by={Vários Artistas %ARTISTA
},album={@walyssondosreis},
id={GB[167.23.168.153] %COD.ID.: XXNNNN
},rev={0}, %REVISÃO
qr={ %LINK
}]
%-----------------------------------------------------------------
\tom{G}{D}
%=================================================================
%\newchords{verse1.XX0000X} % Registrador de Acordes em Sequência
%\newchords{chorus1.XX0000X} % Registrador de Acordes em Sequência
%-----------------------------------------------------------------
\act{\bf\color{red} ATRAI O MEU CORAÇÃO - Filhos do Homem\color{black}}{}{Música Inteira}
\seq{Intro}{G9 G7M(9) G4(9) C/E}{2x}
%\act{}{}{}
%-----------------------------------------------------------------
%\beginverse \endverse
%\beginchorus \endchorus
\beginverse
\[G] Tu és minha \[C]vida, Je\[D/F\#]sus
És meu a\[G]migo \[(Em)]
E a tua von\[C]tade, Doce Es\[D/F\#]pírito
Meu ali\[G]mento
\endverse
\beginverse
Sem \[C]ti não há valor em \[G/B]mim, sou como um vaso de \[Am]barro
Pronto a ser que\[C]brado para ser o que \[D4]queres de \[D]mim
\endverse
\beginverse
A tua pre\[C]sença é tudo que eu pre\[G/B]ciso
A tua pre\[C]sença é o meu maior va\[G/B]lor
\endverse
\beginchorus
Atrai o meu \[C]coração
Atrai o meu \[Am]coração
És tudo que eu \[G]quero
Atrai o meu \[C]coração
Atrai o meu \[Am]coração
Eu posso te to\[G]car
\endchorus
\seq{Interlúdio}{C D/F\# C/E D4(7) G}{1x}
\act{Retomar}{Verso 1}{1x}
\act{Repetir}{Refrão}{+1x}
\seq{Interlúdio}{C D/F\# C/E D4(7) G}{1x}
\seq{\bf\color{red} DEIXA QUEIMAR - Alessandro Vilas Boas \color{black}}{G D Em7 C}{ Música Inteira}

\act{\bf\color{red} QUÃO GRANDE É O MEU DEUS - Soraya Moraes\color{black}}{}{Apenas Refrão}
\beginchorus
Quão \[G]grande é o meu Deus
Cantarei quão \[Em]grande é o meu Deus
E todos hão de \[C]ver
Quão \[D]grande é o meu \[G]Deus \[D]
\endchorus

\act{\bf\color{red} CONSAGRAÇÃO - Aline Barros\color{black}}{}{Apenas Refrão}
\beginchorus
A \[G]honra, a \[D/F\#]glória, a \[Em]força
E \[Bm]o poder ao rei Je\[C]sus \[G/B]
E o lou\[Am]vor \[G] ao \[F]rei \[C/E] Je\[D]sus
\endchorus
%-----------------------------------------------------------------
\begin{comment}
\lstset{basicstyle=\scriptsize\bf} % Parâmetros da TAB
%-----------------------------------------------------------------
\tab{Solo 1}
\begin{lstlisting}
E|-----------------------------------------------------|
B|-----------------------------------------------------|
G|-----------------------------------------------------|
D|-----------------------------------------------------|
A|-----------------------------------------------------|
E|-----------------------------------------------------|
\end{lstlisting}
%-----------------------------------------------------------------
\end{comment}
%=================================================================
\begin{comment}

%-----------------------------------------------------------------
\color{drawChord}\gtab{\color{nameChord} G}{}% 
\color{drawChord}\gtab{\color{nameChord} G4(9)}{}%
\color{drawChord}\gtab{\color{nameChord} G/B}{}% 
\color{drawChord}\gtab{\color{nameChord} G9}{}% 
\color{drawChord}\gtab{\color{nameChord} G7M(9)}{}% 
\color{drawChord}\gtab{\color{nameChord} Am}{}% 
\color{drawChord}\gtab{\color{nameChord} Bm}{}%
\color{drawChord}\gtab{\color{nameChord} C}{}% 
\color{drawChord}\gtab{\color{nameChord} C/E}{}% 
\color{drawChord}\gtab{\color{nameChord} D}{}% 
\color{drawChord}\gtab{\color{nameChord} D4}{}%
\color{drawChord}\gtab{\color{nameChord} D4(7)}{}% 
\color{drawChord}\gtab{\color{nameChord} D/F\#}{}% 
\color{drawChord}\gtab{\color{nameChord} Em}{}% 
\color{drawChord}\gtab{\color{nameChord} Em7}{}%
\color{drawChord}\gtab{\color{nameChord} F}{}%
\end{comment}
%-----------------------------------------------------------------
% PADRÃO: [TonalidadeMaior+NOTAX+Variações] .Ex:[X50] [X57V1V7]
% OBS: Variações são alterações do acorde em relação ao campo harmônico.
%-----------------------------------------------------------------
% Tipos de Variações de Acordes:
% V0 - Variação Diversa
% V1 - Menor (m)
% V2 - Maior (M)
% V3 - Meio Tom Abaixo (Bemol)
% V4 - Com Quarta (ex:C4)
% V5 - Com Quinta (ex:C5)
% V6 - Com Sexta (ex:C6)
% V7 - Com Sétima Menor (ex:C7)
% V8 - Com baixo dois Tons Acima (ex:D/F#)
% V9 - Com Nona (ex:C9)
% V10 - Meio Tom Acima (Sustenido)
% V11 - Com Sétima Maior (ex:C7M)
% V12 - Suspenso (Sus)
% V13 - Com baixo dois Tons e Meio Acima (ex:A/E)
% V14 - Com baixo um Tom e Meio Acima (ex:D9/F) 
% V15 - Meio-Diminuto (m7b5)
% N15 - NÃO Meio-Diminuto
% V16 - Diminuto (º)
% N16 - NÃO Diminuto
% V17 - Com baixo um Tom Acima (ex: C/D)
% V18 - Com baixo um Tom Abaixo (ex: Em/D)
% V19 - Com baixo dois Tons e meio Abaixo (ex: G/D)
%=================================================================
\endsong
%=================================================================
\begin{comment}

\end{comment}