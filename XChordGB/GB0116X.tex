%=================================================================
\songcolumns{2}
\beginsong
{Teu Amor Não Falha %TÍTULO
}[by={Nívea Soares %ARTISTA
},album={@walyssondosreis},
id={GB0116 %COD.ID.: GB0000
},rev={0}, %REVISÃO
qr={https://drive.google.com/open?id=1QMENuHxYe36hIa8JVfLgNnB7fKeUGmVA %LINK
}]
%-----------------------------------------------------------------
\tom{X1}{C}
%=================================================================
%\newchords{verse1.GB0000X} % Registrador de Acordes em Sequência
%\newchords{chorus1.GB0000X} % Registrador de Acordes em Sequência
%-----------------------------------------------------------------
\seq{Intro P1}{X6 X4 X1 X5}{2x}
\seq{Intro P2}{X4 X6 X5 }{2x}
%-----------------------------------------------------------------
%\beginverse* \endverse
%\beginchorus \endchorus
\beginverse
\[X6] Nada vai me \[X4]separar \[X1]
Mesmo se eu me \[X5]abalar
T\[X6]eu amor não \[X4]falha \[X1]\[X5]
\endverse
\beginverse
^ Mesmo sem ^merecer ^
Tua graça se derrama ^sobre mim
T^eu amor não ^falha ^^
\endverse
\beginchorus
\[X4] Tu és o \[X1]mesmo pra sempre \[X5]
T\[X2]eu amor não \[X6]muda
Se o choro \[X1]dura uma noite
A ale\[X5]gria vem pela \[X2]manhã \[X6]
Se o m\[X1]ar se enfurec\[X5]er
Eu não \[X2]tenho o que tem\[X6]er
Porque eu \[X1]sei que me amas \[X5]
T\[X5]eu amor não \[X4]falha \[X6 X5 X4 X6 X5]
\endchorus
\beginverse
\[X6] Se o vento é forte e pro\[X4]fundo o mar \[X1]
Tua presença vem me \[X5]amparar
T\[X6]eu amor não \[X4]falha \[X1]\[X5]
\endverse
\beginverse
^ Difícil era o ^caminhar ^
Nunca pensei que eu fosse ^alcançar
Mas t^eu amor não ^falha ^^
\endverse
\beginverse
\[(X5)] Tu fazes \[X4] que tudo 
Coo\[X6]pere para o m\[X5]eu bem
\endverse
%-----------------------------------------------------------------
\vspace{4em} % Regulador de Espaçamento
%-----------------------------------------------------------------
\begin{comment}
\lstset{basicstyle=\scriptsize\bf} % Parâmetros da TAB
%-----------------------------------------------------------------
\tab{Solo 1}
\begin{lstlisting}
E|-----------------------------------------------------|
B|-----------------------------------------------------|
G|-----------------------------------------------------|
D|-----------------------------------------------------|
A|-----------------------------------------------------|
E|-----------------------------------------------------|
\end{lstlisting}
%-----------------------------------------------------------------
\end{comment}
%=================================================================
 
%-----------------------------------------------------------------
\color{drawChord}\gtab{\color{nameChord} X1}{}% 
\color{drawChord}\gtab{\color{nameChord} X2}{}% 
\color{drawChord}\gtab{\color{nameChord} X4}{}% 
\color{drawChord}\gtab{\color{nameChord} X5}{}%
\color{drawChord}\gtab{\color{nameChord} X6}{}% 
%-----------------------------------------------------------------
% PADRÃO: [TonalidadeMaior+NOTAX+Variações] .Ex:[X50] [X57V1V7]
% OBS: Variações são alterações do acorde em relação ao campo harmônico.
%-----------------------------------------------------------------
% Tipos de Variações de Acordes:
% V0 - Variação Diversa
% V1 - Menor (m)
% V2 - Maior (M)
% V3 - Meio Tom Abaixo (Bemol)
% V4 - Com Quarta (ex:C4)
% V5 - Com Quinta (ex:C5)
% V6 - Com Sexta (ex:C6)
% V7 - Com Sétima Menor (ex:C7)
% V8 - Com baixo dois Tons Acima (ex:D/F#)
% V9 - Com Nona (ex:C9)
% V10 - Meio Tom Acima (Sustenido)
% V11 - Com Sétima Maior (ex:C7M)
% V12 - Suspenso (Sus)
% V13 - Com baixo dois Tons e Meio Acima (ex:A/E)
% V14 - Com baixo um Tom e Meio Acima (ex:D9/F) 
% V15 - Meio-Diminuto (m7b5)
% N15 - NÃO Meio-Diminuto
% V16 - Diminuto (º)
% N16 - NÃO Diminuto
%=================================================================
\endsong
%=================================================================
\begin{comment}

\end{comment}