%=================================================================
\songcolumns{2}
\beginsong
{Corpo e Família %TÍTULO
}[by={Priscila Angel %ARTISTA
},album={@walyssondosreis},
id={GB0020 %COD.ID.: GB0000
},rev={3}, %REVISÃO
qr={https://drive.google.com/open?id=1blUg5nqzoMF1siC1EBDmNyKgxa8e1hg1 %LINK
}]
%-----------------------------------------------------------------
\tom{G}{F}
%=================================================================
%\newchords{verse1.GB0000X} % Registrador de Acordes em Sequência
%\newchords{chorus1.GB0000X} % Registrador de Acordes em Sequência
%-----------------------------------------------------------------
\seq{Intro}{ [ G C D C ] [ G C D C D ]}{}
%-----------------------------------------------------------------
%\beginverse* \endverse
%\beginchorus \endchorus
\beginverse
Rece\[G]bi um novo \[C]coração do \[G]pai \[D/F\#]
Cora\[Em]ção regene\[Bm]rado, cora\[C]ção transfor\[G/B]mado
Cora\[Am]ção que é inspi\[G]rado por Je\[D/F\#]sus \[D4]\[D]
\endverse
\beginverse
Como ^fruto deste ^novo cora^ção, ^
Eu de^claro a paz de ^Cristo 
Te aben^çôo meu ir^mão, preci^osa
É a ^nossa comu\[D]nhão \[D4]
\endverse
\beginchorus
Somos \[G]corpo, e \[C/E]assim bem ajus\[D/F\#]tado,
To\[Bm]talmente li\[Em]gado
U\[C]nido, vi\[Am]vendo em \[D4]amor,
Um\[D]a fa\[G]mília, sem \[C]qualquer \[Em]falsi\[D/F\#]dade
Vi\[Bm]vendo a ver\[Em]dade, expre\[C]ssando
A \[Am]glória do Se\[D4]nhor
Um\[D]a fa\[G]mília vi\[C]vendo o 
\[Em]Compro\[D/F\#]misso no \[Bm]grande amor de \[Em]Cristo:
Eu pre\[C]ciso de \[Am]ti que\[D4]rido \[D]ir\[G]mão
\endchorus
\seq{Riff 1}{G C D}{2x}
\act{Retomar}{Verso 2}{1x}
\beginverse
...,Preci\[Em]oso és para \[C]mim que\[D4]rido \[D]ir\[G]mão
Eu pre\[Em]ciso de \[C]ti que\[D4]rido \[D]ir\[G]mão,
Preci\[Em]oso és para \[C]mim que\[D4]rido \[D]ir\[G]mão
\endverse
%-----------------------------------------------------------------
\begin{comment}
\lstset{basicstyle=\scriptsize\bf} % Parâmetros da TAB
%-----------------------------------------------------------------
\tab{Solo 1}
\begin{lstlisting}
E|-----------------------------------------------------|
B|-----------------------------------------------------|
G|-----------------------------------------------------|
D|-----------------------------------------------------|
A|-----------------------------------------------------|
E|-----------------------------------------------------|
\end{lstlisting}
%-----------------------------------------------------------------
\end{comment}
%=================================================================
 
%-----------------------------------------------------------------
\color{drawChord}\gtab{\color{nameChord} G}{}%
\color{drawChord}\gtab{\color{nameChord} G/B}{}%
\color{drawChord}\gtab{\color{nameChord} Am}{}%
\color{drawChord}\gtab{\color{nameChord} Bm}{}%
\color{drawChord}\gtab{\color{nameChord} C}{}%
\color{drawChord}\gtab{\color{nameChord} C/E}{}\\%
\color{drawChord}\gtab{\color{nameChord} D}{}% 
\color{drawChord}\gtab{\color{nameChord} D4}{}%
\color{drawChord}\gtab{\color{nameChord} D/F\#}{}%
\color{drawChord}\gtab{\color{nameChord} Em}{}%
%-----------------------------------------------------------------
% PADRÃO: [TonalidadeMaior+NOTAX+Variações] .Ex:[X50] [X57V1V7]
% OBS: Variações são alterações do acorde em relação ao campo harmônico.
%-----------------------------------------------------------------
% Tipos de Variações de Acordes:
% V0 - Variação Diversa
% V1 - Menor (m)
% V2 - Maior (M)
% V3 - Meio Tom Abaixo (Bemol)
% V4 - Com Quarta (ex:C4)
% V5 - Com Quinta (ex:C5)
% V6 - Com Sexta (ex:C6)
% V7 - Com Sétima Menor (ex:C7)
% V8 - Com baixo dois Tons Acima (ex:D/F\#)
% V9 - Com Nona (ex:C9)
% V10 - Meio Tom Acima (Sustenido)
% V11 - Com Sétima Maior (ex:C7M)
% V12 - Suspenso (Sus)
% V13 - Com baixo dois Tons e Meio Acima (ex:A/E)
% V14 - Com baixo um Tom e Meio Acima (ex:D9/F) 
% V15 - Meio-Diminuto (m7b5)
% N15 - NÃO Meio-Diminuto
% V16 - Diminuto (º)
% N16 - NÃO Diminuto
%=================================================================
\endsong
%=================================================================
\begin{comment}

\end{comment}