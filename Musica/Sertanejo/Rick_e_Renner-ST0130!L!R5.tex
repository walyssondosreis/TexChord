%=================================================================
\songcolumns{2}
\beginsong
{Nos Bares da Cidade %TÍTULO
}[by={Rick e Renner %ARTISTA
},album={@walyssondosreis},
id={\href{ https://music.youtube.com/watch?v=c6wJ0Mk1ZvY&feature=share
}{ ST0130  %COD.ID.: XXNNNN
}},rev={5}, %REVISÃO
qr={ https://music.youtube.com/watch?v=c6wJ0Mk1ZvY&feature=share %LINK
}]
%-----------------------------------------------------------------
\tom{}{D}
%=================================================================
%\newchords{verse1.XX0000X} % Registrador de Acordes em Sequência
%\newchords{chorus1.XX0000X} % Registrador de Acordes em Sequência
%-----------------------------------------------------------------
\seq{Intro}{...}{2x}
%\act{}{}{}
%-----------------------------------------------------------------
%\beginverse \endverse
%\beginchorus \endchorus
\beginverse
Garçom, me traga 
Outra garrafa de cerveja
Vou ficar sozinho nessa mesa
Eu quero beber e chorar por ela 
\endverse
\beginverse
Garçom, a minha vida agora
Tá de ponta cabeça
Já tentei mas nada 
Faz com que eu esqueça
Os olhos e os lábios
Daquela mulher 
\endverse
\beginverse
Garçom, ela saiu de vez
Da minha vida
E agora eu busco uma saída
Minha história de amor 
Acaba em solidão
Garçom, se eu ficar muito chato
E der algum vexame
Pegue toda a minha cerveja e derrame
Faça o que ela fez
Com a minha paixão 
\endverse
\beginchorus
Derrama cerveja, derrama
Derrama a tristeza do meu coração
Que essa angustia é uma bebida
Misturada, batida
Com a solidão
Derrama cerveja, derrama
Enquanto eu derramo
Toda essa saudade
Eu sou apenas um qualquer
Bebendo por mulher
Nos bares da cidade
\endchorus
\act{Riff Intro}{Repetir}{1x}
\act{Verso 1}{Retomar}{1x}
\act{Refrão}{Repetir}{+1x}
\beginverse
... Eu sou apenas um qualquer
Bebendo por mulher
Nos bares da cidade
\endverse
\act{Verso 4}{Repetir}{+1x}
%-----------------------------------------------------------------
\vspace{4em} % Regulador de Espaçamento
%-----------------------------------------------------------------
\begin{comment}
\lstset{basicstyle=\scriptsize\bf} % Parâmetros da TAB
%-----------------------------------------------------------------
\tab{Solo 1}
\begin{lstlisting}
E|-----------------------------------------------------|
B|-----------------------------------------------------|
G|-----------------------------------------------------|
D|-----------------------------------------------------|
A|-----------------------------------------------------|
E|-----------------------------------------------------|
\end{lstlisting}
%-----------------------------------------------------------------
\end{comment}
%=================================================================
\begin{comment}
\color{drawChord}\gtab{\color{nameChord} X}{}% 
\color{drawChord}\gtab{\color{nameChord} X}{}% 
\color{drawChord}\gtab{\color{nameChord} X}{}% 
\color{drawChord}\gtab{\color{nameChord} X}{}% 
\end{comment}
%=================================================================
% PADRÃO: [TonalidadeMaior+NOTAX+Variações] .Ex:[X50] [X57V1V7]
% OBS: Variações são alterações do acorde em relação ao campo harmônico.
%-----------------------------------------------------------------
% Tipos de Variações de Acordes:
% V0 - Variação Diversa
% V1 - Menor (m)
% V2 - Maior (M)
% V3 - Meio Tom Abaixo (Bemol)
% V4 - Com Quarta (ex:C4)
% V5 - Com Quinta (ex:C5)
% V6 - Com Sexta (ex:C6)
% V7 - Com Sétima Menor (ex:C7)
% V8 - Com baixo dois Tons Acima (ex:D/F#)
% V9 - Com Nona (ex:C9)
% V10 - Meio Tom Acima (Sustenido)
% V11 - Com Sétima Maior (ex:C7M)
% V12 - Suspenso (Sus)
% V13 - Com baixo dois Tons e Meio Acima (ex:A/E)
% V14 - Com baixo um Tom e Meio Acima (ex:D9/F) 
% V15 - Meio-Diminuto (m7b5)
% N15 - NÃO Meio-Diminuto
% V16 - Diminuto (º)
% N16 - NÃO Diminuto
% V17 - Com baixo um Tom Acima (ex: C/D)
% V18 - Com baixo um Tom Abaixo (ex: Em/D)
% V19 - Com baixo dois Tons e meio Abaixo (ex: G/D)
%=================================================================
%\vspace{4em} % Regulador de Espaçamento
%=================================================================
\endsong
%=================================================================
