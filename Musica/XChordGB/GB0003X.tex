%=================================================================
\songcolumns{1}
\beginsong
{Abraça-me %TÍTULO
}[by={André Valadão %ARTISTA
},album={@walyssondosreis},
id={GB0003 %COD.ID.: XXNNNN
},rev={3}, %REVISÃO
qr={https://drive.google.com/open?id=1f4RPhzEFQUgKJ64Hnavj9FRqjfAr-2oW %LINK
}]
%-----------------------------------------------------------------
\tom{X1}{G}
%=================================================================
%\newchords{verse1.GB0000X} % Registrador de Acordes em Sequência
%\newchords{chorus1.GB0000X} % Registrador de Acordes em Sequência
%-----------------------------------------------------------------
%\seq{Intro}{}{}
%-----------------------------------------------------------------
%\beginverse* \endverse
%\beginchorus \endchorus
\beginverse
En\[X1]sina-me a sentir teu coração
Je\[X4]sus, quero ouvir teu respirar
Ti\[X6V7]rar teu fôlego com minha fé
E te ado\[X5]rar \[X5V4]\[X5]
\endverse
\beginverse
Je^sus, tu és o pão que me alimenta
O ^verbo vivo que desceu do céu
Vem ^aquecer meu frio o coração
Com teu a^mor ^^
\endverse
\beginchorus
Abraça-\[X4]me, abraça-\[X4]me
Cura-\[X5]me, cura-\[X5]me
Unge-\[X4]me, unge-\[X4]me
Toca-\[X5]me, toca-\[X5]me
\endchorus
\act{Retomar}{Verso 1}{1x}
\beginverse
Vem \[X1]sobre mim com o teu \[X6V7]manto
\[X4]Reina em mim com tua \[X2]gló\[X5]ria
Pois \[X1]ao teu lado é o \[X6V7]meu lugar
A\[X4]leluia, a\[X2]lelu\[X5]ia
\endverse
\act{Repetir}{Verso 3}{+1x}
\act{Repetir}{Refrão}{2x}
\act{Repetir}{Verso 3}{2x}
%-----------------------------------------------------------------
\vspace{2em} % Regulador de Espaçamento
%-----------------------------------------------------------------
\begin{comment}
\lstset{basicstyle=\scriptsize\bf} % Parâmetros da TAB
%-----------------------------------------------------------------
\tab{Solo 1}
\begin{lstlisting}
E|-----------------------------------------------------|
B|-----------------------------------------------------|
G|-----------------------------------------------------|
D|-----------------------------------------------------|
A|-----------------------------------------------------|
E|-----------------------------------------------------|
\end{lstlisting}
%-----------------------------------------------------------------
\end{comment}
%=================================================================
 
%-----------------------------------------------------------------
\color{drawChord}\gtab{\color{nameChord} X1}{}% 
\color{drawChord}\gtab{\color{nameChord} X2}{}% 
\color{drawChord}\gtab{\color{nameChord} X4}{}% 
\color{drawChord}\gtab{\color{nameChord} X5}{}% 
\color{drawChord}\gtab{\color{nameChord} X5V4}{}% 
\color{drawChord}\gtab{\color{nameChord} X6V7}{}% 
%-----------------------------------------------------------------
% PADRÃO: [TonalidadeMaior+NOTAX+Variações] .Ex:[X50] [X57V1V7]
% OBS: Variações são alterações do acorde em relação ao campo harmônico.
%-----------------------------------------------------------------
% Tipos de Variações de Acordes:
% V0 - Variação Diversa
% V1 - Menor (m)
% V2 - Maior (M)
% V3 - Meio Tom Abaixo (Bemol)
% V4 - Com Quarta (ex:C4)
% V5 - Com Quinta (ex:C5)
% V6 - Com Sexta (ex:C6)
% V7 - Com Sétima Menor (ex:C7)
% V8 - Com baixo dois Tons Acima (ex:D/F#)
% V9 - Com Nona (ex:C9)
% V10 - Meio Tom Acima (Sustenido)
% V11 - Com Sétima Maior (ex:C7M)
% V12 - Suspenso (Sus)
% V13 - Com baixo dois Tons e Meio Acima (ex:A/E)
% V14 - Com baixo um Tom e Meio Acima (ex:D9/F) 
% V15 - Meio-Diminuto (m7b5)
% N15 - NÃO Meio-Diminuto
% V16 - Diminuto (º)
% N16 - NÃO Diminuto
%=================================================================
\endsong
%=================================================================
\begin{comment}

\end{comment}