%=================================================================
\songcolumns{2}
\beginsong
{Ouvir O Teu Falar %TÍTULO
}[by={Discopraise %ARTISTA
},album={@walyssondosreis},
id={GB0083 %COD.ID.: GB0000
},rev={0}, %REVISÃO
qr={https://drive.google.com/open?id=1HLb8oymShMVK6tL-Otgl1okJ0z4HKEGT %LINK
}]
%-----------------------------------------------------------------
\tom{X1}{E}
%=================================================================
%\newchords{verse1.GB0000X} % Registrador de Acordes em Sequência
%\newchords{chorus1.GB0000X} % Registrador de Acordes em Sequência
%-----------------------------------------------------------------
%\seq{Intro}{}{}
%-----------------------------------------------------------------
%\beginverse* \endverse
%\beginchorus \endchorus
\beginverse*
Quanto mais eu corro, mais cansado eu fico
Não importa o quanto eu me sacrifico
Meus objetivos não consigo alcançar
Já trilhei em todos os caminhos
Já gritei por todos mas fiquei sozinho
Sinto minha força se esgotar
Vem me ajudar
\endverse
\beginverse*
Tenho ouvido falar das tuas maravilhas
E tens poder pra salvar
Escreve um novo começo pra minha vida
Creio que ainda hoje tudo vai mudar
\endverse
\beginchorus
Levantarei meu olhos
E clamarei socorro
Tua palavra diz que vais me escutar
Levantarei meus olhos
E clamarei socorro
Meu coração só quer ouvir o teu falar
Ouvir o teu falar
\endchorus
\beginverse*
O que espera em ti
Jamais se abalará
Jamais se cansará
Jamais desistirá
O teu sonho agora é o que eu espero alcançar
Em teu caminho andar
Meu maior desejo é te agradar
\endverse

%-----------------------------------------------------------------
\begin{comment}
\lstset{basicstyle=\scriptsize\bf} % Parâmetros da TAB
%-----------------------------------------------------------------
\tab{Solo 1}
\begin{lstlisting}
E|-----------------------------------------------------|
B|-----------------------------------------------------|
G|-----------------------------------------------------|
D|-----------------------------------------------------|
A|-----------------------------------------------------|
E|-----------------------------------------------------|
\end{lstlisting}
%-----------------------------------------------------------------
\end{comment}
%=================================================================
\vspace{2em} 
%-----------------------------------------------------------------
\color{drawChord}\gtab{\color{nameChord} X}{}% 
\color{drawChord}\gtab{\color{nameChord} X}{}% 
\color{drawChord}\gtab{\color{nameChord} X}{}% 
\color{drawChord}\gtab{\color{nameChord} X}{}% 
%-----------------------------------------------------------------
% PADRÃO: [TonalidadeMaior+NOTAX+Variações] .Ex:[X50] [X57V1V7]
% OBS: Variações são alterações do acorde em relação ao campo harmônico.
%-----------------------------------------------------------------
% Tipos de Variações de Acordes:
% V0 - Variação Diversa
% V1 - Menor (m)
% V2 - Maior (M)
% V3 - Meio Tom Abaixo (Bemol)
% V4 - Com Quarta (ex:C4)
% V5 - Com Quinta (ex:C5)
% V6 - Com Sexta (ex:C6)
% V7 - Com Sétima Menor (ex:C7)
% V8 - Com baixo dois Tons Acima (ex:D/F#)
% V9 - Com Nona (ex:C9)
% V10 - Meio Tom Acima (Sustenido)
% V11 - Com Sétima Maior (ex:C7M)
% V12 - Suspenso (Sus)
% V13 - Com baixo dois Tons e Meio Acima (ex:A/E)
% V14 - Com baixo um Tom e Meio Acima (ex:D9/F) 
% V15 - Meio-Diminuto (m7b5)
% N15 - NÃO Meio-Diminuto
% V16 - Diminuto (º)
% N16 - NÃO Diminuto
%=================================================================
\endsong
%=================================================================
\begin{comment}

\end{comment}