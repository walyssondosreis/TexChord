%=================================================================
\songcolumns{2}
\beginsong
{Chuva de Avivamento %TÍTULO
}[by={Alda Célia %ARTISTA
},album={@walyssondosreis},
id={GB0019 %COD.ID.: GB0000
},rev={1}, %REVISÃO
qr={ %LINK
}]
%-----------------------------------------------------------------
\tom{X1}{X1}
%=================================================================
\newchords{verse1.GB0019X} % Registrador de Acordes em Sequência
%\newchords{chorus1.GB0000X} % Registrador de Acordes em Sequência
%-----------------------------------------------------------------
\seq{Intro}{X1 X1V4}{}
%-----------------------------------------------------------------
%\beginverse* \endverse
%\beginchorus \endchorus
\beginverse\memorize[verse1.GB0019X]
Nos \[X1]últimos dias \[X1V4]diz o Senhor
\[X5]Derramarei Meu Es\[X1]pírito sobre toda a \[X4]terra
Copiosa\[X1]mente \[X1V4]
\[X1]Poços secos \[X1V4]jorrarão
E \[X5]até o deserto de \[X1]novo frutifica\[X4]rá
Abundante\[X1]mente
\endverse
\beginverse
É o \[X6]som do nosso lou\[X5]vor
Que sobe aos \[X6]céus como um va\[X4]por
E se con\[X6]densa na nuvem de \[X5]glória
Sheki\[X7V3N15]nah sobre \[X6V2V3]nós se derrama\[X7V3N15]rá \[X6V2V3V17]
\endverse
\beginchorus
Em abundante \[X3V2V3]chuva, \[X7V3V8N15]chuva
De\[X4V1]rrama sobre nós esta \[X1V1]chuva
\[X2V2V3]Abre as comportas dos \[X4V1]céus, Senhor
\[X5V1]Faz \[X1V1]cho\[X7V3N15]ver
Abundante \[X3V2V3]chuva, \[X7V3V8N15]chuva
De\[X4V1]rrama sobre nós esta \[X1V1]chuva
To\[X2V2V3]rrente de águas \[X6V2V3]sobre o sedento
\[X4V1]Chuva de a\[X7V3N15]viva\[X1]mento, \[X6V2V3]chuva de a\[X7V3N15]viva\[X1]mento
\endchorus
\beginverse\replay[verse1.GB0019X]
E ^esta chuva con^verterá
O ^coração dos pais aos ^filhos e dos filhos aos ^pais
No poder do Es^pírito
\endverse
\act{Retormar}{Verso 2}{1x}
\beginverse
\[X1V1V7]Chuva de aviva\[X6V2V3]mento (derrama)
\[X1V1V7]Chuva de aviva\[X6V2V3]mento (derrama)
\[X]Chuva de aviva\[X6V2V3]mento \[X7V3N15]vem sobre \[X1V1V7]nós \[( X6V2V3 X7V3N15 )]
\endverse
\act{Repetir}{Verso 4}{+1x}
\act{Repetir}{Refrão}{1x}
\beginverse
... \[X6V2V3]Chuva de a\[X7V3N15]viva\[X1]mento
\endverse


%-----------------------------------------------------------------
\begin{comment}
\lstset{basicstyle=\scriptsize\bf} % Parâmetros da TAB
%-----------------------------------------------------------------
\tab{Solo 1}
\begin{lstlisting}
E|-----------------------------------------------------|
B|-----------------------------------------------------|
G|-----------------------------------------------------|
D|-----------------------------------------------------|
A|-----------------------------------------------------|
E|-----------------------------------------------------|
\end{lstlisting}
%-----------------------------------------------------------------
\end{comment}
%=================================================================
\vspace{2em} 
%-----------------------------------------------------------------
\gtab{\color{black} X1}{}% 
\gtab{\color{black} X1V4}{}% 
\gtab{\color{black} X1V1}{}% 
\gtab{\color{black} X1V1V7}{}% 
\gtab{\color{black} X2V2V3}{}% 
\gtab{\color{black} X3V2V3}{}\\% 
\gtab{\color{black} X4}{}% 
\gtab{\color{black} X4V1}{}% 
\gtab{\color{black} X5}{}% 
\gtab{\color{black} X5V1}{}% 
\gtab{\color{black} X6V2V3}{}% 
\gtab{\color{black} X6V2V3V17}{}\\% 
\gtab{\color{black} X6}{}% 
\gtab{\color{black} X7V3N15}{}% 
\gtab{\color{black} X7V3V8N15}{}% 

%-----------------------------------------------------------------
% PADRÃO: [TonalidadeMaior+NOTAX+Variações] .Ex:[X50] [X57V1V7]
% OBS: Variações são alterações do acorde em relação ao campo harmônico.
%-----------------------------------------------------------------
% Tipos de Variações de Acordes:
% V0 - Variação Diversa
% V1 - Menor (m)
% V2 - Maior (M)
% V3 - Meio Tom Abaixo (Bemol)
% V4 - Com Quarta (ex:C4)
% V5 - Com Quinta (ex:C5)
% V6 - Com Sexta (ex:C6)
% V7 - Com Sétima Menor (ex:C7)
% V8 - Com baixo dois Tons Acima (ex:D/F#)
% V9 - Com Nona (ex:C9)
% V10 - Meio Tom Acima (Sustenido)
% V11 - Com Sétima Maior (ex:C7M)
% V12 - Suspenso (Sus)
% V13 - Com baixo dois Tons e Meio Acima (ex:A/E)
% V14 - Com baixo um Tom e Meio Acima (ex:D9/F) 
% V15 - Meio-Diminuto (m7b5)
% N15 - NÃO Meio-Diminuto
% V16 - Diminuto (º)
% N16 - NÃO Diminuto
% V17 - Com baixo um Tom Acima (ex: C/D)
%=================================================================
\endsong
%=================================================================
\begin{comment}

\end{comment}