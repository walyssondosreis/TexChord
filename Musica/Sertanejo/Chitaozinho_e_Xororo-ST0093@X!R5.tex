%=================================================================
\songcolumns{2}
\beginsong
{Fio de Cabelo %TÍTULO
}[by={Chitãozinho e Xororó %ARTISTA
},album={@walyssondosreis},
id={\href{https://music.youtube.com/watch?v=zpp0t3scZYs&feature=share %LINK
}{ST0093 %COD.ID.: XXNNNN
}},rev={5}, %REVISÃO
qr={https://music.youtube.com/watch?v=zpp0t3scZYs&feature=share %LINK
}]
%-----------------------------------------------------------------
\tom{X1}{Gb}
%=================================================================
%\newchords{verse1.XX0000X} % Registrador de Acordes em Sequência
%\newchords{chorus1.XX0000X} % Registrador de Acordes em Sequência
%-----------------------------------------------------------------
\seq{Intro}{X5V7 X1 X5 X5V7 X1}{1x}
%\act{}{}{}
%-----------------------------------------------------------------
%\beginverse \endverse
%\beginchorus \endchorus
\beginverse
\[X1] Quando a gente ama
Qualquer coisa \[X5V7]serve para relem\[X1]brar
Um vestido velho da mulher a\[X1V7]mada
Tem muito va\[X4]lor
Aquele restinho do perfume \[X5V7]dela que ficou no \[X1]rasco
Sobre a penteadeira mostrando que o \[X5V7]quarto
Já foi o cenário de um grande a\[X1]mor
\endverse
\beginchorus
E \[X5]hoje o que encontrei me deixou mais \[X1]triste
Um pedacinho dela que e\[X5V7]xiste
Um fio de ca\[X4]belo no meu pale\[X1]tó
Lem\[X5]brei de tudo entre nós, do amor vi\[X1]vido
Aquele fio de cabelo com\[X5V7]prido
Já esteve gru\[X4]dado em \[X5V7]nosso su\[X1]or
\endchorus
\act{Riff Intro}{Repetir}{1x}
\beginverse
^ Quando a gente ama
E não vive ^junto da mulher a^mada
Uma coisa à toa
É um bom mo^tivo pra gente cho^rar
Apagam-se as luzes ao chegar a ^hora
De ir para a ^cama
A gente começa a esperar por quem ^ama
Na impressão que ela venha se dei^tar
\endverse
\act{Refrão}{Repetir}{1x}
%-----------------------------------------------------------------
\vspace{4em} % Regulador de Espaçamento
%-----------------------------------------------------------------
\begin{comment}
\lstset{basicstyle=\scriptsize\bf} % Parâmetros da TAB
%-----------------------------------------------------------------
\tab{Solo 1}
\begin{lstlisting}
E|-----------------------------------------------------|
B|-----------------------------------------------------|
G|-----------------------------------------------------|
D|-----------------------------------------------------|
A|-----------------------------------------------------|
E|-----------------------------------------------------|
\end{lstlisting}
%-----------------------------------------------------------------
\end{comment}
%=================================================================


\color{drawChord}\gtab{\color{nameChord} X1}{}% 
\color{drawChord}\gtab{\color{nameChord} X1V7}{}% 
\color{drawChord}\gtab{\color{nameChord} X4}{}% 
\color{drawChord}\gtab{\color{nameChord} X5}{}% 
\color{drawChord}\gtab{\color{nameChord} X5V7}{}% 


%=================================================================
% PADRÃO: [TonalidadeMaior+NOTAX+Variações] .Ex:[X50] [X57V1V7]
% OBS: Variações são alterações do acorde em relação ao campo harmônico.
%-----------------------------------------------------------------
% Tipos de Variações de Acordes:
% V0 - Variação Diversa
% V1 - Menor (m)
% V2 - Maior (M)
% V3 - Meio Tom Abaixo (Bemol)
% V4 - Com Quarta (ex:C4)
% V5 - Com Quinta (ex:C5)
% V6 - Com Sexta (ex:C6)
% V7 - Com Sétima Menor (ex:C7)
% V8 - Com baixo dois Tons Acima (ex:D/F#)
% V9 - Com Nona (ex:C9)
% V10 - Meio Tom Acima (Sustenido)
% V11 - Com Sétima Maior (ex:C7M)
% V12 - Suspenso (Sus)
% V13 - Com baixo dois Tons e Meio Acima (ex:A/E)
% V14 - Com baixo um Tom e Meio Acima (ex:D9/F) 
% V15 - Meio-Diminuto (m7b5)
% N15 - NÃO Meio-Diminuto
% V16 - Diminuto (º)
% N16 - NÃO Diminuto
% V17 - Com baixo um Tom Acima (ex: C/D)
% V18 - Com baixo um Tom Abaixo (ex: Em/D)
% V19 - Com baixo dois Tons e meio Abaixo (ex: G/D)
%=================================================================
%\vspace{4em} % Regulador de Espaçamento
%=================================================================
\endsong
%=================================================================
