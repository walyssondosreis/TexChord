%=================================================================
\songcolumns{1}
\beginsong
{Vitória No Deserto %TÍTULO
}[by={Aline Barros %ARTISTA
},album={@walyssondosreis},
id={GB0129 %COD.ID.: GB0000
},rev={0}, %REVISÃO
qr={https://drive.google.com/open?id=1FE1BVpp5dl4MRLgudpjRV4QAuJNWAGiZ %LINK
}]
%-----------------------------------------------------------------
\tom{X1}{G}
%=================================================================
%\newchords{verse1.GB0000X} % Registrador de Acordes em Sequência
%\newchords{chorus1.GB0000X} % Registrador de Acordes em Sequência
%-----------------------------------------------------------------
%\seq{Intro}{}{}
%-----------------------------------------------------------------
%\beginverse* \endverse
%\beginchorus \endchorus
\beginverse*
Quando a noite fria cair sobre mim (nanana)
E num deserto eu me encontrar
Me ver cercado por egípcios e por faraó (ohohoh)
Sendo impedido de prosseguir
\endverse
\beginverse*
Sei que o teu fogo cairá sobre mim (heyey)
Sei que o teu fogo cairá sobre mim
E me levará a em ti confiar (ahahah)
E me levará a em ti confiar
\endverse
\beginchorus
Então eu direi, então eu direi, ououh, abra-se o mar
E eu passarei pulando e dançando em sua presença
Então eu direi, então eu direi, ououh, abra-se o mar
E eu passarei pulando e dançando em sua presença
\endchorus
\beginverse*
Por isso eu pulo, pulo, pulo, pulo, pulo
Na presença do rei
Por isso eu danço, danço, danço, danço, danço
Na presença do rei
Um grito de júbilo
Por isso eu grito, grito, grito, grito, grito
Na presença do rei
Por isso eu corro, corro, corro, corro, corro
Na presença do rei
\endverse
%-----------------------------------------------------------------
\vspace{4em} % Regulador de Espaçamento
%-----------------------------------------------------------------
\begin{comment}
\lstset{basicstyle=\scriptsize\bf} % Parâmetros da TAB
%-----------------------------------------------------------------
\tab{Solo 1}
\begin{lstlisting}
E|-----------------------------------------------------|
B|-----------------------------------------------------|
G|-----------------------------------------------------|
D|-----------------------------------------------------|
A|-----------------------------------------------------|
E|-----------------------------------------------------|
\end{lstlisting}
%-----------------------------------------------------------------
\end{comment}
%=================================================================
 
%-----------------------------------------------------------------
\color{drawChord}\gtab{\color{nameChord} X}{}% 
\color{drawChord}\gtab{\color{nameChord} X}{}% 
\color{drawChord}\gtab{\color{nameChord} X}{}% 
\color{drawChord}\gtab{\color{nameChord} X}{}% 
%-----------------------------------------------------------------
% PADRÃO: [TonalidadeMaior+NOTAX+Variações] .Ex:[X50] [X57V1V7]
% OBS: Variações são alterações do acorde em relação ao campo harmônico.
%-----------------------------------------------------------------
% Tipos de Variações de Acordes:
% V0 - Variação Diversa
% V1 - Menor (m)
% V2 - Maior (M)
% V3 - Meio Tom Abaixo (Bemol)
% V4 - Com Quarta (ex:C4)
% V5 - Com Quinta (ex:C5)
% V6 - Com Sexta (ex:C6)
% V7 - Com Sétima Menor (ex:C7)
% V8 - Com baixo dois Tons Acima (ex:D/F#)
% V9 - Com Nona (ex:C9)
% V10 - Meio Tom Acima (Sustenido)
% V11 - Com Sétima Maior (ex:C7M)
% V12 - Suspenso (Sus)
% V13 - Com baixo dois Tons e Meio Acima (ex:A/E)
% V14 - Com baixo um Tom e Meio Acima (ex:D9/F) 
% V15 - Meio-Diminuto (m7b5)
% N15 - NÃO Meio-Diminuto
% V16 - Diminuto (º)
% N16 - NÃO Diminuto
%=================================================================
\endsong
%=================================================================
\begin{comment}

\end{comment}