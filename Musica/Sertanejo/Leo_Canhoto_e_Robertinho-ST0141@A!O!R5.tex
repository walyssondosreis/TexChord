%=================================================================
\songcolumns{1}
\beginsong
{Meu Velho Pai %TÍTULO
}[by={Léo Canhoto e Robertinho %ARTISTA
},album={@walyssondosreis},
id={\href{https://music.youtube.com/watch?v=0UIv8dv9O5I&feature=share %LINK
}{ST0141 %COD.ID.: XXNNNN
}},rev={5}, %REVISÃO
qr={https://music.youtube.com/watch?v=0UIv8dv9O5I&feature=share %LINK
}]
%-----------------------------------------------------------------
\tom{A}{A}
%=================================================================
%\newchords{verse1.XX0000X} % Registrador de Acordes em Sequência
%\newchords{chorus1.XX0000X} % Registrador de Acordes em Sequência
%-----------------------------------------------------------------
\seq{Intro}{E D A D A E A}{1x}
%\act{}{}{}
%-----------------------------------------------------------------
%\beginverse \endverse
%\beginchorus \endchorus
\beginverse
\[A] Meu velho pai, preste aten\[D]ção no que lhe \[A]digo
Meu pobre papai querido enxugue as \[D]lágrimas do \[E]rosto
Porque, pa\[Bm]pai, que você \[D]chora tão so\[E]zinho
Me con\[D]ta, meu papai\[E]zinho
O que \[D]lhe causa des\[A]gosto
\endverse
\beginverse
\[A] Estou notando que vo\[D]cê está can\[A]sado
Meu pobre velho adorado, é seu fi\[A7]lho que está fa\[D]lando
Quero sa\[Bm]ber qual é a \[E]tristeza que e\[A]xiste
Não quero ver você \[E]triste
Por que é que está cho\[A]rando?
\endverse
\act{Riff Intro}{Repetir}{1x}
\beginverse
\[A] Quando lhe vejo, tão tris\[D]tonho desse \[A]jeito
Sinto estremecer meu peito ao pul\[D]sar meu cora\[E]ção
Meu pobre \[Bm]pai, você so\[D]freu pra me cri\[E]ar
Agora \[D]eu vou lhe cui\[E]dar
Esta é \[D]minha obriga\[A]ção
\endverse
\beginverse
\[A] Não tenha medo, meu ve\[D]lhinho ado\[A]rado
Estarei sempre ao seu \[A7]lado, não lhe deixarei ja\[D]mais
Eu sou o \[Bm]sangue do seu \[E]sangue, papai\[A]zinho
Não vou lhe deixar so\[E]zinho, não tenha medo, meu \[A]pai
\endverse
\act{Riff Intro}{Repetir}{1x}
\beginverse
\[A] Você sofreu quando eu \[D]era ainda cri\[A]ança
A sua grande esperança era me \[D]ver homem for\[E]mado
Eu fiquei \[Bm]grande, estou se\[D]guindo o meu ca\[E]minho
E vo\[D]cê ficou ve\[E]lhinho, mas es\[D]tou sempre ao seu \[A]lado
\endverse
\beginverse
\[A] Meu pobre pai, seus passos \[D]longos silenci\[A]aram
Seus cabelos branquiaram, seu o\[A7]lhar se escure\[D]ceu
A sua \[Bm]voz quase que \[E]não se ouve \[A]mais
Não tenha medo, meu \[E]pai, quem cuida de você sou \[A]eu
\endverse
\beginchorus
\[A] Meu papai\[E]zinho, não pre\[D]cisa mais cho\[A]rar
Saiba que não vou dei\[E]xar você so\[D]zinho, abando\[A]nado \[A7]
Eu sou seu \[D]guia, sou seu tempo, sou seus \[A]passos
Sou sua luz e sou seus \[E]braços
Sou seu filho idola\[A]trado
\endchorus
\act{Riff Intro}{Repetir}{1x}

%-----------------------------------------------------------------
\vspace{4em} % Regulador de Espaçamento
%-----------------------------------------------------------------
\begin{comment}
\lstset{basicstyle=\scriptsize\bf} % Parâmetros da TAB
%-----------------------------------------------------------------
\tab{Solo 1}
\begin{lstlisting}
E|-----------------------------------------------------|
B|-----------------------------------------------------|
G|-----------------------------------------------------|
D|-----------------------------------------------------|
A|-----------------------------------------------------|
E|-----------------------------------------------------|
\end{lstlisting}
%-----------------------------------------------------------------
\end{comment}
%=================================================================


\color{drawChord}\gtab{\color{nameChord} A}{~:X02220}% 
\color{drawChord}\gtab{\color{nameChord} A7}{~:X02020}% 
\color{drawChord}\gtab{\color{nameChord} Bm}{2:X02210}% 
\color{drawChord}\gtab{\color{nameChord} D}{~:XX0232}%
\color{drawChord}\gtab{\color{nameChord} E}{~:022100}% 


%=================================================================
% PADRÃO: [TonalidadeMaior+NOTAX+Variações] .Ex:[E0] [E7V1V7]
% OBS: Variações são alterações do acorde em relação ao campo harmônico.
%-----------------------------------------------------------------
% Tipos de Variações de Acordes:
% V0 - Variação Diversa
% V1 - Menor (m)
% V2 - Maior (M)
% V3 - Meio Tom Abaixo (Bemol)
% V4 - Com Quarta (ex:C4)
% V5 - Com Quinta (ex:C5)
% V6 - Com Sexta (ex:C6)
% V7 - Com Sétima Menor (ex:C7)
% V8 - Com baixo dois Tons Acima (ex:D/F#)
% V9 - Com Nona (ex:C9)
% V10 - Meio Tom Acima (Sustenido)
% V11 - Com Sétima Maior (ex:C7M)
% V12 - Suspenso (Sus)
% V13 - Com baixo dois Tons e Meio Acima (ex:A/E)
% V14 - Com baixo um Tom e Meio Acima (ex:D9/F) 
% V15 - Meio-Diminuto (m7b5)
% N15 - NÃO Meio-Diminuto
% V16 - Diminuto (º)
% N16 - NÃO Diminuto
% V17 - Com baixo um Tom Acima (ex: C/D)
% V18 - Com baixo um Tom Abaixo (ex: Em/D)
% V19 - Com baixo dois Tons e meio Abaixo (ex: G/D)
%=================================================================
%\vspace{4em} % Regulador de Espaçamento
%=================================================================
\endsong
%=================================================================
