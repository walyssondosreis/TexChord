%=================================================================
\songcolumns{1}
\beginsong
{Aquele Que Está Feliz %TÍTULO
}[by={Comunidade de Nilópolis %ARTISTA
},album={@walyssondosreis},
id={GB0011 %COD.ID.: GB0000
},rev={0}, %REVISÃO
qr={https://drive.google.com/open?id=14_FdEFViePVeoi2uSbTdehuxibU_yhc_ %LINK
}]
%-----------------------------------------------------------------
\tom{X1}{A}
%=================================================================
%\newchords{verse1.GB0000X} % Registrador de Acordes em Sequência
%\newchords{chorus1.GB0000X} % Registrador de Acordes em Sequência
%-----------------------------------------------------------------
\seq{Intro}{X1 X2 X3 X2 X1}{}
%-----------------------------------------------------------------
%\beginverse* \endverse
%\beginchorus \endchorus
\beginverse*
A\[X1]quele que es\[X2]tá fe\[X3]liz, \[X2]diga a\[X1]mém! \[X2]\[X3]\[X2]
A\[X1]quele que es\[X2]tá fe\[X3]liz, \[X2]grite ale\[X1]luia! \[X2]\[X3]\[X2]
A\[X1]quele que es\[X2]tá fe\[X3]liz, \[X2]bata \[X1]palma assim \[X2]\[X3]\[X2]
A\[X1]quele que es\[X2]tá fe\[X3]liz, \[X2]dance co\[X1]migo assim
\endverse
\beginverse*
\[X6] Com Jesus no coração, a gente \[X3]é feliz (é fe\[X4]liz)
\[(X4)] Com Jesus na condu\[X2]ção, tudo é muito \[X5]bom!
\endverse
\beginchorus
Jesus é \[X1]alegria, \[X6]euforia, \[X4]companhia \[X2]todo dia! \[X5]\[X5]
Jesus é o mo\[X5]ti\[X5]vo da nossa ale\[X1]gria!
( Jesus é o mo\[X5]ti\[X5]vo da nossa ale\[X1]gria! )
\endchorus
%-----------------------------------------------------------------
\vspace{4em} % Regulador de Espaçamento
%-----------------------------------------------------------------
\begin{comment}
\lstset{basicstyle=\scriptsize\bf} % Parâmetros da TAB
%-----------------------------------------------------------------
\tab{Solo 1}
\begin{lstlisting}
E|-----------------------------------------------------|
B|-----------------------------------------------------|
G|-----------------------------------------------------|
D|-----------------------------------------------------|
A|-----------------------------------------------------|
E|-----------------------------------------------------|
\end{lstlisting}
%-----------------------------------------------------------------
\end{comment}
%=================================================================
 
%-----------------------------------------------------------------
\color{drawChord}\gtab{\color{nameChord} X1}{}% 
\color{drawChord}\gtab{\color{nameChord} X2}{}% 
\color{drawChord}\gtab{\color{nameChord} X3}{}% 
\color{drawChord}\gtab{\color{nameChord} X4}{}% 
\color{drawChord}\gtab{\color{nameChord} X5}{}%
\color{drawChord}\gtab{\color{nameChord} X6}{}% 
%-----------------------------------------------------------------
% PADRÃO: [TonalidadeMaior+NOTAX+Variações] .Ex:[X50] [X57V1V7]
% OBS: Variações são alterações do acorde em relação ao campo harmônico.
%-----------------------------------------------------------------
% Tipos de Variações de Acordes:
% V0 - Variação Diversa
% V1 - Menor (m)
% V2 - Maior (M)
% V3 - Meio Tom Abaixo (Bemol)
% V4 - Com Quarta (ex:C4)
% V5 - Com Quinta (ex:C5)
% V6 - Com Sexta (ex:C6)
% V7 - Com Sétima Menor (ex:C7)
% V8 - Com baixo dois Tons Acima (ex:D/F#)
% V9 - Com Nona (ex:C9)
% V10 - Meio Tom Acima (Sustenido)
% V11 - Com Sétima Maior (ex:C7M)
% V12 - Suspenso (Sus)
% V13 - Com baixo dois Tons e Meio Acima (ex:A/E)
% V14 - Com baixo um Tom e Meio Acima (ex:D9/F) 
% V15 - Meio-Diminuto (m7b5)
% N15 - NÃO Meio-Diminuto
% V16 - Diminuto (º)
% N16 - NÃO Diminuto
%=================================================================
\endsong
%=================================================================
\begin{comment}

\end{comment}