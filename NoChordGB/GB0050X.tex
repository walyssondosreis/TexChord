%=================================================================
\songcolumns{1}
\beginsong
{Há Um Lugar %TÍTULO
}[by={Heloísa Rosa %ARTISTA
},album={@walyssondosreis},
id={GB0050 %COD.ID.: GB0000
},rev={0}, %REVISÃO
qr={https://drive.google.com/open?id=1B00ZLK_1chChLwBh0-oLSWpkM40S7ENg %LINK
}]
%-----------------------------------------------------------------
\tom{X1}{G}
%=================================================================
%\newchords{verse1.GB0000X} % Registrador de Acordes em Sequência
%\newchords{chorus1.GB0000X} % Registrador de Acordes em Sequência
%-----------------------------------------------------------------
%\seq{Intro}{}{}
%-----------------------------------------------------------------
%\beginverse* \endverse
%\beginchorus \endchorus
\beginverse*
Há um lugar de descanso em Ti
Há um lugar de refrigério em Ti
Há um lugar onde a verdade reina
Esse lugar é no Senhor
\endverse
\beginverse*
Há um lugar onde as pessoas não me influenciam
Há um lugar onde eu ouço teu Espírito
Há um lugar de vitória em meio à guerra
Esse lugar é no Senhor
\endverse
\beginchorus
Esse lugar é no Senhor
\endchorus
\beginverse*
Há um lugar onde a inconstância não me domina
Há um lugar onde minha fé é fortalecida
Há um lugar onde a paz é quem governa
Esse lugar é no Senhor
\endverse
\beginverse*
Há um lugar onde os sonhos não se abortam
Há um lugar onde os seus medos não te dominam
Há um lugar que quando se perde é que se ganha
Esse lugar é no Senhor
\endverse
\beginverse*
Tu és tudo o que eu preciso, Jesus!
\endverse

% Verso de preenchimento
\beginverse*\color{white}
.
\endverse
%-----------------------------------------------------------------
\begin{comment}
\lstset{basicstyle=\scriptsize\bf} % Parâmetros da TAB
%-----------------------------------------------------------------
\tab{Solo 1}
\begin{lstlisting}
E|-----------------------------------------------------|
B|-----------------------------------------------------|
G|-----------------------------------------------------|
D|-----------------------------------------------------|
A|-----------------------------------------------------|
E|-----------------------------------------------------|
\end{lstlisting}
%-----------------------------------------------------------------
\end{comment}
%=================================================================
 
%-----------------------------------------------------------------
\color{drawChord}\gtab{\color{nameChord} X}{}% 
\color{drawChord}\gtab{\color{nameChord} X}{}% 
\color{drawChord}\gtab{\color{nameChord} X}{}% 
\color{drawChord}\gtab{\color{nameChord} X}{}% 
%-----------------------------------------------------------------
% PADRÃO: [TonalidadeMaior+NOTAX+Variações] .Ex:[X50] [X57V1V7]
% OBS: Variações são alterações do acorde em relação ao campo harmônico.
%-----------------------------------------------------------------
% Tipos de Variações de Acordes:
% V0 - Variação Diversa
% V1 - Menor (m)
% V2 - Maior (M)
% V3 - Meio Tom Abaixo (Bemol)
% V4 - Com Quarta (ex:C4)
% V5 - Com Quinta (ex:C5)
% V6 - Com Sexta (ex:C6)
% V7 - Com Sétima Menor (ex:C7)
% V8 - Com baixo dois Tons Acima (ex:D/F#)
% V9 - Com Nona (ex:C9)
% V10 - Meio Tom Acima (Sustenido)
% V11 - Com Sétima Maior (ex:C7M)
% V12 - Suspenso (Sus)
% V13 - Com baixo dois Tons e Meio Acima (ex:A/E)
% V14 - Com baixo um Tom e Meio Acima (ex:D9/F) 
% V15 - Meio-Diminuto (m7b5)
% N15 - NÃO Meio-Diminuto
% V16 - Diminuto (º)
% N16 - NÃO Diminuto
%=================================================================
\endsong
%=================================================================
\begin{comment}

\end{comment}