%=================================================================
\songcolumns{1}
\beginsong
{Tudo é Teu %TÍTULO
}[by={Aline Barros %ARTISTA
},album={@walyssondosreis},
id={GB0123 %COD.ID.: GB0000
(Rev.0) %REVISÃO.: 0...N
}]
%-----------------------------------------------------------------
\tom{X1}
%=================================================================
%\newchords{verse0.GB0000} % Registrador de Acordes em Sequência
%-----------------------------------------------------------------
\seq{Intro}{X1 X3 X1}{}
%-----------------------------------------------------------------
\begin{verse*}
\[X1]De todo lugar perdidos virão 
\chordsoff Em uma só voz 
Livres vamos cantar 
Levaste a cruz,
\chordson Morreste e vivo est\[X6]ás
\[X6]Meu \[X6]Deus, \[X4]meu \[X4]tudo sempre te da\[X1]rei 
\end{verse*}
%-----------------------------------------------------------------
\seq{Riff Intro}{X1 X3 X1}{}
%-----------------------------------------------------------------
\begin{verse*}
\[X1]Mandaste Jesus do céu até nós 
\chordsoff A todos livrou 
Sempre se ouvirá 
Busquei a verdade 
\chordson E a Ti eu encont\[X6]rei
\[X6]Meu \[X6]Deus, \[X4]meu \[X4]tudo sempre te da\[X1]rei 
\end{verse*}
%-----------------------------------------------------------------
\begin{chorus}
\[X1]Jesus, por \[X5]Ti eu  viverei 
\[X6]Nunca me envergonha\[X4]rei ô ô ô!!
\[X1]Meu ser e \[X5]todo meu louvor 
\[X2]Teu, \[X2]Teu, \[X2]tudo é \[X2]Teu!! 
\[X2]Teu, \[X2]Teu, \[X2]tudo é \[X2]Teu!!  \[(X1)]
\end{chorus}
%-----------------------------------------------------------------
\begin{verse*}
\[X2]Bus\[X2]co a Ti e \[X6]\[X6]agora \[X5]\[X5]posso \[X4]\[X4]ver 
\[X2]\[X2]Seguirei a \[X6]\[X6]Tua \[X5]\[X5]luz
\[X2]\[X2]Pois em Tuas \[X6]\[X6]mãos \[X5]es\[X5]tá \[X4]Se\[X4]nhor a salva\[X1]ção!!\[(X1)]
\end{verse*}
%-----------------------------------------------------------------
\begin{comment}
\lstset{basicstyle=\scriptsize\bf} % Parâmetros da TAB
%-----------------------------------------------------------------
\tab{Solo 1}
\begin{lstlisting}
E|-----------------------------------------------------|
B|-----------------------------------------------------|
G|-----------------------------------------------------|
D|-----------------------------------------------------|
A|-----------------------------------------------------|
E|-----------------------------------------------------|
\end{lstlisting}
%-----------------------------------------------------------------
\end{comment}
%=================================================================
\vspace{2em}
%-----------------------------------------------------------------
\gtab{\color{black} X1}{}% G [X1]
\gtab{\color{black} X2}{}% Am [X2]
\gtab{\color{black} X3}{}% B [X3]
\gtab{\color{black} X4}{}% C [X4]
\gtab{\color{black} X5}{}% D [X5]
\gtab{\color{black} X6}{}% Em [X6]
%-----------------------------------------------------------------
% PADRÃO [TonalidadeMaiorNOTAX.Variação] .Ex:[X50] [X50V1]
% PADRÃO [TonalidadeMenorNOTAX.Variação] .Ex:[mX50] [mX50V1]
% OBS: Variações são alterações do acorde em relação ao campo harmônico.
%-----------------------------------------------------------------
% TIPOS DE VARIAÇÂO DOS ACORDES:
% V0 - ACORDE COM VARIAÇÃO DIVERSA
% V1 - ACORDE MENOR (m)
% V2 - ACORDE MAIOR (M)
% V3 - ACORDE MEIO TOM ABAIXO (Bemois)
% V4 - ACORDE COM QUARTA (C4)
% V5 - ACORDE COM QUINTA (C5)
% V6 - ACORDE COM SEXTA (C6)
% V7 - ACORDE COM SÉTIMA MENOR (C7)
% V8 - ACORDE COM BAIXO DOIS TONS ACIMA (D/F#)
% V9 - ACORDE COM NONA (C9)
% V10 - ACORDE MEIO TOM ACIMA (Sustenidos)
% V11 - ACORDE COM SÉTIMA MAIOR (C7M)
% V12 - ACORDE SUSPENSO (Sus)
% V13 - ACORDE COM BAIXO DOIS TONS E MEIO ACIMA (A/E)
% V14 - ACORDE UM TOM E MEIO ACIMA (D9/F)
%=================================================================
\endsong
%=================================================================








