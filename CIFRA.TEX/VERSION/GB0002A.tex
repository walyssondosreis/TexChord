%=================================================================
\songcolumns{1}
\beginsong
{A Alegria do Senhor %TÍTULO
}[by={Fernandinho %ARTISTA
},album={@walyssondosreis},
id={GB0002 %COD.ID.: GB0000
}] 
%-----------------------------------------------------------------
\tom{A}
%=================================================================
%\newchords{verse0.GB0000} % Registrador de Acordes em Sequência
%-----------------------------------------------------------------
\seq{Intro}{B5 A5 F\#5}{2x}
%-----------------------------------------------------------------
\begin{verse*}
\[F\#m]Vento sopra forte
\chordsoff Tuas águas não podem me afogar
\chordson \[F\#m]Vento sopra forte
\chordsoff E em suas mãos vou segurar
\chordson\end{verse*}
%-----------------------------------------------------------------
\seq{Riff Intro}{B5 A5 F\#5}{2x}
%-----------------------------------------------------------------
\begin{verse*}
\[Bm7]E não me guio pelo que vejo
\[F\#m7]Mas eu sigo pelo que creio
\[Bm7]Eu não olho as circunstâncias
\[D5]Eu vejo o teu a\[C\#5]mor \[D5 C\#5 D5 C\#5]
\end{verse*}
%-----------------------------------------------------------------
\begin{chorus}
\[F\#5]A alegria do Se\[A5]nhor é a nossa \[D5]força\[B5]
\[F\#5]A alegria do Se\[A5]nhor é a nossa \[D5]força\[B5]
\[F\#5]A alegria do Se\[A5]nhor é a nossa \[D5]força\[B5]
\[F\#5]A alegria do Se\[A5]nhor é a nossa \[D5]força\[B5]
\end{chorus}
%-----------------------------------------------------------------
\begin{verse*}
Essa ale\[A5]gria não vai mais sa\[D5]ir
Essa ale\[B5]gria não vai mais sa\[F\#5]ir
Essa ale\[A5]gria não vai mais sa\[D5]ir
De \[E5]dentro do meu cora\[A5]ção
\end{verse*}
%-----------------------------------------------------------------
\begin{comment}
\lstset{basicstyle=\scriptsize\bf} % Parâmetros da TAB
%-----------------------------------------------------------------
\tab{Solo 1}
\begin{lstlisting}
E|-----------------------------------------------------|
B|-----------------------------------------------------|
G|-----------------------------------------------------|
D|-----------------------------------------------------|
A|-----------------------------------------------------|
E|-----------------------------------------------------|
\end{lstlisting}
%-----------------------------------------------------------------
\end{comment}
%=================================================================
\vspace{2em}
%-----------------------------------------------------------------
% A = A [A]
\gtab{\color{black} A5}{}% A5 [A5]
\gtab{\color{black} B5}{}% B5 [B5]
\gtab{\color{black} Bm7}{}% Bm7 [Bm7]
\gtab{\color{black} C\#5}{}% C#5 [C\#5]
\gtab{\color{black} D5}{}% D5 [D5]
\gtab{\color{black} E5}{}% E5 [E5]
\gtab{\color{black} F\#m}{}% F#m [F\#m]
\gtab{\color{black} F\#m7}{}% F#m7 [F\#m7]
\gtab{\color{black} F\#5}{}% F#5 [F\#5]
%-----------------------------------------------------------------
% PADRÃO [TonalidadeMaiorNOTAX.Variação] .Ex:[X50] [X50V1]
% PADRÃO [TonalidadeMenorNOTAX.Variação] .Ex:[mX50] [mX50V1]
% OBS: Variações são alterações do acorde em relação ao campo harmônico.
%-----------------------------------------------------------------
% TIPOS DE VARIAÇÂO DOS ACORDES:
% V0 - ACORDE COM VARIAÇÃO DIVERSA
% V1 - ACORDE MENOR (m)
% V2 - ACORDE MAIOR (M)
% V3 - ACORDE MEIO TOM ABAIXO (Bemois)
% V4 - ACORDE COM QUARTA (C4)
% V5 - ACORDE COM QUINTA (C5)
% V6 - ACORDE COM SEXTA (C6)
% V7 - ACORDE COM SÉTIMA MENOR (C7)
% V8 - ACORDE COM BAIXO DOIS TONS ACIMA (D/F#)
% V9 - ACORDE COM NONA (C9)
% V10 - ACORDE MEIO TOM ACIMA (Sustenidos)
% V11 - ACORDE COM SÉTIMA MAIOR (C7M)
% V12 - ACORDE SUSPENSO (Sus)
% V13 - ACORDE COM BAIXO DOIS TONS E MEIO ACIMA (A/E)
% V14 - ACORDE UM TOM E MEIO ACIMA (D9/F)
%=================================================================
\endsong
%=================================================================
