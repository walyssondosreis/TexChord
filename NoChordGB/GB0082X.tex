%=================================================================
\songcolumns{2}
\beginsong
{Ousado Amor %TÍTULO
}[by={Isaias Saad %ARTISTA
},album={@walyssondosreis},
id={GB0082 %COD.ID.: GB0000
},rev={0}, %REVISÃO
qr={https://drive.google.com/open?id=1-UIWmVbXIFvXGtvpdx2ruqT9f6p83KdK %LINK
}]
%-----------------------------------------------------------------
\tom{X1}{F\#}
%=================================================================
%\newchords{verse1.GB0000X} % Registrador de Acordes em Sequência
%\newchords{chorus1.GB0000X} % Registrador de Acordes em Sequência
%-----------------------------------------------------------------
%\seq{Intro}{}{}
%-----------------------------------------------------------------
%\beginverse* \endverse
%\beginchorus \endchorus
\beginverse*
Antes de eu falar
Tu cantavas sobre mim
Tu tens sido tão, tão bom pra mim
Antes de eu respirar
Sopraste Tua vida em mim
Tu tens sido tão, tão bom pra mim
\endverse
\beginchorus
Oh, impressionante, infinito
E ousado amor de Deus
Oh, que deixa as noventa e nove
Só pra me encontrar
Não posso comprá-lo
Nem merecê-lo
Mesmo assim se entregou
Oh, impressionante, infinito
E ousado amor de Deus
\endchorus
\beginverse*
Inimigo eu fui
Mas Teu amor lutou por mim
Tu tens sido tão, tão bom pra mim
Não tinha valor
Mas tudo pagou por mim
Tu tens sido tão, tão bom pra mim
\endverse
\beginverse*
Traz luz para as sombras
Escala montanhas
Pra me encontrar
Derruba muralhas
Destrói as mentiras
Pra me encontrar
\endverse

%-----------------------------------------------------------------
\begin{comment}
\lstset{basicstyle=\scriptsize\bf} % Parâmetros da TAB
%-----------------------------------------------------------------
\tab{Solo 1}
\begin{lstlisting}
E|-----------------------------------------------------|
B|-----------------------------------------------------|
G|-----------------------------------------------------|
D|-----------------------------------------------------|
A|-----------------------------------------------------|
E|-----------------------------------------------------|
\end{lstlisting}
%-----------------------------------------------------------------
\end{comment}
%=================================================================
\vspace{2em} 
%-----------------------------------------------------------------
\color{drawChord}\gtab{\color{nameChord} X}{}% 
\color{drawChord}\gtab{\color{nameChord} X}{}% 
\color{drawChord}\gtab{\color{nameChord} X}{}% 
\color{drawChord}\gtab{\color{nameChord} X}{}% 
%-----------------------------------------------------------------
% PADRÃO: [TonalidadeMaior+NOTAX+Variações] .Ex:[X50] [X57V1V7]
% OBS: Variações são alterações do acorde em relação ao campo harmônico.
%-----------------------------------------------------------------
% Tipos de Variações de Acordes:
% V0 - Variação Diversa
% V1 - Menor (m)
% V2 - Maior (M)
% V3 - Meio Tom Abaixo (Bemol)
% V4 - Com Quarta (ex:C4)
% V5 - Com Quinta (ex:C5)
% V6 - Com Sexta (ex:C6)
% V7 - Com Sétima Menor (ex:C7)
% V8 - Com baixo dois Tons Acima (ex:D/F#)
% V9 - Com Nona (ex:C9)
% V10 - Meio Tom Acima (Sustenido)
% V11 - Com Sétima Maior (ex:C7M)
% V12 - Suspenso (Sus)
% V13 - Com baixo dois Tons e Meio Acima (ex:A/E)
% V14 - Com baixo um Tom e Meio Acima (ex:D9/F) 
% V15 - Meio-Diminuto (m7b5)
% N15 - NÃO Meio-Diminuto
% V16 - Diminuto (º)
% N16 - NÃO Diminuto
%=================================================================
\endsong
%=================================================================
\begin{comment}

\end{comment}