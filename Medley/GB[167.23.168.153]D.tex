%=================================================================
\songcolumns{2}
\beginsong
{Medley %TÍTULO
}[by={Vários Artistas %ARTISTA
},album={@walyssondosreis},
id={GB[167.23.168.153] %COD.ID.: XXNNNN
},rev={0}, %REVISÃO
qr={ %LINK
}]
%-----------------------------------------------------------------
\tom{D}{D}
%=================================================================
%\newchords{verse1.XX0000X} % Registrador de Acordes em Sequência
%\newchords{chorus1.XX0000X} % Registrador de Acordes em Sequência
%-----------------------------------------------------------------
\act{\bf\color{red} ATRAI O MEU CORAÇÃO - Filhos do Homem\color{black}}{}{Música Inteira}
\seq{Intro}{D9 D7M(9) D4(9) G/B}{2x}
%\act{}{}{}
%-----------------------------------------------------------------
%\beginverse \endverse
%\beginchorus \endchorus
\beginverse
\[D] Tu és minha \[G]vida, Je\[A/C\#]sus
És meu a\[D]migo \[(Bm)]
E a tua von\[G]tade, Doce Es\[A/C\#]pírito
Meu ali\[D]mento
\endverse
\beginverse
Sem \[G]ti não há valor em \[D/F\#]mim, sou como um vaso de \[Em]barro
Pronto a ser que\[G]brado para ser o que \[A4]queres de \[A]mim
\endverse
\beginverse
A tua pre\[G]sença é tudo que eu pre\[D/F\#]ciso
A tua pre\[G]sença é o meu maior va\[D/F\#]lor
\endverse
\beginchorus
Atrai o meu \[G]coração
Atrai o meu \[Em]coração
És tudo que eu \[D]quero
Atrai o meu \[G]coração
Atrai o meu \[Em]coração
Eu posso te to\[D]car
\endchorus
\seq{Interlúdio}{G A/C\# G/B A4(7) D}{1x}
\act{Retomar}{Verso 1}{1x}
\act{Repetir}{Refrão}{+1x}
\seq{Interlúdio}{G A/C\# G/B A4(7) D}{1x}
\seq{\bf\color{red} DEIXA QUEIMAR - Alessandro Vilas Boas \color{black}}{D A Bm7 G}{ Música Inteira}

\act{\bf\color{red} QUÃO GRANDE É O MEU DEUS - Soraya Moraes\color{black}}{}{Apenas Refrão}
\beginchorus
Quão \[D]grande é o meu Deus
Cantarei quão \[Bm]grande é o meu Deus
E todos hão de \[G]ver
Quão \[A]grande é o meu \[D]Deus \[A]
\endchorus

\act{\bf\color{red} CONSAGRAÇÃO - Aline Barros\color{black}}{}{Apenas Refrão}
\beginchorus
A \[D]honra, a \[A/C\#]glória, a \[Bm]força
E \[F\#m]o poder ao rei Je\[G]sus \[D/F\#]
E o lou\[Em]vor \[D] ao \[C]rei \[G/B] Je\[A]sus
\endchorus
%-----------------------------------------------------------------
\begin{comment}
\lstset{basicstyle=\scriptsize\bf} % Parâmetros da TAB
%-----------------------------------------------------------------
\tab{Solo 1}
\begin{lstlisting}
E|-----------------------------------------------------|
B|-----------------------------------------------------|
G|-----------------------------------------------------|
D|-----------------------------------------------------|
A|-----------------------------------------------------|
E|-----------------------------------------------------|
\end{lstlisting}
%-----------------------------------------------------------------
\end{comment}
%=================================================================
\begin{comment}

%-----------------------------------------------------------------
\color{drawChord}\gtab{\color{nameChord} D}{}% 
\color{drawChord}\gtab{\color{nameChord} D4(9)}{}%
\color{drawChord}\gtab{\color{nameChord} D/F\#}{}% 
\color{drawChord}\gtab{\color{nameChord} D9}{}% 
\color{drawChord}\gtab{\color{nameChord} D7M(9)}{}% 
\color{drawChord}\gtab{\color{nameChord} Em}{}% 
\color{drawChord}\gtab{\color{nameChord} F\#m}{}%
\color{drawChord}\gtab{\color{nameChord} G}{}% 
\color{drawChord}\gtab{\color{nameChord} G/B}{}% 
\color{drawChord}\gtab{\color{nameChord} A}{}% 
\color{drawChord}\gtab{\color{nameChord} A4}{}%
\color{drawChord}\gtab{\color{nameChord} A4(7)}{}% 
\color{drawChord}\gtab{\color{nameChord} A/C\#}{}% 
\color{drawChord}\gtab{\color{nameChord} Bm}{}% 
\color{drawChord}\gtab{\color{nameChord} Bm7}{}%
\color{drawChord}\gtab{\color{nameChord} C}{}%
\end{comment}
%-----------------------------------------------------------------
% PADRÃO: [TonalidadeMaior+NOTAX+Variações] .Ex:[X50] [X57V1V7]
% OBS: Variações são alterações do acorde em relação ao campo harmônico.
%-----------------------------------------------------------------
% Tipos de Variações de Acordes:
% V0 - Variação Diversa
% V1 - Menor (m)
% V2 - Maior (M)
% V3 - Meio Tom Abaixo (Bemol)
% V4 - Com Quarta (ex:C4)
% V5 - Com Quinta (ex:C5)
% V6 - Com Sexta (ex:C6)
% V7 - Com Sétima Menor (ex:C7)
% V8 - Com baixo dois Tons Acima (ex:D/F#)
% V9 - Com Nona (ex:C9)
% V10 - Meio Tom Acima (Sustenido)
% V11 - Com Sétima Maior (ex:C7M)
% V12 - Suspenso (Sus)
% V13 - Com baixo dois Tons e Meio Acima (ex:A/E)
% V14 - Com baixo um Tom e Meio Acima (ex:D9/F) 
% V15 - Meio-Diminuto (m7b5)
% N15 - NÃO Meio-Diminuto
% V16 - Diminuto (º)
% N16 - NÃO Diminuto
% V17 - Com baixo um Tom Acima (ex: C/D)
% V18 - Com baixo um Tom Abaixo (ex: Em/D)
% V19 - Com baixo dois Tons e meio Abaixo (ex: G/D)
%=================================================================
\endsong
%=================================================================
\begin{comment}

\end{comment}