%=================================================================
\songcolumns{2}
\beginsong
{Wonder %TÍTULO
}[by={Hillsong United  %ARTISTA
},album={@walyssondosreis},
id={GI0007 %COD.ID.: XX0000
},rev={0}, %REVISÃO
qr={ %LINK
}]
%-----------------------------------------------------------------
\tom{X1}{X1}
%=================================================================
%\newchords{verse1.XX0000X} % Registrador de Acordes em Sequência
%\newchords{chorus1.XX0000X} % Registrador de Acordes em Sequência
%-----------------------------------------------------------------
%\seq{Intro}{}{}
%\act{}{}{}
%-----------------------------------------------------------------
%\beginverse \endverse
%\beginchorus \endchorus
\beginverse
Have you ever seen the wonder
In the glimmer of first sight
As the eyes begin to open
And the blindness meets the light
If you have so say
\endverse
\beginchorus
I see the world in light
I see the world in wonder
I see the world in life
Bursting in living colour
I see the world Your way
And I'm walking in the light
\endchorus
\beginverse
Have you ever seen the wonder
In the air of second life
Having come out of the waters
With the old one left behind
If you have so say
\endverse
\beginverse
I see the world in grace
I see the world in gospel
I see the world Your way
And I'm walking in the light
I'm walking in the wonder
You're the wonder in the wild
Turning wilderness to wonder
If You have so say
I see the world in love
I see the world in freedom
I see the Jesus way
You're the wonder in the wild
\endverse
\beginverse
I see the world Your way
And I'm not afraid to follow
I see the world Your way
And I'm not ashamed to say so
I see the Jesus way
And I'm walking in the light
\endverse

%-----------------------------------------------------------------
\begin{comment}
\lstset{basicstyle=\scriptsize\bf} % Parâmetros da TAB
%-----------------------------------------------------------------
\tab{Solo 1}
\begin{lstlisting}
E|-----------------------------------------------------|
B|-----------------------------------------------------|
G|-----------------------------------------------------|
D|-----------------------------------------------------|
A|-----------------------------------------------------|
E|-----------------------------------------------------|
\end{lstlisting}
%-----------------------------------------------------------------
\end{comment}
%=================================================================
\vspace{2em} 
%-----------------------------------------------------------------
\color{drawChord}\gtab{\color{nameChord} X}{}% 
\color{drawChord}\gtab{\color{nameChord} X}{}% 
\color{drawChord}\gtab{\color{nameChord} X}{}% 
\color{drawChord}\gtab{\color{nameChord} X}{}% 
%-----------------------------------------------------------------
% PADRÃO: [TonalidadeMaior+NOTAX+Variações] .Ex:[X50] [X57V1V7]
% OBS: Variações são alterações do acorde em relação ao campo harmônico.
%-----------------------------------------------------------------
% Tipos de Variações de Acordes:
% V0 - Variação Diversa
% V1 - Menor (m)
% V2 - Maior (M)
% V3 - Meio Tom Abaixo (Bemol)
% V4 - Com Quarta (ex:C4)
% V5 - Com Quinta (ex:C5)
% V6 - Com Sexta (ex:C6)
% V7 - Com Sétima Menor (ex:C7)
% V8 - Com baixo dois Tons Acima (ex:D/F#)
% V9 - Com Nona (ex:C9)
% V10 - Meio Tom Acima (Sustenido)
% V11 - Com Sétima Maior (ex:C7M)
% V12 - Suspenso (Sus)
% V13 - Com baixo dois Tons e Meio Acima (ex:A/E)
% V14 - Com baixo um Tom e Meio Acima (ex:D9/F) 
% V15 - Meio-Diminuto (m7b5)
% N15 - NÃO Meio-Diminuto
% V16 - Diminuto (º)
% N16 - NÃO Diminuto
% V17 - Com baixo um Tom Acima (ex: C/D)
%=================================================================
\endsong
%=================================================================
\begin{comment}

\end{comment}