%=================================================================
\songcolumns{2}
\beginsong
{Toda Sorte de Bençãos %TÍTULO
}[by={Davi Sacer %ARTISTA
},album={@walyssondosreis},
id={GB0120 %COD.ID.: GB0000
},rev={0}, %REVISÃO
qr={https://drive.google.com/open?id=1Nu5r7rhQg9SR1C6uyum6NT0X9SWoClP5 %LINK
}]
%-----------------------------------------------------------------
\tom{X1}{A}
%=================================================================
%\newchords{verse1.GB0000X} % Registrador de Acordes em Sequência
%\newchords{chorus1.GB0000X} % Registrador de Acordes em Sequência
%-----------------------------------------------------------------
\seq{Intro}{X1 X5 X4}{2x}
%-----------------------------------------------------------------
%\beginverse* \endverse
%\beginchorus \endchorus
\beginverse*
\[X1] Por onde eu \[X5]for a tua \[X4]bênção me segui\[X1]rá
Onde eu \[X5]colocar as \[X4]minhas mãos prospera\[X1]rá
A minha en\[X5]trada e a minha sa\[X4]ída bendita se\[X1]rá
Pois sobre \[X5]mim há uma pro\[X4]messa
\[X2]Pros\[X6]pera\[X5]rei, \[X2]trans\[X6]borda\[X5]rei
\endverse
\beginverse*
^ Os meus ce^leiros farta^mente se enche^rão
A minha ^casa terá ^sempre tua provi^são
Onde eu pu^ser a planta ^dos meus pés, possui^rei
Pois sobre ^mim há uma pro^messa
^Pros^pera^rei, ^trans^borda^rei
\endverse
\beginchorus
Para di\[X1]reita, \[X5] para es\[X6]querda \[X4]
A minha \[X1]frente \[X5]
E para \[X4]trás
Por todo \[X1]lado, \[X5]uoo\[X6]o
Sou \[X4]abenço\[X1]ado, \[X5]iéé\[X6]é
Em \[X4]tudo o que eu \[X1]faço, \[X5]uoo\[X6]o
Sou \[X4]abenço\[X1]ado, \[X5]iéé\[X4]é
\endchorus
\beginverse*
Toda \[X1]sorte \[X5] de \[X6]bençãos \[X4]
O se\[X1]nhor prepa\[X5]rou para \[X6]mim \[X4]
E em \[X1]todas \[X5] as \[X6]coisas \[X4]
Eu sou \[X1]mais do \[X5]que vence\[X4]dor
\endverse

%-----------------------------------------------------------------
\begin{comment}
\lstset{basicstyle=\scriptsize\bf} % Parâmetros da TAB
%-----------------------------------------------------------------
\tab{Solo 1}
\begin{lstlisting}
E|-----------------------------------------------------|
B|-----------------------------------------------------|
G|-----------------------------------------------------|
D|-----------------------------------------------------|
A|-----------------------------------------------------|
E|-----------------------------------------------------|
\end{lstlisting}
%-----------------------------------------------------------------
\end{comment}
%=================================================================
\vspace{2em} 
%-----------------------------------------------------------------
\color{drawChord}\gtab{\color{nameChord} X1}{}% 
\color{drawChord}\gtab{\color{nameChord} X2}{}% 
\color{drawChord}\gtab{\color{nameChord} X4}{}% 
\color{drawChord}\gtab{\color{nameChord} X5}{}%
\color{drawChord}\gtab{\color{nameChord} X6}{}%
%-----------------------------------------------------------------
% PADRÃO: [TonalidadeMaior+NOTAX+Variações] .Ex:[X50] [X57V1V7]
% OBS: Variações são alterações do acorde em relação ao campo harmônico.
%-----------------------------------------------------------------
% Tipos de Variações de Acordes:
% V0 - Variação Diversa
% V1 - Menor (m)
% V2 - Maior (M)
% V3 - Meio Tom Abaixo (Bemol)
% V4 - Com Quarta (ex:C4)
% V5 - Com Quinta (ex:C5)
% V6 - Com Sexta (ex:C6)
% V7 - Com Sétima Menor (ex:C7)
% V8 - Com baixo dois Tons Acima (ex:D/F#)
% V9 - Com Nona (ex:C9)
% V10 - Meio Tom Acima (Sustenido)
% V11 - Com Sétima Maior (ex:C7M)
% V12 - Suspenso (Sus)
% V13 - Com baixo dois Tons e Meio Acima (ex:A/E)
% V14 - Com baixo um Tom e Meio Acima (ex:D9/F) 
% V15 - Meio-Diminuto (m7b5)
% N15 - NÃO Meio-Diminuto
% V16 - Diminuto (º)
% N16 - NÃO Diminuto
%=================================================================
\endsong
%=================================================================
\begin{comment}

\end{comment}