%=================================================================
\songcolumns{2}
\beginsong
{O Poder do Criador %TÍTULO
}[by={Goiano e Paranaense %ARTISTA
},album={@walyssondosreis},
id={SB0132 %COD.ID.: XXNNNN
},rev={3}, %REVISÃO
qr={https://music.youtube.com/watch?v=jOADeghWRhA&feature=share %LINK
}]
%-----------------------------------------------------------------
\tom{E}{E}
%=================================================================
%\newchords{verse1.XX0000X} % Registrador de Acordes em Sequência
%\newchords{chorus1.XX0000X} % Registrador de Acordes em Sequência
%-----------------------------------------------------------------
\seq{Intro}{E  B  E  E7  A | E  B  E  B  E}{1x}
%\act{}{}{}
%-----------------------------------------------------------------
\lstset{basicstyle=\scriptsize\bf} % Parâmetros da TAB
%-----------------------------------------------------------------
\act{Solo}{Parte 1}{1x}
\act{Solo}{Parte 2}{1x}
%-----------------------------------------------------------------
\beginverse 
\[E] Hora triste foi aquela, Que Jesus Cristo fa\[B]lou
\[B] Mãe está chegando a hora, A senhora fica e eu \[E]vou
\[E] Com certeza mãe e filho, Neste momento cho\[B]rou
\[B] Hora triste dolo\[E]rida Porque a dor da despe\[B]dida
Só co\[A]nhece \[B]quem pa\[E]ssou
\endverse
\act{Repetir}{Solo: Parte 2}{1x}
\beginverse
\[E] Maria disse meu filho, Faz tudo que o Pai man\[B]dou
\[B] Pra salvar a humanidade, Ele lhe determi\[E]nou
\[E] Com as lágrimas caindo, O seu rosto ele bei\[B]jou
\[B] Pra cumprir a profe\[E]cia Naquele instante o Me\[B]ssias
Todo \[A]pecado \[B]abra\[E]çou
\endverse
\act{Repetir}{Solo: Parte 1 | Parte 2}{1x}
\beginverse
\[E] Nas margens do Rio Jordão, Jesus Cristo cami\[B]nhou
\[B] Para encontrar João, Aquele que testemu\[E]nhou
\[E] O encontro foi tão lindo, Que o povo se emocio\[B]nou
\[B] Também foi nessa vi\[E]sita Que nas mãos de João Ba\[B]tista
Jesus \[A]Cristo \[B]bati\[E]zou
\endverse
\act{Repetir}{Solo: Parte 2}{1x}
\beginverse
\[E] Na mesa da Santa Ceia, Jesus Cristo orde\[B]nou
\[B] Ensine os meus mandamentos, Que onde está meu Pai eu \[E]vou
\[E] Se o mundo lhes odiar, Também já me odi\[B]ou
\[B] Faça o bem sem ver a \[E]quem A sua recompensa \[B]vem
É o que Je\[A]sus p\[B]rofeti\[E]zou
\endverse

%-----------------------------------------------------------------
\vspace{4em} % Regulador de Espaçamento
%=================================================================
\color{drawChord}\gtab{\color{black} B}{}
\color{drawChord}\gtab{\color{black} E}{}
\color{drawChord}\gtab{\color{black} A}{}

%=================================================================
% PADRÃO: [TonalidadeMaior+NOTAX+Variações] .Ex:[B0] [B7V1V7]
% OBS: Variações são alterações do acorde em relação ao campo harmônico.
%-----------------------------------------------------------------
% Tipos de Variações de Acordes:
% V0 - Variação Diversa
% V1 - Menor (m)
% V2 - Maior (M)
% V3 - Meio Tom Abaixo (Bemol)
% V4 - Com Quarta (ex:C4)
% V5 - Com Quinta (ex:C5)
% V6 - Com Sexta (ex:C6)
% V7 - Com Sétima Menor (ex:C7)
% V8 - Com baixo dois Tons Acima (ex:D/F#)
% V9 - Com Nona (ex:C9)
% V10 - Meio Tom Acima (Sustenido)
% V11 - Com Sétima Maior (ex:C7M)
% V12 - Suspenso (Sus)
% V13 - Com baixo dois Tons e Meio Acima (ex:A/E)
% V14 - Com baixo um Tom e Meio Acima (ex:D9/F) 
% V15 - Meio-Diminuto (m7b5)
% N15 - NÃO Meio-Diminuto
% V16 - Diminuto (º)
% N16 - NÃO Diminuto
% V17 - Com baixo um Tom Acima (ex: C/D)
% V18 - Com baixo um Tom Abaixo (ex: Em/D)
% V19 - Com baixo dois Tons e meio Abaixo (ex: G/D)
%=================================================================
%\vspace{4em} % Regulador de Espaçamento
%=================================================================
\endsong
%=================================================================
