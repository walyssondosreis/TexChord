%=================================================================
\songcolumns{1}
\beginsong
{Infinitamente Mais %TÍTULO
}[by={Fernandinho %ARTISTA
},album={@walyssondosreis},
id={GB0053 %COD.ID.: GB0000
},rev={3}, %REVISÃO
qr={https://drive.google.com/open?id=1BSXxOnvJexKv7kjbZplZOIeXLMJ4fFxK %LINK
}]
%-----------------------------------------------------------------
\tom{X1}{Ab}
%=================================================================
%\newchords{verse1.GB0000X} % Registrador de Acordes em Sequência
%\newchords{chorus1.GB0000X} % Registrador de Acordes em Sequência
%-----------------------------------------------------------------
\seq{Intro}{X6}{}
%-----------------------------------------------------------------
%\beginverse* \endverse
%\beginchorus \endchorus
\beginverse
\[X6]Minhas lágrimas \[X4V9]enchem os Seus odres
\[X6]Meu clamor está es\[X4V9]crito no Seu livro
\endverse
\act{Executar}{Solo 1}{}
\beginverse
\[X6]Uooo\[X4V9]oh!
\[X6]Uooo\[X4V9]oh!
\endverse
\beginverse
\[X6]Minhas lágrimas \[X4V9]enchem os Seus odres
\[X6]Meu clamor está es\[X4V9]crito no Seu livro
\[X2] As muitas águas não \[X6]podem me afogar, não
\[X2] Estou esperando Tua \[X4V9]luz em meio à \[X5V9]noite \[X4V9]
\endverse
\beginchorus
Mas eu \[X1]sei em quem eu tenho crido
Também \[X5V9]sei que é Poderoso pra fa\[X6]zer
Infi\[X4V9]nitamente \[X1]mais, mais, \[X5V9]mais
\endchorus
\act{Repetir}{Refrão}{+2x}
\act{Repetir}{Verso 2}{1x}
\act{Retomar}{Verso 3}{1x}
\act{Repetir}{Refrão}{+1x}
\beginverse*
.
.
.
.
\endverse
%-----------------------------------------------------------------
\begin{comment}
\lstset{basicstyle=\scriptsize\bf} % Parâmetros da TAB
%-----------------------------------------------------------------
\tab{Solo 1}
\begin{lstlisting}
E|-----------------------------------------------------|
B|-----------------------------------------------------|
G|-----------------------------------------------------|
D|-----------------------------------------------------|
A|-----------------------------------------------------|
E|-----------------------------------------------------|
\end{lstlisting}
%-----------------------------------------------------------------
\end{comment}
%=================================================================
\vspace{2em} 
%-----------------------------------------------------------------
\gtab{\color{black} X1}{}%  
\gtab{\color{black} X2}{}%  
\gtab{\color{black} X4V9}{}%  
\gtab{\color{black} X5V9}{}%  
\gtab{\color{black} X6}{}% 
%-----------------------------------------------------------------
% PADRÃO: [TonalidadeMaior+NOTAX+Variações] .Ex:[X50] [X57V1V7]
% OBS: Variações são alterações do acorde em relação ao campo harmônico.
%-----------------------------------------------------------------
% Tipos de Variações de Acordes:
% V0 - Variação Diversa
% V1 - Menor (m)
% V2 - Maior (M)
% V3 - Meio Tom Abaixo (Bemol)
% V4 - Com Quarta (ex:C4)
% V5 - Com Quinta (ex:C5)
% V6 - Com Sexta (ex:C6)
% V7 - Com Sétima Menor (ex:C7)
% V8 - Com baixo dois Tons Acima (ex:D/F#)
% V9 - Com Nona (ex:C9)
% V10 - Meio Tom Acima (Sustenido)
% V11 - Com Sétima Maior (ex:C7M)
% V12 - Suspenso (Sus)
% V13 - Com baixo dois Tons e Meio Acima (ex:A/E)
% V14 - Com baixo um Tom e Meio Acima (ex:D9/F) 
% V15 - Meio-Diminuto (m7b5)
% N15 - NÃO Meio-Diminuto
% V16 - Diminuto (º)
% N16 - NÃO Diminuto
%=================================================================
\endsong
%=================================================================
\begin{comment}

\end{comment}