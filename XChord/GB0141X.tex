%=================================================================
\songcolumns{2}
\beginsong
{O Fogo Nunca Dorme %TÍTULO
}[by={Alessandro Vilas Boas %ARTISTA
},album={@walyssondosreis},
id={GB0141 %COD.ID.: GB0000
},rev={3}, %REVISÃO
qr={https://drive.google.com/open?id=1n3ZpXReGtyJbddLbVc2eHC15Tz1ZWhkK %LINK
}]
%-----------------------------------------------------------------
\tom{X1}{C}
%=================================================================
%\newchords{verse1.GB0000X} % Registrador de Acordes em Sequência
%\newchords{chorus1.GB0000X} % Registrador de Acordes em Sequência
%-----------------------------------------------------------------
\seq{Intro}{X6 X4}{2x}
%-----------------------------------------------------------------
%\beginverse* \endverse
%\beginchorus \endchorus
\beginverse
\[X6] Existe um trono colocado no mais \[X4]alto lugar
\[X6] Rodeado de anjos
Onde Deus \[X4]reina
\endverse
\beginverse
\[X6] E a minha \[X5]vida Ele go\[X4]verna
\[X6 ]E a minha \[X5] vida Ele go\[X4]verna
\endverse
\beginverse
\[X1] E esse Rei veio a \[X6]mim
\[X5] Com Seus olhos em \[X4]chamas
\[X1] E eu perguntei sobre Seus \[X6]olhos
So\[X5]rrindo me disse
\endverse
\beginchorus
O \[X4]fogo nunca dorme
O fogo nunca apaga\[X6]
Nunca apaga\[X1]rá não, não, \[X5]não
\[X4]Deus nunca morre
Deus nunca é ven\[X6]cido
Nunca é ven\[X1]cido\[X5]
\endchorus
\beginverse
\[X6] Existe um amor mais ciumento que a \[X4]morte
\[X6] Existe um amor mais ciumento que a \[X4]morte
\endverse
\act{Repetir}{Verso 2}{1x}
\beginverse
\[X1] E esse Amor veio a \[X6]mim
\[X5] Com Seus olhos em \[X4]chamas
\[X1] E eu perguntei sobre Seus \[X6]olhos
Ele so\[X5]rrindo me disse
\endverse
\act{Repetir}{Refrão}{3x}

%-----------------------------------------------------------------
\begin{comment}
\lstset{basicstyle=\scriptsize\bf} % Parâmetros da TAB
%-----------------------------------------------------------------
\tab{Solo 1}
\begin{lstlisting}
E|-----------------------------------------------------|
B|-----------------------------------------------------|
G|-----------------------------------------------------|
D|-----------------------------------------------------|
A|-----------------------------------------------------|
E|-----------------------------------------------------|
\end{lstlisting}
%-----------------------------------------------------------------
\end{comment}
%=================================================================
\vspace{2em} 
%-----------------------------------------------------------------
\color{act-gray}\gtab{\color{act-gray} X1}{}% 
\color{act-gray}\gtab{\color{act-gray} X4}{}% 
\color{act-gray}\gtab{\color{act-gray} X5}{}% 
\color{act-gray}\gtab{\color{act-gray} X6}{}% 
%-----------------------------------------------------------------
% PADRÃO: [TonalidadeMaior+NOTAX+Variações] .Ex:[X50] [X57V1V7]
% OBS: Variações são alterações do acorde em relação ao campo harmônico.
%-----------------------------------------------------------------
% Tipos de Variações de Acordes:
% V0 - Variação Diversa
% V1 - Menor (m)
% V2 - Maior (M)
% V3 - Meio Tom Abaixo (Bemol)
% V4 - Com Quarta (ex:C4)
% V5 - Com Quinta (ex:C5)
% V6 - Com Sexta (ex:C6)
% V7 - Com Sétima Menor (ex:C7)
% V8 - Com baixo dois Tons Acima (ex:D/F#)
% V9 - Com Nona (ex:C9)
% V10 - Meio Tom Acima (Sustenido)
% V11 - Com Sétima Maior (ex:C7M)
% V12 - Suspenso (Sus)
% V13 - Com baixo dois Tons e Meio Acima (ex:A/E)
% V14 - Com baixo um Tom e Meio Acima (ex:D9/F) 
% V15 - Meio-Diminuto (m7b5)
% N15 - NÃO Meio-Diminuto
% V16 - Diminuto (º)
% N16 - NÃO Diminuto
%=================================================================
\endsong
%=================================================================
\begin{comment}

\end{comment}