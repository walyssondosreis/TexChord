%=================================================================
\songcolumns{1}
\beginsong
{Faz Chover %TÍTULO
}[by={Fernandinho %ARTISTA
},album={@walyssondosreis},
id={GB0043 %COD.ID.: GB0000
},rev={3}, %REVISÃO
qr={https://drive.google.com/open?id=1LNR_pNtM8JnORzK_gLmbF_zNXb-HyDx4 %LINK
}]
%-----------------------------------------------------------------
\tom{X1}{D}
%=================================================================
%\newchords{verse1.GB0000X} % Registrador de Acordes em Sequência
%\newchords{chorus1.GB0000X} % Registrador de Acordes em Sequência
%-----------------------------------------------------------------
\seq{Intro}{X6 X4 X1 X5V9}{2x}
\seq{Intro}{X4V9}{}
%-----------------------------------------------------------------
%\beginverse* \endverse
%\beginchorus \endchorus

\beginverse
As\[X1]sim como a corça
An\[X5V9]seia por \[X6]água
\[X4]Como terra \[X1V9V8]seca
Pre\[X2]cisa da \[X5V9]chuva
Meu \[X4]cora\[X5V9]ção tem \[X1V9V8]sede de \[X4V9]ti
Rei \[X1]meu e Deus \[X5V9]meu
\endverse
\act{Repetir}{Verso 1}{+1x}
\beginchorus
Faz cho\[X6]ver \[X4] Se\[X1]nhor Je\[X5V9]sus
Derrama a \[X6]chuva \[X4]neste lu\[X5V9]gar
Vem com teu \[X6]rio \[X4] Se\[X1]nhor Je\[X5V9]sus
Inun\[X6]dar o \[X4]meu cora\[X5V9]ção
\endchorus
\act{Repetir}{Refrão}{+1x}
\act{Retomar}{Verso 1}{1x}
\beginverse
...Faz cho\[X6]ver \[X4 X1 X5V9]
\endverse
\act{Repetir}{Verso 2}{1x}
\beginverse*
.
.
.
.
\endverse
%-----------------------------------------------------------------
\begin{comment}
\lstset{basicstyle=\scriptsize\bf} % Parâmetros da TAB
%-----------------------------------------------------------------
\tab{Solo 1}
\begin{lstlisting}
E|-----------------------------------------------------|
B|-----------------------------------------------------|
G|-----------------------------------------------------|
D|-----------------------------------------------------|
A|-----------------------------------------------------|
E|-----------------------------------------------------|
\end{lstlisting}
%-----------------------------------------------------------------
\end{comment}
%=================================================================
\vspace{2em}
%-----------------------------------------------------------------
\gtab{\color{black} X1}{}% D [X1]
\gtab{\color{black} X1V9V8}{}% D9/F# [X1V9V8]
\gtab{\color{black} X2}{}% Em [X2]
\gtab{\color{black} X4}{}% G [X4]
\gtab{\color{black} X4V9}{}% G9 [X4V9]
\gtab{\color{black} X5V9}{}% A9 [X5V9]
\gtab{\color{black} X6}{}% Bm [X6]
%-----------------------------------------------------------------
% PADRÃO [TonalidadeMaiorNOTAX.Variação] .Ex:[X50] [X50V1]
% PADRÃO [TonalidadeMenorNOTAX.Variação] .Ex:[mX50] [mX50V1]
% OBS: Variações são alterações do acorde em relação ao campo harmônico.
%-----------------------------------------------------------------
% TIPOS DE VARIAÇÂO DOS ACORDES:
% V0 - ACORDE COM VARIAÇÃO DIVERSA
% V1 - ACORDE MENOR (m)
% V2 - ACORDE MAIOR (M)
% V3 - ACORDE MEIO TOM ABAIXO (Bemois)
% V4 - ACORDE COM QUARTA (C4)
% V5 - ACORDE COM QUINTA (C5)
% V6 - ACORDE COM SEXTA (C6)
% V7 - ACORDE COM SÉTIMA MENOR (C7)
% V8 - ACORDE COM BAIXO DOIS TONS ACIMA (D/F#)
% V9 - ACORDE COM NONA (C9)
% V10 - ACORDE MEIO TOM ACIMA (Sustenidos)
% V11 - ACORDE COM SÉTIMA MAIOR (C7M)
% V12 - ACORDE SUSPENSO (Sus)
% V13 - ACORDE COM BAIXO DOIS TONS E MEIO ACIMA (A/E)
% V14 - ACORDE UM TOM E MEIO ACIMA (D9/F)
%=================================================================
\endsong
%=================================================================
\begin{comment}

\end{comment}