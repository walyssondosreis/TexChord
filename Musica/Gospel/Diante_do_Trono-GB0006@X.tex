%=================================================================
\songcolumns{2}
\beginsong
{Aclame Ao Senhor %TÍTULO
}[by={Diante do Trono %ARTISTA
},album={@walyssondosreis},
id={GB0006 %COD.ID.: XXNNNN
},rev={3}, %REVISÃO
qr={https://drive.google.com/open?id=1oc7xmGkAdxx3SRjtyY3W9WeXnvwXpsFB %LINK
}]
%-----------------------------------------------------------------
\tom{X1}{A}
%=================================================================
%\newchords{verse1.GB0000X} % Registrador de Acordes em Sequência
%\newchords{chorus1.GB0000X} % Registrador de Acordes em Sequência
%-----------------------------------------------------------------
\seq{Intro}{X1 X6}{2x}
%-----------------------------------------------------------------
%\beginverse \endverse
%\beginchorus \endchorus
\beginverse
\[X1V9] Meu Jesus,\[X5] salvador
\[X6]Outro i\[X5]gual não \[X4]há
Todos os \[X1V8]dias \[X4]quero lou\[X1]var
As \[X6]maravilhas \[X7V3N15]de \[X4V8]teu a\[X5V4]mor \[X5]
\endverse
\beginverse
^ Consolo,^ abrigo
^Força e re^fúgio é o Se^nhor
Com todo o meu ^ser
Com ^tudo o que ^sou
^Sempre Te a^do^ra^rei ^
\endverse
\beginchorus
A\[X1V9]clame ao Se\[X6]nhor toda a \[X4]terra e can\[X5V4]te\[X5]mos
Po\[X1V9]der, majes\[X6]tade e lou\[X4]vores ao \[X5]Rei \[X5V4]
Mon\[X6]tanhas se \[X5]prostrem e \[X4]rujam os mares
Ao \[X5]som \[X4V8]de teu \[X5V8]nome
A\[X1V9]legre Te \[X6]louvo por \[X4]Teus grandes \[X5V4]fei\[X5]tos
Fir\[X1V9]mado esta\[X6]rei, sempre \[X4]te ama\[X5]rei \[X5V4]
\[X6]Incompa\[X5]ráveis são \[X4]tuas pro\[X5]messas 
Pra \[X1V9]mim
\endchorus
\seq{Riff Intro}{X1 X6}{1x}
\act{Retomar}{Verso 1}{1x}
\act{Repetir:Ao final da penúltima linha}{Refrão}{2x}
\act{Executar:Ao final da penúltima linha}{Verso 3}{}
\beginverse
...\[X6]Incompa\[X5]ráveis são \[X4]Tuas pro\[X5]messas
\endverse
\act{Repetir}{Verso 3}{4x}
\beginverse
...Pra \[X1V9]mim
\endverse
%-----------------------------------------------------------------
\vspace{4em} % Regulador de Espaçamento
%-----------------------------------------------------------------
\begin{comment}
\lstset{basicstyle=\scriptsize\bf} % Parâmetros da TAB
%-----------------------------------------------------------------
\tab{Solo 1}
\begin{lstlisting}
E|-----------------------------------------------------|
B|-----------------------------------------------------|
G|-----------------------------------------------------|
D|-----------------------------------------------------|
A|-----------------------------------------------------|
E|-----------------------------------------------------|
\end{lstlisting}
%-----------------------------------------------------------------
\end{comment}
%=================================================================
 
%-----------------------------------------------------------------
\color{drawChord}\gtab{\color{nameChord} X1}{}%
\color{drawChord}\gtab{\color{nameChord} X1V8}{}%
\color{drawChord}\gtab{\color{nameChord} X1V9}{}% 
\color{drawChord}\gtab{\color{nameChord} X4}{}%
\color{drawChord}\gtab{\color{nameChord} X4V8}{}%
\color{drawChord}\gtab{\color{nameChord} X5}{}\\%
\color{drawChord}\gtab{\color{nameChord} X5V4}{}%
\color{drawChord}\gtab{\color{nameChord} X5V8}{}%
\color{drawChord}\gtab{\color{nameChord} X6}{}% 
\color{drawChord}\gtab{\color{nameChord} X7V3N15}{}%

%-----------------------------------------------------------------
% PADRÃO: [TonalidadeMaior+NOTAX+Variações] .Ex:[X50] [X57V1V7]
% OBS: Variações são alterações do acorde em relação ao campo harmônico.
%-----------------------------------------------------------------
% Tipos de Variações de Acordes:
% V0 - Variação Diversa
% V1 - Menor (m)
% V2 - Maior (M)
% V3 - Meio Tom Abaixo (Bemol)
% V4 - Com Quarta (ex:C4)
% V5 - Com Quinta (ex:C5)
% V6 - Com Sexta (ex:C6)
% V7 - Com Sétima Menor (ex:C7)
% V8 - Com baixo dois Tons Acima (ex:D/F#)
% V9 - Com Nona (ex:C9)
% V10 - Meio Tom Acima (Sustenido)
% V11 - Com Sétima Maior (ex:C7M)
% V12 - Suspenso (Sus)
% V13 - Com baixo dois Tons e Meio Acima (ex:A/E)
% V14 - Com baixo um Tom e Meio Acima (ex:D9/F) 
% V15 - Meio-Diminuto (m7b5)
% N15 - NÃO Meio-Diminuto
% V16 - Diminuto (º)
% N16 - NÃO Diminuto
%=================================================================
\endsong
%=================================================================
\begin{comment}

\end{comment}