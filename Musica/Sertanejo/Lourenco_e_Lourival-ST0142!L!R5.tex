%=================================================================
\songcolumns{2}
\beginsong
{Velha Porteira %TÍTULO
}[by={Lourenço e Lourival %ARTISTA
},album={@walyssondosreis},
id={\href{https://music.youtube.com/watch?v=USV5l8Kwokw&feature=share %LINK
}{ST0142 %COD.ID.: XXNNNN
}},rev={5}, %REVISÃO
qr={https://music.youtube.com/watch?v=USV5l8Kwokw&feature=share %LINK
}]
%-----------------------------------------------------------------
\tom{}{E}
%=================================================================
%\newchords{verse1.XX0000X} % Registrador de Acordes em Sequência
%\newchords{chorus1.XX0000X} % Registrador de Acordes em Sequência
%-----------------------------------------------------------------
\seq{Intro}{...}{1x}
\act{Solo}{Parte 1 | Parte 2}{1x}
%-----------------------------------------------------------------
%\beginverse \endverse
%\beginchorus \endchorus
\beginverse
Ao passar pela velha porteira
Senti minha terra mais perto de mim
De emoção eu estava chorando
Porque minha angústia chegava ao fim
\endverse
\beginverse
Eu confesso que era meu sonho
Rever a fazenda onde me criei
Não via chegar o momento 
De abraçar de novo
Meu querido povo que um dia eu deixei
\endverse
\act{Solo: Parte 1 | Parte 2}{Repetir}{1x}
\beginverse
Que surpresa cruel me aguardava
Ao ver a fazenda como transformou
Quase todos dali se mudaram
E a velha colônia deserta ficou
\endverse
\beginverse
Os amigos que ali permanecem
Transformaram tanto que nem conheci
E eles não me conheceram e nem perceberam
Que os anos passaram e eu envelheci
\endverse
\act{Solo: Parte 1 | Parte 2}{Repetir}{1x}
\beginverse
E você, minha velha porteira
Também não está como outrora deixei
Seus mourões pelo tempo roído
No solo caídos também encontrei
\endverse
\beginverse
Já não ouço as suas batidas
Seu triste rangido lembrança me traz
Porteira na realidade, você é a saudade
Do tempo da infância que não volta mais
\endverse
\act{Solo: Parte 2}{Repetir}{1x}

%-----------------------------------------------------------------
\vspace{4em} % Regulador de Espaçamento
%-----------------------------------------------------------------
\begin{comment}
\lstset{basicstyle=\scriptsize\bf} % Parâmetros da TAB
%-----------------------------------------------------------------
\tab{Solo 1}
\begin{lstlisting}
E|-----------------------------------------------------|
B|-----------------------------------------------------|
G|-----------------------------------------------------|
D|-----------------------------------------------------|
A|-----------------------------------------------------|
E|-----------------------------------------------------|
\end{lstlisting}
%-----------------------------------------------------------------
\end{comment}
%=================================================================
\begin{comment}

\color{drawChord}\gtab{\color{nameChord} X}{}% 
\color{drawChord}\gtab{\color{nameChord} X}{}% 
\color{drawChord}\gtab{\color{nameChord} X}{}% 
\color{drawChord}\gtab{\color{nameChord} X}{}% 

\end{comment}
%=================================================================
% PADRÃO: [TonalidadeMaior+NOTAX+Variações] .Ex:[X50] [X57V1V7]
% OBS: Variações são alterações do acorde em relação ao campo harmônico.
%-----------------------------------------------------------------
% Tipos de Variações de Acordes:
% V0 - Variação Diversa
% V1 - Menor (m)
% V2 - Maior (M)
% V3 - Meio Tom Abaixo (Bemol)
% V4 - Com Quarta (ex:C4)
% V5 - Com Quinta (ex:C5)
% V6 - Com Sexta (ex:C6)
% V7 - Com Sétima Menor (ex:C7)
% V8 - Com baixo dois Tons Acima (ex:D/F#)
% V9 - Com Nona (ex:C9)
% V10 - Meio Tom Acima (Sustenido)
% V11 - Com Sétima Maior (ex:C7M)
% V12 - Suspenso (Sus)
% V13 - Com baixo dois Tons e Meio Acima (ex:A/E)
% V14 - Com baixo um Tom e Meio Acima (ex:D9/F) 
% V15 - Meio-Diminuto (m7b5)
% N15 - NÃO Meio-Diminuto
% V16 - Diminuto (º)
% N16 - NÃO Diminuto
% V17 - Com baixo um Tom Acima (ex: C/D)
% V18 - Com baixo um Tom Abaixo (ex: Em/D)
% V19 - Com baixo dois Tons e meio Abaixo (ex: G/D)
%=================================================================
%\vspace{4em} % Regulador de Espaçamento
%=================================================================
\endsong
%=================================================================
