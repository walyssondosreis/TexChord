%=================================================================
\songcolumns{1}
\beginsong
{Seu Nome É Jesus %TÍTULO
}[by={Fernandinho %ARTISTA
},album={@walyssondosreis},
id={GB0109 %COD.ID.: GB0000
},rev={3}, %REVISÃO
qr={https://drive.google.com/open?id=1ZlaBnuzOvo9IhCCVWAPExOZC0uum6Jha %LINK
}]
%-----------------------------------------------------------------
\tom{X1}{G}
%=================================================================
%\newchords{verse0.GB0000} % Registrador de Acordes em Sequência
%-----------------------------------------------------------------
\seq{Intro}{X6 X5 X4}{}
%-----------------------------------------------------------------

\beginverse
\[X6]Os que andavam\[X4] desgarrados como o\[X6]velhas
Os que viviam\[X4] encarcerados em pri\[X6]sões
Os que choravam\[X4] e não tinham consolo\[X6]
Os cansados\[X4] viram a grande \[X6]luz \[X4]\[X1]
\[X5]Viram a grande \[X6]luz \[X4]\[X1]\[X5]
\endverse
\seq{Riff Intro}{X6 X5 X4}{}
\act{Repetir}{Verso 1}{1x}

\beginchorus
\[X6]Onde está oh morte, \[X4]a sua vitória?
\[X1]Meu senhor ressusci\[X5]tou
\[X6]Apanhou a chave da \[X4]morte e do inferno
\[X1]Meu senhor ressusci\[X5]tou
\endchorus
\act{Repetir}{Refrão}{+1x}
\beginverse
Seu nome é Je\[X2]sus\[X6]
Seu nome é Je\[X1]sus\[X5]
Seu nome é Je\[X2]sus\[X6]
Seu nome é Je\[X1]sus\[X5]
\endverse
\act{Repetir}{Verso 2}{+1x}
\act{Repetir}{Refrão}{1x}
\beginverse*\color{white}
.
\endverse
%-----------------------------------------------------------------
\begin{comment}
\lstset{basicstyle=\scriptsize\bf} % Parâmetros da TAB
%-----------------------------------------------------------------
\tab{Solo 1} % SOLO PARA O TOM DE SOL
\begin{lstlisting}
E|-------12-----------------12----------------------|
B|----12------10h12------12------10h12--12h13--12~--|
G|-12-----------------12----------------------------|
D|--------------------------------------------------|
A|--------------------------------------------------|
E|--------------------------------------------------|
\end{lstlisting}
%-----------------------------------------------------------------
\end{comment}
%=================================================================
\vspace{2em}
%-----------------------------------------------------------------
\color{drawChord}\gtab{\color{nameChord} X1}{}% G [X1]
\color{drawChord}\gtab{\color{nameChord} X2}{}% Am [X2]
\color{drawChord}\gtab{\color{nameChord} X4}{}% C [X4]
\color{drawChord}\gtab{\color{nameChord} X5}{}% D [X5]
\color{drawChord}\gtab{\color{nameChord} X6}{}% Em [X6]
%-----------------------------------------------------------------
% PADRÃO [TonalidadeMaiorNOTAX.Variação] .Ex:[X50] [X50V1]
% PADRÃO [TonalidadeMenorNOTAX.Variação] .Ex:[mX50] [mX50V1]
% OBS: Variações são alterações do acorde em relação ao campo harmônico.
%-----------------------------------------------------------------
% TIPOS DE VARIAÇÂO DOS ACORDES:
% V0 - ACORDE COM VARIAÇÃO DIVERSA
% V1 - ACORDE MENOR (m)
% V2 - ACORDE MAIOR (M)
% V3 - ACORDE MEIO TOM ABAIXO (Bemois)
% V4 - ACORDE COM QUARTA (C4)
% V5 - ACORDE COM QUINTA (C5)
% V6 - ACORDE COM SEXTA (C6)
% V7 - ACORDE COM SÉTIMA MENOR (C7)
% V8 - ACORDE COM BAIXO DOIS TONS ACIMA (D/F#)
% V9 - ACORDE COM NONA (C9)
% V10 - ACORDE MEIO TOM ACIMA (Sustenidos)
% V11 - ACORDE COM SÉTIMA MAIOR (C7M)
% V12 - ACORDE SUSPENSO (Sus)
% V13 - ACORDE COM BAIXO DOIS TONS E MEIO ACIMA (A/E)
% V14 - ACORDE UM TOM E MEIO ACIMA (D9/F)
%=================================================================
\endsong
%=================================================================