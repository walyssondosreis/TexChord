%=================================================================
\songcolumns{2}
\beginsong
{Em espírito, em verdade\\Meu prazer %TÍTULO
}[by={Ministério Koinonya %ARTISTA
},album={@walyssondosreis},
id={GB0032 %COD.ID.: GB0000
},rev={3}, %REVISÃO
qr={https://drive.google.com/open?id=1Zow-pYsWJrr4sj6kBDEsyMh52slxbxx6 %LINK
}]
%-----------------------------------------------------------------
\tom{G}{A}
%=================================================================
%\newchords{verse0.GB0000} % Registrador de Acordes em Sequência
%-----------------------------------------------------------------
\seq{Intro}{}{2x}
%-----------------------------------------------------------------
%\beginverse* \endverse
%\beginchorus \endchorus

\beginverse
\[G]Em espírito, em ver\[D]dade
Te ado\[C]ramos,  te ado\[G]ramos \[D]
\[G]Em espírito, em ver\[D]dade
Te ado\[C]ramos,  te ado\[G]ramos \[D]
\endverse

\beginverse
Rei dos \[Em]reis  e se\[G]nhor 
Te entre\[C]gamos \[D]nosso vi\[G]ver \[D]
Rei dos \[Em]reis  e se\[G]nhor 
Te entre\[C]gamos \[Am]nosso vi\[D]ver \[D4]
\endverse

\beginchorus 
Pra te ado\[C]rar oh! \[Am]rei dos \[D]reis \[D4]
Foi que eu nas\[C]ci oh! \[Am]rei Je\[D]sus!
Meu pra\[B7]zer é te lou\[Em]var
Meu pra\[D]zer é estar\[C]
Nos \[D]átrios do se\[G]nhor
Meu pra\[D]zer é vi\[Em]ver
Na \[D]casa de \[C]Deus
Onde \[D]flui o a\[G]mor \[(D)]
\endchorus
\act{Executar}{Solo 1}{}
\act{Retomar}{Verso 1}{1x}
\act{Repetir}{Refrão}{1x}
\beginverse
... Onde \[D]flui o a\[G]mor \[(D)]

\endverse
\act{Repetir}{Verso 3}{+4x}

%-----------------------------------------------------------------
\begin{comment}
\lstset{basicstyle=\scriptsize\bf} % Parâmetros da TAB
%-----------------------------------------------------------------
\tab{Solo 1}
\begin{lstlisting}
E|-----------------------------------------------------|
B|-----------------------------------------------------|
D|-----------------------------------------------------|
D|-----------------------------------------------------|
A|-----------------------------------------------------|
E|-----------------------------------------------------|
\end{lstlisting}
%-----------------------------------------------------------------
\end{comment}
%=================================================================

%-----------------------------------------------------------------
\color{drawChord}\gtab{\color{nameChord} G}{}% C [G]
\color{drawChord}\gtab{\color{nameChord} Am}{}% Dm [Am]
\color{drawChord}\gtab{\color{nameChord} B7}{}% E7 [B7]
\color{drawChord}\gtab{\color{nameChord} C}{}% F [C]
\color{drawChord}\gtab{\color{nameChord} D}{}% G [D]
\color{drawChord}\gtab{\color{nameChord} D4}{}\\% G4 [D4]
\color{drawChord}\gtab{\color{nameChord} Em}{}% Am [Em]
%-----------------------------------------------------------------
% PADRÃO [TonalidadeMaiorNOTAX.Variação] .Ex:[X50] [X50V1]
% PADRÃO [TonalidadeMenorNOTAX.Variação] .Ex:[mX50] [mX50V1]
% OBS: Variações são alterações do acorde em relação ao campo harmônico.
%-----------------------------------------------------------------
% TIPOS DE VARIAÇÂO DOS ACORDES:
% V0 - ACORDE COM VARIAÇÃO DIVERSA
% V1 - ACORDE MENOR (m)
% V2 - ACORDE MAIOR (M)
% V3 - ACORDE MEIO TOM ABAIXO (Bemois)
% V4 - ACORDE COM QUARTA (C4)
% V5 - ACORDE COM QUINTA (C5)
% V6 - ACORDE COM SEXTA (C6)
% V7 - ACORDE COM SÉTIMA MENOR (C7)
% V8 - ACORDE COM BAIXO DOIS TONS ACIMA (D/F#)
% V9 - ACORDE COM NONA (C9)
% V10 - ACORDE MEIO TOM ACIMA (Sustenidos)
% V11 - ACORDE COM SÉTIMA MAIOR (C7M)
% V12 - ACORDE SUSPENSO (Sus)
% V13 - ACORDE COM BAIXO DOIS TONS E MEIO ACIMA (A/E)
% V14 - ACORDE UM TOM E MEIO ACIMA (D9/F)
%=================================================================
\endsong
%=================================================================

\begin{comment}

\end{comment}