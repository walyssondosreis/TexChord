%=================================================================
\songcolumns{2}
\beginsong
{Pra Onde Iremos? %TÍTULO
}[by={Gabriela Rocha %ARTISTA
},album={@walyssondosreis},
id={GB0157 %COD.ID.: GB0000
},rev={3}, %REVISÃO
qr={https://drive.google.com/open?id=1N4Dqvlol8t-fbttsd_u8tc1T9_ptPV5w %LINK
}]
%-----------------------------------------------------------------
\tom{X1}{F\#}
%=================================================================
%\newchords{verse1.GB0000X} % Registrador de Acordes em Sequência
%\newchords{chorus1.GB0000X} % Registrador de Acordes em Sequência
%-----------------------------------------------------------------
\seq{Intro}{X1 X6V7 X5 X4V9}{}
%-----------------------------------------------------------------
%\beginverse* \endverse
%\beginchorus \endchorus

\beginverse
Pra \[X1]onde iremos \[X6V7]nós?
Só \[X3V7]tu tens a \[X4V9]vida eterna
Pra \[X1]onde iremos \[X6V7]nós?
Só \[X3V7]tu tens a \[X4V9]vida eterna
\endverse

\beginverse
\[X2V7]Tu és o \[X1]Pão que des\[X5V8]ceu do céu
\[X2V7]Fonte de \[X1V8]vida, \[X5]Emanuel
\endverse

\beginchorus
Com \[X1]tua glória \[X3V7]sobre mim
Enche-me \[X6V7]de ti até trans\[X4V9]bordar
Eu \[X1V8]nunca vou me \[X3V7]saciar
Enche-me de\[X6V7] ti até trans\[X4V9]bordar
\endchorus

\act{Retomar}{Verso 2}{1x}

\beginverse
Pra \[X1]on...\[X3V7]de iremos \[X6V7]nós?
Só \[X4V9]tu tens as palavras
De \[X1V8]vi...\[X5V8]da eter\[X6V7]na \[X4V9]Ohhohoh
\endverse

\beginverse
\[X1] Tu és o pão do céu
\[X5] O Deus Emanuel
\[X6V7]Enche-me, enche-me 
\[X4V9]Até transbordar
\[X1V8] Tu és o pão do céu
\[X5V8] O Deus Emanuel
\[X6V7]Enche-me, enche-me 
\[X4V9]Até transbordar
\endverse
\act{Repetir}{Verso 4}{+1x}
\act{Repetir}{Refrão}{1x}
\act{Repetir}{Verso 3}{1x}
%-----------------------------------------------------------------
\begin{comment}
\lstset{basicstyle=\scriptsize\bf} % Parâmetros da TAB
%-----------------------------------------------------------------
\tab{Solo 1}
\begin{lstlisting}
E|-----------------------------------------------------|
B|-----------------------------------------------------|
G|-----------------------------------------------------|
D|-----------------------------------------------------|
A|-----------------------------------------------------|
E|-----------------------------------------------------|
\end{lstlisting}
%-----------------------------------------------------------------
\end{comment}
%=================================================================
\vspace{2em} 
%-----------------------------------------------------------------
\color{drawChord}\gtab{\color{nameChord} X1}{}% E
\color{drawChord}\gtab{\color{nameChord} X1V8}{}% E/G#
\color{drawChord}\gtab{\color{nameChord} X2V7}{}% F#m7
\color{drawChord}\gtab{\color{nameChord} X3V7}{}% G#m7
\color{drawChord}\gtab{\color{nameChord} X4V9}{}% A9 
\color{drawChord}\gtab{\color{nameChord} X5}{}\\%  B
\color{drawChord}\gtab{\color{nameChord} X5V8}{}% B/D#
\color{drawChord}\gtab{\color{nameChord} X6V7}{}% C#m7
%-----------------------------------------------------------------
% PADRÃO: [TonalidadeMaior+NOTAX+Variações] .Ex:[X50] [X57V1V7]
% OBS: Variações são alterações do acorde em relação ao campo harmônico.
%-----------------------------------------------------------------
% Tipos de Variações de Acordes:
% V0 - Variação Diversa
% V1 - Menor (m)
% V2 - Maior (M)
% V3 - Meio Tom Abaixo (Bemol)
% V4 - Com Quarta (ex:C4)
% V5 - Com Quinta (ex:C5)
% V6 - Com Sexta (ex:C6)
% V7 - Com Sétima Menor (ex:C7)
% V8 - Com baixo dois Tons Acima (ex:D/F#)
% V9 - Com Nona (ex:C9)
% V10 - Meio Tom Acima (Sustenido)
% V11 - Com Sétima Maior (ex:C7M)
% V12 - Suspenso (Sus)
% V13 - Com baixo dois Tons e Meio Acima (ex:A/E)
% V14 - Com baixo um Tom e Meio Acima (ex:D9/F) 
% V15 - Meio-Diminuto (m7b5)
% N15 - NÃO Meio-Diminuto
% V16 - Diminuto (º)
% N16 - NÃO Diminuto
%=================================================================
\endsong
%=================================================================
\begin{comment}

\end{comment}