%=================================================================
\songcolumns{2}
\beginsong
{Chuva de Avivamento %TÍTULO
}[by={Alda Célia %ARTISTA
},album={@walyssondosreis},
id={GB0019 %COD.ID.: GB0000
},rev={3}, %REVISÃO
qr={https://drive.google.com/open?id=1zWPc_MckF2FMwh3oMt7X4vOXTugCutR1 %LINK
}]
%-----------------------------------------------------------------
\tom{E}{E}
%=================================================================
\newchords{verse1.GB0019X} % Registrador de Acordes em Sequência
%\newchords{chorus1.GB0000X} % Registrador de Acordes em Sequência
%-----------------------------------------------------------------
\seq{Intro}{E E4}{}
%-----------------------------------------------------------------
%\beginverse* \endverse
%\beginchorus \endchorus
\beginverse\memorize[verse1.GB0019X]
Nos \[E]últimos dias \[E4]diz o Senhor
\[B]Derramarei Meu Es\[E]pírito sobre toda a \[A]terra
Copiosa\[E]mente \[E4]
\[E]Poços secos \[E4]jorrarão
E \[B]até o deserto de \[E]novo frutifica\[A]rá
Abundante\[E]mente
\endverse
\beginverse
É o \[C\#m]som do nosso lou\[B]vor
Que sobe aos \[C\#m]céus como um va\[A]por
E se con\[C\#m]densa na nuvem de \[B]glória
Sheki\[D]nah sobre \[C]nós se derrama\[D]rá \[C/D]
\endverse
\beginchorus
Em abundante \[G]chuva, \[D/F\#]chuva
De\[Am]rrama sobre nós esta \[Em]chuva
\[F]Abre as comportas dos \[Am]céus, Senhor
\[Bm]Faz \[Em]cho\[D]ver
Abundante \[G]chuva, \[D/F\#]chuva
De\[Am]rrama sobre nós esta \[Em]chuva
To\[F]rrente de águas \[C]sobre o sedento
\[Am]Chuva de a\[D]viva\[E]mento, \[C]chuva de a\[D]viva\[E]mento
\endchorus
\beginverse\replay[verse1.GB0019X]
E ^esta chuva con^verterá
O ^coração dos pais aos ^filhos e dos filhos aos ^pais
No poder do Es^pírito
\endverse
\act{Retormar}{Verso 2}{1x}
\beginverse
\[Em7]Chuva de aviva\[C]mento (derrama)
\[Em7]Chuva de aviva\[C]mento (derrama)
\[X]Chuva de aviva\[C]mento \[D]vem sobre \[Em7]nós \[( C D )]
\endverse
\act{Repetir}{Verso 4}{+1x}
\act{Repetir}{Refrão}{1x}
\beginverse
... \[C]Chuva de a\[D]viva\[E]mento
\endverse


%-----------------------------------------------------------------
\begin{comment}
\lstset{basicstyle=\scriptsize\bf} % Parâmetros da TAB
%-----------------------------------------------------------------
\tab{Solo 1}
\begin{lstlisting}
E|-----------------------------------------------------|
B|-----------------------------------------------------|
G|-----------------------------------------------------|
D|-----------------------------------------------------|
A|-----------------------------------------------------|
E|-----------------------------------------------------|
\end{lstlisting}
%-----------------------------------------------------------------
\end{comment}
%=================================================================
 
%-----------------------------------------------------------------
\color{drawChord}\gtab{\color{nameChord} E}{~:022100}% E
\color{drawChord}\gtab{\color{nameChord} E4}{~:022200}% E4
\color{drawChord}\gtab{\color{nameChord} Em}{~:022000}% Em
\color{drawChord}\gtab{\color{nameChord} Em7}{~:022030}% Em7
\color{drawChord}\gtab{\color{nameChord} F}{1:022100}% F
\color{drawChord}\gtab{\color{nameChord} G}{~:320003}\\% G
\color{drawChord}\gtab{\color{nameChord} A}{~:X02220}% A
\color{drawChord}\gtab{\color{nameChord} Am}{~:X02210}% Am
\color{drawChord}\gtab{\color{nameChord} B}{2:X02220}% B
\color{drawChord}\gtab{\color{nameChord} Bm}{2:X02210}% Bm
\color{drawChord}\gtab{\color{nameChord} C}{~:X32010}% C
\color{drawChord}\gtab{\color{nameChord} C/D}{}\\% C/D
\color{drawChord}\gtab{\color{nameChord} C\#m}{4:X02210}% C#m
\color{drawChord}\gtab{\color{nameChord} D}{~:XX0232}% D
\color{drawChord}\gtab{\color{nameChord} D/F\#}{~:200232}% D/F#

%-----------------------------------------------------------------
% PADRÃO: [TonalidadeMaior+NOTAX+Variações] .Ex:[X50] [X57V1V7]
% OBS: Variações são alterações do acorde em relação ao campo harmônico.
%-----------------------------------------------------------------
% Tipos de Variações de Acordes:
% V0 - Variação Diversa
% V1 - Menor (m)
% V2 - Maior (M)
% V3 - Meio Tom Abaixo (Bemol)
% V4 - Com Quarta (ex:C4)
% V5 - Com Quinta (ex:C5)
% V6 - Com Sexta (ex:C6)
% V7 - Com Sétima Menor (ex:C7)
% V8 - Com baixo dois Tons Acima (ex:D/F#)
% V9 - Com Nona (ex:C9)
% V10 - Meio Tom Acima (Sustenido)
% V11 - Com Sétima Maior (ex:C7M)
% V12 - Suspenso (Sus)
% V13 - Com baixo dois Tons e Meio Acima (ex:A/E)
% V14 - Com baixo um Tom e Meio Acima (ex:D9/F) 
% V15 - Meio-Diminuto (m7b5)
% N15 - NÃO Meio-Diminuto
% V16 - Diminuto (º)
% N16 - NÃO Diminuto
% V17 - Com baixo um Tom Acima (ex: C/D)
%=================================================================
\endsong
%=================================================================
\begin{comment}

\end{comment}