%=================================================================
\songcolumns{2}
\beginsong
{Diante de Ti %TÍTULO
}[by={Quatro Por Um %ARTISTA
},album={@walyssondosreis},
id={GB0028 %COD.ID.: GB0000
},rev={3}, %REVISÃO
qr={https://drive.google.com/open?id=1Dqilwhm-t7JWiCaFZ7tHiNiSbuhcNz-y %LINK
}]
%-----------------------------------------------------------------
\tom{X1}{E}
%=================================================================
%\newchords{verse1.GB0000X} % Registrador de Acordes em Sequência
%\newchords{chorus1.GB0000X} % Registrador de Acordes em Sequência
%-----------------------------------------------------------------
\seq{Intro}{X1 X1V4}{2x}
%-----------------------------------------------------------------
%\beginverse* \endverse
%\beginchorus \endchorus
\beginverse
\[X1]Vem, Se\[X1V4]nhor, encher este lu\[X1]gar \[X1V4]
\[X1]Vem, Se\[X1V4]nhor, encher este lu\[X1]gar \[X1V4]
Com tua \[X5]gló...ria, com tua \[X4]gló...\[X1]ria
Com tua \[X5]gló...ria, \[X4]com tua \[X1]glória \[X1]\[X1V4]
\endverse
\beginverse
^Fala-^me, eu quero te ^ouvir ^
^Toca-^me, eu quero te ^sentir ^
Vem e a^braça-me, vem e a^braça-^me
Vem e a^braça-me, ^vem e a^braça-me \[(X1)]
\endverse
\beginverse
\[X5]Todo dia é \[X6]dia de adorar ao \[X4]Se\[X1]nhor
\[X5]Eu conto os se\[X6]gundos só pra te en\[X4]con\[X1]trar
Quando es\[X2]tou em \[X3]tua pre\[X4]sença \[(X4)]
\endverse
\beginchorus
Dá vontade de pu\[X1]la...\[X5]ar, dá vontade de dan\[X6]ça...\[X4]ar
Dá vontade de gri\[X1]ta...\[X5]ar, dá vontade de co\[X6]rre...\[X4]er
Diante de \[X1]ti
Dá vontade de pu\[X5]lar, dá vontade de dan\[X6]çar
Dá vontade de gri\[X4]tar, dá vontade de co\[X1]rrer
Dá vontade de pu\[X5]lar, dá vontade de dan\[X6]çar \[X4]
(Hey yeah) 
\endchorus
\act{Executar}{Solo 1}{}
\act{Retomar}{Verso 2}{1x}
\beginverse
Na tua pre\[X1]sença, Senhor
Dá von\[X5]tade de correr, de sal\[X6]tar de alegria
De \[X4]te conhecer, Senhor
\[X1] De erguer minhas \[X5]mãos
E te ado\[X6]rar, e te ado\[X4]rar, Senhor
Da von\[X1]tade
Dá vontade de pu\[X5]lar, dá vontade de dan\[X6]çar
Dá vontade de gri\[X4]tar, dá vontade de co\[X1]rrer
Dá vontade de pu\[X5]lar, dá vontade de dan\[X6]çar \[X4]
Diante de \[X1]ti
\endverse
\act{Executar}{Solo 2}{}
%-----------------------------------------------------------------
\vspace{4em} % Regulador de Espaçamento
%-----------------------------------------------------------------
\begin{comment}
\lstset{basicstyle=\scriptsize\bf} % Parâmetros da TAB
%-----------------------------------------------------------------
\tab{Solo 1}
\begin{lstlisting}
E|-----------------------------------------------------|
B|-----------------------------------------------------|
G|-----------------------------------------------------|
D|-----------------------------------------------------|
A|-----------------------------------------------------|
E|-----------------------------------------------------|
\end{lstlisting}
%-----------------------------------------------------------------
\end{comment}
%=================================================================
 
%-----------------------------------------------------------------
\color{drawChord}\gtab{\color{nameChord} X1}{}% E  
\color{drawChord}\gtab{\color{nameChord} X1V4}{}% E4
\color{drawChord}\gtab{\color{nameChord} X2}{}% F#m
\color{drawChord}\gtab{\color{nameChord} X3}{}% G#m
\color{drawChord}\gtab{\color{nameChord} X4}{}% A
\color{drawChord}\gtab{\color{nameChord} X5}{}\\% B  
\color{drawChord}\gtab{\color{nameChord} X6}{}% C#m
%-----------------------------------------------------------------
% PADRÃO: [TonalidadeMaior+NOTAX+Variações] .Ex:[X50] [X57V1V7]
% OBS: Variações são alterações do acorde em relação ao campo harmônico.
%-----------------------------------------------------------------
% Tipos de Variações de Acordes:
% V0 - Variação Diversa
% V1 - Menor (m)
% V2 - Maior (M)
% V3 - Meio Tom Abaixo (Bemol)
% V4 - Com Quarta (ex:C4)
% V5 - Com Quinta (ex:C5)
% V6 - Com Sexta (ex:C6)
% V7 - Com Sétima Menor (ex:C7)
% V8 - Com baixo dois Tons Acima (ex:D/F#)
% V9 - Com Nona (ex:C9)
% V10 - Meio Tom Acima (Sustenido)
% V11 - Com Sétima Maior (ex:C7M)
% V12 - Suspenso (Sus)
% V13 - Com baixo dois Tons e Meio Acima (ex:A/E)
% V14 - Com baixo um Tom e Meio Acima (ex:D9/F) 
% V15 - Meio-Diminuto (m7b5)
% N15 - NÃO Meio-Diminuto
% V16 - Diminuto (º)
% N16 - NÃO Diminuto
%=================================================================
\endsong
%=================================================================
\begin{comment}

\end{comment}