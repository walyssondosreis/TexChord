%=================================================================
\songcolumns{2}
\beginsong
{Chuva de Avivamento %TÍTULO
}[by={Alda Célia %ARTISTA
},album={@walyssondosreis},
id={GB0019 %COD.ID.: GB0000
(Rev.1) %REVISÃO.: 0...N
}]
%-----------------------------------------------------------------
\tom{X1}
%=================================================================
%\newchords{verse1.GB0000X} % Registrador de Acordes em Sequência
%\newchords{chorus1.GB0000X} % Registrador de Acordes em Sequência
%-----------------------------------------------------------------
\seq{Intro}{}{}
%-----------------------------------------------------------------
%\beginverse* \endverse
%\beginchorus \endchorus
\beginverse
Nos últimos dias diz o Senhor
Derramarei Meu Espírito sobre toda a terra
Copiosamente
Poços secos jorrarão
E até o deserto de novo frutificará
Abundantemente
\endverse
\beginverse
É o som do nosso louvor
Que sobe aos céus como um vapor
E se condensa na nuvem de glória
Shekinah sobre nós se derramará
\endverse
\beginchorus
Em abundante chuva, chuva
Derrama sobre nós esta chuva
Abre as comportas dos céus, Senhor
Faz chover
Abundante chuva, chuva
Derrama sobre nós esta chuva
Torrente de águas sobre o sedento
Chuva de avivamento, chuva de avivamento
\endchorus
\beginverse
E esta chuva converterá
O coração dos pais aos filhos e dos filhos aos pais
No poder do Espírito
\endverse
\act{Retormar}{Verso 2}{1x}
\beginverse*
Chuva de avivamento
Chuva de avivamento
Chuva de avivamento vem sobre nós
Chuva de avivamento (derrama)
Chuva de avivamento (derrama)
Chuva de avivamento vem sobre nós
\endverse
\act{Repetir}{Refrão}{1x}
\beginverse
... Chuva de avivamento
\endverse


%-----------------------------------------------------------------
\begin{comment}
\lstset{basicstyle=\scriptsize\bf} % Parâmetros da TAB
%-----------------------------------------------------------------
\tab{Solo 1}
\begin{lstlisting}
E|-----------------------------------------------------|
B|-----------------------------------------------------|
G|-----------------------------------------------------|
D|-----------------------------------------------------|
A|-----------------------------------------------------|
E|-----------------------------------------------------|
\end{lstlisting}
%-----------------------------------------------------------------
\end{comment}
%=================================================================
\vspace{2em} 
%-----------------------------------------------------------------
\gtab{\color{black} X}{}% 
\gtab{\color{black} X}{}% 
\gtab{\color{black} X}{}% 
\gtab{\color{black} X}{}% 
%-----------------------------------------------------------------
% PADRÃO: [TonalidadeMaior+NOTAX+Variações] .Ex:[X50] [X57V1V7]
% OBS: Variações são alterações do acorde em relação ao campo harmônico.
%-----------------------------------------------------------------
% Tipos de Variações de Acordes:
% V0 - Variação Diversa
% V1 - Menor (m)
% V2 - Maior (M)
% V3 - Meio Tom Abaixo (Bemol)
% V4 - Com Quarta (ex:C4)
% V5 - Com Quinta (ex:C5)
% V6 - Com Sexta (ex:C6)
% V7 - Com Sétima Menor (ex:C7)
% V8 - Com baixo dois Tons Acima (ex:D/F#)
% V9 - Com Nona (ex:C9)
% V10 - Meio Tom Acima (Sustenido)
% V11 - Com Sétima Maior (ex:C7M)
% V12 - Suspenso (Sus)
% V13 - Com baixo dois Tons e Meio Acima (ex:A/E)
% V14 - Com baixo um Tom e Meio Acima (ex:D9/F) 
% V15 - Meio-Diminuto (m7b5)
% N15 - NÃO Meio-Diminuto
% V16 - Diminuto (º)
% N16 - NÃO Diminuto
%=================================================================
\endsong
%=================================================================
\begin{comment}

\end{comment}