%=================================================================
\songcolumns{1}
\beginsong
{Quebrantado %TÍTULO
}[by={Vineyard Brasil %ARTISTA
},album={@walyssondosreis},
id={GB0095 %COD.ID.: GB0000
},rev={0}, %REVISÃO
qr={https://drive.google.com/open?id=19e8ijGdG-CEKY7T2ENbSYJ7oj6FzNPtp %LINK
}]
%-----------------------------------------------------------------
\tom{X1}{C}
%=================================================================
%\newchords{verse1.GB0000X} % Registrador de Acordes em Sequência
%\newchords{chorus1.GB0000X} % Registrador de Acordes em Sequência
%-----------------------------------------------------------------
\seq{Intro}{X1 X5 X6 X4}{} 
%-----------------------------------------------------------------
%\beginverse* \endverse
%\beginchorus \endchorus
\beginverse*
Eu olho para a \[X1]cruz
E para a cruz eu \[X5]vou
Do seu sofrer partici\[X6]par
Da Tua obra vou can\[X4]tar
Meu salva\[X1]dor
Na cruz mos\[X5]trou
O amor do \[X6]pai
Do justo \[X4]Deus
\endverse
\beginchorus
\[X1] Pela cruz me chamou \[X6]
Gentilmente me atraiu
E \[X5]eu, sem palavras, me aproximo \[X4]
Quebrantado por seu amor \[(X1)]
\endchorus
\beginverse*
Imerecida ^vida, de graça rece^bi
Por sua ^cruz da morte me li^vrou
Trouxe-me a ^vida, eu estava conde^nado
Mas agora pela ^cruz eu fui reconcili^ado
\endverse
\beginverse*
\[X2] Impressionante é o seu a\[X4]mor
\[X2] Me redimiu e me mos\[X4]trou o quanto é fi\[X5]el
\endverse

%-----------------------------------------------------------------
\begin{comment}
\lstset{basicstyle=\scriptsize\bf} % Parâmetros da TAB
%-----------------------------------------------------------------
\tab{Solo 1}
\begin{lstlisting}
E|-----------------------------------------------------|
B|-----------------------------------------------------|
G|-----------------------------------------------------|
D|-----------------------------------------------------|
A|-----------------------------------------------------|
E|-----------------------------------------------------|
\end{lstlisting}
%-----------------------------------------------------------------
\end{comment}
%=================================================================
 
%-----------------------------------------------------------------
\color{drawChord}\gtab{\color{nameChord} X1}{}% 
\color{drawChord}\gtab{\color{nameChord} X2}{}% 
\color{drawChord}\gtab{\color{nameChord} X4}{}% 
\color{drawChord}\gtab{\color{nameChord} X5}{}%
\color{drawChord}\gtab{\color{nameChord} X6}{}%
%-----------------------------------------------------------------
% PADRÃO: [TonalidadeMaior+NOTAX+Variações] .Ex:[X50] [X57V1V7]
% OBS: Variações são alterações do acorde em relação ao campo harmônico.
%-----------------------------------------------------------------
% Tipos de Variações de Acordes:
% V0 - Variação Diversa
% V1 - Menor (m)
% V2 - Maior (M)
% V3 - Meio Tom Abaixo (Bemol)
% V4 - Com Quarta (ex:C4)
% V5 - Com Quinta (ex:C5)
% V6 - Com Sexta (ex:C6)
% V7 - Com Sétima Menor (ex:C7)
% V8 - Com baixo dois Tons Acima (ex:D/F#)
% V9 - Com Nona (ex:C9)
% V10 - Meio Tom Acima (Sustenido)
% V11 - Com Sétima Maior (ex:C7M)
% V12 - Suspenso (Sus)
% V13 - Com baixo dois Tons e Meio Acima (ex:A/E)
% V14 - Com baixo um Tom e Meio Acima (ex:D9/F) 
% V15 - Meio-Diminuto (m7b5)
% N15 - NÃO Meio-Diminuto
% V16 - Diminuto (º)
% N16 - NÃO Diminuto
%=================================================================
\endsong
%=================================================================
\begin{comment}

\end{comment}