%=================================================================
\songcolumns{2}
\beginsong
{Temos Que Ser Um %TÍTULO
}[by={Fernandinho %ARTISTA
},album={@walyssondosreis},
id={GB0115 %COD.ID.: GB0000
},rev={3}, %REVISÃO
qr={https://drive.google.com/open?id=1TAHLV9x4tIM6ZB-rgx-tYalaFTCy3D5N %LINK
}]
%-----------------------------------------------------------------
\tom{X1}{C}
%=================================================================
%\newchords{verse1.GB0000X} % Registrador de Acordes em Sequência
%\newchords{chorus1.GB0000X} % Registrador de Acordes em Sequência
%-----------------------------------------------------------------
\seq{Intro}{X6 X4 X5 X2 | X6 X4 X5}{}
%-----------------------------------------------------------------
%\beginverse \endverse
%\beginchorus \endchorus
\beginverse
A\[X6]mai-vos \[X4]uns aos \[X5]outros \[X2]
Sujei\[X6]tai-vos \[X4]uns aos \[X5]outros
Aquele que qui\[X6]ser ser \[X4]o pri\[X5]meiro
Sirva a \[X6]todo\[X4 X5]s
\endverse
\beginverse
Temos um lu^gar no ^cora^ção de Deus ^
Importa que Ele ^cresça e ^eu dimi^nua
Eu quero ser^vir aos ^meus ir^mãos
Assim como J^esu^s
\endverse
\act{Retomar}{Verso 1}{1x}
\beginchorus
Temos que ser \[X1]um \[X5]
Como o Pai em \[X6]Cristo \[X4]é
Temos que ser \[X1]um \[X5]
Pra que o mundo creia que o Pai o \[X6]envi\[X4]ou
\endchorus
\act{Repetir}{Refrão}{+1x}
\beginverse
Temos que ser \[X1 X5]umm \[X6 X4]hoooooo
\endverse
\act{Repetir}{Verso 3}{+3x}
\act{Repetir}{Refrão}{4x}
\act{Repetir}{Verso 3}{4x}
\act{Repetir}{Refrão}{2x}
\act{Repetir}{Verso 3}{4x}
\beginverse
...Temos que ser \[X1]um
\endverse

%-----------------------------------------------------------------
\begin{comment}
\lstset{basicstyle=\scriptsize\bf} % Parâmetros da TAB
%-----------------------------------------------------------------
\tab{Solo 1}
\begin{lstlisting}
E|-----------------------------------------------------|
B|-----------------------------------------------------|
G|-----------------------------------------------------|
D|-----------------------------------------------------|
A|-----------------------------------------------------|
E|-----------------------------------------------------|
\end{lstlisting}
%-----------------------------------------------------------------
\end{comment}
%=================================================================
 
%-----------------------------------------------------------------
\color{drawChord}\gtab{\color{nameChord} X1}{}% C [X1] 
\color{drawChord}\gtab{\color{nameChord} X2}{}% Dm  [X2]
\color{drawChord}\gtab{\color{nameChord} X4}{}% F  [X4]
\color{drawChord}\gtab{\color{nameChord} X5}{}% G  [X5]
\color{drawChord}\gtab{\color{nameChord} X6}{}% Am  [X6]
%-----------------------------------------------------------------
% PADRÃO [TonalidadeMaiorNOTAX.Variação] .Ex:[X50] [X50V1]
% PADRÃO [TonalidadeMenorNOTAX.Variação] .Ex:[mX50] [mX50V1]
% OBS: Variações são alterações do acorde em relação ao campo harmônico.
%-----------------------------------------------------------------
% TIPOS DE VARIAÇÂO DOS ACORDES:
% V0 - ACORDE COM VARIAÇÃO DIVERSA
% V1 - ACORDE MENOR (m)
% V2 - ACORDE MAIOR (M)
% V3 - ACORDE MEIO TOM ABAIXO (Bemois)
% V4 - ACORDE COM QUARTA (C4)
% V5 - ACORDE COM QUINTA (C5)
% V6 - ACORDE COM SEXTA (C6)
% V7 - ACORDE COM SÉTIMA MENOR (C7)
% V8 - ACORDE COM BAIXO DOIS TONS ACIMA (D/F#)
% V9 - ACORDE COM NONA (C9)
% V10 - ACORDE MEIO TOM ACIMA (Sustenidos)
% V11 - ACORDE COM SÉTIMA MAIOR (C7M)
% V12 - ACORDE SUSPENSO (Sus)
% V13 - ACORDE COM BAIXO DOIS TONS E MEIO ACIMA (A/E)
% V14 - ACORDE UM TOM E MEIO ACIMA (D9/F)
%=================================================================
\endsong
%=================================================================
\begin{comment}

\end{comment}