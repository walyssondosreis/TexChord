%=================================================================
\songcolumns{1}
\beginsong
{Temos Que Ser Um %TÍTULO
}[by={Fernandinho %ARTISTA
},album={@walyssondosreis},
id={GB0115 %COD.ID.: GB0000
}] 
%-----------------------------------------------------------------
\tom{C}
%=================================================================
%\newchords{verse1.GB0000X} % Registrador de Acordes em Sequência
%\newchords{chorus1.GB0000X} % Registrador de Acordes em Sequência
%-----------------------------------------------------------------
\seq{Intro}{Am F G Dm | Am F G}{2x}
%-----------------------------------------------------------------
%\beginverse* \endverse
%\beginchorus \endchorus
\beginverse*
A\[Am]mai-vos \[F]uns aos \[G]outros \[Dm]
Sujei\[Am]tai-vos \[F]uns aos \[G]outros
Aquele que qui\[Am]ser ser \[F]o pri\[G]meiro
Sirva a \[Am]todo\[F G]s
\endverse
\beginverse*
Temos um lu^gar no ^cora^ção de Deus ^
Importa que Ele ^cresça e ^eu dimi^nua
Eu quero ser^vir aos ^meus ir^mãos
Assim como J^esu^s
\endverse
\beginchorus
Temos que ser \[C]um \[G]
Como o Pai em \[Am]Cristo \[F]é
Temos que ser \[C]um \[G]
Pra que o mundo creia que o Pai o \[Am]envi\[F]ou
\{ Temos que ser \[C G]umm \[Am F]hoooooo \}
\endchorus

%-----------------------------------------------------------------
\begin{comment}
\lstset{basicstyle=\scriptsize\bf} % Parâmetros da TAB
%-----------------------------------------------------------------
\tab{Solo 1}
\begin{lstlisting}
E|-----------------------------------------------------|
B|-----------------------------------------------------|
G|-----------------------------------------------------|
D|-----------------------------------------------------|
A|-----------------------------------------------------|
E|-----------------------------------------------------|
\end{lstlisting}
%-----------------------------------------------------------------
\end{comment}
%=================================================================
\vspace{2em} 
%-----------------------------------------------------------------
\gtab{\color{black} C}{}% C [C] 
\gtab{\color{black} Dm}{}% Dm  [Dm]
\gtab{\color{black} F}{}% F  [F]
\gtab{\color{black} G}{}% G  [G]
\gtab{\color{black} Am}{}% Am  [Am]
%-----------------------------------------------------------------
% PADRÃO [TonalidadeMaiorNOTAX.Variação] .Ex:[X50] [X50V1]
% PADRÃO [TonalidadeMenorNOTAX.Variação] .Ex:[mX50] [mX50V1]
% OBS: Variações são alterações do acorde em relação ao campo harmônico.
%-----------------------------------------------------------------
% TIPOS DE VARIAÇÂO DOS ACORDES:
% V1 - ACORDE MENOR (m)
% V2 - ACORDE MAIOR (M)
% V3 - ACORDE MEIO TOM ABAIXO (Bemois)
% V4 - ACORDE COM QUARTA (C4)
% V5 - ACORDE COM QUINTA (C5)
% V6 - ACORDE COM SEXTA (C6)
% V7 - ACORDE COM SÉTIMA MENOR (C7)
% V8 - ACODE COM BAIXO DOIS TONS ACIMA (ex:D/F#)
% V9 - ACORDE COM NONA (C9)
% V10 - ACORDE MEIO TOM ACIMA (Sustenidos)
% V11 - ACORDE COM SÉTIMA MAIOR (C7M)
% V12 - ACORDE SUSPENSO (Sus)
% V13 - ACORDE COM VARIAÇÃO DIVERSA
%=================================================================
\endsong
%=================================================================
\begin{comment}

\end{comment}