%=================================================================
\songcolumns{2}
\beginsong
{No Dia em que sai de casa %TÍTULO
}[by={Zezé Di Camargo e Luciano %ARTISTA
},album={@walyssondosreis},
id={\href{https://music.youtube.com/watch?v=iSH1B-zK1nc&feature=share %LINK
}{ST0028 %COD.ID.: XXNNNN
}},rev={5}, %REVISÃO
qr={https://music.youtube.com/watch?v=iSH1B-zK1nc&feature=share %LINK
}]
%-----------------------------------------------------------------
\tom{A}{A}
%=================================================================
%\newchords{verse1.XX0000X} % Registrador de Acordes em Sequência
%\newchords{chorus1.XX0000X} % Registrador de Acordes em Sequência
%-----------------------------------------------------------------
\seq{Intro}{A Bm E A E A}{1x}
%\act{}{}{}
%-----------------------------------------------------------------
%\beginverse \endverse
%\beginchorus \endchorus
\beginverse
No \[A]dia em que eu saí de casa
Minha mãe me disse: Filho, vem \[D]cá
Pa\[E]ssou a mão em meus cabelos
Olhou em meus olhos
Começou fa\[A]lar
Por onde você for, eu sigo com meu pensa\[A7]mento
Sempre onde esti\[D]ver
Em minhas orações, eu \[A]vou pedir a Deus
Que \[E]ilumine os passos \[A]seus \[A7]
\endverse

\beginchorus
Eu sei que \[D]ela nunca compreendeu
Os meus motivos de sair de \[A]lá
Mas ela sabe que depois que \[E]cresce
O filho vira \[D]passarinho e quer vo\[A]ar \[A7]
Eu bem que\[D]ria continuar ali
Mas o destino quis me contra\[A]riar
E o olhar de minha mãe na \[E]porta
Eu deixei, cho\[D]rando, a me a\[E]benço\[A]ar
\endchorus
\act{Riff Intro}{Repetir}{1x}
\beginverse
A \[A]minha mãe naquele dia
Me falou do mundo como \[A7]ele \[D]é
Pa\[E]rece que ela conhecia
Cada pedra que eu iria pôr o \[A]pé
Sempre ao lado do meu pai
Da pequena ci\[A7]dade, ela jamais sa\[D]iu
Ela me disse assim: Meu \[A]filho, vá com Deus
Que \[E]esse mundo inteiro é \[A]seu \[A7]
\endverse
\act{Refrão}{Repetir}{1x}
\beginverse
... E o olhar de minha mãe na \[E]porta
Eu deixei, cho\[D]rando, a me a\[E]benço\[A]ar
\endverse
\act{Verso 3}{Repetir}{+1x}

%-----------------------------------------------------------------
\vspace{4em} % Regulador de Espaçamento
%-----------------------------------------------------------------
\begin{comment}
\lstset{basicstyle=\scriptsize\bf} % Parâmetros da TAB
%-----------------------------------------------------------------
\tab{Solo 1}
\begin{lstlisting}
E|-----------------------------------------------------|
B|-----------------------------------------------------|
G|-----------------------------------------------------|
D|-----------------------------------------------------|
A|-----------------------------------------------------|
E|-----------------------------------------------------|
\end{lstlisting}
%-----------------------------------------------------------------
\end{comment}
%=================================================================

\color{drawChord}\gtab{\color{nameChord} A}{~:X02220}% 
\color{drawChord}\gtab{\color{nameChord} A7}{~:X02020}% 
\color{drawChord}\gtab{\color{nameChord} Bm}{2:X02210}% 
\color{drawChord}\gtab{\color{nameChord} D}{~:XX0232}% 
\color{drawChord}\gtab{\color{nameChord} E}{~:022100}% 

%=================================================================
% PADRÃO: [TonalidadeMaior+NOTAX+Variações] .Ex:[E0] [E7V1V7]
% OBS: Variações são alterações do acorde em relação ao campo harmônico.
%-----------------------------------------------------------------
% Tipos de Variações de Acordes:
% V0 - Variação Diversa
% V1 - Menor (m)
% V2 - Maior (M)
% V3 - Meio Tom Abaixo (Bemol)
% V4 - Com Quarta (ex:C4)
% V5 - Com Quinta (ex:C5)
% V6 - Com Sexta (ex:C6)
% V7 - Com Sétima Menor (ex:C7)
% V8 - Com baixo dois Tons Acima (ex:D/F#)
% V9 - Com Nona (ex:C9)
% V10 - Meio Tom Acima (Sustenido)
% V11 - Com Sétima Maior (ex:C7M)
% V12 - Suspenso (Sus)
% V13 - Com baixo dois Tons e Meio Acima (ex:A/E)
% V14 - Com baixo um Tom e Meio Acima (ex:D9/F) 
% V15 - Meio-Diminuto (m7b5)
% N15 - NÃO Meio-Diminuto
% V16 - Diminuto (º)
% N16 - NÃO Diminuto
% V17 - Com baixo um Tom Acima (ex: C/D)
% V18 - Com baixo um Tom Abaixo (ex: Em/D)
% V19 - Com baixo dois Tons e meio Abaixo (ex: G/D)
%=================================================================
%\vspace{4em} % Regulador de Espaçamento
%=================================================================
\endsong
%=================================================================
