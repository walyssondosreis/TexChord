%=================================================================
\songcolumns{1}
\beginsong
{Marca da Promessa %TÍTULO
}[by={Ministerio Trazendo a Arca %ARTISTA
},album={@walyssondosreis},
id={GB0061 %COD.ID.: GB0000
},rev={0}, %REVISÃO
qr={https://drive.google.com/open?id=1LKhNfXWQRHltHccTiu40lde8woqovnhI %LINK
}]
%-----------------------------------------------------------------
\tom{X1}{G}
%=================================================================
%\newchords{verse1.GB0000X} % Registrador de Acordes em Sequência
%\newchords{chorus1.GB0000X} % Registrador de Acordes em Sequência
%-----------------------------------------------------------------
%\seq{Intro}{}{}
%-----------------------------------------------------------------
%\beginverse* \endverse
%\beginchorus \endchorus
\beginverse*
Se tentam destruir-me zombando da minha fé
E até tramam contra mim
Querem entulhar meus poços
Querem frustrar meus sonhos e me fazer desistir
\endverse
\beginverse*
Mas quem vai apagar, o selo que há em mim
A marca da promessa, que Ele me fez
E quem vai me impedir, se decidido estou,
Pois quem me prometeu é fiel pra cumprir
\endverse
\beginchorus
O meu Deus nunca falhará, eu sei que chegará minha vez
Minha sorte Ele mudará, diante dos meus olhos
\endchorus
\beginverse*
Prepara-me uma mesa na presença dos meus inimigos
Unge minha cabeça e o meu cálice faz transbordar
\endverse


% Verso de preenchimento
\beginverse*\color{white}
.
.
.
\endverse
%-----------------------------------------------------------------
\begin{comment}
\lstset{basicstyle=\scriptsize\bf} % Parâmetros da TAB
%-----------------------------------------------------------------
\tab{Solo 1}
\begin{lstlisting}
E|-----------------------------------------------------|
B|-----------------------------------------------------|
G|-----------------------------------------------------|
D|-----------------------------------------------------|
A|-----------------------------------------------------|
E|-----------------------------------------------------|
\end{lstlisting}
%-----------------------------------------------------------------
\end{comment}
%=================================================================
\vspace{2em} 
%-----------------------------------------------------------------
\color{drawChord}\gtab{\color{nameChord} X}{}% 
\color{drawChord}\gtab{\color{nameChord} X}{}% 
\color{drawChord}\gtab{\color{nameChord} X}{}% 
\color{drawChord}\gtab{\color{nameChord} X}{}% 
%-----------------------------------------------------------------
% PADRÃO: [TonalidadeMaior+NOTAX+Variações] .Ex:[X50] [X57V1V7]
% OBS: Variações são alterações do acorde em relação ao campo harmônico.
%-----------------------------------------------------------------
% Tipos de Variações de Acordes:
% V0 - Variação Diversa
% V1 - Menor (m)
% V2 - Maior (M)
% V3 - Meio Tom Abaixo (Bemol)
% V4 - Com Quarta (ex:C4)
% V5 - Com Quinta (ex:C5)
% V6 - Com Sexta (ex:C6)
% V7 - Com Sétima Menor (ex:C7)
% V8 - Com baixo dois Tons Acima (ex:D/F#)
% V9 - Com Nona (ex:C9)
% V10 - Meio Tom Acima (Sustenido)
% V11 - Com Sétima Maior (ex:C7M)
% V12 - Suspenso (Sus)
% V13 - Com baixo dois Tons e Meio Acima (ex:A/E)
% V14 - Com baixo um Tom e Meio Acima (ex:D9/F) 
% V15 - Meio-Diminuto (m7b5)
% N15 - NÃO Meio-Diminuto
% V16 - Diminuto (º)
% N16 - NÃO Diminuto
%=================================================================
\endsong
%=================================================================
\begin{comment}

\end{comment}