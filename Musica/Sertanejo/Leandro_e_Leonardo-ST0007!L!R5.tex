%=================================================================
\songcolumns{2}
\beginsong
{Temporal de Amor %TÍTULO
}[by={Leandro e Leonardo %ARTISTA
},album={@walyssondosreis},
id={\href{https://music.youtube.com/watch?v=O7q6aepVnLU&feature=share %LINK
}{ST0007 %COD.ID.: XXNNNN
}},rev={5}, %REVISÃO
qr={https://music.youtube.com/watch?v=O7q6aepVnLU&feature=share %LINK
}]
%-----------------------------------------------------------------
\tom{}{A}
%=================================================================
%\newchords{verse1.XX0000X} % Registrador de Acordes em Sequência
%\newchords{chorus1.XX0000X} % Registrador de Acordes em Sequência
%-----------------------------------------------------------------
\seq{Intro}{...}{1x}
%\act{}{}{}
%-----------------------------------------------------------------
%\beginverse \endverse
%\beginchorus \endchorus
\beginverse
Chuva no telhado
Vento no portão
E eu aqui
Nesta solidão
\endverse

\beginverse
Fecho a janela
Tá frio o nosso quarto
E eu aqui
Sem o teu abraço
\endverse

\beginverse
Doido pra sentir seu cheiro
Doido pra sentir seu gosto
Louco pra beijar seu beijo
Matar a saudade
Esse meu desejo
\endverse

\beginverse
Vê se não demora muito
Coração tá reclamando
Traga logo o seu carinho
Tô aqui sozinho
Tô te esperando
\endverse

\beginchorus
Quando você chegar
Tira essa roupa molhada
Quero ser a toalha
E o seu cobertor
Quando você chegar
Manda a saudade sair
Vai trovejar, vai cair
Um temporal de amor
\endchorus
\seq{Solo}{...}{1x}
\act{Verso 3}{Retomar}{1x}
\act{Refrão}{Repetir}{+1x}
\beginverse
...Um temporal de amor
Um temporal de amor
\endverse

%-----------------------------------------------------------------
\vspace{4em} % Regulador de Espaçamento
%-----------------------------------------------------------------
\begin{comment}
\lstset{basicstyle=\scriptsize\bf} % Parâmetros da TAB
%-----------------------------------------------------------------
\tab{Solo 1}
\begin{lstlisting}
E|-----------------------------------------------------|
B|-----------------------------------------------------|
G|-----------------------------------------------------|
D|-----------------------------------------------------|
A|-----------------------------------------------------|
E|-----------------------------------------------------|
\end{lstlisting}
%-----------------------------------------------------------------
\end{comment}
%=================================================================
\begin{comment}

\color{drawChord}\gtab{\color{nameChord} X}{}% 
\color{drawChord}\gtab{\color{nameChord} X}{}% 
\color{drawChord}\gtab{\color{nameChord} X}{}% 
\color{drawChord}\gtab{\color{nameChord} X}{}% 

\end{comment}
%=================================================================
% PADRÃO: [TonalidadeMaior+NOTAX+Variações] .Ex:[X50] [X57V1V7]
% OBS: Variações são alterações do acorde em relação ao campo harmônico.
%-----------------------------------------------------------------
% Tipos de Variações de Acordes:
% V0 - Variação Diversa
% V1 - Menor (m)
% V2 - Maior (M)
% V3 - Meio Tom Abaixo (Bemol)
% V4 - Com Quarta (ex:C4)
% V5 - Com Quinta (ex:C5)
% V6 - Com Sexta (ex:C6)
% V7 - Com Sétima Menor (ex:C7)
% V8 - Com baixo dois Tons Acima (ex:D/F#)
% V9 - Com Nona (ex:C9)
% V10 - Meio Tom Acima (Sustenido)
% V11 - Com Sétima Maior (ex:C7M)
% V12 - Suspenso (Sus)
% V13 - Com baixo dois Tons e Meio Acima (ex:A/E)
% V14 - Com baixo um Tom e Meio Acima (ex:D9/F) 
% V15 - Meio-Diminuto (m7b5)
% N15 - NÃO Meio-Diminuto
% V16 - Diminuto (º)
% N16 - NÃO Diminuto
% V17 - Com baixo um Tom Acima (ex: C/D)
% V18 - Com baixo um Tom Abaixo (ex: Em/D)
% V19 - Com baixo dois Tons e meio Abaixo (ex: G/D)
%=================================================================
%\vspace{4em} % Regulador de Espaçamento
%=================================================================
\endsong
%=================================================================
