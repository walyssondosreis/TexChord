%=================================================================
\songcolumns{2}
\beginsong
{Foi na Cruz %TÍTULO
}[by={André Valadão %ARTISTA
},album={@walyssondosreis},
id={GB0154 %COD.ID.: GB0000
(Rev.0) %REVISÃO.: 0...N
}]
%-----------------------------------------------------------------
\tom{X1}
%=================================================================
%\newchords{verse0.GB0000} % Registrador de Acordes em Sequência
%-----------------------------------------------------------------
%\seq{Intro}{}{}
%-----------------------------------------------------------------
%\beginverse* \endverse
%\beginchorus \endchorus

\beginverse*
Oh, quão \[X1]cego eu andei e per\[X5]dido vaguei
Longe, \[X4]longe do \[X5]meu Salva\[X1]dor \[X1V7]
Mas da \[X4]glória des\[X4V1]ceu e Seu \[X1]sangue ver\[X6]teu
Pra sal\[X2]var um tão \[X5]pobre peca\[X1]dor \[(X5)]
\endverse

\beginchorus
Foi na ^cruz, foi na cruz
Onde um ^dia eu vi
Meus pe^cados casti^gados em Je^sus ^
Foi ^ali, pela ^fé, que meus ^olhos a^bri
E a^gora me a^legro em Sua ^luz ^
\endchorus

\beginverse*
Eu ou^via falar dessa ^graça sem par
Que do ^céu ^trouxe-nos ^Jesus ^
Mas eu ^surdo me ^fiz, conver^ter-me não ^quis
Ao Se^nhor que por ^mim morreu na ^cruz ^
\endverse

\beginverse*
Mas um ^dia senti meus pe^cados e vi
Sobre ^mim o cas^tigo da ^lei ^
Mas de^pressa fu^gi, em Je^sus me escon^di
E re^fúgio se^guro nele a^chei ^
\endverse

\beginverse*
Oh, que ^grande prazer inun^dou o meu ser
Conhe^cendo esse ^tão grande a^mor ^
Que le^vou meu Je^sus ao so^frer lá na ^cruz
Pra sal^var um tão ^pobre peca^dor ^
\endverse

%=================================================================
\vspace{2em}
%-----------------------------------------------------------------
\gtab{\color{black} X1}{}% A [X1]
\gtab{\color{black} X1V7}{}% A7 [X1V7]
\gtab{\color{black} X2}{}% Bm [X2]
\gtab{\color{black} X4}{}% D [X4]
\gtab{\color{black} X4V1}{}% Dm [X4V1]
\gtab{\color{black} X5}{}\\% E [X5]
\gtab{\color{black} X6}{}% F#m [X6]
%-----------------------------------------------------------------
% PADRÃO [TonalidadeMaiorNOTAX.Variação] .Ex:[X50] [X50V1]
% PADRÃO [TonalidadeMenorNOTAX.Variação] .Ex:[mX50] [mX50V1]
% OBS: Variações são alterações do acorde em relação ao campo harmônico.
%-----------------------------------------------------------------
% TIPOS DE VARIAÇÂO DOS ACORDES:
% V0 - ACORDE COM VARIAÇÃO DIVERSA
% V1 - ACORDE MENOR (m)
% V2 - ACORDE MAIOR (M)
% V3 - ACORDE MEIO TOM ABAIXO (Bemois)
% V4 - ACORDE COM QUARTA (C4)
% V5 - ACORDE COM QUINTA (C5)
% V6 - ACORDE COM SEXTA (C6)
% V7 - ACORDE COM SÉTIMA MENOR (C7)
% V8 - ACORDE COM BAIXO DOIS TONS ACIMA (D/F#)
% V9 - ACORDE COM NONA (C9)
% V10 - ACORDE MEIO TOM ACIMA (Sustenidos)
% V11 - ACORDE COM SÉTIMA MAIOR (C7M)
% V12 - ACORDE SUSPENSO (Sus)
% V13 - ACORDE COM BAIXO DOIS TONS E MEIO ACIMA (A/E)
% V14 - ACORDE UM TOM E MEIO ACIMA (D9/F)
%=================================================================
\endsong
%=================================================================
\begin{comment}

\end{comment}