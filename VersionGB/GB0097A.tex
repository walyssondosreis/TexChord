%=================================================================
\songcolumns{2}
\beginsong
{Rede Ao Mar %TÍTULO
}[by={Adoração e Adoradores %ARTISTA
},album={@walyssondosreis},
id={GB0097 %COD.ID.: GB0000
},rev={3}, %REVISÃO
qr={https://drive.google.com/open?id=10vODSbP2UWRX0JrdK66ypEzU81J2Gg_4 %LINK
}]
%-----------------------------------------------------------------
\tom{A}{A}
%=================================================================
%\newchords{verse1.GB0000X} % Registrador de Acordes em Sequência
%\newchords{chorus1.GB0000X} % Registrador de Acordes em Sequência
%-----------------------------------------------------------------
\seq{Intro}{A E D}{2x}
\seq{Intro}{F\#m E D}{2x}
%-----------------------------------------------------------------
%\beginverse* \endverse
%\beginchorus \endchorus
\beginverse
\[A] Não po\[Bm]dia enten\[F\#m]der \[D] 
\[A] Que vo\[Bm]cê ia me que\[F\#m]rer \[D]
\[A] Lançou a \[Bm]rede ao mar \[F\#m]
E que\[D]rendo me pe\[A]gar \[Bm]
Pegou meu \[F\#m]coração \[D]
\endverse
\seq{Riff 1}{A E D}{2x}
\beginverse
^ Hoje ^eu estou a^qui ^ 
^ Pois vo^cê me esco^lheu ^
^ Agora ^posso enten^der 
E o ^mesmo eu vou fa^zer
Vou lançar ^
A minha ^rede ao mar ^
\endverse
\beginchorus
\[A] Vou lan\[E]çar a minha \[D]rede ao mar
\[A] Muitas \[E]almas também \[D]vou ganhar
\[F\#m] Tantas \[E]que eu não pode\[D]rei contar
\[F\#m]Almas como as \[E]areias do \[D]mar
\chordsoff(Almas como as areias do mar)
\chordson
\endchorus
\seq{Riff 1}{A E D}{}
\act{Repetir}{Verso 1}{1x}
\act{Repetir}{Verso 2}{1x}
\act{Repetir}{Refrão}{2x}
\beginverse
\[F\#m]Ide, \[E] fazei dis\[D]cípulos
De \[F\#m]todas \[E] as na\[D]ções \[(E)]
\endverse
\act{Repetir}{Verso 3}{+4x}
\beginchorus
\act{Variar}{+1 Tom}{}
\[B] Vou lan\[F\#]çar a minha \[E]rede ao mar
\[B] Muitas \[F\#]almas também \[E]vou ganhar
\[G\#m] Tantas \[F\#]que eu não pode\[E]rei contar
\[G\#m]Almas como as \[F\#]areias do \[E]mar
\chordsoff(Almas como as areias do mar)
\chordson
\endchorus
\act{Repetir}{Refrão}{+1x}
\beginverse
...\[G\#m]Almas como as \[F\#]areias do \[E]mar
\chordsoff(Almas como as areias do mar)
\chordson \[G\#m]Almas como as \[F\#]areias do \[E]mar
\chordsoff(Almas como as areias do mar)
\chordson
\endverse
%-----------------------------------------------------------------
\vspace{4em} % Regulador de Espaçamento
%-----------------------------------------------------------------
\begin{comment}
\lstset{basicstyle=\scriptsize\bf} % Parâmetros da TAB
%-----------------------------------------------------------------
\tab{Solo 1}
\begin{lstlisting}
E|-----------------------------------------------------|
B|-----------------------------------------------------|
G|-----------------------------------------------------|
D|-----------------------------------------------------|
A|-----------------------------------------------------|
E|-----------------------------------------------------|
\end{lstlisting}
%-----------------------------------------------------------------
\end{comment}
%=================================================================
 
%-----------------------------------------------------------------
\color{drawChord}\gtab{\color{nameChord} A}{}% 
\color{drawChord}\gtab{\color{nameChord} B}{}% 
\color{drawChord}\gtab{\color{nameChord} Bm}{}% 
\color{drawChord}\gtab{\color{nameChord} D}{}%
\color{drawChord}\gtab{\color{nameChord} E}{}% 
\color{drawChord}\gtab{\color{nameChord} F\#}{}\\% 
\color{drawChord}\gtab{\color{nameChord} F\#m}{}%
\color{drawChord}\gtab{\color{nameChord} G\#m}{}% 
%-----------------------------------------------------------------
% PADRÃO: [TonalidadeMaior+NOTAX+Variações] .Ex:[X50] [X57V1V7]
% OBS: Variações são alterações do acorde em relação ao campo harmônico.
%-----------------------------------------------------------------
% Tipos de Variações de Acordes:
% V0 - Variação Diversa
% V1 - Menor (m)
% V2 - Maior (M)
% V3 - Meio Tom Abaixo (Bemol)
% V4 - Com Quarta (ex:C4)
% V5 - Com Quinta (ex:C5)
% V6 - Com Sexta (ex:C6)
% V7 - Com Sétima Menor (ex:C7)
% V8 - Com baixo dois Tons Acima (ex:D/F#)
% V9 - Com Nona (ex:C9)
% V10 - Meio Tom Acima (Sustenido)
% V11 - Com Sétima Maior (ex:C7M)
% V12 - Suspenso (Sus)
% V13 - Com baixo dois Tons e Meio Acima (ex:A/E)
% V14 - Com baixo um Tom e Meio Acima (ex:D9/F) 
% V15 - Meio-Diminuto (m7b5)
% N15 - NÃO Meio-Diminuto
% V16 - Diminuto (º)
% N16 - NÃO Diminuto
%=================================================================
\endsong
%=================================================================
\begin{comment}

\end{comment}