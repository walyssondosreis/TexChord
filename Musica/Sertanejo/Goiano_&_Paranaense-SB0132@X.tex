%=================================================================
\songcolumns{2}
\beginsong
{O Poder do Criador %TÍTULO
}[by={Goiano e Paranaense %ARTISTA
},album={@walyssondosreis},
id={\href{https://music.youtube.com/watch?v=jOADeghWRhA&feature=share
}{ SB0132 %COD.ID.: XXNNNN
}},rev={4}, %REVISÃO
qr={https://music.youtube.com/watch?v=jOADeghWRhA&feature=share %LINK
}]
%-----------------------------------------------------------------
\tom{X1}{Bb}
%=================================================================
%\newchords{verse1.XX0000X} % Registrador de Acordes em Sequência
%\newchords{chorus1.XX0000X} % Registrador de Acordes em Sequência
%-----------------------------------------------------------------
\seq{Intro}{X1  X5  X1  X17  X4 | X1  X5  X1  X5  X1}{1x}
\act{Solo}{Parte 1}{1x}
\act{Solo}{Parte 2}{1x}

%-----------------------------------------------------------------
\lstset{basicstyle=\scriptsize\bf} % Parâmetros da TAB

%-----------------------------------------------------------------
\beginverse 
\[X1] Hora triste foi aquela
Que Jesus Cristo fa\[X5]lou
\[X5] Mãe está chegando a hora
A senhora fica e eu \[X1]vou
\[X1] Com certeza mãe e filho
Neste momento cho\[X5]rou
\[X5] Hora triste dolo\[X1]rida 
Porque a dor da despe\[X5]dida
Só co\[X4]nhece \[X5]quem pa\[X1]ssou
\endverse
\act{Solo: Parte 2}{Repetir}{1x}
\beginverse
\[X1] Maria disse meu filho 
Faz tudo que o Pai man\[X5]dou
\[X5] Pra salvar a humanidade
Ele lhe determi\[X1]nou
\[X1] Com as lágrimas caindo
O seu rosto ele bei\[X5]jou
\[X5] Pra cumprir a profe\[X1]cia 
Naquele instante o Me\[X5]ssias
Todo \[X4]pecado \[X5]abra\[X1]çou
\endverse
\act{Solo: Parte 1 | Parte 2}{Repetir}{1x}
\beginverse
\[X1] Nas margens do Rio Jordão
Jesus Cristo cami\[X5]nhou
\[X5] Para encontrar João
Aquele que testemu\[X1]nhou
\[X1] O encontro foi tão lindo
Que o povo se emocio\[X5]nou
\[X5] Também foi nessa vi\[X1]sita
Que nas mãos de João Ba\[X5]tista
Jesus \[X4]Cristo \[X5]bati\[X1]zou
\endverse
\act{Solo: Parte 2}{Repetir}{1x}
\beginverse
\[X1] Na mesa da Santa Ceia
Jesus Cristo orde\[X5]nou
\[X5] Ensine os meus mandamentos
Que onde está meu Pai eu \[X1]vou
\[X1] Se o mundo lhes odiar
Também já me odi\[X5]ou
\[X5] Faça o bem sem ver a \[X1]quem 
A sua recompensa \[X5]vem
É o que Je\[X4]sus p\[X5]rofeti\[X1]zou
\endverse

%-----------------------------------------------------------------
\vspace{4em} % Regulador de Espaçamento
%=================================================================
\color{drawChord}\gtab{\color{black} X5}{}
\color{drawChord}\gtab{\color{black} X1}{}
\color{drawChord}\gtab{\color{black} X4}{}

%=================================================================
% PADRÃO: [TonalidadeMaior+NOTAX+Variações] .Ex:[X50] [X57V1V7]
% OBS: Variações são alterações do acorde em relação ao campo harmônico.
%-----------------------------------------------------------------
% Tipos de Variações de Acordes:
% V0 - Variação Diversa
% V1 - Menor (m)
% V2 - Maior (M)
% V3 - Meio Tom Abaixo (Bemol)
% V4 - Com Quarta (ex:C4)
% V5 - Com Quinta (ex:C5)
% V6 - Com Sexta (ex:C6)
% V7 - Com Sétima Menor (ex:C7)
% V8 - Com baixo dois Tons Acima (ex:D/F#)
% V9 - Com Nona (ex:C9)
% V10 - Meio Tom Acima (Sustenido)
% V11 - Com Sétima Maior (ex:C7M)
% V12 - Suspenso (Sus)
% V13 - Com baixo dois Tons e Meio Acima (ex:A/E)
% V14 - Com baixo um Tom e Meio Acima (ex:D9/F) 
% V15 - Meio-Diminuto (m7b5)
% N15 - NÃO Meio-Diminuto
% V16 - Diminuto (º)
% N16 - NÃO Diminuto
% V17 - Com baixo um Tom Acima (ex: C/D)
% V18 - Com baixo um Tom Abaixo (ex: Em/D)
% V19 - Com baixo dois Tons e meio Abaixo (ex: G/D)
%=================================================================
%\vspace{4em} % Regulador de Espaçamento
%=================================================================
\endsong
%=================================================================
