%=================================================================
\songcolumns{3}
\beginsong
{Meu Velho Pai %TÍTULO
}[by={Léo Canhoto e Robertinho %ARTISTA
},album={@walyssondosreis},
id={\href{https://music.youtube.com/watch?v=0UIv8dv9O5I&feature=share %LINK
}{ST0141 %COD.ID.: XXNNNN
}},rev={0}, %REVISÃO
qr={https://music.youtube.com/watch?v=0UIv8dv9O5I&feature=share %LINK
}]
%-----------------------------------------------------------------
\tom{}{}
%=================================================================
%\newchords{verse1.XX0000X} % Registrador de Acordes em Sequência
%\newchords{chorus1.XX0000X} % Registrador de Acordes em Sequência
%-----------------------------------------------------------------
%\seq{Intro}{}{}
%\act{}{}{}
%-----------------------------------------------------------------
%\beginverse \endverse
%\beginchorus \endchorus
\beginverse
Meu velho pai, preste atenção no que lhe digo
Meu pobre papai querido
Enxugue as lágrimas do rosto
Porque, papai, que você chora tão sozinho
Me conta, meu papaizinho
O que lhe causa desgosto
\endverse
\beginverse
Estou notando que você está cansado
Meu pobre velho adorado, é seu filho que está falando
Quero saber qual é a tristeza que existe
Não quero ver você triste
Por que é que está chorando?
\endverse
\beginverse
Quando lhe vejo, tão tristonho desse jeito
Sinto estremecer meu peito ao pulsar meu coração
Meu pobre pai, você sofreu pra me criar
Agora eu vou lhe cuidar
Esta é minha obrigação
\endverse
\beginverse
Não tenha medo, meu velhinho adorado
Estarei sempre ao seu lado, não lhe deixarei jamais
Eu sou o sangue do seu sangue, papaizinho
Não vou lhe deixar sozinho, não tenha medo, meu pai
\endverse
\beginverse
Você sofreu quando eu era ainda criança
A sua grande esperança era me ver homem formado
Eu fiquei grande, estou seguindo o meu caminho
E você ficou velhinho, mas estou sempre ao seu lado
\endverse
\beginverse
Meu pobre pai, seus passos longos silenciaram
Seus cabelos branquiaram, seu olhar se escureceu
A sua voz quase que não se ouve mais
Não tenha medo, meu pai, quem cuida de você sou eu
\endverse
\beginchorus
Meu papaizinho, não precisa mais chorar
Saiba que não vou deixar você sozinho, abandonado
Eu sou seu guia, sou seu tempo, sou seus passos
Sou sua luz e sou seus braços
Sou seu filho idolatrado
\endchorus

%-----------------------------------------------------------------
\vspace{4em} % Regulador de Espaçamento
%-----------------------------------------------------------------
\begin{comment}
\lstset{basicstyle=\scriptsize\bf} % Parâmetros da TAB
%-----------------------------------------------------------------
\tab{Solo 1}
\begin{lstlisting}
E|-----------------------------------------------------|
B|-----------------------------------------------------|
G|-----------------------------------------------------|
D|-----------------------------------------------------|
A|-----------------------------------------------------|
E|-----------------------------------------------------|
\end{lstlisting}
%-----------------------------------------------------------------
\end{comment}
%=================================================================
\begin{comment}

\color{drawChord}\gtab{\color{nameChord} X}{}% 
\color{drawChord}\gtab{\color{nameChord} X}{}% 
\color{drawChord}\gtab{\color{nameChord} X}{}% 
\color{drawChord}\gtab{\color{nameChord} X}{}% 

\end{comment}
%=================================================================
% PADRÃO: [TonalidadeMaior+NOTAX+Variações] .Ex:[X50] [X57V1V7]
% OBS: Variações são alterações do acorde em relação ao campo harmônico.
%-----------------------------------------------------------------
% Tipos de Variações de Acordes:
% V0 - Variação Diversa
% V1 - Menor (m)
% V2 - Maior (M)
% V3 - Meio Tom Abaixo (Bemol)
% V4 - Com Quarta (ex:C4)
% V5 - Com Quinta (ex:C5)
% V6 - Com Sexta (ex:C6)
% V7 - Com Sétima Menor (ex:C7)
% V8 - Com baixo dois Tons Acima (ex:D/F#)
% V9 - Com Nona (ex:C9)
% V10 - Meio Tom Acima (Sustenido)
% V11 - Com Sétima Maior (ex:C7M)
% V12 - Suspenso (Sus)
% V13 - Com baixo dois Tons e Meio Acima (ex:A/E)
% V14 - Com baixo um Tom e Meio Acima (ex:D9/F) 
% V15 - Meio-Diminuto (m7b5)
% N15 - NÃO Meio-Diminuto
% V16 - Diminuto (º)
% N16 - NÃO Diminuto
% V17 - Com baixo um Tom Acima (ex: C/D)
% V18 - Com baixo um Tom Abaixo (ex: Em/D)
% V19 - Com baixo dois Tons e meio Abaixo (ex: G/D)
%=================================================================
%\vspace{4em} % Regulador de Espaçamento
%=================================================================
\endsong
%=================================================================
