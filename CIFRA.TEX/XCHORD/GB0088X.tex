%=================================================================
\songcolumns{2}
\beginsong
{Pra Sempre %TÍTULO
}[by={Fernandinho  %ARTISTA
},album={@walyssondosreis},
id={GB0088 %COD.ID.: GB0000
}] 
%-----------------------------------------------------------------
\tom{X1}
%=================================================================
%\newchords{verse0.GB0000} % Registrador de Acordes em Sequência
%-----------------------------------------------------------------
\seq{Intro}{X1 X5 X6 X4}{}
%-----------------------------------------------------------------
%\beginverse* \endverse
%\beginchorus \endchorus

\beginverse*
\[X1]O universo \[X5]chora, o Sol se apa\[X6V7]gou
Ali estava \[X4]morto o Salva\[X1]dor
\[X1]Seu corpo lá na \[X5]cruz, seu sangue derra\[X6V7]mou
O peso do pe\[X4]cado ele le\[X1]vou
Le\[X5]vou, le\[X2V7]vou \[X4]
\endverse

\beginverse* 
^Deus Pai o abando^nou, cessou seu respi^rar
Em trevas se encon^trou o Fi^lho
^A guerra come^çou, a morte Ele enfren^tou
Todo o poder das ^trevas vencido \[X1]foi
\endverse

\beginverse* 
^A terra estreme^ceu, sepulcro se a^briu
Nada vence^rá seu grande a^mor
^Ó morte, onde es^tás? O Rei ressusci^tou
Ele venceu pra ^sempre
\endverse

\beginchorus
Pra ^sempre exal^tado é
Pra ^sempre ado^rado é
Pra ^sempre Ele \[X5]vive
Ressusci\[X6]tou, ressusci\[X4]tou!
\endchorus

\beginverse* 
^Cantamos ale^luia! Cantamos ale^luia!
Cantamos ale^luia! O cordeiro ven^ceu!
\endverse

%-----------------------------------------------------------------
\begin{comment}
\lstset{basicstyle=\scriptsize\bf} % Parâmetros da TAB
%-----------------------------------------------------------------
\tab{Solo 1}
\begin{lstlisting}
E|-----------------------------------------------------|
B|-----------------------------------------------------|
G|-----------------------------------------------------|
D|-----------------------------------------------------|
A|-----------------------------------------------------|
E|-----------------------------------------------------|
\end{lstlisting}
%-----------------------------------------------------------------
\end{comment}
%=================================================================
\vspace{2em}
%-----------------------------------------------------------------
\gtab{\color{black} X1}{}% C [X1]
\gtab{\color{black} X2V7}{}% Dm7 [X2V7]
\gtab{\color{black} X4}{}% F [X4]
\gtab{\color{black} X5}{}% G [X5]
\gtab{\color{black} X6}{}% Am [X6]
\gtab{\color{black} X6V7}{}% Am7 [X6V7]
%-----------------------------------------------------------------
% PADRÃO [TonalidadeMaiorNOTAX.Variação] .Ex:[X50] [X50V1]
% PADRÃO [TonalidadeMenorNOTAX.Variação] .Ex:[mX50] [mX50V1]
% OBS: Variações são alterações do acorde em relação ao campo harmônico.
%-----------------------------------------------------------------
% TIPOS DE VARIAÇÂO DOS ACORDES:
% V0 - ACORDE COM VARIAÇÃO DIVERSA
% V1 - ACORDE MENOR (m)
% V2 - ACORDE MAIOR (M)
% V3 - ACORDE MEIO TOM ABAIXO (Bemois)
% V4 - ACORDE COM QUARTA (C4)
% V5 - ACORDE COM QUINTA (C5)
% V6 - ACORDE COM SEXTA (C6)
% V7 - ACORDE COM SÉTIMA MENOR (C7)
% V8 - ACORDE COM BAIXO DOIS TONS ACIMA (D/F#)
% V9 - ACORDE COM NONA (C9)
% V10 - ACORDE MEIO TOM ACIMA (Sustenidos)
% V11 - ACORDE COM SÉTIMA MAIOR (C7M)
% V12 - ACORDE SUSPENSO (Sus)
% V13 - ACORDE COM BAIXO DOIS TONS E MEIO ACIMA (A/E)
% V14 - ACORDE UM TOM E MEIO ACIMA (D9/F)
%=================================================================
\endsong
%=================================================================
\begin{comment}

\end{comment}