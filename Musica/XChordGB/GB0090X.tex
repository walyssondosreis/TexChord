%=================================================================
\songcolumns{1}
\beginsong
{Quando Ele Vem %TÍTULO
}[by={André Aquino %ARTISTA
},album={@walyssondosreis},
id={GB0090 %COD.ID.: GB0000
},rev={0}, %REVISÃO
qr={https://music.youtube.com/watch?v=V2HupH7IM0Q&feature=share %LINK
}]
%-----------------------------------------------------------------
\tom{X1}{A}
%=================================================================
%\newchords{verse1.GB0000X} % Registrador de Acordes em Sequência
%\newchords{chorus1.GB0000X} % Registrador de Acordes em Sequência
%-----------------------------------------------------------------
\seq{Intro}{X4V9 X1 X5}{2x}
%-----------------------------------------------------------------
%\beginverse* \endverse
%\beginchorus \endchorus
\beginverse*
\[X4V9] Nosso amigo Santo Espíri\[X1]to \[X5]
\[X4V9] Venha aquecer os cora\[X1]ções \[X5]
\[X4V9] Vem com cura e todo Seu po\[X1]der \[X5]
\[X4V9] Vem manifestar os Seus si\[X1]nais \[X5]
\endverse
\beginchorus
^ O que dizer?
O que fazer quando Ele ^vem a^qui?
^Mais que ser bem-vindo
Nós Te desejamos outra ^vez a^qui
\endchorus
\beginverse*
^ Nosso amigo Santo Espíri^to ^
^ Esperamos pelo seu so^prar ^
^ Com seu óleo venha nos un^gir ^
^ E com seu fogo vem nos bati^zar ^
\endverse
\seq{Riff 1}{X4V9 X1 X1V8}{}
\beginverse*
\[X4V9]Faz de novo, faz de novo
\[X1]O Seu povo \[X1V8]clama a Ti mais uma \[X4V9]vez
\endverse
%-----------------------------------------------------------------
\vspace{4em} % Regulador de Espaçamento
%-----------------------------------------------------------------
\begin{comment}
\lstset{basicstyle=\scriptsize\bf} % Parâmetros da TAB
%-----------------------------------------------------------------
\tab{Solo 1}
\begin{lstlisting}
E|-----------------------------------------------------|
B|-----------------------------------------------------|
G|-----------------------------------------------------|
D|-----------------------------------------------------|
A|-----------------------------------------------------|
E|-----------------------------------------------------|
\end{lstlisting}
%-----------------------------------------------------------------
\end{comment}
%=================================================================
 
%-----------------------------------------------------------------
\color{drawChord}\gtab{\color{nameChord} X1}{}% 
\color{drawChord}\gtab{\color{nameChord} X1V8}{}% 
\color{drawChord}\gtab{\color{nameChord} X4V9}{}% 
\color{drawChord}\gtab{\color{nameChord} X5}{}% 
%-----------------------------------------------------------------
% PADRÃO: [TonalidadeMaior+NOTAX+Variações] .Ex:[X50] [X57V1V7]
% OBS: Variações são alterações do acorde em relação ao campo harmônico.
%-----------------------------------------------------------------
% Tipos de Variações de Acordes:
% V0 - Variação Diversa
% V1 - Menor (m)
% V2 - Maior (M)
% V3 - Meio Tom Abaixo (Bemol)
% V4 - Com Quarta (ex:C4)
% V5 - Com Quinta (ex:C5)
% V6 - Com Sexta (ex:C6)
% V7 - Com Sétima Menor (ex:C7)
% V8 - Com baixo dois Tons Acima (ex:D/F#)
% V9 - Com Nona (ex:C9)
% V10 - Meio Tom Acima (Sustenido)
% V11 - Com Sétima Maior (ex:C7M)
% V12 - Suspenso (Sus)
% V13 - Com baixo dois Tons e Meio Acima (ex:A/E)
% V14 - Com baixo um Tom e Meio Acima (ex:D9/F) 
% V15 - Meio-Diminuto (m7b5)
% N15 - NÃO Meio-Diminuto
% V16 - Diminuto (º)
% N16 - NÃO Diminuto
%=================================================================
\endsong
%=================================================================
\begin{comment}

\end{comment}