%=================================================================
\songcolumns{1}
\beginsong
{Eu Me Rendo %TÍTULO
}[by={Leonardo Gonçalves  %ARTISTA
},album={@walyssondosreis},
id={GB0037 %COD.ID.: GB0000
}] 
%-----------------------------------------------------------------
\tom{X1}
%=================================================================
%\newchords{verse0.GB0000} % Registrador de Acordes em Sequência
%-----------------------------------------------------------------
%\seq{Intro}{}{}
%-----------------------------------------------------------------
%\beginverse* \endverse
%\beginchorus \endchorus

\beginverse* 
\[X1]A ti eu vou cla\[X5]mar
Pois tudo \[X6]vem de ti
E tudo es\[X4]tá em ti
\[X1]Por ti vou cami\[X5]nhar
Tu és a \[X6]direção
O sol a \[X4]me guiar
\[X2]Tudo pode pa\[X5]ssar
Teu a\[X6]mor ja\[X5]mais me de\[X4]ixará
\[X2]Sempre há de exis\[X5]tir
Novo ama\[X6]nhã prepa\[X5]rado pra \[X4]mim, \[X6]
Prepa\[X5]rado pra \[X4]mim \[(X5)]
\endverse

\beginchorus
Eu me \[X1]rendo aos teus \[X5]pés
És tudo \[X6]que eu preciso \[X4]pra viver
Eu me \[X1]lanço aos teus \[X5]braços
Onde en\[X6]contro meu re\[X4]fúgio
Je\[X1]su\[X3V7]s, eis-me a\[X6]qui \[X4]
Je\[X1]su\[X5]s, eis-me a\[X6]qui \[X4]
\endchorus


%=================================================================
\vspace{2em}
%-----------------------------------------------------------------
\gtab{\color{black} X1}{}% E [X1]
\gtab{\color{black} X2}{}% F#m [X2]
\gtab{\color{black} X3V7}{}% G#m7 [X3V7]
\gtab{\color{black} X4}{}% A [X4]
\gtab{\color{black} X5}{}% B [X5]
\gtab{\color{black} X6}{}% C#m [X6]
%-----------------------------------------------------------------
% PADRÃO [TonalidadeMaiorNOTAX.Variação] .Ex:[X50] [X50V1]
% PADRÃO [TonalidadeMenorNOTAX.Variação] .Ex:[mX50] [mX50V1]
% OBS: Variações são alterações do acorde em relação ao campo harmônico.
%-----------------------------------------------------------------
% TIPOS DE VARIAÇÂO DOS ACORDES:
% V0 - ACORDE COM VARIAÇÃO DIVERSA
% V1 - ACORDE MENOR (m)
% V2 - ACORDE MAIOR (M)
% V3 - ACORDE MEIO TOM ABAIXO (Bemois)
% V4 - ACORDE COM QUARTA (C4)
% V5 - ACORDE COM QUINTA (C5)
% V6 - ACORDE COM SEXTA (C6)
% V7 - ACORDE COM SÉTIMA MENOR (C7)
% V8 - ACORDE COM BAIXO DOIS TONS ACIMA (D/F#)
% V9 - ACORDE COM NONA (C9)
% V10 - ACORDE MEIO TOM ACIMA (Sustenidos)
% V11 - ACORDE COM SÉTIMA MAIOR (C7M)
% V12 - ACORDE SUSPENSO (Sus)
% V13 - ACORDE COM BAIXO DOIS TONS E MEIO ACIMA (A/E)
% V14 - ACORDE UM TOM E MEIO ACIMA (D9/F)
%=================================================================
\endsong
%=================================================================
\begin{comment}

\end{comment}